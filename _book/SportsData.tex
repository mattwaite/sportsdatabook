\PassOptionsToPackage{unicode=true}{hyperref} % options for packages loaded elsewhere
\PassOptionsToPackage{hyphens}{url}
%
\documentclass[]{book}
\usepackage{lmodern}
\usepackage{amssymb,amsmath}
\usepackage{ifxetex,ifluatex}
\usepackage{fixltx2e} % provides \textsubscript
\ifnum 0\ifxetex 1\fi\ifluatex 1\fi=0 % if pdftex
  \usepackage[T1]{fontenc}
  \usepackage[utf8]{inputenc}
  \usepackage{textcomp} % provides euro and other symbols
\else % if luatex or xelatex
  \usepackage{unicode-math}
  \defaultfontfeatures{Ligatures=TeX,Scale=MatchLowercase}
\fi
% use upquote if available, for straight quotes in verbatim environments
\IfFileExists{upquote.sty}{\usepackage{upquote}}{}
% use microtype if available
\IfFileExists{microtype.sty}{%
\usepackage[]{microtype}
\UseMicrotypeSet[protrusion]{basicmath} % disable protrusion for tt fonts
}{}
\IfFileExists{parskip.sty}{%
\usepackage{parskip}
}{% else
\setlength{\parindent}{0pt}
\setlength{\parskip}{6pt plus 2pt minus 1pt}
}
\usepackage{hyperref}
\hypersetup{
            pdftitle={Sports Data Analysis and Visualization},
            pdfauthor={By Matt Waite},
            pdfborder={0 0 0},
            breaklinks=true}
\urlstyle{same}  % don't use monospace font for urls
\usepackage{color}
\usepackage{fancyvrb}
\newcommand{\VerbBar}{|}
\newcommand{\VERB}{\Verb[commandchars=\\\{\}]}
\DefineVerbatimEnvironment{Highlighting}{Verbatim}{commandchars=\\\{\}}
% Add ',fontsize=\small' for more characters per line
\usepackage{framed}
\definecolor{shadecolor}{RGB}{248,248,248}
\newenvironment{Shaded}{\begin{snugshade}}{\end{snugshade}}
\newcommand{\AlertTok}[1]{\textcolor[rgb]{0.94,0.16,0.16}{#1}}
\newcommand{\AnnotationTok}[1]{\textcolor[rgb]{0.56,0.35,0.01}{\textbf{\textit{#1}}}}
\newcommand{\AttributeTok}[1]{\textcolor[rgb]{0.77,0.63,0.00}{#1}}
\newcommand{\BaseNTok}[1]{\textcolor[rgb]{0.00,0.00,0.81}{#1}}
\newcommand{\BuiltInTok}[1]{#1}
\newcommand{\CharTok}[1]{\textcolor[rgb]{0.31,0.60,0.02}{#1}}
\newcommand{\CommentTok}[1]{\textcolor[rgb]{0.56,0.35,0.01}{\textit{#1}}}
\newcommand{\CommentVarTok}[1]{\textcolor[rgb]{0.56,0.35,0.01}{\textbf{\textit{#1}}}}
\newcommand{\ConstantTok}[1]{\textcolor[rgb]{0.00,0.00,0.00}{#1}}
\newcommand{\ControlFlowTok}[1]{\textcolor[rgb]{0.13,0.29,0.53}{\textbf{#1}}}
\newcommand{\DataTypeTok}[1]{\textcolor[rgb]{0.13,0.29,0.53}{#1}}
\newcommand{\DecValTok}[1]{\textcolor[rgb]{0.00,0.00,0.81}{#1}}
\newcommand{\DocumentationTok}[1]{\textcolor[rgb]{0.56,0.35,0.01}{\textbf{\textit{#1}}}}
\newcommand{\ErrorTok}[1]{\textcolor[rgb]{0.64,0.00,0.00}{\textbf{#1}}}
\newcommand{\ExtensionTok}[1]{#1}
\newcommand{\FloatTok}[1]{\textcolor[rgb]{0.00,0.00,0.81}{#1}}
\newcommand{\FunctionTok}[1]{\textcolor[rgb]{0.00,0.00,0.00}{#1}}
\newcommand{\ImportTok}[1]{#1}
\newcommand{\InformationTok}[1]{\textcolor[rgb]{0.56,0.35,0.01}{\textbf{\textit{#1}}}}
\newcommand{\KeywordTok}[1]{\textcolor[rgb]{0.13,0.29,0.53}{\textbf{#1}}}
\newcommand{\NormalTok}[1]{#1}
\newcommand{\OperatorTok}[1]{\textcolor[rgb]{0.81,0.36,0.00}{\textbf{#1}}}
\newcommand{\OtherTok}[1]{\textcolor[rgb]{0.56,0.35,0.01}{#1}}
\newcommand{\PreprocessorTok}[1]{\textcolor[rgb]{0.56,0.35,0.01}{\textit{#1}}}
\newcommand{\RegionMarkerTok}[1]{#1}
\newcommand{\SpecialCharTok}[1]{\textcolor[rgb]{0.00,0.00,0.00}{#1}}
\newcommand{\SpecialStringTok}[1]{\textcolor[rgb]{0.31,0.60,0.02}{#1}}
\newcommand{\StringTok}[1]{\textcolor[rgb]{0.31,0.60,0.02}{#1}}
\newcommand{\VariableTok}[1]{\textcolor[rgb]{0.00,0.00,0.00}{#1}}
\newcommand{\VerbatimStringTok}[1]{\textcolor[rgb]{0.31,0.60,0.02}{#1}}
\newcommand{\WarningTok}[1]{\textcolor[rgb]{0.56,0.35,0.01}{\textbf{\textit{#1}}}}
\usepackage{longtable,booktabs}
% Fix footnotes in tables (requires footnote package)
\IfFileExists{footnote.sty}{\usepackage{footnote}\makesavenoteenv{longtable}}{}
\usepackage{graphicx,grffile}
\makeatletter
\def\maxwidth{\ifdim\Gin@nat@width>\linewidth\linewidth\else\Gin@nat@width\fi}
\def\maxheight{\ifdim\Gin@nat@height>\textheight\textheight\else\Gin@nat@height\fi}
\makeatother
% Scale images if necessary, so that they will not overflow the page
% margins by default, and it is still possible to overwrite the defaults
% using explicit options in \includegraphics[width, height, ...]{}
\setkeys{Gin}{width=\maxwidth,height=\maxheight,keepaspectratio}
\setlength{\emergencystretch}{3em}  % prevent overfull lines
\providecommand{\tightlist}{%
  \setlength{\itemsep}{0pt}\setlength{\parskip}{0pt}}
\setcounter{secnumdepth}{5}
% Redefines (sub)paragraphs to behave more like sections
\ifx\paragraph\undefined\else
\let\oldparagraph\paragraph
\renewcommand{\paragraph}[1]{\oldparagraph{#1}\mbox{}}
\fi
\ifx\subparagraph\undefined\else
\let\oldsubparagraph\subparagraph
\renewcommand{\subparagraph}[1]{\oldsubparagraph{#1}\mbox{}}
\fi

% set default figure placement to htbp
\makeatletter
\def\fps@figure{htbp}
\makeatother

\usepackage{booktabs}
\usepackage{etoolbox}
\makeatletter
\providecommand{\subtitle}[1]{% add subtitle to \maketitle
  \apptocmd{\@title}{\par {\large #1 \par}}{}{}
}
\makeatother
\usepackage[]{natbib}
\bibliographystyle{apalike}

\title{Sports Data Analysis and Visualization}
\providecommand{\subtitle}[1]{}
\subtitle{Code, data, visuals and the Tidyverse for journalists and other storytellers}
\author{By Matt Waite}
\date{July 29, 2019}

\begin{document}
\maketitle

{
\setcounter{tocdepth}{1}
\tableofcontents
}
\hypertarget{throwing-cold-water-on-hot-takes}{%
\chapter{Throwing cold water on hot takes}\label{throwing-cold-water-on-hot-takes}}

The 2018 season started out disastrously for the Nebraska Cornhuskers. The first game against a probably overmatched opponent? Called on account of an epic thunderstorm that plowed right over Memorial Stadium. The next game? Loss. The one following? Loss. The next four? All losses, after the fanbase was whipped into a hopeful frenzy by the hiring of Scott Frost, national title winning quarterback turned hot young coach come back home to save a mythical football program from the mediocrity it found itself mired in.

All that excitement lay in tatters.

On sports talk radio, on the sports pages and across social media and cafe conversations, one topic kept coming up again and again to explain why the team was struggling: Penalties. The team was just committing too many of them. In fact, six games and no wins into the season, they were dead last in the FBS penalty yards.

Worse yet for this line of reasoning? Nebraska won game 7, against Minnesota, committing only six penalties for 43 yards, just about half their average over the season. Then they won game 8 against FCS patsy Bethune Cookman, committing only five penalties for 35 yards. That's a whopping 75 yards less than when they were losing. See? Cut the penalties, win games screamed the radio show callers.

The problem? It's not true. Penalties might matter for a single drive. They may even throw a single game. But if you look at every top-level college football team since 2009, the number of penalty yards the team racks up means absolutely nothing to the total number of points they score. There's no relationship between them. Penalty yards have no discernible influence on points beyond just random noise.

Put this another way: If you were Scott Frost, and a major college football program was paying you \$5 million a year to make your team better, what should you focus on in practice? If you had growled at some press conference that you're going to work on penalties in practice until your team stops committing them, the results you'd get from all that wasted practice time would be impossible to separate from just random chance. You very well may reduce your penalty yards and still lose.

How do I know this? Simple statistics.

That's one of the three pillars of this book: Simple stats. The three pillars are:

\begin{enumerate}
\def\labelenumi{\arabic{enumi}.}
\tightlist
\item
  Simple, easy to understand statistics \ldots{}
\item
  \ldots{} extracted using simple code \ldots{}
\item
  \ldots{} visualized simply to reveal new and interesting things in sports.
\end{enumerate}

Do you need to be a math whiz to read this book? No.~I'm not one either. What we're going to look at is pretty basic, but that's also why it's so powerful.

Do you need to be a computer science major to write code? Nope. I'm not one of those either. But anyone can think logically, and write simple code that is repeatable and replicable.

Do you need to be an artist to create compelling visuals? I think you see where this is going. No.~I can barely draw stick figures, but I've been paid to make graphics in my career. With a little graphic design know how, you can create publication worthy graphics with code.

\hypertarget{requirements-and-conventions}{%
\section{Requirements and Conventions}\label{requirements-and-conventions}}

This book is all in the R statistical language. To follow along, you'll do the following:

\begin{enumerate}
\def\labelenumi{\arabic{enumi}.}
\item
  Install the R language on your computer. Go to the \href{https://www.r-project.org/}{R Project website}, click download R and select a mirror closest to your location. Then download the version for your computer.
\item
  Install \href{https://www.rstudio.com/products/rstudio/\#Desktop}{R Studio Desktop}. The free version is great.
\end{enumerate}

Going forward, you'll see passages like this:

\begin{Shaded}
\begin{Highlighting}[]
\KeywordTok{install.packages}\NormalTok{(}\StringTok{"tidyverse"}\NormalTok{)}
\end{Highlighting}
\end{Shaded}

Don't do it now, but that is code that you'll need to run in your R Studio. When you see that, you'll know what to do.

\hypertarget{about-this-book}{%
\section{About this book}\label{about-this-book}}

This book is the collection of class materials for the author's Sports Data Analysis and Visualization class at the University of Nebraska-Lincoln's College of Journalism and Mass Communications. There's some things you should know about it:

\begin{itemize}
\tightlist
\item
  It is free for students.
\item
  The topics will remain the same but the text is going to be constantly tinkered with.
\item
  What is the work of the author is copyright Matt Waite 2019.
\item
  The text is \href{https://creativecommons.org/licenses/by-nc-sa/4.0/}{Attribution-NonCommercial-ShareAlike 4.0 International} Creative Commons licensed. That means you can share it and change it, but only if you share your changes with the same license and it cannot be used for commercial purposes. I'm not making money on this so you can't either.\\
\item
  As such, the whole book -- authored in Bookdown -- is \href{https://github.com/mattwaite/sportsdatabook}{open sourced on Github}. Pull requests welcomed!
\end{itemize}

\hypertarget{the-very-basics}{%
\chapter{The very basics}\label{the-very-basics}}

R is a programming language, one specifically geared toward statistical analysis. Like all programming languages, it has certain built-in functions and you can interact with it in multiple ways. The first, and most basic, is the console.

\includegraphics[width=18.97in]{images/verybasics1}

Think of the console like talking directly to R. It's direct, but it has some drawbacks and some quirks we'll get into later. For now, try typing this into the console and hit enter:

\begin{Shaded}
\begin{Highlighting}[]
\DecValTok{2}\OperatorTok{+}\DecValTok{2}
\end{Highlighting}
\end{Shaded}

\begin{verbatim}
## [1] 4
\end{verbatim}

Congrats, you've run some code. It's not very complex, and you knew the answer before hand, but you get the idea. We can compute things. We can also store things. \textbf{In programming languages, these are called variables}. We can assign things to variables using \texttt{\textless{}-}. And then we can do things with them. \textbf{The \texttt{\textless{}-} is a called an assignment operator}.

\begin{Shaded}
\begin{Highlighting}[]
\NormalTok{number <-}\StringTok{ }\DecValTok{2}

\NormalTok{number }\OperatorTok{*}\StringTok{ }\NormalTok{number}
\end{Highlighting}
\end{Shaded}

\begin{verbatim}
## [1] 4
\end{verbatim}

Now assign a different number to the variable number. Try run \texttt{number\ *\ number} again. Get what you expected?

We can have as many variables as we can name. \textbf{We can even reuse them (but be careful you know you're doing that or you'll introduce errors)}. Try this in your console.

\begin{Shaded}
\begin{Highlighting}[]
\NormalTok{firstnumber <-}\StringTok{ }\DecValTok{1}
\NormalTok{secondnumber <-}\DecValTok{2} 

\NormalTok{(firstnumber }\OperatorTok{+}\StringTok{ }\NormalTok{secondnumber) }\OperatorTok{*}\StringTok{ }\NormalTok{secondnumber}
\end{Highlighting}
\end{Shaded}

\begin{verbatim}
## [1] 6
\end{verbatim}

\textbf{We can store anything in a variable}. A whole table. An array of numbers. A single word. A whole book. All the books of the 18th century. They're really powerful. We'll explore them at length.

\hypertarget{adding-libraries-part-1}{%
\section{Adding libraries, part 1}\label{adding-libraries-part-1}}

The real strength of any given programming language is the external libraries that power it. The base language can do a lot, but it's the external libraries that solve many specific problems -- even making the base language easier to use.

For this class, we're going to need several external libraries.

The first library we're going to use is called Swirl. So in the console, type \texttt{install.packages(\textquotesingle{}swirl\textquotesingle{})} and hit enter. That installs swirl.

Now, to use the library, type \texttt{library(swirl)} and hit enter. That loads swirl. Then type \texttt{swirl()} and hit enter. Now you're running swirl. Follow the directions on the screen. When you are asked, you want to install course 1 R Programming: The basics of programming in R. Then, when asked, you want to do option 1, R Programming, in that course.

When you are finished with the course -- it will take just a few minutes -- it will first ask you if you want credit on Coursera. You do not. Then type 0 to exit (it will not be very clear that's what you do when you are done).

\hypertarget{adding-libraries-part-2}{%
\section{Adding libraries, part 2}\label{adding-libraries-part-2}}

We'll mostly use two libraries for analysis -- \texttt{dplyr} and \texttt{ggplot2}. To get them, and several other useful libraries, we can install a single collection of libraries called the tidyverse. Type this into your console: \texttt{install.packages(\textquotesingle{}tidyverse\textquotesingle{})}

\textbf{NOTE}: This is a pattern. You should always install libraries in the console.

Then, to help us with learning and replication, we're going to use R Notebooks. So we need to install that library. Type this into your console: \texttt{install.packages(\textquotesingle{}rmarkdown\textquotesingle{})}

\hypertarget{notebooks}{%
\section{Notebooks}\label{notebooks}}

For the rest of the class, we're going to be working in notebooks. In notebooks, you will both run your code and explain each step, much as I am doing here.

To start a notebook, you click on the green plus in the top left corner and go down to R Notebook. Do that now.

\includegraphics[width=11.08in]{images/verybasics2}

You will see that the notebook adds a lot of text for you. It tells you how to work in notebooks -- and you should read it. The most important parts are these:

To add text, simply type. To add code you can click on the \emph{Insert} button on the toolbar or by pressing \emph{Cmd+Option+I} on Mac or \emph{Ctl+Alt+I} on Windows.

Highlight all that text and delete it. You should have a blank document. This document is called a R Markdown file -- it's a special form of text, one that you can style, and one you can include R in the middle of it. Markdown is a simple markup format that you can use to create documents. So first things first, let's give our notebook a big headline. Add this:

\texttt{\#\ My\ awesome\ notebook}

Now, under that, without any markup, just type This is my awesome notebook.

Under that, you can make text bold by writing \texttt{It\ is\ **really**\ awesome}.

If you want it italics, just do this on the next line: \texttt{No,\ it\textquotesingle{}s\ \_really\_\ awesome.\ I\ swear.}

To see what it looks like without the markup, click the Preview or Knit button in the toolbar. That will turn your notebook into a webpage, with the formatting included.

Throughout this book, we're going to use this markdown to explain what we are doing and, more importantly, why we are doing it. Explaining your thinking is a vital part of understanding what you are doing.

That explaination, plus the code, is the real power of notebooks. To add a block of code, follow the instructions from above: click on the \emph{Insert} button on the toolbar or by pressing \emph{Cmd+Option+I} on Mac or \emph{Ctl+Alt+I} on Windows.

In that window, use some of the code from above and add two numbers together. To see it run, click the green triangle on the right. That runs the chunk. You should see the answer to your addition problem.

And that, just that, is the foundation you need to start this book.

\hypertarget{data-structures-and-types}{%
\chapter{Data, structures and types}\label{data-structures-and-types}}

Data are everywhere (and data is plural of datum, thus the use of are in that statement). It surrounds you. Every time you use your phone, you are creating data. Lots of it. Your online life. Any time you buy something. It's everywhere. Sports, like life, is no different. Sports is drowning in data, and more comes along all the time.

In sports, and in this class, we'll be dealing largely with two kinds of data: event level data and summary data. It's not hard to envision event level data in sports. A pitch in baseball. A hit. A play in football. A pass in soccer. They are the events that make up the game. Combine them together -- summarize them -- and you'll have some notion of how the game went. What we usually see is summary data -- who wants to scroll through 50 pitches to find out a player went 2-3 with a double and an RBI? Who wants to scroll through hundreds of pitches to figure out the Rays beat the Yankees?

To start with, we need to understand the shape of data.

\begin{quote}
EXERCISE: Try scoring a child's board game. For example, Chutes and Ladders. If you were placed in charge of analytics for the World Series of Chutes and Ladders, what is your event level data? What summary data do you keep? If you've got the game, try it.
\end{quote}

\hypertarget{rows-and-columns}{%
\section{Rows and columns}\label{rows-and-columns}}

Data, oversimplifying it a bit, is information organized. Generally speaking, it's organized into rows and columns. Rows, generally, are individual elements. A team. A player. A game. Columns, generally, are components of the data, sometimes called variables. So if each row is a player, the first column might be their name. The second is their position. The third is their batting average. And so on.

\includegraphics[width=22in]{images/data1}

One of the critical components of data analysis, especially for beginners, is having a mental picture of your data. What does each row mean? What does each column in each row signify? How many rows do you have? How many columns?

\hypertarget{types}{%
\section{Types}\label{types}}

There are scores of data types in the world, and R has them. In this class, we're primarily going to be dealing with data frames, and each element of our data frames will have a data type.

Typically, they'll be one of four types of data:

\begin{itemize}
\tightlist
\item
  Numeric: a number, like the number of touchdown passes in a season or a batting average.
\item
  Character: Text, like a name, a team, a conference.
\item
  Date: Fully formed dates -- 2019-01-01 -- have a special date type. Elements of a date, like a year (ex. 2019) are not technically dates, so they'll appear as numeric data types.
\item
  Logical: Rare, but every now and then we'll have a data type that's Yes or No, True or False, etc.
\end{itemize}

\textbf{Question:} Is a zip code a number? Is a jersey number a number? Trick question, because the answer is no. Numbers are things we do math on. If the thing you want is not something you're going to do math on -- can you add two phone numbers together? -- then make it a character type. If you don't, most every software system on the planet will drop leading zeros. For example, every zip code in Boston starts with 0. If you record that as a number, your zip code will become a four digit number, which isn't a zip code anymore.

\hypertarget{a-simple-way-to-get-data}{%
\section{A simple way to get data}\label{a-simple-way-to-get-data}}

One good thing about sports is that there's lots of interest in it. And that means there's outlets that put sports data on the internet. Now I'm going to show you a trick to getting it easily.

The site sports-reference.com takes NCAA (and other league) stats and puts them online. For instance, \href{https://www.sports-reference.com/cbb/schools/nebraska/2019-gamelogs.html}{here's their page on Nebraska basketball's game logs}, which you should open now.

Now, in a new tab, log into Google Docs/Drive and open a new spreadsheet. In the first cell of the first row, copy and paste this formula in:

\begin{verbatim}
=IMPORTHTML("https://www.sports-reference.com/cbb/schools/nebraska/2019-gamelogs.html", "table", 1)
\end{verbatim}

If it worked right, you've got the data from that page in a spreadsheet.

\hypertarget{cleaning-the-data}{%
\section{Cleaning the data}\label{cleaning-the-data}}

The first thing we need to do is recognize that we don't have data, really. We have the results of a formula. You can tell by putting your cursor on that field, where you'll see the formula again. This is where you'd look:

\includegraphics[width=33.28in]{images/clean1}

The solution is easy:

Edit \textgreater{} Select All or type command/control A
Edit \textgreater{} Copy or type command/control c
Edit \textgreater{} Paste Special \textgreater{} Values Only or type command/control shift v

You can verify that it worked by looking in that same row 1 column A, where you'll see the formula is gone.

\includegraphics[width=36.81in]{images/clean2}

Now you have data, but your headers are all wrong. You want your headers to be one line -- not two, like they have. And the header names repeat -- first for our team, then for theirs. So you have to change each header name to be UsORB or TeamORB and OpponentORB instead of just ORB.

After you've done that, note we have repeating headers. There's two ways to deal with that -- you could just hightlight it and go up to Edit \textgreater{} Delete Rows XX-XX depending on what rows you highlighted. That's the easy way with our data.

But what if you had hundreds of repeating headers like that? Deleting them would take a long time.

You can use sorting to get rid of anything that's not data. So click on Data \textgreater{} Sort Range. You'll want to check the ``Data has header row'' field. Then hit Sort.

\includegraphics[width=21.61in]{images/clean3}

Now all you need to do is search through the data for where your junk data -- extra headers, blanks, etc. -- got sorted and delete it. After you've done that, you can export it for use in R. Go to File \textgreater{} Download as \textgreater{} Comma Separated Values. Remember to put it in the same directory as your R Notebook file so you can import the data easily.

\hypertarget{aggregates}{%
\chapter{Aggregates}\label{aggregates}}

R is a statistical programming language that is purpose built for data analysis.

Base R does a lot, but there are a mountain of external libraries that do things to make R better/easier/more fully featured. We already installed the tidyverse -- or you should have if you followed the instructions for the last assignment -- which isn't exactly a library, but a collection of libraries. Together, they make up the tidyverse. Individually, they are extraordinarily useful for what they do. We can load them all at once using the tidyverse name, or we can load them individually. Let's start with individually.

The two libraries we are going to need for this assignement are \texttt{readr} and \texttt{dplyr}. The library \texttt{readr} reads different types of data in. For this assignment, we're going to read in csv data or Comma Separated Values data. That's data that has a comma between each column of data.

Then we're going to use \texttt{dplyr} to analyze it.

To use a library, you need to import it. Good practice -- one I'm going to insist on -- is that you put all your library steps at the top of your notebooks.

That code looks like this:

\begin{Shaded}
\begin{Highlighting}[]
\KeywordTok{library}\NormalTok{(readr)}
\end{Highlighting}
\end{Shaded}

To load them both, you need to run that code twice:

\begin{Shaded}
\begin{Highlighting}[]
\KeywordTok{library}\NormalTok{(readr)}
\KeywordTok{library}\NormalTok{(dplyr)}
\end{Highlighting}
\end{Shaded}

You can keep doing that for as many libraries as you need. I've seen notebooks with 10 or more library imports.

\hypertarget{basic-data-analysis-group-by-and-count}{%
\section{Basic data analysis: Group By and Count}\label{basic-data-analysis-group-by-and-count}}

The first thing we need to do is get some data to work with. We do that by reading it in. In our case, we're going to read data from a csv file -- a comma-separated values file.

The CSV file we're going to read from is a \href{https://unl.box.com/s/xjipgkesl9rjmng4weg77vb73xt41apf}{Nebraska Game and Parks Commission dataset} of confirmed mountain lion sightings in Nebraska. There are, on occasion, fierce debates about mountain lions and if they should be hunted in Nebraska. This dataset can tell us some interesting things about that debate.

So step 1 is to import the data. The code looks \emph{something} like this, but hold off copying it just yet:

\texttt{mountainlions\ \textless{}-\ read\_csv("\textasciitilde{}/Documents/Data/mountainlions.csv")}

Let's unpack that.

The first part -- mountainlions -- is the name of your variable. A variable is just a name of a thing. In this case, our variable is a data frame, which is R's way of storing data. We can call this whatever we want. I always want to name data frames after what is in it. In this case, we're going to import a dataset of mountain lion sightings from the Nebraska Game and Parks Commission. Variable names, by convention are one word all lower case. You can end a variable with a number, but you can't start one with a number.

The \textless{}- bit is the variable assignment operator. It's how we know we're assigning something to a word. Think of the arrow as saying ``Take everything on the right of this arrow and stuff it into the thing on the left.'' So we're creating an empty vessel called mountainlions and stuffing all this data into it.

The \texttt{read\_csv} bits are pretty obvious, except for one thing. What happens in the quote marks is the path to the data. In there, I have to tell R where it will find the data. The easiest thing to do, if you are confused about how to find your data, is to put your data in the same folder as as your notebook (you'll have to save that notebook first). If you do that, then you just need to put the name of the file in there (mountainlions.csv). In my case, I've got a folder called Documents in my home directory (that's the \texttt{\textasciitilde{}} part), and in there is a folder called Data that has the file called mountainlions.csv in it. Some people -- insane people -- leave the data in their downloads folder. The data path then would be \texttt{\textasciitilde{}/Downloads/nameofthedatafilehere.csv} on PC or Mac.

\textbf{What you put in there will be different from mine}. So your first task is to import the data.

\begin{Shaded}
\begin{Highlighting}[]
\NormalTok{mountainlions <-}\StringTok{ }\KeywordTok{read_csv}\NormalTok{(}\StringTok{"data/mountainlions.csv"}\NormalTok{)}
\end{Highlighting}
\end{Shaded}

\begin{verbatim}
## Parsed with column specification:
## cols(
##   ID = col_double(),
##   `Cofirm Type` = col_character(),
##   COUNTY = col_character(),
##   Date = col_character()
## )
\end{verbatim}

Now we can inspect the data we imported. What does it look like? To do that, we use \texttt{head(mountainlions)} to show the headers and the first six rows of data. If we wanted to see them all, we could just simply enter \texttt{mountainlions} and run it.

To get the number of records in our dataset, we run \texttt{nrow(mountainlions)}

\begin{Shaded}
\begin{Highlighting}[]
\KeywordTok{head}\NormalTok{(mountainlions)}
\end{Highlighting}
\end{Shaded}

\begin{verbatim}
## # A tibble: 6 x 4
##      ID `Cofirm Type` COUNTY       Date    
##   <dbl> <chr>         <chr>        <chr>   
## 1     1 Track         Dawes        9/14/91 
## 2     2 Mortality     Sioux        11/10/91
## 3     3 Mortality     Scotts Bluff 4/21/96 
## 4     4 Mortality     Sioux        5/9/99  
## 5     5 Mortality     Box Butte    9/29/99 
## 6     6 Track         Scotts Bluff 11/12/99
\end{verbatim}

\begin{Shaded}
\begin{Highlighting}[]
\KeywordTok{nrow}\NormalTok{(mountainlions)}
\end{Highlighting}
\end{Shaded}

\begin{verbatim}
## [1] 393
\end{verbatim}

So what if we wanted to know how many mountain lion sightings there were in each county? To do that by hand, we'd have to take each of the 393 records and sort them into a pile. We'd put them in groups and then count them.

\texttt{dplyr} has a group by function in it that does just this. A massive amount of data analysis involves grouping like things together at some point. So it's a good place to start.

So to do this, we'll take our dataset and we'll introduce a new operator: \%\textgreater{}\%. The best way to read that operator, in my opinion, is to interpret that as ``and then do this.'' Here's the code:

\begin{Shaded}
\begin{Highlighting}[]
\NormalTok{mountainlions }\OperatorTok
\StringTok{  }\KeywordTok{group_by}\NormalTok{(COUNTY) }\OperatorTok
\StringTok{  }\KeywordTok{summarise}\NormalTok{(}
    \DataTypeTok{total =} \KeywordTok{n}\NormalTok{()}
\NormalTok{  )}
\end{Highlighting}
\end{Shaded}

\begin{verbatim}
## # A tibble: 42 x 2
##    COUNTY    total
##    <chr>     <int>
##  1 Banner        6
##  2 Blaine        3
##  3 Box Butte     4
##  4 Brown        15
##  5 Buffalo       3
##  6 Cedar         1
##  7 Cherry       30
##  8 Custer        8
##  9 Dakota        3
## 10 Dawes       111
## # ... with 32 more rows
\end{verbatim}

So let's walk through that. We start with our dataset -- \texttt{mountainlions} -- and then we tell it to group the data by a given field in the data. In this case, we wanted to group together all the counties, signified by the field name COUNTY, which you could get from looking at \texttt{head(mountainlions)}. After we group the data, we need to count them up. In dplyr, we use \texttt{summarize} \href{http://dplyr.tidyverse.org/reference/summarise.html}{which can do more than just count things}. Inside the parentheses in summarize, we set up the summaries we want. In this case, we just want a count of the counties: \texttt{count\ =\ n(),} says create a new field, called \texttt{total} and set it equal to \texttt{n()}, which might look weird, but it's common in stats. The number of things in a dataset? Statisticians call in n. There are n number of incidents in this dataset. So \texttt{n()} is a function that counts the number of things there are.

And when we run that, we get a list of counties with a count next to them. But it's not in any order. So we'll add another And Then Do This \%\textgreater{}\% and use \texttt{arrange}. Arrange does what you think it does -- it arranges data in order. By default, it's in ascending order -- smallest to largest. But if we want to know the county with the most mountain lion sightings, we need to sort it in descending order. That looks like this:

\begin{Shaded}
\begin{Highlighting}[]
\NormalTok{mountainlions }\OperatorTok
\StringTok{  }\KeywordTok{group_by}\NormalTok{(COUNTY) }\OperatorTok
\StringTok{  }\KeywordTok{summarise}\NormalTok{(}
    \DataTypeTok{count =} \KeywordTok{n}\NormalTok{()}
\NormalTok{  ) }\OperatorTok\StringTok{ }\KeywordTok{arrange}\NormalTok{(}\KeywordTok{desc}\NormalTok{(count))}
\end{Highlighting}
\end{Shaded}

\begin{verbatim}
## # A tibble: 42 x 2
##    COUNTY       count
##    <chr>        <int>
##  1 Dawes          111
##  2 Sioux           52
##  3 Sheridan        35
##  4 Cherry          30
##  5 Scotts Bluff    26
##  6 Keya Paha       20
##  7 Brown           15
##  8 Rock            11
##  9 Lincoln         10
## 10 Custer           8
## # ... with 32 more rows
\end{verbatim}

We can, if we want, group by more than one thing. So how are these sightings being confirmed? To do that, we can group by County and ``Cofirm Type'', which is how the state misspelled Confirm. But note something in this example below:

\begin{Shaded}
\begin{Highlighting}[]
\NormalTok{mountainlions }\OperatorTok
\StringTok{  }\KeywordTok{group_by}\NormalTok{(COUNTY, }\StringTok{`}\DataTypeTok{Cofirm Type}\StringTok{`}\NormalTok{) }\OperatorTok
\StringTok{  }\KeywordTok{summarise}\NormalTok{(}
    \DataTypeTok{count =} \KeywordTok{n}\NormalTok{()}
\NormalTok{  ) }\OperatorTok\StringTok{ }\KeywordTok{arrange}\NormalTok{(}\KeywordTok{desc}\NormalTok{(count))}
\end{Highlighting}
\end{Shaded}

\begin{verbatim}
## # A tibble: 93 x 3
## # Groups:   COUNTY [42]
##    COUNTY       `Cofirm Type`      count
##    <chr>        <chr>              <int>
##  1 Dawes        Trail Camera Photo    41
##  2 Sioux        Trail Camera Photo    40
##  3 Dawes        Track                 19
##  4 Keya Paha    Trail Camera Photo    18
##  5 Cherry       Trail Camera Photo    17
##  6 Dawes        Mortality             17
##  7 Sheridan     Trail Camera Photo    16
##  8 Dawes        Photo                 13
##  9 Dawes        DNA                   11
## 10 Scotts Bluff Trail Camera Photo    11
## # ... with 83 more rows
\end{verbatim}

See it? When you have a field name that has two words, \texttt{readr} wraps it in backticks, which is next to the 1 key on your keyboard. You can figure out which fields have backticks around it by looking at the output of \texttt{readr}. Pay attention to that, because it's coming up again in the next section and will be a part of your homework.

\hypertarget{other-aggregates-mean-and-median}{%
\section{Other aggregates: Mean and median}\label{other-aggregates-mean-and-median}}

In the last example, we grouped some data together and counted it up, but there's so much more you can do. You can do multiple measures in a single step as well.

Let's look at some \href{https://unl.box.com/s/09t2u4qoncfh6qlv2156flzlxb8ruzpq}{salary data from the University of Nebraska}.

\begin{Shaded}
\begin{Highlighting}[]
\NormalTok{salaries <-}\StringTok{ }\KeywordTok{read_csv}\NormalTok{(}\StringTok{"data/nusalaries1819.csv"}\NormalTok{)}
\end{Highlighting}
\end{Shaded}

\begin{verbatim}
## Parsed with column specification:
## cols(
##   Employee = col_character(),
##   Position = col_character(),
##   Campus = col_character(),
##   Department = col_character(),
##   `Budgeted Annual Salary` = col_number(),
##   `Salary from State Aided Funds` = col_number(),
##   `Salary from Other Funds` = col_number()
## )
\end{verbatim}

\begin{Shaded}
\begin{Highlighting}[]
\KeywordTok{head}\NormalTok{(salaries)}
\end{Highlighting}
\end{Shaded}

\begin{verbatim}
## # A tibble: 6 x 7
##   Employee Position Campus Department `Budgeted Annua~ `Salary from St~
##   <chr>    <chr>    <chr>  <chr>                 <dbl>            <dbl>
## 1 Abbey, ~ Associa~ UNK    Kinesiolo~            61276            61276
## 2 Abbott,~ Staff S~ UNL    FM&P Faci~            37318               NA
## 3 Abboud,~ Adminis~ UNMC   Surgery-U~            76400            76400
## 4 Abdalla~ Asst Pr~ UNMC   Pathology~            74774            71884
## 5 Abdelka~ Post-Do~ UNMC   Surgery-T~            43516               NA
## 6 Abdel-M~ Researc~ UNL    Public Po~            58502               NA
## # ... with 1 more variable: `Salary from Other Funds` <dbl>
\end{verbatim}

In summarize, we can calculate any number of measures. Here, we'll use R's built in mean and median functions to calculate \ldots{} well, you get the idea.

\begin{Shaded}
\begin{Highlighting}[]
\NormalTok{salaries }\OperatorTok
\StringTok{  }\KeywordTok{summarise}\NormalTok{(}
    \DataTypeTok{count =} \KeywordTok{n}\NormalTok{(),}
    \DataTypeTok{mean_salary =} \KeywordTok{mean}\NormalTok{(}\StringTok{`}\DataTypeTok{Budgeted Annual Salary}\StringTok{`}\NormalTok{),}
    \DataTypeTok{median_salary =} \KeywordTok{median}\NormalTok{(}\StringTok{`}\DataTypeTok{Budgeted Annual Salary}\StringTok{`}\NormalTok{)}
\NormalTok{  )}
\end{Highlighting}
\end{Shaded}

\begin{verbatim}
## # A tibble: 1 x 3
##   count mean_salary median_salary
##   <int>       <dbl>         <dbl>
## 1 13039      62065.         51343
\end{verbatim}

So there's 13,039 employees in the database, spread across four campuses plus the system office. The mean or average salary is about \$62,000, but the median salary is slightly more than \$51,000.

Why?

Let's let sort help us.

\begin{Shaded}
\begin{Highlighting}[]
\NormalTok{salaries }\OperatorTok\StringTok{ }\KeywordTok{arrange}\NormalTok{(}\KeywordTok{desc}\NormalTok{(}\StringTok{`}\DataTypeTok{Budgeted Annual Salary}\StringTok{`}\NormalTok{))}
\end{Highlighting}
\end{Shaded}

\begin{verbatim}
## # A tibble: 13,039 x 7
##    Employee Position Campus Department `Budgeted Annua~ `Salary from St~
##    <chr>    <chr>    <chr>  <chr>                 <dbl>            <dbl>
##  1 Frost, ~ Head Co~ UNL    Athletics           5000000               NA
##  2 Miles, ~ Head Co~ UNL    Athletics           2375000               NA
##  3 Moos, W~ Athleti~ UNL    Athletics           1000000               NA
##  4 Gold, J~ Chancel~ UNMC   Office of~           853338           853338
##  5 Chinand~ Assista~ UNL    Athletics            800000               NA
##  6 Walters~ Assista~ UNL    Athletics            700000               NA
##  7 Cook, J~ Head Co~ UNL    Athletics            675000               NA
##  8 William~ Head Co~ UNL    Athletics            626750               NA
##  9 Bounds,~ Preside~ UNCA   Office of~           540000           540000
## 10 Austin ~ Assista~ UNL    Athletics            475000               NA
## # ... with 13,029 more rows, and 1 more variable: `Salary from Other
## #   Funds` <dbl>
\end{verbatim}

Oh, right. In this dataset, the university pays a football coach \$5 million. Extremes influence averages, not medians, and now you have your answer.

So when choosing a measure of the middle, you have to ask yourself -- could I have extremes? Because a median won't be sensitive to extremes. It will be the point at which half the numbers are above and half are below. The average or mean will be a measure of the middle, but if you have a bunch of low paid people and then one football coach, the average will be wildly skewed. Here, because there's so few highly paid football coaches compared to people who make a normal salary, the number is only slightly skewed in the grand scheme, but skewed nonetheless.

\hypertarget{even-more-aggregates}{%
\section{Even more aggregates}\label{even-more-aggregates}}

There's a ton of things we can do in summarize -- we'll work with more of them as the course progresses -- but here's a few other questions you can ask.

Which department on campus has the highest wage bill? And what is the highest and lowest salary in the department? And how wide is the spread between salaries? We can find that with \texttt{sum} to add up the salaries to get the total wage bill, \texttt{min} to find the minumum salary, \texttt{max} to find the maximum salary and \texttt{sd} to find the standard deviation in the numbers.

\begin{Shaded}
\begin{Highlighting}[]
\NormalTok{salaries }\OperatorTok\StringTok{ }
\StringTok{  }\KeywordTok{group_by}\NormalTok{(Campus, Department) }\OperatorTok\StringTok{ }
\StringTok{  }\KeywordTok{summarise}\NormalTok{(}
    \DataTypeTok{total =} \KeywordTok{sum}\NormalTok{(}\StringTok{`}\DataTypeTok{Budgeted Annual Salary}\StringTok{`}\NormalTok{), }
    \DataTypeTok{avgsalary =} \KeywordTok{mean}\NormalTok{(}\StringTok{`}\DataTypeTok{Budgeted Annual Salary}\StringTok{`}\NormalTok{), }
    \DataTypeTok{minsalary =} \KeywordTok{min}\NormalTok{(}\StringTok{`}\DataTypeTok{Budgeted Annual Salary}\StringTok{`}\NormalTok{),}
    \DataTypeTok{maxsalary =} \KeywordTok{max}\NormalTok{(}\StringTok{`}\DataTypeTok{Budgeted Annual Salary}\StringTok{`}\NormalTok{),}
    \DataTypeTok{stdev =} \KeywordTok{sd}\NormalTok{(}\StringTok{`}\DataTypeTok{Budgeted Annual Salary}\StringTok{`}\NormalTok{)) }\OperatorTok\StringTok{ }\KeywordTok{arrange}\NormalTok{(}\KeywordTok{desc}\NormalTok{(total))}
\end{Highlighting}
\end{Shaded}

\begin{verbatim}
## # A tibble: 804 x 7
## # Groups:   Campus [5]
##    Campus Department                  total avgsalary minsalary maxsalary  stdev
##    <chr>  <chr>                       <dbl>     <dbl>     <dbl>     <dbl>  <dbl>
##  1 UNL    Athletics                  3.56e7   118508.     12925   5000000 3.33e5
##  2 UNMC   Pathology/Microbiology     1.36e7    63158.      1994    186925 3.41e4
##  3 UNL    Agronomy & Horticulture    8.98e6    66496.      5000    208156 4.01e4
##  4 UNMC   Anesthesiology             7.90e6    78237.     10000    245174 3.59e4
##  5 UNL    School of Natural Resour~  6.86e6    65995.      2400    194254 3.28e4
##  6 UNL    College of Law             6.70e6    77953.      1000    326400 7.23e4
##  7 UNL    University Television      6.44e6    55542.     16500    221954 2.75e4
##  8 UNL    University Libraries       6.27e6    51390.      1200    215917 2.68e4
##  9 UNMC   Pharmacology/Exp Neurosc~  6.24e6    58911.      2118    248139 4.29e4
## 10 UNMC   CON-Omaha Division         6.11e6    78304.      3000    172522 4.48e4
## # ... with 794 more rows
\end{verbatim}

So again, no surprise, the UNL athletic department has the single largest wage bill at nearly \$36 million. The average salary in the department is \$118,508 -- more than double the univeristy as a whole, again thanks to Scott Frost's paycheck.

\hypertarget{mutating-data}{%
\chapter{Mutating data}\label{mutating-data}}

One of the most common data analysis techniques is to look at change over time. The most common way of comparing change over time is through percent change. The math behind calculating percent change is very simple, and you should know it off the top of your head. The easy way to remember it is:

\texttt{(new\ -\ old)\ /\ old}

Or new minus old divided by old. Your new number minus the old number, the result of which is divided by the old number. To do that in R, we can use \texttt{dplyr} and \texttt{mutate} to calculate new metrics in a new field using existing fields of data.

So first we'll import the tidyverse so we can read in our data and begin to work with it.

\begin{Shaded}
\begin{Highlighting}[]
\KeywordTok{library}\NormalTok{(tidyverse)}
\end{Highlighting}
\end{Shaded}

Now we'll import a common and \href{https://unl.box.com/s/etqna5bfvf3b0gsnw0kcjjn1rxx9335s}{simple dataset of total attendance} at NCAA football games over the last few seasons.

\begin{Shaded}
\begin{Highlighting}[]
\NormalTok{attendance <-}\StringTok{ }\KeywordTok{read_csv}\NormalTok{(}\StringTok{'data/attendance.csv'}\NormalTok{)}
\end{Highlighting}
\end{Shaded}

\begin{verbatim}
## Parsed with column specification:
## cols(
##   Institution = col_character(),
##   Conference = col_character(),
##   `2013` = col_double(),
##   `2014` = col_double(),
##   `2015` = col_double(),
##   `2016` = col_double(),
##   `2017` = col_double(),
##   `2018` = col_double()
## )
\end{verbatim}

\begin{Shaded}
\begin{Highlighting}[]
\KeywordTok{head}\NormalTok{(attendance)}
\end{Highlighting}
\end{Shaded}

\begin{verbatim}
## # A tibble: 6 x 8
##   Institution     Conference      `2013` `2014` `2015` `2016` `2017` `2018`
##   <chr>           <chr>            <dbl>  <dbl>  <dbl>  <dbl>  <dbl>  <dbl>
## 1 Air Force       MWC             228562 168967 156158 177519 174924 166205
## 2 Akron           MAC             107101  55019 108588  62021 117416  92575
## 3 Alabama         SEC             710538 710736 707786 712747 712053 710931
## 4 Appalachian St. FBS Independent 149366     NA     NA     NA     NA     NA
## 5 Appalachian St. Sun Belt            NA 138995 128755 156916 154722 131716
## 6 Arizona         Pac-12          285713 354973 308355 338017 255791 318051
\end{verbatim}

The code to calculate percent change is pretty simple. Remember, with \texttt{summarize}, we used \texttt{n()} to count things. With \texttt{mutate}, we use very similar syntax to calculate a new value using other values in our dataset. So in this case, we're trying to do (new-old)/old, but we're doing it with fields. If we look at what we got when we did \texttt{head}, you'll see there's `2018` as the new data, and we'll use `2017` as the old data. So we're looking at one year. Then, to help us, we'll use arrange again to sort it, so we get the fastest growing school over one year.

\begin{Shaded}
\begin{Highlighting}[]
\NormalTok{attendance }\OperatorTok\StringTok{ }\KeywordTok{mutate}\NormalTok{(}
  \DataTypeTok{change =}\NormalTok{ (}\StringTok{`}\DataTypeTok{2018}\StringTok{`} \OperatorTok{-}\StringTok{ `}\DataTypeTok{2017}\StringTok{`}\NormalTok{)}\OperatorTok{/}\StringTok{`}\DataTypeTok{2017}\StringTok{`}
\NormalTok{) }
\end{Highlighting}
\end{Shaded}

\begin{verbatim}
## # A tibble: 150 x 9
##    Institution   Conference   `2013` `2014` `2015` `2016` `2017` `2018`   change
##    <chr>         <chr>         <dbl>  <dbl>  <dbl>  <dbl>  <dbl>  <dbl>    <dbl>
##  1 Air Force     MWC          228562 168967 156158 177519 174924 166205 -0.0498 
##  2 Akron         MAC          107101  55019 108588  62021 117416  92575 -0.212  
##  3 Alabama       SEC          710538 710736 707786 712747 712053 710931 -0.00158
##  4 Appalachian ~ FBS Indepen~ 149366     NA     NA     NA     NA     NA NA      
##  5 Appalachian ~ Sun Belt         NA 138995 128755 156916 154722 131716 -0.149  
##  6 Arizona       Pac-12       285713 354973 308355 338017 255791 318051  0.243  
##  7 Arizona St.   Pac-12       501509 343073 368985 286417 359660 291091 -0.191  
##  8 Arkansas      SEC          431174 399124 471279 487067 442569 367748 -0.169  
##  9 Arkansas St.  Sun Belt     149477 149163 138043 136200 119538 119001 -0.00449
## 10 Army West Po~ FBS Indepen~ 169781 171310 185946 163267 185543 190156  0.0249 
## # ... with 140 more rows
\end{verbatim}

What do we see right away? Do those numbers look like we expect them to? No.~They're a decimal expressed as a percentage. So let's fix that by multiplying by 100.

\begin{Shaded}
\begin{Highlighting}[]
\NormalTok{attendance }\OperatorTok\StringTok{ }\KeywordTok{mutate}\NormalTok{(}
  \DataTypeTok{change =}\NormalTok{ ((}\StringTok{`}\DataTypeTok{2018}\StringTok{`} \OperatorTok{-}\StringTok{ `}\DataTypeTok{2017}\StringTok{`}\NormalTok{)}\OperatorTok{/}\StringTok{`}\DataTypeTok{2017}\StringTok{`}\NormalTok{)}\OperatorTok{*}\DecValTok{100}
\NormalTok{) }
\end{Highlighting}
\end{Shaded}

\begin{verbatim}
## # A tibble: 150 x 9
##    Institution    Conference   `2013` `2014` `2015` `2016` `2017` `2018`  change
##    <chr>          <chr>         <dbl>  <dbl>  <dbl>  <dbl>  <dbl>  <dbl>   <dbl>
##  1 Air Force      MWC          228562 168967 156158 177519 174924 166205  -4.98 
##  2 Akron          MAC          107101  55019 108588  62021 117416  92575 -21.2  
##  3 Alabama        SEC          710538 710736 707786 712747 712053 710931  -0.158
##  4 Appalachian S~ FBS Indepen~ 149366     NA     NA     NA     NA     NA  NA    
##  5 Appalachian S~ Sun Belt         NA 138995 128755 156916 154722 131716 -14.9  
##  6 Arizona        Pac-12       285713 354973 308355 338017 255791 318051  24.3  
##  7 Arizona St.    Pac-12       501509 343073 368985 286417 359660 291091 -19.1  
##  8 Arkansas       SEC          431174 399124 471279 487067 442569 367748 -16.9  
##  9 Arkansas St.   Sun Belt     149477 149163 138043 136200 119538 119001  -0.449
## 10 Army West Poi~ FBS Indepen~ 169781 171310 185946 163267 185543 190156   2.49 
## # ... with 140 more rows
\end{verbatim}

Now, does this ordering do anything for us? No.~Let's fix that with arrange.

\begin{Shaded}
\begin{Highlighting}[]
\NormalTok{attendance }\OperatorTok\StringTok{ }\KeywordTok{mutate}\NormalTok{(}
  \DataTypeTok{change =}\NormalTok{ ((}\StringTok{`}\DataTypeTok{2018}\StringTok{`} \OperatorTok{-}\StringTok{ `}\DataTypeTok{2017}\StringTok{`}\NormalTok{)}\OperatorTok{/}\StringTok{`}\DataTypeTok{2017}\StringTok{`}\NormalTok{)}\OperatorTok{*}\DecValTok{100}
\NormalTok{) }\OperatorTok\StringTok{ }\KeywordTok{arrange}\NormalTok{(}\KeywordTok{desc}\NormalTok{(change))}
\end{Highlighting}
\end{Shaded}

\begin{verbatim}
## # A tibble: 150 x 9
##    Institution   Conference `2013` `2014` `2015` `2016` `2017` `2018` change
##    <chr>         <chr>       <dbl>  <dbl>  <dbl>  <dbl>  <dbl>  <dbl>  <dbl>
##  1 Ga. Southern  Sun Belt       NA 105510 124681 104095  61031 100814   65.2
##  2 La.-Monroe    Sun Belt    85177  90540  58659  67057  49640  71048   43.1
##  3 Louisiana     Sun Belt   129878 154652 129577 121346  78754 111303   41.3
##  4 Hawaii        MWC        185931 192159 164031 170299 145463 205455   41.2
##  5 Buffalo       MAC        136418 122418 110743 104957  80102 110280   37.7
##  6 California    Pac-12     345303 286051 292797 279769 219290 300061   36.8
##  7 UCF           AAC        252505 226869 180388 214814 257924 352148   36.5
##  8 UTSA          C-USA      175282 165458 138048 138226 114104 148257   29.9
##  9 Eastern Mich. MAC         20255  75127  29381 106064  73649  95632   29.8
## 10 Louisville    ACC            NA 317829 294413 324391 276957 351755   27.0
## # ... with 140 more rows
\end{verbatim}

So who had the most growth last year from the year before? Something going on at Georgia Southern.

\hypertarget{a-more-complex-example}{%
\section{A more complex example}\label{a-more-complex-example}}

There's metric in basketball that's easy to understand -- shooting percentage. It's the number of shots made divided by the number of shots attempted. Simple, right? Except it's a little too simple. Because what about three point shooters? They tend to be more vailable because the three point shot is worth more. What about players who get to the line? In shooting percentage, free throws are nowhere to be found.

Basketball nerds, because of these weaknesses, have created a new metric called \href{https://en.wikipedia.org/wiki/True_shooting_percentage}{True Shooting Percentage}. True shooting percentage takes into account all aspects of a players shooting to determine who the real shooters are.

Using \texttt{dplyr} and \texttt{mutate}, we can calculate true shooting percentage. So let's look at a new dataset, one of \href{https://unl.box.com/s/s1wzw61u9ia50qmirfhuvprgpmmah9rj}{every college basketball player's season stats in 2018-19 season}. It's a dataset of 5,386 players, and we've got 59 variables -- one of them is True Shooting Percentage, but we're going to ignore that.

\begin{Shaded}
\begin{Highlighting}[]
\NormalTok{players <-}\StringTok{ }\KeywordTok{read_csv}\NormalTok{(}\StringTok{"data/players19.csv"}\NormalTok{)}
\end{Highlighting}
\end{Shaded}

\begin{verbatim}
## Warning: Missing column names filled in: 'X1' [1]
\end{verbatim}

\begin{verbatim}
## Parsed with column specification:
## cols(
##   .default = col_double(),
##   Team = col_character(),
##   Conference = col_character(),
##   Player = col_character(),
##   Class = col_character(),
##   Pos = col_character(),
##   Height = col_character(),
##   Hometown = col_character(),
##   `High School` = col_character(),
##   Summary = col_character()
## )
\end{verbatim}

\begin{verbatim}
## See spec(...) for full column specifications.
\end{verbatim}

The basic true shooting percentage formula is \texttt{(Points\ /\ (2*(FieldGoalAttempts\ +\ (.44\ *\ FreeThrowAttempts))))\ *\ 100}. Let's talk that through. Points divided by a lot. It's really field goal attempts plus 44 percent of the free throw attempts. Why? Because that's about what a free throw is worth, compared to other ways to score. After adding those things together, you double it. And after you divide points by that number, you multiply the whole lot by 100.

In our data, we need to be able to find the fields so we can complete the formula. To do that, one way is to use the Environment tab in R Studio. In the Environment tab is a listing of all the data you've imported, and if you click the triangle next to it, it'll list all the field names, giving you a bit of information about each one.

\includegraphics[width=18.14in]{images/environment}

So what does True Shooting Percentage look like in code?

Let's think about this differently. Who had the best true shooting season last year?

\begin{Shaded}
\begin{Highlighting}[]
\NormalTok{players }\OperatorTok
\StringTok{  }\KeywordTok{mutate}\NormalTok{(}\DataTypeTok{trueshooting =}\NormalTok{ (PTS}\OperatorTok{/}\NormalTok{(}\DecValTok{2}\OperatorTok{*}\NormalTok{(FGA }\OperatorTok{+}\StringTok{ }\NormalTok{(.}\DecValTok{44}\OperatorTok{*}\NormalTok{FTA))))}\OperatorTok{*}\DecValTok{100}\NormalTok{) }\OperatorTok
\StringTok{  }\KeywordTok{arrange}\NormalTok{(}\KeywordTok{desc}\NormalTok{(trueshooting))}
\end{Highlighting}
\end{Shaded}

\begin{verbatim}
## # A tibble: 5,386 x 60
##       X1 Team  Conference Player   `#` Class Pos   Height Weight Hometown
##    <dbl> <chr> <chr>      <chr>  <dbl> <chr> <chr> <chr>   <dbl> <chr>   
##  1   579 Texa~ Big 12     Drayt~     4 JR    G     6-0       156 Austin,~
##  2   843 Ston~ AEC        Nick ~    42 FR    F     6-7       240 Port Je~
##  3  1059 Sout~ Southland  Patri~    22 SO    F     6-3       210 Folsom,~
##  4  4269 Dayt~ A-10       Camro~    52 SO    G     5-7       160 Country~
##  5  4681 Cali~ Pac-12     David~    21 JR    G     6-4       185 Newbury~
##  6   326 Virg~ ACC        Grant~     1 FR    G     <NA>       NA Charlot~
##  7   410 Vand~ SEC        Mac H~    42 FR    G     6-6       182 Chattan~
##  8  1390 Sain~ A-10       Jack ~    31 JR    G     6-6       205 Mattoon~
##  9  2230 NJIT~ A-Sun      Patri~     3 SO    G     5-9       160 West Or~
## 10   266 Wash~ Pac-12     Reaga~    34 FR    F     6-6       225 Santa A~
## # ... with 5,376 more rows, and 50 more variables: `High School` <chr>,
## #   Summary <chr>, Rk.x <dbl>, G <dbl>, GS <dbl>, MP <dbl>, FG <dbl>,
## #   FGA <dbl>, `FG%` <dbl>, `2P` <dbl>, `2PA` <dbl>, `2P%` <dbl>, `3P` <dbl>,
## #   `3PA` <dbl>, `3P%` <dbl>, FT <dbl>, FTA <dbl>, `FT%` <dbl>, ORB <dbl>,
## #   DRB <dbl>, TRB <dbl>, AST <dbl>, STL <dbl>, BLK <dbl>, TOV <dbl>, PF <dbl>,
## #   PTS <dbl>, Rk.y <dbl>, PER <dbl>, `TS%` <dbl>, `eFG%` <dbl>, `3PAr` <dbl>,
## #   FTr <dbl>, PProd <dbl>, `ORB%` <dbl>, `DRB%` <dbl>, `TRB%` <dbl>,
## #   `AST%` <dbl>, `STL%` <dbl>, `BLK%` <dbl>, `TOV%` <dbl>, `USG%` <dbl>,
## #   OWS <dbl>, DWS <dbl>, WS <dbl>, `WS/40` <dbl>, OBPM <dbl>, DBPM <dbl>,
## #   BPM <dbl>, trueshooting <dbl>
\end{verbatim}

You'll be forgiven if you did not hear about Texas Longhorns shooting sensation Drayton Whiteside. He played in six games, took one shot and actually hit it. It happened to be a three pointer, which is one more three pointer than I've hit in college basketball. So props to him. Does that mean he had the best true shooting season in college basketball last year? Not hardly.

We'll talk about how to narrow the pile and filter out data in the next chapter.

\hypertarget{filters-and-selections}{%
\chapter{Filters and selections}\label{filters-and-selections}}

More often than not, we have more data than we want. Sometimes we need to be rid of that data. In \texttt{dplyr}, there's two ways to go about this: filtering and selecting.

\textbf{Filtering creates a subset of the data based on criteria}. All records where the count is greater than 10. All records that match ``Nebraska''. Something like that.

\textbf{Selecting simply returns only the fields named}. So if you only want to see School and Attendance, you select those fields. When you look at your data again, you'll have two columns. If you try to use one of your columns that you had before you used select, you'll get an error.

Let's work with our \href{https://unl.box.com/s/etqna5bfvf3b0gsnw0kcjjn1rxx9335s}{football attendance} data to show some examples.

\begin{Shaded}
\begin{Highlighting}[]
\KeywordTok{library}\NormalTok{(tidyverse)}
\end{Highlighting}
\end{Shaded}

\begin{Shaded}
\begin{Highlighting}[]
\NormalTok{attendance <-}\StringTok{ }\KeywordTok{read_csv}\NormalTok{(}\StringTok{'data/attendance.csv'}\NormalTok{)}
\end{Highlighting}
\end{Shaded}

\begin{verbatim}
## Parsed with column specification:
## cols(
##   Institution = col_character(),
##   Conference = col_character(),
##   `2013` = col_double(),
##   `2014` = col_double(),
##   `2015` = col_double(),
##   `2016` = col_double(),
##   `2017` = col_double(),
##   `2018` = col_double()
## )
\end{verbatim}

So, first things first, let's say we don't care about all this Air Force, Akron, Alabama crap and just want to see Dear Old Nebraska U. We do that with \texttt{filter} and then we pass it a condition.

Before we do that, a note about conditions. Most of the conditional operators you'll understand -- greater than and less than are \textgreater{} and \textless{}. The tough one to remember is equal to. In conditional statements, equal to is == not =. If you haven't noticed, = is a variable assignment operator, not a conditional statement. So equal is == and NOT equal is !=.

So if you want to see Institutions equal to Nebraska, you do this:

\begin{Shaded}
\begin{Highlighting}[]
\NormalTok{attendance }\OperatorTok\StringTok{ }\KeywordTok{filter}\NormalTok{(Institution }\OperatorTok{==}\StringTok{ "Nebraska"}\NormalTok{)}
\end{Highlighting}
\end{Shaded}

\begin{verbatim}
## # A tibble: 1 x 8
##   Institution Conference `2013` `2014` `2015` `2016` `2017` `2018`
##   <chr>       <chr>       <dbl>  <dbl>  <dbl>  <dbl>  <dbl>  <dbl>
## 1 Nebraska    Big Ten    727466 638744 629983 631402 628583 623240
\end{verbatim}

Or if we want to see schools that had more than half a million people buy tickets to a football game in a season, we do the following. NOTE THE BACKTICKS.

\begin{Shaded}
\begin{Highlighting}[]
\NormalTok{attendance }\OperatorTok\StringTok{ }\KeywordTok{filter}\NormalTok{(}\StringTok{`}\DataTypeTok{2018}\StringTok{`} \OperatorTok{>=}\StringTok{ }\DecValTok{500000}\NormalTok{)}
\end{Highlighting}
\end{Shaded}

\begin{verbatim}
## # A tibble: 17 x 8
##    Institution    Conference `2013` `2014` `2015` `2016` `2017` `2018`
##    <chr>          <chr>       <dbl>  <dbl>  <dbl>  <dbl>  <dbl>  <dbl>
##  1 Alabama        SEC        710538 710736 707786 712747 712053 710931
##  2 Auburn         SEC        685252 612157 612157 695498 605120 591236
##  3 Clemson        ACC        574333 572262 588266 566787 565412 562799
##  4 Florida        SEC        524638 515001 630457 439229 520290 576299
##  5 Georgia        SEC        556476 649222 649222 556476 556476 649222
##  6 LSU            SEC        639927 712063 654084 708618 591034 705733
##  7 Michigan       Big Ten    781144 734364 771174 883741 669534 775156
##  8 Michigan St.   Big Ten    506294 522765 522628 522666 507398 508088
##  9 Nebraska       Big Ten    727466 638744 629983 631402 628583 623240
## 10 Ohio St.       Big Ten    734528 744075 750705 750944 752464 713630
## 11 Oklahoma       Big 12     508334 510972 512139 521142 519119 607146
## 12 Penn St.       Big Ten    676112 711358 698590 701800 746946 738396
## 13 South Carolina SEC        576805 569664 472934 538441 550099 515396
## 14 Tennessee      SEC        669087 698276 704088 706776 670454 650887
## 15 Texas          Big 12     593857 564618 540210 587283 556667 586277
## 16 Texas A&M      SEC        697003 630735 725354 713418 691612 698908
## 17 Wisconsin      Big Ten    552378 556642 546099 476144 551766 540072
\end{verbatim}

But what if we want to see all of the Power Five conferences? We \emph{could} use conditional logic in our filter. The conditional logic operators are \texttt{\textbar{}} for OR and \texttt{\&} for AND. NOTE: AND means all conditions have to be met. OR means any of the conditions work. So be careful about boolean logic.

\begin{Shaded}
\begin{Highlighting}[]
\NormalTok{attendance }\OperatorTok\StringTok{ }\KeywordTok{filter}\NormalTok{(Conference }\OperatorTok{==}\StringTok{ "Big 10"} \OperatorTok{|}\StringTok{ }\NormalTok{Conference }\OperatorTok{==}\StringTok{ "SEC"} \OperatorTok{|}\StringTok{ }\NormalTok{Conference }\OperatorTok{==}\StringTok{ "Pac-12"} \OperatorTok{|}\StringTok{ }\NormalTok{Conference }\OperatorTok{==}\StringTok{ "ACC"} \OperatorTok{|}\StringTok{ }\NormalTok{Conference }\OperatorTok{==}\StringTok{ "Big 12"}\NormalTok{)}
\end{Highlighting}
\end{Shaded}

\begin{verbatim}
## # A tibble: 51 x 8
##    Institution    Conference `2013` `2014` `2015` `2016` `2017` `2018`
##    <chr>          <chr>       <dbl>  <dbl>  <dbl>  <dbl>  <dbl>  <dbl>
##  1 Alabama        SEC        710538 710736 707786 712747 712053 710931
##  2 Arizona        Pac-12     285713 354973 308355 338017 255791 318051
##  3 Arizona St.    Pac-12     501509 343073 368985 286417 359660 291091
##  4 Arkansas       SEC        431174 399124 471279 487067 442569 367748
##  5 Auburn         SEC        685252 612157 612157 695498 605120 591236
##  6 Baylor         Big 12     321639 280257 276960 275029 262978 248017
##  7 Boston College ACC        198035 239893 211433 192942 215546 263363
##  8 California     Pac-12     345303 286051 292797 279769 219290 300061
##  9 Clemson        ACC        574333 572262 588266 566787 565412 562799
## 10 Colorado       Pac-12     230778 226670 236331 279652 282335 274852
## # ... with 41 more rows
\end{verbatim}

But that's a lot of repetitive code. And a lot of typing. And typing is the devil. So what if we could create a list and pass it into the filter? It's pretty simple.

We can create a new variable -- remember variables can represent just about anything -- and create a list. To do that we use the \texttt{c} operator, which stands for concatenate. That just means take all the stuff in the parenthesis after the c and bunch it into a list.

Note here: text is in quotes. If they were numbers, we wouldn't need the quotes.

\begin{Shaded}
\begin{Highlighting}[]
\NormalTok{powerfive <-}\StringTok{ }\KeywordTok{c}\NormalTok{(}\StringTok{"SEC"}\NormalTok{, }\StringTok{"Big Ten"}\NormalTok{, }\StringTok{"Pac-12"}\NormalTok{, }\StringTok{"Big 12"}\NormalTok{, }\StringTok{"ACC"}\NormalTok{)}
\end{Highlighting}
\end{Shaded}

Now with a list, we can use the \%in\% operator. It does what you think it does -- it gives you data that matches things IN the list you give it.

\begin{Shaded}
\begin{Highlighting}[]
\NormalTok{attendance }\OperatorTok\StringTok{ }\KeywordTok{filter}\NormalTok{(Conference }\OperatorTok\StringTok{ }\NormalTok{powerfive)}
\end{Highlighting}
\end{Shaded}

\begin{verbatim}
## # A tibble: 65 x 8
##    Institution    Conference `2013` `2014` `2015` `2016` `2017` `2018`
##    <chr>          <chr>       <dbl>  <dbl>  <dbl>  <dbl>  <dbl>  <dbl>
##  1 Alabama        SEC        710538 710736 707786 712747 712053 710931
##  2 Arizona        Pac-12     285713 354973 308355 338017 255791 318051
##  3 Arizona St.    Pac-12     501509 343073 368985 286417 359660 291091
##  4 Arkansas       SEC        431174 399124 471279 487067 442569 367748
##  5 Auburn         SEC        685252 612157 612157 695498 605120 591236
##  6 Baylor         Big 12     321639 280257 276960 275029 262978 248017
##  7 Boston College ACC        198035 239893 211433 192942 215546 263363
##  8 California     Pac-12     345303 286051 292797 279769 219290 300061
##  9 Clemson        ACC        574333 572262 588266 566787 565412 562799
## 10 Colorado       Pac-12     230778 226670 236331 279652 282335 274852
## # ... with 55 more rows
\end{verbatim}

\hypertarget{selecting-data-to-make-it-easier-to-read}{%
\section{Selecting data to make it easier to read}\label{selecting-data-to-make-it-easier-to-read}}

So now we have our Power Five list. What if we just wanted to see attendance from the most recent season and ignore all the rest? Select to the rescue.

\begin{Shaded}
\begin{Highlighting}[]
\NormalTok{attendance }\OperatorTok\StringTok{ }\KeywordTok{filter}\NormalTok{(Conference }\OperatorTok\StringTok{ }\NormalTok{powerfive) }\OperatorTok\StringTok{ }\KeywordTok{select}\NormalTok{(Institution, Conference, }\StringTok{`}\DataTypeTok{2018}\StringTok{`}\NormalTok{)}
\end{Highlighting}
\end{Shaded}

\begin{verbatim}
## # A tibble: 65 x 3
##    Institution    Conference `2018`
##    <chr>          <chr>       <dbl>
##  1 Alabama        SEC        710931
##  2 Arizona        Pac-12     318051
##  3 Arizona St.    Pac-12     291091
##  4 Arkansas       SEC        367748
##  5 Auburn         SEC        591236
##  6 Baylor         Big 12     248017
##  7 Boston College ACC        263363
##  8 California     Pac-12     300061
##  9 Clemson        ACC        562799
## 10 Colorado       Pac-12     274852
## # ... with 55 more rows
\end{verbatim}

If you have truly massive data, Select has tools to help you select fields that start\_with the same things or ends with a certain word. \href{https://dplyr.tidyverse.org/reference/select.html}{The documentation will guide you} if you need those someday. For 90 plus percent of what we do, just naming the fields will be sufficient.

\hypertarget{using-conditional-filters-to-set-limits}{%
\section{Using conditional filters to set limits}\label{using-conditional-filters-to-set-limits}}

Let's return to the blistering season of Drayton Whiteside using our dataset of \href{https://unl.box.com/s/s1wzw61u9ia50qmirfhuvprgpmmah9rj}{every college basketball player's season stats in 2018-19 season}. How can we set limits in something like a question of who had the best season? Let's get our Drayton Whiteside data from the previous chapter back up.

\begin{Shaded}
\begin{Highlighting}[]
\NormalTok{players <-}\StringTok{ }\KeywordTok{read_csv}\NormalTok{(}\StringTok{"data/players19.csv"}\NormalTok{)}
\end{Highlighting}
\end{Shaded}

\begin{verbatim}
## Warning: Missing column names filled in: 'X1' [1]
\end{verbatim}

\begin{verbatim}
## Parsed with column specification:
## cols(
##   .default = col_double(),
##   Team = col_character(),
##   Conference = col_character(),
##   Player = col_character(),
##   Class = col_character(),
##   Pos = col_character(),
##   Height = col_character(),
##   Hometown = col_character(),
##   `High School` = col_character(),
##   Summary = col_character()
## )
\end{verbatim}

\begin{verbatim}
## See spec(...) for full column specifications.
\end{verbatim}

\begin{Shaded}
\begin{Highlighting}[]
\NormalTok{players }\OperatorTok
\StringTok{  }\KeywordTok{mutate}\NormalTok{(}\DataTypeTok{trueshooting =}\NormalTok{ (PTS}\OperatorTok{/}\NormalTok{(}\DecValTok{2}\OperatorTok{*}\NormalTok{(FGA }\OperatorTok{+}\StringTok{ }\NormalTok{(.}\DecValTok{44}\OperatorTok{*}\NormalTok{FTA))))}\OperatorTok{*}\DecValTok{100}\NormalTok{) }\OperatorTok
\StringTok{  }\KeywordTok{arrange}\NormalTok{(}\KeywordTok{desc}\NormalTok{(trueshooting))}
\end{Highlighting}
\end{Shaded}

\begin{verbatim}
## # A tibble: 5,386 x 60
##       X1 Team  Conference Player   `#` Class Pos   Height Weight Hometown
##    <dbl> <chr> <chr>      <chr>  <dbl> <chr> <chr> <chr>   <dbl> <chr>   
##  1   579 Texa~ Big 12     Drayt~     4 JR    G     6-0       156 Austin,~
##  2   843 Ston~ AEC        Nick ~    42 FR    F     6-7       240 Port Je~
##  3  1059 Sout~ Southland  Patri~    22 SO    F     6-3       210 Folsom,~
##  4  4269 Dayt~ A-10       Camro~    52 SO    G     5-7       160 Country~
##  5  4681 Cali~ Pac-12     David~    21 JR    G     6-4       185 Newbury~
##  6   326 Virg~ ACC        Grant~     1 FR    G     <NA>       NA Charlot~
##  7   410 Vand~ SEC        Mac H~    42 FR    G     6-6       182 Chattan~
##  8  1390 Sain~ A-10       Jack ~    31 JR    G     6-6       205 Mattoon~
##  9  2230 NJIT~ A-Sun      Patri~     3 SO    G     5-9       160 West Or~
## 10   266 Wash~ Pac-12     Reaga~    34 FR    F     6-6       225 Santa A~
## # ... with 5,376 more rows, and 50 more variables: `High School` <chr>,
## #   Summary <chr>, Rk.x <dbl>, G <dbl>, GS <dbl>, MP <dbl>, FG <dbl>,
## #   FGA <dbl>, `FG%` <dbl>, `2P` <dbl>, `2PA` <dbl>, `2P%` <dbl>, `3P` <dbl>,
## #   `3PA` <dbl>, `3P%` <dbl>, FT <dbl>, FTA <dbl>, `FT%` <dbl>, ORB <dbl>,
## #   DRB <dbl>, TRB <dbl>, AST <dbl>, STL <dbl>, BLK <dbl>, TOV <dbl>, PF <dbl>,
## #   PTS <dbl>, Rk.y <dbl>, PER <dbl>, `TS%` <dbl>, `eFG%` <dbl>, `3PAr` <dbl>,
## #   FTr <dbl>, PProd <dbl>, `ORB%` <dbl>, `DRB%` <dbl>, `TRB%` <dbl>,
## #   `AST%` <dbl>, `STL%` <dbl>, `BLK%` <dbl>, `TOV%` <dbl>, `USG%` <dbl>,
## #   OWS <dbl>, DWS <dbl>, WS <dbl>, `WS/40` <dbl>, OBPM <dbl>, DBPM <dbl>,
## #   BPM <dbl>, trueshooting <dbl>
\end{verbatim}

In most contests like the batting title in Major League Baseball, there's a minimum number of X to qualify. In baseball, it's at bats. In basketball, it attempts. So let's set a floor and see how it changes. What if we said you had to have played 100 minutes in a season? The top players in college basketball play more than 1000 minutes in a season. So 100 is not that much. Let's try it and see.

\begin{Shaded}
\begin{Highlighting}[]
\NormalTok{players }\OperatorTok
\StringTok{  }\KeywordTok{mutate}\NormalTok{(}\DataTypeTok{trueshooting =}\NormalTok{ (PTS}\OperatorTok{/}\NormalTok{(}\DecValTok{2}\OperatorTok{*}\NormalTok{(FGA }\OperatorTok{+}\StringTok{ }\NormalTok{(.}\DecValTok{44}\OperatorTok{*}\NormalTok{FTA))))}\OperatorTok{*}\DecValTok{100}\NormalTok{) }\OperatorTok
\StringTok{  }\KeywordTok{arrange}\NormalTok{(}\KeywordTok{desc}\NormalTok{(trueshooting)) }\OperatorTok
\StringTok{  }\KeywordTok{filter}\NormalTok{(MP }\OperatorTok{>}\StringTok{ }\DecValTok{100}\NormalTok{)}
\end{Highlighting}
\end{Shaded}

\begin{verbatim}
## # A tibble: 3,659 x 60
##       X1 Team  Conference Player   `#` Class Pos   Height Weight Hometown
##    <dbl> <chr> <chr>      <chr>  <dbl> <chr> <chr> <chr>   <dbl> <chr>   
##  1  4634 Cent~ Southland  Jorda~    33 JR    G     6-1       185 Harriso~
##  2  3623 Hart~ AEC        Max T~    20 SR    G     6-5       200 Rye, NY 
##  3  2675 Mich~ Big Ten    Thoma~    15 FR    F     6-8       225 Clarkst~
##  4  5175 Litt~ Sun Belt   Kris ~    32 SO    F     6-8       194 Dewitt,~
##  5  5205 Ariz~ Pac-12     De'Qu~    32 SR    F     6-10      225 St. Tho~
##  6  4099 ETSU~ Southern   Lucas~    25 JR    C     7-0       220 De Lier~
##  7  3006 Loui~ Sun Belt   Brand~     0 SR    G     6-4       180 Hawthor~
##  8   570 Texa~ Big 12     Jaxso~    10 FR    F     6-11      220 Lovelan~
##  9  1704 Pepp~ WCC        Victo~    34 FR    C     6-9       200 Owerri,~
## 10  4056 East~ MAC        Jalen~    30 SO    F     6-9       215 Pasco, ~
## # ... with 3,649 more rows, and 50 more variables: `High School` <chr>,
## #   Summary <chr>, Rk.x <dbl>, G <dbl>, GS <dbl>, MP <dbl>, FG <dbl>,
## #   FGA <dbl>, `FG%` <dbl>, `2P` <dbl>, `2PA` <dbl>, `2P%` <dbl>, `3P` <dbl>,
## #   `3PA` <dbl>, `3P%` <dbl>, FT <dbl>, FTA <dbl>, `FT%` <dbl>, ORB <dbl>,
## #   DRB <dbl>, TRB <dbl>, AST <dbl>, STL <dbl>, BLK <dbl>, TOV <dbl>, PF <dbl>,
## #   PTS <dbl>, Rk.y <dbl>, PER <dbl>, `TS%` <dbl>, `eFG%` <dbl>, `3PAr` <dbl>,
## #   FTr <dbl>, PProd <dbl>, `ORB%` <dbl>, `DRB%` <dbl>, `TRB%` <dbl>,
## #   `AST%` <dbl>, `STL%` <dbl>, `BLK%` <dbl>, `TOV%` <dbl>, `USG%` <dbl>,
## #   OWS <dbl>, DWS <dbl>, WS <dbl>, `WS/40` <dbl>, OBPM <dbl>, DBPM <dbl>,
## #   BPM <dbl>, trueshooting <dbl>
\end{verbatim}

Now you get Central Arkansas Bears Junior Jordan Grant, who played in 25 games and was on the floor for 152 minutes. So he played regularly. But in that time, he only attempted 16 shots, and made 68 percent of them. In other words, when he shot, he probably scored. He just rarely shot.

So is 100 minutes our level? Here's the truth -- there's not really an answer here. We're picking a cutoff. If you can cite a reason for it and defend it, then it probably works.

\hypertarget{top-list}{%
\section{Top list}\label{top-list}}

One last little dplyr trick that's nice to have in the toolbox is a shortcut for selecting only the top values for your dataset. Want to make a Top 10 List? Or Top 25? Or Top Whatever You Want? It's easy.

So what are the top 10 Power Five schools by season attendance. All we're doing here is chaining commands together with what we've already got. We're \emph{filtering} by our list of Power Five conferences, we're \emph{selecting} the three fields we need, now we're going to \emph{arrange} it by total attendance and then we'll introduce the new function: \texttt{top\_n}. The \texttt{top\_n} function just takes a number. So we want a top 10 list? We do it like this:

\begin{Shaded}
\begin{Highlighting}[]
\NormalTok{attendance }\OperatorTok\StringTok{ }\KeywordTok{filter}\NormalTok{(Conference }\OperatorTok\StringTok{ }\NormalTok{powerfive) }\OperatorTok\StringTok{ }\KeywordTok{select}\NormalTok{(Institution, Conference, }\StringTok{`}\DataTypeTok{2018}\StringTok{`}\NormalTok{) }\OperatorTok\StringTok{ }\KeywordTok{arrange}\NormalTok{(}\KeywordTok{desc}\NormalTok{(}\StringTok{`}\DataTypeTok{2018}\StringTok{`}\NormalTok{)) }\OperatorTok\StringTok{ }\KeywordTok{top_n}\NormalTok{(}\DecValTok{10}\NormalTok{)}
\end{Highlighting}
\end{Shaded}

\begin{verbatim}
## Selecting by 2018
\end{verbatim}

\begin{verbatim}
## # A tibble: 10 x 3
##    Institution Conference `2018`
##    <chr>       <chr>       <dbl>
##  1 Michigan    Big Ten    775156
##  2 Penn St.    Big Ten    738396
##  3 Ohio St.    Big Ten    713630
##  4 Alabama     SEC        710931
##  5 LSU         SEC        705733
##  6 Texas A&M   SEC        698908
##  7 Tennessee   SEC        650887
##  8 Georgia     SEC        649222
##  9 Nebraska    Big Ten    623240
## 10 Oklahoma    Big 12     607146
\end{verbatim}

That's all there is to it. Just remember -- for it to work correctly, you need to sort your data BEFORE you run top\_n. Otherwise, you're just getting the first 10 values in the list. The function doesn't know what field you want the top values of. You have to do it.

\hypertarget{transforming-data}{%
\chapter{Transforming data}\label{transforming-data}}

Sometimes long data needs to be wide, and sometimes wide data needs to be long. I'll explain.

You are soon going to discover that long before you can visualize data, \textbf{you need to have it in a form that the visualization library can deal with}. One of the ways that isn't immediately obvious is \textbf{how your data is cast}. Most of the data you will encounter will be \textbf{wide -- each row will represent a single entity with multiple measures for that entity}. So think of states. Your row of your dataset could have the state name, population, average life expectancy and other demographic data.

But what if your visualization library needs one row for each measure? So state, data type and the data. Nebraska, Population, 1,929,000. That's one row. Then the next row is Nebraska, Average Life Expectancy, 76. That's the next row. That's where recasting your data comes in.

We can use a library called \texttt{tidyr} to \texttt{pivot\_longer} or \texttt{pivot\_wider} the data, depending on what we need. We'll use a \href{https://unl.box.com/s/etqna5bfvf3b0gsnw0kcjjn1rxx9335s}{dataset of college football attendance} to demonstrate. First we need some libraries.

\begin{Shaded}
\begin{Highlighting}[]
\KeywordTok{library}\NormalTok{(tidyverse)}
\end{Highlighting}
\end{Shaded}

Now we'll load the data.

\begin{Shaded}
\begin{Highlighting}[]
\NormalTok{attendance <-}\StringTok{ }\KeywordTok{read_csv}\NormalTok{(}\StringTok{'data/attendance.csv'}\NormalTok{)}
\end{Highlighting}
\end{Shaded}

\begin{verbatim}
## Parsed with column specification:
## cols(
##   Institution = col_character(),
##   Conference = col_character(),
##   `2013` = col_double(),
##   `2014` = col_double(),
##   `2015` = col_double(),
##   `2016` = col_double(),
##   `2017` = col_double(),
##   `2018` = col_double()
## )
\end{verbatim}

\begin{Shaded}
\begin{Highlighting}[]
\NormalTok{attendance}
\end{Highlighting}
\end{Shaded}

\begin{verbatim}
## # A tibble: 150 x 8
##    Institution     Conference      `2013` `2014` `2015` `2016` `2017` `2018`
##    <chr>           <chr>            <dbl>  <dbl>  <dbl>  <dbl>  <dbl>  <dbl>
##  1 Air Force       MWC             228562 168967 156158 177519 174924 166205
##  2 Akron           MAC             107101  55019 108588  62021 117416  92575
##  3 Alabama         SEC             710538 710736 707786 712747 712053 710931
##  4 Appalachian St. FBS Independent 149366     NA     NA     NA     NA     NA
##  5 Appalachian St. Sun Belt            NA 138995 128755 156916 154722 131716
##  6 Arizona         Pac-12          285713 354973 308355 338017 255791 318051
##  7 Arizona St.     Pac-12          501509 343073 368985 286417 359660 291091
##  8 Arkansas        SEC             431174 399124 471279 487067 442569 367748
##  9 Arkansas St.    Sun Belt        149477 149163 138043 136200 119538 119001
## 10 Army West Point FBS Independent 169781 171310 185946 163267 185543 190156
## # ... with 140 more rows
\end{verbatim}

So as you can see, each row represents a school, and then each column represents a year. This is great for calculating the percent change -- we can subtract a column from a column and divide by that column. But later, when we want to chart each school's attendance over the years, we have to have each row be one team for one year. Nebraska in 2013, then Nebraska in 2014, and Nebraska in 2015 and so on.

To do that, we use \texttt{pivot\_longer} because we're making wide data long. Since all of the columns we want to make rows start with 20, we can use that in our \texttt{cols} directive. Then we give that column a name -- Year -- and the values for each year need a name too. Those are the attendance figure. We can see right away how this works.

\begin{Shaded}
\begin{Highlighting}[]
\NormalTok{attendance }\OperatorTok\StringTok{ }\KeywordTok{pivot_longer}\NormalTok{(}\DataTypeTok{cols =} \KeywordTok{starts_with}\NormalTok{(}\StringTok{"20"}\NormalTok{), }\DataTypeTok{names_to =} \StringTok{"Year"}\NormalTok{, }\DataTypeTok{values_to =} \StringTok{"Attendance"}\NormalTok{)}
\end{Highlighting}
\end{Shaded}

\begin{verbatim}
## # A tibble: 900 x 4
##    Institution Conference Year  Attendance
##    <chr>       <chr>      <chr>      <dbl>
##  1 Air Force   MWC        2013      228562
##  2 Air Force   MWC        2014      168967
##  3 Air Force   MWC        2015      156158
##  4 Air Force   MWC        2016      177519
##  5 Air Force   MWC        2017      174924
##  6 Air Force   MWC        2018      166205
##  7 Akron       MAC        2013      107101
##  8 Akron       MAC        2014       55019
##  9 Akron       MAC        2015      108588
## 10 Akron       MAC        2016       62021
## # ... with 890 more rows
\end{verbatim}

We've gone from 150 rows to 900, but that's expected when we have 6 years for each team.

\hypertarget{making-long-data-wide}{%
\section{Making long data wide}\label{making-long-data-wide}}

We can reverse this process using \texttt{pivot\_wider}, which makes long data wide.

Why do any of this?

In some cases, you're going to be given long data and you need to calculate some metric using two of the years -- a percent change for instance. So you'll need to make the data wide to do that. You might then have to re-lengthen the data now with the percent change. Some project require you to do all kinds of flexing like this. It just depends on the data.

So let's take what we made above and turn it back into wide data.

\begin{Shaded}
\begin{Highlighting}[]
\NormalTok{longdata <-}\StringTok{ }\NormalTok{attendance }\OperatorTok\StringTok{ }\KeywordTok{pivot_longer}\NormalTok{(}\DataTypeTok{cols =} \KeywordTok{starts_with}\NormalTok{(}\StringTok{"20"}\NormalTok{), }\DataTypeTok{names_to =} \StringTok{"Year"}\NormalTok{, }\DataTypeTok{values_to =} \StringTok{"Attendance"}\NormalTok{)}

\NormalTok{longdata}
\end{Highlighting}
\end{Shaded}

\begin{verbatim}
## # A tibble: 900 x 4
##    Institution Conference Year  Attendance
##    <chr>       <chr>      <chr>      <dbl>
##  1 Air Force   MWC        2013      228562
##  2 Air Force   MWC        2014      168967
##  3 Air Force   MWC        2015      156158
##  4 Air Force   MWC        2016      177519
##  5 Air Force   MWC        2017      174924
##  6 Air Force   MWC        2018      166205
##  7 Akron       MAC        2013      107101
##  8 Akron       MAC        2014       55019
##  9 Akron       MAC        2015      108588
## 10 Akron       MAC        2016       62021
## # ... with 890 more rows
\end{verbatim}

To \texttt{pivot\_wider}, we just need to say where our column names are coming from -- the Year -- and where the data under it should come from -- Attendance.

\begin{Shaded}
\begin{Highlighting}[]
\NormalTok{longdata }\OperatorTok\StringTok{ }\KeywordTok{pivot_wider}\NormalTok{(}\DataTypeTok{names_from =}\NormalTok{ Year, }\DataTypeTok{values_from =}\NormalTok{ Attendance)}
\end{Highlighting}
\end{Shaded}

\begin{verbatim}
## # A tibble: 150 x 8
##    Institution     Conference      `2013` `2014` `2015` `2016` `2017` `2018`
##    <chr>           <chr>            <dbl>  <dbl>  <dbl>  <dbl>  <dbl>  <dbl>
##  1 Air Force       MWC             228562 168967 156158 177519 174924 166205
##  2 Akron           MAC             107101  55019 108588  62021 117416  92575
##  3 Alabama         SEC             710538 710736 707786 712747 712053 710931
##  4 Appalachian St. FBS Independent 149366     NA     NA     NA     NA     NA
##  5 Appalachian St. Sun Belt            NA 138995 128755 156916 154722 131716
##  6 Arizona         Pac-12          285713 354973 308355 338017 255791 318051
##  7 Arizona St.     Pac-12          501509 343073 368985 286417 359660 291091
##  8 Arkansas        SEC             431174 399124 471279 487067 442569 367748
##  9 Arkansas St.    Sun Belt        149477 149163 138043 136200 119538 119001
## 10 Army West Point FBS Independent 169781 171310 185946 163267 185543 190156
## # ... with 140 more rows
\end{verbatim}

And just like that, we're back.

\hypertarget{why-this-matters}{%
\section{Why this matters}\label{why-this-matters}}

This matters because certain visualization types need wide or long data. A significant hurdle you will face for the rest of the semester is getting the data in the right format for what you want to do.

So let me walk you through an example using this data.

Let's look at Nebraska's attendance over the time period. In order to do that, I need long data because that's what the charting library, \texttt{ggplot2}, needs. You're going to learn a lot more about ggplot later.

\begin{Shaded}
\begin{Highlighting}[]
\NormalTok{nebraska <-}\StringTok{ }\NormalTok{longdata }\OperatorTok\StringTok{ }\KeywordTok{filter}\NormalTok{(Institution }\OperatorTok{==}\StringTok{ "Nebraska"}\NormalTok{)}
\end{Highlighting}
\end{Shaded}

Now that we have long data for just Nebraska, we can chart it.

\begin{Shaded}
\begin{Highlighting}[]
\KeywordTok{ggplot}\NormalTok{(nebraska, }\KeywordTok{aes}\NormalTok{(}\DataTypeTok{x=}\NormalTok{Year, }\DataTypeTok{y=}\NormalTok{Attendance, }\DataTypeTok{group=}\DecValTok{1}\NormalTok{)) }\OperatorTok{+}\StringTok{ }
\StringTok{  }\KeywordTok{geom_line}\NormalTok{() }\OperatorTok{+}\StringTok{ }
\StringTok{  }\KeywordTok{scale_y_continuous}\NormalTok{(}\DataTypeTok{labels =}\NormalTok{ scales}\OperatorTok{::}\NormalTok{comma) }\OperatorTok{+}\StringTok{ }
\StringTok{  }\KeywordTok{labs}\NormalTok{(}\DataTypeTok{x=}\StringTok{"Year"}\NormalTok{, }\DataTypeTok{y=}\StringTok{"Attendance"}\NormalTok{, }\DataTypeTok{title=}\StringTok{"We'll all stick together?"}\NormalTok{, }\DataTypeTok{subtitle=}\StringTok{"It's not as bad as you think -- they widened the seats, cutting the number."}\NormalTok{, }\DataTypeTok{caption=}\StringTok{"Source: NCAA | By Matt Waite"}\NormalTok{, }\DataTypeTok{color =} \StringTok{"Outcome"}\NormalTok{) }\OperatorTok{+}
\StringTok{  }\KeywordTok{theme_minimal}\NormalTok{() }\OperatorTok{+}\StringTok{ }
\StringTok{  }\KeywordTok{theme}\NormalTok{(}
    \DataTypeTok{plot.title =} \KeywordTok{element_text}\NormalTok{(}\DataTypeTok{size =} \DecValTok{16}\NormalTok{, }\DataTypeTok{face =} \StringTok{"bold"}\NormalTok{),}
    \DataTypeTok{axis.title =} \KeywordTok{element_text}\NormalTok{(}\DataTypeTok{size =} \DecValTok{10}\NormalTok{),}
    \DataTypeTok{axis.title.y =} \KeywordTok{element_blank}\NormalTok{(),}
    \DataTypeTok{axis.text =} \KeywordTok{element_text}\NormalTok{(}\DataTypeTok{size =} \DecValTok{7}\NormalTok{),}
    \DataTypeTok{axis.ticks =} \KeywordTok{element_blank}\NormalTok{(),}
    \DataTypeTok{panel.grid.minor =} \KeywordTok{element_blank}\NormalTok{(),}
    \DataTypeTok{panel.grid.major.x =} \KeywordTok{element_blank}\NormalTok{(),}
    \DataTypeTok{legend.position=}\StringTok{"bottom"}
\NormalTok{  )}
\end{Highlighting}
\end{Shaded}

\includegraphics{SportsData_files/figure-latex/unnamed-chunk-49-1.pdf}

\hypertarget{simulations}{%
\chapter{Simulations}\label{simulations}}

Two seasons ago, James Palmer Jr.~shot 139 three point attempts and made 43 of them for a .309 shooting percentage last year. A few weeks into last season, he was 7 for 39 -- a paltry .179. Is something wrong or is this just bad luck?

Luck is something that comes up a lot in sports. Is a team unlucky? Or a player? One way we can get to this, we can get to that is by simulating things based on their typical percentages. Simulations work by choosing random values within a range based on a distribution. The most common distribution is the normal or binomial distribution. The normal distribution is where the most cases appear around the mean, 66 percent of cases are within one standard deviation from the mean, and the further away from the mean you get, the more rare things become.

\includegraphics[width=17.64in]{images/simulations2}

Let's simulate 39 three point attempts 1000 times with his season long shooting percentage and see if this could just be random chance or something else.

We do this using a base R function called \texttt{rbinom} or binomial distribution. So what that means is there's a normally distrubuted chance that James Palmer Jr.~is going to shoot above and below his career three point shooting percentage. If we randomly assign values in that distribution 1000 times, how many times will it come up 7, like this example?

\begin{Shaded}
\begin{Highlighting}[]
\KeywordTok{set.seed}\NormalTok{(}\DecValTok{1234}\NormalTok{)}

\NormalTok{simulations <-}\StringTok{ }\KeywordTok{rbinom}\NormalTok{(}\DataTypeTok{n =} \DecValTok{1000}\NormalTok{, }\DataTypeTok{size =} \DecValTok{39}\NormalTok{, }\DataTypeTok{prob =} \FloatTok{.309}\NormalTok{)}

\KeywordTok{table}\NormalTok{(simulations)}
\end{Highlighting}
\end{Shaded}

\begin{verbatim}
## simulations
##   3   4   5   6   7   8   9  10  11  12  13  14  15  16  17  18  19  20  21  22 
##   1   4   5  12  35  44  76 117 134 135 135  99  71  53  37  21  15   2   3   1
\end{verbatim}

How do we read this? The first row and the second row form a pair. The top row is the number of shots made. The number immediately under it is the number of simulations where that occurred.

\includegraphics[width=23.06in]{images/simulations1}

So what we see is given his season long shooting percentage, it's not out of the realm of randomness that with just 39 attempts for Palmer, he's only hit only 7. In 1000 simulations, it comes up 35 times. Is he below where he should be? Yes. Will he likely improve and soon? Unless something is very wrong, yes. And indeed, by the end of the season, he finished with a .313 shooting percentage from 3 point range. So we can say he was just unlucky.

\hypertarget{cold-streaks}{%
\section{Cold streaks}\label{cold-streaks}}

During the Western Illinois game, the team, shooting .329 on the season from behind the arc, went 0-15 in the second half. How strange is that?

\begin{Shaded}
\begin{Highlighting}[]
\KeywordTok{set.seed}\NormalTok{(}\DecValTok{1234}\NormalTok{)}

\NormalTok{simulations <-}\StringTok{ }\KeywordTok{rbinom}\NormalTok{(}\DataTypeTok{n =} \DecValTok{1000}\NormalTok{, }\DataTypeTok{size =} \DecValTok{15}\NormalTok{, }\DataTypeTok{prob =} \FloatTok{.329}\NormalTok{)}

\KeywordTok{hist}\NormalTok{(simulations)}
\end{Highlighting}
\end{Shaded}

\includegraphics{SportsData_files/figure-latex/unnamed-chunk-53-1.pdf}

\begin{Shaded}
\begin{Highlighting}[]
\KeywordTok{table}\NormalTok{(simulations)}
\end{Highlighting}
\end{Shaded}

\begin{verbatim}
## simulations
##   0   1   2   3   4   5   6   7   8   9  10  11 
##   5  16  59 132 200 218 172  92  65  34   4   3
\end{verbatim}

Short answer: Really weird. If you simulate 15 threes 1000 times, sometimes you'll see them miss all of them, but only a few times -- five times, in this case. Most of the time, the team won't go 0-15 even once. So going ice cold is not totally out of the realm of random chance, but it's highly unlikely.

\hypertarget{correlations-and-regression}{%
\chapter{Correlations and regression}\label{correlations-and-regression}}

Throughout sports, you will find no shortage of opinions. From people yelling at their TV screens to an entire industry of people paid to have opinions, there are no shortage of reasons why this team sucks and that player is great. They may have their reasons, but a better question is, does that reason really matter?

Can we put some numbers behind that? Can we prove it or not?

This is what we're going to start to answer. And we'll do it with correlations and regressions.

First, we need libraries and \href{https://unl.box.com/s/zlxoptqixkt98gubk3i6316qun99l49r}{data}.

\begin{Shaded}
\begin{Highlighting}[]
\KeywordTok{library}\NormalTok{(tidyverse)}
\end{Highlighting}
\end{Shaded}

\begin{Shaded}
\begin{Highlighting}[]
\NormalTok{correlations <-}\StringTok{ }\KeywordTok{read_csv}\NormalTok{(}\StringTok{"data/correlations.csv"}\NormalTok{)}
\end{Highlighting}
\end{Shaded}

\begin{verbatim}
## Parsed with column specification:
## cols(
##   Name = col_character(),
##   OffPoints = col_double(),
##   OffPointsG = col_double(),
##   DefPoints = col_double(),
##   DefPointsG = col_double(),
##   Pen. = col_double(),
##   Yards = col_double(),
##   `Pen./G` = col_double(),
##   OffConversions = col_double(),
##   OffConversionPct = col_double(),
##   DefConversions = col_double(),
##   DefConversionPct = col_double()
## )
\end{verbatim}

To do this, we need all FBS college football teams and their season stats from last year. How much, over the course of a season, does a thing matter? That's the question you're going to answer.

In our case, we want to know how much does a team's accumulated penalties influence the number of points they score in a season? How much difference can we explain in points with penalties?

We're going to use two different methods here and they're closely related. Correlations -- specifically the Pearson Correlation Coefficient -- is a measure of how related two numbers are in a linear fashion. In other words -- if our X value goes up one, what happens to Y? If it also goes up 1, that's a perfect correlation. X goes up 1, Y goes up 1. Every time. Correlation coefficients are a number between 0 and 1, with zero being no correlation and 1 being perfect correlation \textbf{if our data is linear}. We'll soon go over scatterplots to visually determine if our data is linear, but for now, we have a hypothesis: More penalties are bad. Penalties hurt. So if a team gets lots of them, they should have worse outcomes than teams that get few of them. That is an argument for a linear relationship between them.

But is there one?

We're going create a new dataframe called newcorrelations that takes our data that we imported and adds a column called \texttt{differential} because we don't have separate offense and defense penalties, and then we'll use correlations to see how related those two things are.

\begin{Shaded}
\begin{Highlighting}[]
\NormalTok{newcorrelations <-}\StringTok{ }\NormalTok{correlations }\OperatorTok\StringTok{ }
\StringTok{  }\KeywordTok{mutate}\NormalTok{(}\DataTypeTok{differential =}\NormalTok{ OffPoints }\OperatorTok{-}\StringTok{ }\NormalTok{DefPoints)}
\end{Highlighting}
\end{Shaded}

In R, there is a \texttt{cor} function, and it works much the same as \texttt{mean} or \texttt{median}. So we want to see if \texttt{differential} is correlated with \texttt{Yards}, which is the yards of penalties a team gets in a game. We do that by referenceing \texttt{differential} and \texttt{Yards} and specifying we want a \texttt{pearson} correlation. The number we get back is the correlation coefficient.

\begin{Shaded}
\begin{Highlighting}[]
\NormalTok{newcorrelations }\OperatorTok\StringTok{ }\KeywordTok{summarise}\NormalTok{(}\DataTypeTok{correlation =} \KeywordTok{cor}\NormalTok{(differential, Yards, }\DataTypeTok{method=}\StringTok{"pearson"}\NormalTok{))}
\end{Highlighting}
\end{Shaded}

\begin{verbatim}
## # A tibble: 1 x 1
##   correlation
##         <dbl>
## 1       0.201
\end{verbatim}

So on a scale of 0 to 1, penalty yards and whether or not the team scores more points than it give up are at .2. You could say they're 20 percent related. Another way to say it? They're 80 percent not related.

What about the number of penalties instead of the yards?

\begin{Shaded}
\begin{Highlighting}[]
\NormalTok{newcorrelations }\OperatorTok\StringTok{ }
\StringTok{  }\KeywordTok{summarise}\NormalTok{(}\DataTypeTok{correlation =} \KeywordTok{cor}\NormalTok{(differential, }\StringTok{`}\DataTypeTok{Pen.}\StringTok{`}\NormalTok{, }\DataTypeTok{method=}\StringTok{"pearson"}\NormalTok{))}
\end{Highlighting}
\end{Shaded}

\begin{verbatim}
## # A tibble: 1 x 1
##   correlation
##         <dbl>
## 1       0.153
\end{verbatim}

Even less related. What about looking at the average? Penalty yards per game?

\begin{Shaded}
\begin{Highlighting}[]
\NormalTok{newcorrelations }\OperatorTok\StringTok{ }\KeywordTok{summarise}\NormalTok{(}\DataTypeTok{correlation =} \KeywordTok{cor}\NormalTok{(differential, }\StringTok{`}\DataTypeTok{Pen./G}\StringTok{`}\NormalTok{, }\DataTypeTok{method=}\StringTok{"pearson"}\NormalTok{))}
\end{Highlighting}
\end{Shaded}

\begin{verbatim}
## # A tibble: 1 x 1
##   correlation
##         <dbl>
## 1     -0.0331
\end{verbatim}

Not only is it less related, but the relationship is inverted.

So wait, what does that mean?

It means that the number of penalty yards and penalties is actually positively related to differential. Put another way, teams that have more penalties and penalty yards tend to have better outcomes. The average is barely -- 3 percent -- negatively correlated, meaning that teams with higher averages score fewer points.

What? That makes no sense. How can that be?

Enter regression. Regression is how we try to fit our data into a line that explains the relationship the best. Regressions will help us predict things as well -- if we have a team that has so many penalties, what kind of point differential could we expect, given every FBS team? So regressions are about prediction, correlations are about description. Correlations describe a relationship. Regressions help us predict what that relationship means. Specifically, it tells us how much of the change in a dependent variable can be explained by the independent variable.

Another thing regressions do is give us some other tools to evaluate if the relationship is real or not.

Here's an example of using linear modeling to look at yards. Think of the \texttt{\textasciitilde{}} character as saying ``is predicted by''. The output looks like a lot, but what we need is a small part of it.

\begin{Shaded}
\begin{Highlighting}[]
\NormalTok{fit <-}\StringTok{ }\KeywordTok{lm}\NormalTok{(differential }\OperatorTok{~}\StringTok{ }\NormalTok{Yards, }\DataTypeTok{data =}\NormalTok{ newcorrelations)}
\KeywordTok{summary}\NormalTok{(fit)}
\end{Highlighting}
\end{Shaded}

\begin{verbatim}
## 
## Call:
## lm(formula = differential ~ Yards, data = newcorrelations)
## 
## Residuals:
##     Min      1Q  Median      3Q     Max 
## -351.02  -93.49    2.67  107.88  444.42 
## 
## Coefficients:
##               Estimate Std. Error t value Pr(>|t|)  
## (Intercept) -108.73848   59.70868  -1.821   0.0709 .
## Yards          0.19484    0.08399   2.320   0.0219 *
## ---
## Signif. codes:  0 '***' 0.001 '**' 0.01 '*' 0.05 '.' 0.1 ' ' 1
## 
## Residual standard error: 140 on 128 degrees of freedom
## Multiple R-squared:  0.04035,    Adjusted R-squared:  0.03285 
## F-statistic: 5.382 on 1 and 128 DF,  p-value: 0.02193
\end{verbatim}

There's three things we need here:

\begin{enumerate}
\def\labelenumi{\arabic{enumi}.}
\tightlist
\item
  First we want to look at the p-value. It's at the bottom right corner of the output. In the case of Yards, the p-value is .02193. The threshold we're looking for here is .05. If it's less than .05, then the relationship is considered to be \emph{statistically significant}. Significance here does not mean it's a big deal. It means it's not random. That's it. Just that. Not random. So in our case, the relationship between penalty yards and a team's aggregate point differential are not random. It's a real relationship.
\item
  Second, we look at the Adjusted R-squared value. It's right above the p-value. Adjusted R-squared is a measure of how much of the difference between teams aggregate point values can be explained by penalty yards. Our correlation coefficient said they're 20 percent related to each other, but penalty yard's ability to explain the difference between teams? About 3.3 percent. That's \ldots{} not much. It's really nothing.
\item
  The third thing we can look at, and we only bother if the first two are meaningful, is the coefficients. In the middle, you can see the (Intercept) is -108.73848 and the Yards coefficient is .19484. Remember high school algebra? Remember learning the equation of a line? Remember swearing that learning \texttt{y=mx+b} is stupid because you'll never need it again? Surprise. It's useful again. In this case, we could try to predict a team's aggregate score in a season -- will they score more than they give up -- by using \texttt{y=mx+b}. In this case, y is the aggregate score, m is .19484 and b is -108.73848. So we would multiply a teams total penalty yards by .19484 and then subtract 108.73848 from it. The result would tell you what the total aggregate score in the season would be. Chance that your even close with this? About 3 percent.
\end{enumerate}

You can see the problem in a graph. On the X axis is penalty yards, on the y is aggregate score. If these elements had a strong relationship, we'd see a clear pattern moving from right to left, sloping down. On the left would be the teams with lots of penalty yards and a negative point differential. On right would be teams with low penalty yards and high point differentials. Do you see that below?

\includegraphics{SportsData_files/figure-latex/unnamed-chunk-61-1.pdf}

\begin{quote}
\textbf{Your turn}: Try it with the other penalty measures. Total penalties and penalty yards per game. Does anything change? Do either of these meet the .05 threshold for randomness? Are either of these any more predictive?
\end{quote}

\hypertarget{a-more-predictive-example}{%
\section{A more predictive example}\label{a-more-predictive-example}}

So we've firmly established that penalties aren't predictive. But what is? One measure I've found to be highly predictive of a team's success is how well do they do on third down. It's simple really: Succeed on third down, you get to stay on offense. Fail on third down, you are punting (most likely) or settling for a field goal. Either way, you're scoring less than you would by scoring touchdowns. How related are points per game and third down conversion percentage?

\begin{Shaded}
\begin{Highlighting}[]
\NormalTok{newcorrelations }\OperatorTok\StringTok{ }
\StringTok{  }\KeywordTok{summarise}\NormalTok{(}\DataTypeTok{correlation =} \KeywordTok{cor}\NormalTok{(OffPointsG, OffConversionPct, }\DataTypeTok{method=}\StringTok{"pearson"}\NormalTok{))}
\end{Highlighting}
\end{Shaded}

\begin{verbatim}
## # A tibble: 1 x 1
##   correlation
##         <dbl>
## 1       0.666
\end{verbatim}

Answer: 67 percent. More than three times more related than penalty yards. But how meaningful is that relationship and how predictive is it?

\begin{Shaded}
\begin{Highlighting}[]
\NormalTok{third <-}\StringTok{ }\KeywordTok{lm}\NormalTok{(OffPointsG }\OperatorTok{~}\StringTok{ }\NormalTok{OffConversionPct, }\DataTypeTok{data =}\NormalTok{ newcorrelations)}
\KeywordTok{summary}\NormalTok{(third)}
\end{Highlighting}
\end{Shaded}

\begin{verbatim}
## 
## Call:
## lm(formula = OffPointsG ~ OffConversionPct, data = newcorrelations)
## 
## Residuals:
##      Min       1Q   Median       3Q      Max 
## -11.3861  -3.5411  -0.5885   2.9011  13.5188 
## 
## Coefficients:
##                  Estimate Std. Error t value Pr(>|t|)    
## (Intercept)      -4.74024    3.41041   -1.39    0.167    
## OffConversionPct  0.85625    0.08479   10.10   <2e-16 ***
## ---
## Signif. codes:  0 '***' 0.001 '**' 0.01 '*' 0.05 '.' 0.1 ' ' 1
## 
## Residual standard error: 5.111 on 128 degrees of freedom
## Multiple R-squared:  0.4434, Adjusted R-squared:  0.4391 
## F-statistic:   102 on 1 and 128 DF,  p-value: < 2.2e-16
\end{verbatim}

First we check p-value. See that e-16? That means scientific notation. That means our number is 2.2 times 10 to the -16 power. So -.000000000000000022. That's sixteen zeros between the decimal and 22. Is that less than .05? Uh, yeah. So this is really, really, really not random. But anyone who has watched a game of football knows this is true. It makes intuitive sense.

Second, Adjusted R-squared: .4391. So we can predict 44 percent of the difference in the total offensive points per game a team scores by simply looking at their third down conversion percentage.

Third, the coefficients: In this case, our \texttt{y=mx+b} formula looks like \texttt{y\ =\ .85625x-4.74024}. So if we were applying this, let's look at Nebraska's 31-28 loss to Iowa on Black Friday in 2018. Nebraska was 6-15 on third down in that game, or 40 percent (Iowa was 7 of 13 or 54 percent). Given those numbers, our formula predicts Nebraska should have scored how many points?

\begin{Shaded}
\begin{Highlighting}[]
\NormalTok{(}\FloatTok{0.85625} \OperatorTok{*}\StringTok{ }\DecValTok{40}\NormalTok{) }\OperatorTok{-}\StringTok{ }\FloatTok{4.74024} 
\end{Highlighting}
\end{Shaded}

\begin{verbatim}
## [1] 29.50976
\end{verbatim}

That's really close to the 28 they did score. And Iowa?

\begin{Shaded}
\begin{Highlighting}[]
\NormalTok{(}\FloatTok{0.85625} \OperatorTok{*}\StringTok{ }\DecValTok{54}\NormalTok{) }\OperatorTok{-}\StringTok{ }\FloatTok{4.74024} 
\end{Highlighting}
\end{Shaded}

\begin{verbatim}
## [1] 41.49726
\end{verbatim}

By our model, Iowa should have scored 10 more points than they did. But they didn't. Why, besides Iowa is terrible and deserves punishment from the football gods for being Iowa? Remember our model can only explain 44 percent of the points. There's more to football than one metric.

\hypertarget{multiple-regression}{%
\chapter{Multiple regression}\label{multiple-regression}}

Last chapter, we looked at correlations and linear regression to predict how one element of a game would predict the score. But we know that a single variable, in all but the rarest instances, are not going to be that predictive. We need more than one. Enter multiple regression. Multiple regression lets us add -- wait for it -- multiple predictors to our equation to help us get a better

That presents it's own problems. So let's get our libraries and our data, this time of \href{https://unl.box.com/s/u9407jj007fxtnu1vbkybdawaqg6j3fw}{every college basketball game since the 2014-15 season} loaded up.

\begin{Shaded}
\begin{Highlighting}[]
\KeywordTok{library}\NormalTok{(tidyverse)}
\end{Highlighting}
\end{Shaded}

\begin{Shaded}
\begin{Highlighting}[]
\NormalTok{logs <-}\StringTok{ }\KeywordTok{read_csv}\NormalTok{(}\StringTok{"data/logs1519.csv"}\NormalTok{)}
\end{Highlighting}
\end{Shaded}

\begin{verbatim}
## Warning: Missing column names filled in: 'X1' [1]
\end{verbatim}

\begin{verbatim}
## Parsed with column specification:
## cols(
##   .default = col_double(),
##   Date = col_date(format = ""),
##   HomeAway = col_character(),
##   Opponent = col_character(),
##   W_L = col_character(),
##   Blank = col_logical(),
##   Team = col_character(),
##   Conference = col_character(),
##   season = col_character()
## )
\end{verbatim}

\begin{verbatim}
## See spec(...) for full column specifications.
\end{verbatim}

So one way to show how successful a basketball team was for a game is to show the differential between the team's score and the opponent's score. Score a lot more than the opponent = good, score a lot less than the opponent = bad. And, relatively speaking, the more the better. So let's create that differential.

\begin{Shaded}
\begin{Highlighting}[]
\NormalTok{logs <-}\StringTok{ }\NormalTok{logs }\OperatorTok\StringTok{ }\KeywordTok{mutate}\NormalTok{(}\DataTypeTok{Differential =}\NormalTok{ TeamScore }\OperatorTok{-}\StringTok{ }\NormalTok{OpponentScore)}
\end{Highlighting}
\end{Shaded}

The linear model code we used before is pretty straight forward. Its \texttt{field} is predicted by \texttt{field}. Here's a simple linear model that looks at predicting a team's point differential by looking at their offensive shooting percentage.

\begin{Shaded}
\begin{Highlighting}[]
\NormalTok{shooting <-}\StringTok{ }\KeywordTok{lm}\NormalTok{(TeamFGPCT }\OperatorTok{~}\StringTok{ }\NormalTok{Differential, }\DataTypeTok{data=}\NormalTok{logs)}
\KeywordTok{summary}\NormalTok{(shooting)}
\end{Highlighting}
\end{Shaded}

\begin{verbatim}
## 
## Call:
## lm(formula = TeamFGPCT ~ Differential, data = logs)
## 
## Residuals:
##       Min        1Q    Median        3Q       Max 
## -0.260485 -0.040230 -0.001096  0.039038  0.267457 
## 
## Coefficients:
##               Estimate Std. Error t value Pr(>|t|)    
## (Intercept)  4.399e-01  2.487e-04  1768.4   <2e-16 ***
## Differential 2.776e-03  1.519e-05   182.8   <2e-16 ***
## ---
## Signif. codes:  0 '***' 0.001 '**' 0.01 '*' 0.05 '.' 0.1 ' ' 1
## 
## Residual standard error: 0.05949 on 57514 degrees of freedom
##   (4 observations deleted due to missingness)
## Multiple R-squared:  0.3675, Adjusted R-squared:  0.3674 
## F-statistic: 3.341e+04 on 1 and 57514 DF,  p-value: < 2.2e-16
\end{verbatim}

Remember: There's a lot here, but only some of it we care about. What is the Adjusted R-squared value? What's the p-value and is it less than .05? In this case, we can predict 37 percent of the difference in differential with how well a team shoots the ball.

To add more predictors to this mix, we merely add them. But it's not that simple, as you'll see in a moment. So first, let's look at adding how well the other team shot to our prediction model:

\begin{Shaded}
\begin{Highlighting}[]
\NormalTok{model1 <-}\StringTok{ }\KeywordTok{lm}\NormalTok{(Differential }\OperatorTok{~}\StringTok{ }\NormalTok{TeamFGPCT }\OperatorTok{+}\StringTok{ }\NormalTok{OpponentFGPCT, }\DataTypeTok{data=}\NormalTok{logs)}
\KeywordTok{summary}\NormalTok{(model1)}
\end{Highlighting}
\end{Shaded}

\begin{verbatim}
## 
## Call:
## lm(formula = Differential ~ TeamFGPCT + OpponentFGPCT, data = logs)
## 
## Residuals:
##     Min      1Q  Median      3Q     Max 
## -49.591  -6.185  -0.198   5.938  68.344 
## 
## Coefficients:
##                Estimate Std. Error  t value Pr(>|t|)    
## (Intercept)      1.1195     0.3483    3.214  0.00131 ** 
## TeamFGPCT      118.5211     0.5279  224.518  < 2e-16 ***
## OpponentFGPCT -119.9369     0.5252 -228.372  < 2e-16 ***
## ---
## Signif. codes:  0 '***' 0.001 '**' 0.01 '*' 0.05 '.' 0.1 ' ' 1
## 
## Residual standard error: 9.407 on 57513 degrees of freedom
##   (4 observations deleted due to missingness)
## Multiple R-squared:  0.6683, Adjusted R-squared:  0.6683 
## F-statistic: 5.793e+04 on 2 and 57513 DF,  p-value: < 2.2e-16
\end{verbatim}

First things first: What is the adjusted R-squared?

Second: what is the p-value and is it less than .05?

Third: Compare the residual standard error. We went from .05949 to 9.4. The meaning of this is both really opaque and also simple -- we added a lot of error to our model by adding more measures -- 158 times more. Residual standard error is the total distance between what our model would predict and what we actually have in the data. So lots of residual error means the distance between reality and our model is wider. So the width of our predictive range in this example grew pretty dramatically, but so did the amount of the difference we could predict. It's a trade off.

One of the more difficult things to understand about multiple regression is the issue of multicollinearity. What that means is that there is significant correlation overlap between two variables -- the two are related to each other as well as to the target output -- and all you are doing by adding both of them is adding error with no real value to the R-squared. In pure statistics, we don't want any multicollinearity at all. Violating that assumption limits the applicability of what you are doing. So if we have some multicollinearity, it limits our scope of application to college basketball. We can't say this will work for every basketball league and level everywhere. What we need to do is see how correlated each value is to each other and throw out ones that are highly co-correlated.

So to find those, we have to create a correlation matrix that shows us how each value is correlated to our outcome variable, but also with each other. We can do that in the \texttt{Hmisc} library. We install that in the console with \texttt{install.packages("Hmisc")}

\begin{Shaded}
\begin{Highlighting}[]
\KeywordTok{library}\NormalTok{(Hmisc)}
\end{Highlighting}
\end{Shaded}

We can pass in every numeric value to the Hmisc library and get a correlation matrix out of it, but since we have a large number of values -- and many of them character values -- we should strip that down and reorder them. So that's what I'm doing here. I'm saying give me differential first, and then columns 9-24, and then 26-41. Why the skip? There's a blank column in the middle of the data -- a remnant of the scraper I used.

\begin{Shaded}
\begin{Highlighting}[]
\NormalTok{simplelogs <-}\StringTok{ }\NormalTok{logs }\OperatorTok\StringTok{ }\KeywordTok{select}\NormalTok{(Differential, }\DecValTok{9}\OperatorTok{:}\DecValTok{24}\NormalTok{, }\DecValTok{26}\OperatorTok{:}\DecValTok{41}\NormalTok{)}
\end{Highlighting}
\end{Shaded}

Before we proceed, what we're looking to do is follow the Differential column down, looking for correlation values near 1 or -1. Correlations go from -1, meaning perfect negative correlation, to 0, meaning no correlation, to 1, meaning perfect positive correlation. So we're looking for numbers near 1 or -1 for their predictive value. BUT: We then need to see if that value is also highly correlated with something else. If it is, we have a decision to make.

We get our correlation matrix like this:

\begin{Shaded}
\begin{Highlighting}[]
\NormalTok{cormatrix <-}\StringTok{ }\KeywordTok{rcorr}\NormalTok{(}\KeywordTok{as.matrix}\NormalTok{(simplelogs))}

\NormalTok{cormatrix}\OperatorTok{$}\NormalTok{r}
\end{Highlighting}
\end{Shaded}

\begin{verbatim}
##                       Differential       TeamFG      TeamFGA    TeamFGPCT
## Differential           1.000000000  0.584766682  0.107389235  0.606178206
## TeamFG                 0.584766682  1.000000000  0.563220974  0.751715176
## TeamFGA                0.107389235  0.563220974  1.000000000 -0.109620267
## TeamFGPCT              0.606178206  0.751715176 -0.109620267  1.000000000
## Team3P                 0.318300418  0.408787900  0.213352219  0.322872202
## Team3PA                0.056680627  0.179527313  0.426011924 -0.119421368
## Team3PPCT              0.367934059  0.380235821 -0.101463821  0.545986963
## TeamFT                 0.238182740 -0.022308582 -0.137853824  0.084649669
## TeamFTA                0.206075949 -0.027927391 -0.129851346  0.070632302
## TeamFTPCT              0.138833800  0.016247282 -0.044394472  0.056887587
## TeamOffRebounds        0.136095147  0.161626257  0.545231683 -0.234244567
## TeamTotalRebounds      0.470722398  0.328460524  0.470719037  0.018581908
## TeamAssists            0.540398009  0.664057724  0.284659104  0.566152928
## TeamSteals             0.277670288  0.210221346  0.208743124  0.080191710
## TeamBlocks             0.257608076  0.140856644  0.074555286  0.107327505
## TeamTurnovers         -0.180578328 -0.143210529 -0.223971265  0.001901048
## TeamPersonalFouls     -0.194427271 -0.014722266  0.107325560 -0.094653222
## OpponentFG            -0.538515115  0.144061400  0.256737262 -0.020183466
## OpponentFGA            0.001768386  0.302143806  0.301593528  0.126415534
## OpponentFGPCT         -0.614427717 -0.058571888  0.068034775 -0.114791403
## Opponent3P            -0.283754971  0.131517138  0.135290090  0.053105214
## Opponent3PA            0.013910296  0.191131927  0.138445785  0.118723805
## Opponent3PPCT         -0.382427841  0.008026622  0.057261756 -0.031370545
## OpponentFT            -0.269300868  0.019511923  0.157025930 -0.091558712
## OpponentFTA           -0.226064714  0.012937366  0.159529646 -0.101685664
## OpponentFTPCT         -0.175223632  0.007923359  0.023732217 -0.006190565
## OpponentOffRebounds   -0.089347536 -0.036316958  0.002848058 -0.042399744
## OpponentTotalRebounds -0.420010794 -0.225202127  0.316139528 -0.512983306
## OpponentAssists       -0.491676030  0.004558539  0.149320067 -0.106252682
## OpponentSteals        -0.187754380 -0.102436608 -0.131734964 -0.021724636
## OpponentBlocks        -0.262252627 -0.160469663  0.218483865 -0.356255034
## OpponentTurnovers      0.274326954  0.155293275  0.198127970  0.024254833
## OpponentPersonalFouls  0.169025733 -0.023116620 -0.107189301  0.060150658
##                             Team3P     Team3PA     Team3PPCT       TeamFT
## Differential           0.318300418  0.05668063  0.3679340589  0.238182740
## TeamFG                 0.408787900  0.17952731  0.3802358207 -0.022308582
## TeamFGA                0.213352219  0.42601192 -0.1014638212 -0.137853824
## TeamFGPCT              0.322872202 -0.11942137  0.5459869634  0.084649669
## Team3P                 1.000000000  0.70114773  0.7073663404 -0.106344056
## Team3PA                0.701147726  1.00000000  0.0407645751 -0.160515313
## Team3PPCT              0.707366340  0.04076458  1.0000000000  0.005129556
## TeamFT                -0.106344056 -0.16051531  0.0051295561  1.000000000
## TeamFTA               -0.137499074 -0.18150913 -0.0180696209  0.927525817
## TeamFTPCT              0.048777304  0.01119250  0.0553684315  0.387017653
## TeamOffRebounds       -0.062026229  0.12484929 -0.1968568361  0.087168289
## TeamTotalRebounds      0.038344971  0.12095682 -0.0628970009  0.190691619
## TeamAssists            0.519530086  0.28786139  0.4326950943 -0.016343370
## TeamSteals             0.016545254  0.04598400 -0.0246657289  0.088535320
## TeamBlocks             0.004747719 -0.02895321  0.0294277389  0.092392379
## TeamTurnovers         -0.088374940 -0.10883919 -0.0209433827  0.051609207
## TeamPersonalFouls     -0.024028303  0.02499520 -0.0498165852  0.217846416
## OpponentFG             0.123800594  0.15638030  0.0296913406  0.057853338
## OpponentFGA            0.148931744  0.13062824  0.0812237901  0.193116094
## OpponentFGPCT          0.029908235  0.08057726 -0.0264843759 -0.075399282
## Opponent3P             0.079455775  0.07482590  0.0402012413  0.024228311
## Opponent3PA            0.085704376  0.05927299  0.0601150176  0.079894905
## Opponent3PPCT          0.029666235  0.04634676  0.0005076038 -0.035478488
## OpponentFT             0.009796521  0.06316300 -0.0390873639  0.161311559
## OpponentFTA           -0.002503282  0.05474884 -0.0480732723  0.183801456
## OpponentFTPCT          0.022780414  0.02587876  0.0086512859 -0.015688533
## OpponentOffRebounds   -0.007870292 -0.01895081  0.0086776821  0.064938518
## OpponentTotalRebounds -0.062384273  0.20289676 -0.2638845414 -0.064969878
## OpponentAssists        0.029413582  0.08254506 -0.0320289494 -0.057730062
## OpponentSteals        -0.053878305 -0.05298037 -0.0251316716 -0.001883349
## OpponentBlocks        -0.111782062 -0.05804217 -0.0965607977 -0.065055523
## OpponentTurnovers      0.009284106  0.06383515 -0.0488449748  0.136922084
## OpponentPersonalFouls -0.127197007 -0.15536393 -0.0268876881  0.793539202
##                            TeamFTA    TeamFTPCT TeamOffRebounds
## Differential           0.206075949  0.138833800    0.1360951470
## TeamFG                -0.027927391  0.016247282    0.1616262575
## TeamFGA               -0.129851346 -0.044394472    0.5452316831
## TeamFGPCT              0.070632302  0.056887587   -0.2342445674
## Team3P                -0.137499074  0.048777304   -0.0620262290
## Team3PA               -0.181509133  0.011192503    0.1248492948
## Team3PPCT             -0.018069621  0.055368431   -0.1968568361
## TeamFT                 0.927525817  0.387017653    0.0871682888
## TeamFTA                1.000000000  0.053233778    0.1415933172
## TeamFTPCT              0.053233778  1.000000000   -0.0948040467
## TeamOffRebounds        0.141593317 -0.094804047    1.0000000000
## TeamTotalRebounds      0.231278690 -0.037356471    0.6373027887
## TeamAssists           -0.028289202  0.025948025    0.0509277222
## TeamSteals             0.111199125 -0.025969502    0.1195581042
## TeamBlocks             0.104112579 -0.001425412    0.1060163877
## TeamTurnovers          0.072070652 -0.034614485    0.0371728710
## TeamPersonalFouls      0.250787085 -0.025827923    0.0542337992
## OpponentFG             0.043602296  0.036986356   -0.0464694335
## OpponentFGA            0.193466766  0.040334507    0.0242353640
## OpponentFGPCT         -0.091897172  0.012864509   -0.0688833747
## Opponent3P             0.009600704  0.031763685   -0.0063710321
## Opponent3PA            0.071193179  0.032554796    0.0003753868
## Opponent3PPCT         -0.047136861  0.013996880   -0.0056578317
## OpponentFT             0.180010001 -0.009352580    0.0434399899
## OpponentFTA            0.213209437 -0.025707797    0.0584669041
## OpponentFTPCT         -0.032862991  0.028078614   -0.0319032781
## OpponentOffRebounds    0.077003661 -0.016936223   -0.0143325753
## OpponentTotalRebounds  0.004736343 -0.177541483   -0.0603891339
## OpponentAssists       -0.063875391 -0.007401206   -0.0386521955
## OpponentSteals         0.006758108 -0.022033431    0.0326977763
## OpponentBlocks        -0.053973588 -0.041175463    0.1571812909
## OpponentTurnovers      0.169704736 -0.035463921    0.1154717115
## OpponentPersonalFouls  0.866395092  0.018757079    0.1240631120
##                       TeamTotalRebounds   TeamAssists   TeamSteals   TeamBlocks
## Differential                0.470722398  0.5403980088  0.277670288  0.257608076
## TeamFG                      0.328460524  0.6640577242  0.210221346  0.140856644
## TeamFGA                     0.470719037  0.2846591045  0.208743124  0.074555286
## TeamFGPCT                   0.018581908  0.5661529279  0.080191710  0.107327505
## Team3P                      0.038344971  0.5195300862  0.016545254  0.004747719
## Team3PA                     0.120956819  0.2878613903  0.045984003 -0.028953212
## Team3PPCT                  -0.062897001  0.4326950943 -0.024665729  0.029427739
## TeamFT                      0.190691619 -0.0163433697  0.088535320  0.092392379
## TeamFTA                     0.231278690 -0.0282892019  0.111199125  0.104112579
## TeamFTPCT                  -0.037356471  0.0259480253 -0.025969502 -0.001425412
## TeamOffRebounds             0.637302789  0.0509277222  0.119558104  0.106016388
## TeamTotalRebounds           1.000000000  0.2321524530  0.027446991  0.265518873
## TeamAssists                 0.232152453  1.0000000000  0.164837110  0.144764562
## TeamSteals                  0.027446991  0.1648371104  1.000000000  0.065539758
## TeamBlocks                  0.265518873  0.1447645615  0.065539758  1.000000000
## TeamTurnovers               0.109155292 -0.0789200586  0.078278779  0.032775757
## TeamPersonalFouls          -0.007423332 -0.1050900267  0.005151965 -0.054105029
## OpponentFG                 -0.229331788 -0.0022308763 -0.138728115 -0.143969401
## OpponentFGA                 0.360268614  0.1863368268 -0.120696505  0.257245080
## OpponentFGPCT              -0.530432484 -0.1397140493 -0.068951590 -0.353110391
## Opponent3P                 -0.053371243  0.0354785684 -0.062074442 -0.103465578
## Opponent3PA                 0.232049186  0.1116023406 -0.039184667 -0.042234814
## Opponent3PPCT              -0.273572339 -0.0502063543 -0.047114732 -0.099440199
## OpponentFT                 -0.095266106 -0.0835716395 -0.034152581 -0.070920662
## OpponentFTA                -0.022971823 -0.0841605708 -0.022178476 -0.056095076
## OpponentFTPCT              -0.194279344 -0.0278263543 -0.041125993 -0.052504157
## OpponentOffRebounds        -0.052416263 -0.0333847454  0.016707012  0.178200671
## OpponentTotalRebounds      -0.059965631 -0.2225952122  0.035155522  0.037788375
## OpponentAssists            -0.218597433  0.0006884142 -0.053327136 -0.151146052
## OpponentSteals              0.066119486 -0.0288668673  0.055697260  0.028453380
## OpponentBlocks              0.013924890 -0.1657235463 -0.002230784 -0.038978593
## OpponentTurnovers          -0.034355689  0.1314533533  0.730885169  0.031375703
## OpponentPersonalFouls       0.189144014 -0.0267820830  0.071442012  0.080582762
##                       TeamTurnovers TeamPersonalFouls   OpponentFG  OpponentFGA
## Differential           -0.180578328      -0.194427271 -0.538515115  0.001768386
## TeamFG                 -0.143210529      -0.014722266  0.144061400  0.302143806
## TeamFGA                -0.223971265       0.107325560  0.256737262  0.301593528
## TeamFGPCT               0.001901048      -0.094653222 -0.020183466  0.126415534
## Team3P                 -0.088374940      -0.024028303  0.123800594  0.148931744
## Team3PA                -0.108839191       0.024995197  0.156380301  0.130628244
## Team3PPCT              -0.020943383      -0.049816585  0.029691341  0.081223790
## TeamFT                  0.051609207       0.217846416  0.057853338  0.193116094
## TeamFTA                 0.072070652       0.250787085  0.043602296  0.193466766
## TeamFTPCT              -0.034614485      -0.025827923  0.036986356  0.040334507
## TeamOffRebounds         0.037172871       0.054233799 -0.046469434  0.024235364
## TeamTotalRebounds       0.109155292      -0.007423332 -0.229331788  0.360268614
## TeamAssists            -0.078920059      -0.105090027 -0.002230876  0.186336827
## TeamSteals              0.078278779       0.005151965 -0.138728115 -0.120696505
## TeamBlocks              0.032775757      -0.054105029 -0.143969401  0.257245080
## TeamTurnovers           1.000000000       0.220285924  0.081879049  0.155947902
## TeamPersonalFouls       0.220285924       1.000000000 -0.015422966 -0.122639976
## OpponentFG              0.081879049      -0.015422966  1.000000000  0.515517123
## OpponentFGA             0.155947902      -0.122639976  0.515517123  1.000000000
## OpponentFGPCT          -0.023017156       0.078411084  0.754791141 -0.161220379
## Opponent3P             -0.018088322      -0.126817358  0.399027442  0.193563166
## Opponent3PA             0.041669476      -0.167647391  0.144074778  0.418730422
## Opponent3PPCT          -0.063187150      -0.015909552  0.395540055 -0.118020866
## OpponentFT              0.123594852       0.793147614 -0.013421944 -0.156152803
## OpponentFTA             0.154110278       0.865844664 -0.027151720 -0.151706668
## OpponentFTPCT          -0.034267574       0.026877590  0.037049836 -0.043324702
## OpponentOffRebounds     0.074131214       0.122282037  0.120715447  0.519792207
## OpponentTotalRebounds  -0.106168146       0.195017438  0.275438081  0.424276325
## OpponentAssists         0.072644677      -0.022619097  0.638304131  0.231851475
## OpponentSteals          0.709987911       0.064446997  0.140823916  0.165329579
## OpponentBlocks          0.006463872       0.087211248  0.129076992  0.045565883
## OpponentTurnovers       0.188537020       0.101693555 -0.183558009 -0.215633733
## OpponentPersonalFouls   0.131539040       0.322258517  0.015334210  0.136789046
##                       OpponentFGPCT   Opponent3P   Opponent3PA Opponent3PPCT
## Differential           -0.614427717 -0.283754971  0.0139102958 -0.3824278411
## TeamFG                 -0.058571888  0.131517138  0.1911319274  0.0080266219
## TeamFGA                 0.068034775  0.135290090  0.1384457845  0.0572617563
## TeamFGPCT              -0.114791403  0.053105214  0.1187238045 -0.0313705446
## Team3P                  0.029908235  0.079455775  0.0857043764  0.0296662353
## Team3PA                 0.080577258  0.074825900  0.0592729911  0.0463467602
## Team3PPCT              -0.026484376  0.040201241  0.0601150176  0.0005076038
## TeamFT                 -0.075399282  0.024228311  0.0798949051 -0.0354784876
## TeamFTA                -0.091897172  0.009600704  0.0711931792 -0.0471368607
## TeamFTPCT               0.012864509  0.031763685  0.0325547961  0.0139968801
## TeamOffRebounds        -0.068883375 -0.006371032  0.0003753868 -0.0056578317
## TeamTotalRebounds      -0.530432484 -0.053371243  0.2320491861 -0.2735723395
## TeamAssists            -0.139714049  0.035478568  0.1116023406 -0.0502063543
## TeamSteals             -0.068951590 -0.062074442 -0.0391846669 -0.0471147320
## TeamBlocks             -0.353110391 -0.103465578 -0.0422348142 -0.0994401990
## TeamTurnovers          -0.023017156 -0.018088322  0.0416694763 -0.0631871498
## TeamPersonalFouls       0.078411084 -0.126817358 -0.1676473908 -0.0159095518
## OpponentFG              0.754791141  0.399027442  0.1440747785  0.3955400546
## OpponentFGA            -0.161220379  0.193563166  0.4187304220 -0.1180208656
## OpponentFGPCT           1.000000000  0.312295571 -0.1493674362  0.5522792378
## Opponent3P              0.312295571  1.000000000  0.6914518201  0.7094041257
## Opponent3PA            -0.149367436  0.691451820  1.0000000000  0.0303822862
## Opponent3PPCT           0.552279238  0.709404126  0.0303822862  1.0000000000
## OpponentFT              0.106226566 -0.106344743 -0.1743400433  0.0169282910
## OpponentFTA             0.086625216 -0.140194309 -0.1972872368 -0.0080249496
## OpponentFTPCT           0.076650746  0.053774302  0.0101886734  0.0623587723
## OpponentOffRebounds    -0.251623986 -0.085432899  0.0978389488 -0.2013096986
## OpponentTotalRebounds  -0.005789348  0.005903551  0.0810576009 -0.0680836101
## OpponentAssists         0.553535793  0.513869716  0.2641728450  0.4428640799
## OpponentSteals          0.036468797 -0.011661373  0.0214481397 -0.0383569868
## OpponentBlocks          0.111935521 -0.004746412 -0.0495426307  0.0354134646
## OpponentTurnovers      -0.048082678 -0.095218199 -0.0944428800 -0.0421344973
## OpponentPersonalFouls  -0.081776664 -0.011247805  0.0396475169 -0.0466461289
##                         OpponentFT  OpponentFTA OpponentFTPCT
## Differential          -0.269300868 -0.226064714  -0.175223632
## TeamFG                 0.019511923  0.012937366   0.007923359
## TeamFGA                0.157025930  0.159529646   0.023732217
## TeamFGPCT             -0.091558712 -0.101685664  -0.006190565
## Team3P                 0.009796521 -0.002503282   0.022780414
## Team3PA                0.063163000  0.054748838   0.025878762
## Team3PPCT             -0.039087364 -0.048073272   0.008651286
## TeamFT                 0.161311559  0.183801456  -0.015688533
## TeamFTA                0.180010001  0.213209437  -0.032862991
## TeamFTPCT             -0.009352580 -0.025707797   0.028078614
## TeamOffRebounds        0.043439990  0.058466904  -0.031903278
## TeamTotalRebounds     -0.095266106 -0.022971823  -0.194279344
## TeamAssists           -0.083571639 -0.084160571  -0.027826354
## TeamSteals            -0.034152581 -0.022178476  -0.041125993
## TeamBlocks            -0.070920662 -0.056095076  -0.052504157
## TeamTurnovers          0.123594852  0.154110278  -0.034267574
## TeamPersonalFouls      0.793147614  0.865844664   0.026877590
## OpponentFG            -0.013421944 -0.027151720   0.037049836
## OpponentFGA           -0.156152803 -0.151706668  -0.043324702
## OpponentFGPCT          0.106226566  0.086625216   0.076650746
## Opponent3P            -0.106344743 -0.140194309   0.053774302
## Opponent3PA           -0.174340043 -0.197287237   0.010188673
## Opponent3PPCT          0.016928291 -0.008024950   0.062358772
## OpponentFT             1.000000000  0.928286066   0.393203255
## OpponentFTA            0.928286066  1.000000000   0.063446167
## OpponentFTPCT          0.393203255  0.063446167   1.000000000
## OpponentOffRebounds    0.086671729  0.136423744  -0.082982260
## OpponentTotalRebounds  0.197591588  0.232447345  -0.021281750
## OpponentAssists       -0.012378006 -0.031205800   0.041793598
## OpponentSteals         0.077614062  0.097206119  -0.022196700
## OpponentBlocks         0.101422181  0.110063752   0.008946765
## OpponentTurnovers      0.015778567  0.038679394  -0.052040732
## OpponentPersonalFouls  0.215609923  0.251289640  -0.029978048
##                       OpponentOffRebounds OpponentTotalRebounds OpponentAssists
## Differential                 -0.089347536          -0.420010794   -0.4916760300
## TeamFG                       -0.036316958          -0.225202127    0.0045585394
## TeamFGA                       0.002848058           0.316139528    0.1493200670
## TeamFGPCT                    -0.042399744          -0.512983306   -0.1062526818
## Team3P                       -0.007870292          -0.062384273    0.0294135821
## Team3PA                      -0.018950808           0.202896760    0.0825450568
## Team3PPCT                     0.008677682          -0.263884541   -0.0320289494
## TeamFT                        0.064938518          -0.064969878   -0.0577300621
## TeamFTA                       0.077003661           0.004736343   -0.0638753907
## TeamFTPCT                    -0.016936223          -0.177541483   -0.0074012062
## TeamOffRebounds              -0.014332575          -0.060389134   -0.0386521955
## TeamTotalRebounds            -0.052416263          -0.059965631   -0.2185974327
## TeamAssists                  -0.033384745          -0.222595212    0.0006884142
## TeamSteals                    0.016707012           0.035155522   -0.0533271359
## TeamBlocks                    0.178200671           0.037788375   -0.1511460518
## TeamTurnovers                 0.074131214          -0.106168146    0.0726446766
## TeamPersonalFouls             0.122282037           0.195017438   -0.0226190966
## OpponentFG                    0.120715447           0.275438081    0.6383041307
## OpponentFGA                   0.519792207           0.424276325    0.2318514751
## OpponentFGPCT                -0.251623986          -0.005789348    0.5535357935
## Opponent3P                   -0.085432899           0.005903551    0.5138697156
## Opponent3PA                   0.097838949           0.081057601    0.2641728450
## Opponent3PPCT                -0.201309699          -0.068083610    0.4428640799
## OpponentFT                    0.086671729           0.197591588   -0.0123780062
## OpponentFTA                   0.136423744           0.232447345   -0.0312058003
## OpponentFTPCT                -0.082982260          -0.021281750    0.0417935976
## OpponentOffRebounds           1.000000000           0.622115242    0.0095497736
## OpponentTotalRebounds         0.622115242           1.000000000    0.1792668711
## OpponentAssists               0.009549774           0.179266871    1.0000000000
## OpponentSteals                0.081573888          -0.038673692    0.1068223463
## OpponentBlocks                0.096186044           0.258597044    0.1337215898
## OpponentTurnovers             0.017562976           0.073936193   -0.1060361856
## OpponentPersonalFouls         0.071468553           0.020500608   -0.0849725350
##                       OpponentSteals OpponentBlocks OpponentTurnovers
## Differential            -0.187754380  -0.2622526274      0.2743269542
## TeamFG                  -0.102436608  -0.1604696630      0.1552932747
## TeamFGA                 -0.131734964   0.2184838647      0.1981279705
## TeamFGPCT               -0.021724636  -0.3562550337      0.0242548332
## Team3P                  -0.053878305  -0.1117820624      0.0092841059
## Team3PA                 -0.052980367  -0.0580421730      0.0638351465
## Team3PPCT               -0.025131672  -0.0965607977     -0.0488449748
## TeamFT                  -0.001883349  -0.0650555225      0.1369220844
## TeamFTA                  0.006758108  -0.0539735876      0.1697047361
## TeamFTPCT               -0.022033431  -0.0411754626     -0.0354639208
## TeamOffRebounds          0.032697776   0.1571812909      0.1154717115
## TeamTotalRebounds        0.066119486   0.0139248895     -0.0343556886
## TeamAssists             -0.028866867  -0.1657235463      0.1314533533
## TeamSteals               0.055697260  -0.0022307839      0.7308851693
## TeamBlocks               0.028453380  -0.0389785933      0.0313757033
## TeamTurnovers            0.709987911   0.0064638717      0.1885370196
## TeamPersonalFouls        0.064446997   0.0872112484      0.1016935547
## OpponentFG               0.140823916   0.1290769921     -0.1835580089
## OpponentFGA              0.165329579   0.0455658832     -0.2156337333
## OpponentFGPCT            0.036468797   0.1119355214     -0.0480826780
## Opponent3P              -0.011661373  -0.0047464115     -0.0952181989
## Opponent3PA              0.021448140  -0.0495426307     -0.0944428800
## Opponent3PPCT           -0.038356987   0.0354134646     -0.0421344973
## OpponentFT               0.077614062   0.1014221807      0.0157785673
## OpponentFTA              0.097206119   0.1100637520      0.0386793945
## OpponentFTPCT           -0.022196700   0.0089467648     -0.0520407316
## OpponentOffRebounds      0.081573888   0.0961860439      0.0175629757
## OpponentTotalRebounds   -0.038673692   0.2585970440      0.0739361927
## OpponentAssists          0.106822346   0.1337215898     -0.1060361856
## OpponentSteals           1.000000000   0.0443672204      0.0740678539
## OpponentBlocks           0.044367220   1.0000000000      0.0001223389
## OpponentTurnovers        0.074067854   0.0001223389      1.0000000000
## OpponentPersonalFouls    0.030766974  -0.0514541037      0.2252310703
##                       OpponentPersonalFouls
## Differential                     0.16902573
## TeamFG                          -0.02311662
## TeamFGA                         -0.10718930
## TeamFGPCT                        0.06015066
## Team3P                          -0.12719701
## Team3PA                         -0.15536393
## Team3PPCT                       -0.02688769
## TeamFT                           0.79353920
## TeamFTA                          0.86639509
## TeamFTPCT                        0.01875708
## TeamOffRebounds                  0.12406311
## TeamTotalRebounds                0.18914401
## TeamAssists                     -0.02678208
## TeamSteals                       0.07144201
## TeamBlocks                       0.08058276
## TeamTurnovers                    0.13153904
## TeamPersonalFouls                0.32225852
## OpponentFG                       0.01533421
## OpponentFGA                      0.13678905
## OpponentFGPCT                   -0.08177666
## Opponent3P                      -0.01124781
## Opponent3PA                      0.03964752
## Opponent3PPCT                   -0.04664613
## OpponentFT                       0.21560992
## OpponentFTA                      0.25128964
## OpponentFTPCT                   -0.02997805
## OpponentOffRebounds              0.07146855
## OpponentTotalRebounds            0.02050061
## OpponentAssists                 -0.08497254
## OpponentSteals                   0.03076697
## OpponentBlocks                  -0.05145410
## OpponentTurnovers                0.22523107
## OpponentPersonalFouls            1.00000000
\end{verbatim}

Notice right away -- TeamFG is highly correlated. But it's also highly correlated with TeamFGPCT. And that makes sense. A team that doesn't shoot many shots is not going to have a high score differential. But the number of shots taken and the field goal percentage are also highly related. So including both of these measures would be pointless -- they would add error without adding much in the way of predictive power.

\begin{quote}
\textbf{Your turn}: What else do you see? What other values have predictive power and aren't co-correlated?
\end{quote}

We can add more just by simply adding them.

\begin{Shaded}
\begin{Highlighting}[]
\NormalTok{model2 <-}\StringTok{ }\KeywordTok{lm}\NormalTok{(Differential }\OperatorTok{~}\StringTok{ }\NormalTok{TeamFGPCT }\OperatorTok{+}\StringTok{ }\NormalTok{OpponentFGPCT }\OperatorTok{+}\StringTok{ }\NormalTok{TeamTotalRebounds }\OperatorTok{+}\StringTok{ }\NormalTok{OpponentTotalRebounds, }\DataTypeTok{data=}\NormalTok{logs)}
\KeywordTok{summary}\NormalTok{(model2)}
\end{Highlighting}
\end{Shaded}

\begin{verbatim}
## 
## Call:
## lm(formula = Differential ~ TeamFGPCT + OpponentFGPCT + TeamTotalRebounds + 
##     OpponentTotalRebounds, data = logs)
## 
## Residuals:
##     Min      1Q  Median      3Q     Max 
## -44.813  -5.586  -0.109   5.453  60.831 
## 
## Coefficients:
##                         Estimate Std. Error  t value Pr(>|t|)    
## (Intercept)            -3.655461   0.606119   -6.031 1.64e-09 ***
## TeamFGPCT             100.880013   0.560363  180.026  < 2e-16 ***
## OpponentFGPCT         -97.563291   0.565004 -172.677  < 2e-16 ***
## TeamTotalRebounds       0.516176   0.006239   82.729  < 2e-16 ***
## OpponentTotalRebounds  -0.436402   0.006448  -67.679  < 2e-16 ***
## ---
## Signif. codes:  0 '***' 0.001 '**' 0.01 '*' 0.05 '.' 0.1 ' ' 1
## 
## Residual standard error: 8.501 on 57511 degrees of freedom
##   (4 observations deleted due to missingness)
## Multiple R-squared:  0.7291, Adjusted R-squared:  0.7291 
## F-statistic: 3.87e+04 on 4 and 57511 DF,  p-value: < 2.2e-16
\end{verbatim}

Go down the list:

What is the Adjusted R-squared now?
What is the p-value and is it less than .05?
What is the Residual standard error?

The final thing we can do with this is predict things. Look at our coefficients table. See the Estimates? We can build a formula from that, same as we did with linear regressions.

\begin{verbatim}
Differential = (TeamFGPCT*100.880013) + (OpponentFGPCT*-97.563291) + (TeamTotalRebounds*0.516176) + (OpponentTotalRebounds*-0.436402) - 3.655461
\end{verbatim}

How does this apply in the real world? Let's pretend for a minute that you are Fred Hoiberg, and you have just been hired as Nebraska's Mens Basketball Coach. Your job is to win conference titles and go deep into the NCAA tournament. To do that, we need to know what attributes of a team should we emphasize. We can do that by looking at what previous Big Ten conference champions looked like.

So if our goal is to predict a conference champion team, we need to know what those teams did. Here's the regular season conference champions in this dataset.

\begin{Shaded}
\begin{Highlighting}[]
\NormalTok{logs }\OperatorTok\StringTok{ }\KeywordTok{filter}\NormalTok{(Team }\OperatorTok{==}\StringTok{ "Michigan State Spartans"} \OperatorTok{&}\StringTok{ }\NormalTok{season }\OperatorTok{==}\StringTok{ "2018-2019"} \OperatorTok{|}\StringTok{ }\NormalTok{Team }\OperatorTok{==}\StringTok{ "Michigan State Spartans"} \OperatorTok{&}\StringTok{ }\NormalTok{season }\OperatorTok{==}\StringTok{ "2017-2018"} \OperatorTok{|}\StringTok{ }\NormalTok{Team }\OperatorTok{==}\StringTok{ "Purdue Boilermakers"} \OperatorTok{&}\StringTok{ }\NormalTok{season }\OperatorTok{==}\StringTok{ "2016-2017"} \OperatorTok{|}\StringTok{ }\NormalTok{Team }\OperatorTok{==}\StringTok{ "Indiana Hoosiers"} \OperatorTok{&}\StringTok{ }\NormalTok{season }\OperatorTok{==}\StringTok{ "2015-2016"} \OperatorTok{|}\StringTok{ }\NormalTok{Team }\OperatorTok{==}\StringTok{ "Wisconsin Badgers"} \OperatorTok{&}\StringTok{ }\NormalTok{season }\OperatorTok{==}\StringTok{ "2014-2015"}\NormalTok{) }\OperatorTok\StringTok{ }\KeywordTok{summarise}\NormalTok{(}\DataTypeTok{avgfgpct =} \KeywordTok{mean}\NormalTok{(TeamFGPCT), }\DataTypeTok{avgoppfgpct=}\KeywordTok{mean}\NormalTok{(OpponentFGPCT), }\DataTypeTok{avgtotrebound =} \KeywordTok{mean}\NormalTok{(TeamTotalRebounds), }\DataTypeTok{avgopptotrebound=}\KeywordTok{mean}\NormalTok{(OpponentTotalRebounds))}
\end{Highlighting}
\end{Shaded}

\begin{verbatim}
## # A tibble: 1 x 4
##   avgfgpct avgoppfgpct avgtotrebound avgopptotrebound
##      <dbl>       <dbl>         <dbl>            <dbl>
## 1    0.489       0.409          35.3             27.2
\end{verbatim}

Now it's just plug and chug.

\begin{Shaded}
\begin{Highlighting}[]
\NormalTok{(}\FloatTok{0.4886133}\OperatorTok{*}\FloatTok{100.880013}\NormalTok{) }\OperatorTok{+}\StringTok{ }\NormalTok{(}\FloatTok{0.4090221}\OperatorTok{*-}\FloatTok{97.563291}\NormalTok{) }\OperatorTok{+}\StringTok{ }\NormalTok{(}\FloatTok{35.29834}\OperatorTok{*}\FloatTok{0.516176}\NormalTok{) }\OperatorTok{+}\StringTok{ }\NormalTok{(}\FloatTok{27.20994}\OperatorTok{*-}\FloatTok{0.436402}\NormalTok{) }\OperatorTok{-}\StringTok{ }\FloatTok{3.655461}
\end{Highlighting}
\end{Shaded}

\begin{verbatim}
## [1] 12.076
\end{verbatim}

So a team with those numbers is going to average scoring 12 more points per game than their opponent.

How does that compare to Nebraska of this past season? The last of the Tim Miles era?

\begin{Shaded}
\begin{Highlighting}[]
\NormalTok{logs }\OperatorTok\StringTok{ }
\StringTok{  }\KeywordTok{filter}\NormalTok{(}
\NormalTok{    Team }\OperatorTok{==}\StringTok{ "Nebraska Cornhuskers"} \OperatorTok{&}\StringTok{ }\NormalTok{season }\OperatorTok{==}\StringTok{ "2018-2019"}
\NormalTok{    ) }\OperatorTok\StringTok{ }
\StringTok{  }\KeywordTok{summarise}\NormalTok{(}
    \DataTypeTok{avgfgpct =} \KeywordTok{mean}\NormalTok{(TeamFGPCT), }
    \DataTypeTok{avgoppfgpct =} \KeywordTok{mean}\NormalTok{(OpponentFGPCT), }
    \DataTypeTok{avgtotrebound =} \KeywordTok{mean}\NormalTok{(TeamTotalRebounds),}
    \DataTypeTok{avgopptotrebound =} \KeywordTok{mean}\NormalTok{(OpponentTotalRebounds)}
\NormalTok{    )}
\end{Highlighting}
\end{Shaded}

\begin{verbatim}
## # A tibble: 1 x 4
##   avgfgpct avgoppfgpct avgtotrebound avgopptotrebound
##      <dbl>       <dbl>         <dbl>            <dbl>
## 1    0.431       0.423          32.5             34.9
\end{verbatim}

\begin{Shaded}
\begin{Highlighting}[]
\NormalTok{(}\FloatTok{0.4305833}\OperatorTok{*}\FloatTok{100.880013}\NormalTok{) }\OperatorTok{+}\StringTok{ }\NormalTok{(}\FloatTok{0.4226667}\OperatorTok{*-}\FloatTok{97.563291}\NormalTok{) }\OperatorTok{+}\StringTok{ }\NormalTok{(}\FloatTok{32.5}\OperatorTok{*}\FloatTok{0.516176}\NormalTok{) }\OperatorTok{+}\StringTok{ }\NormalTok{(}\FloatTok{34.94444}\OperatorTok{*-}\FloatTok{0.436402}\NormalTok{) }\OperatorTok{-}\StringTok{ }\FloatTok{3.655461}
\end{Highlighting}
\end{Shaded}

\begin{verbatim}
## [1] 0.07093015
\end{verbatim}

By this model, it predicted we would outscore our opponent by .07 points over the season. So we'd win slightly more than we'd lose. Nebraska's overall record? 19-17.

\hypertarget{residuals}{%
\chapter{Residuals}\label{residuals}}

When looking at a linear model of your data, there's a measure you need to be aware of called residuals. The residual is the distance between what the model predicted and what the real outcome is. So if your model predicted a team would score 38 points per game given their third down conversion percentage, and they score 45, then your residual is 7. If they had scored 31, then their residual would be -7.

Residuals can tell you severals things, but most importantly is if a linear model the right model for your data. If the residuals appear to be random, then a linear model is appropriate. If they have a pattern, it means something else is going on in your data and a linear model isn't appropriate.

Residuals can also tell you who is underperforming and overperforming the model. Let's take a look at an example we've used regularly this semester -- third down conversion percentage and penalties.

Let's first attach libraries and \href{https://unl.box.com/s/zlxoptqixkt98gubk3i6316qun99l49r}{get some data}. Note: In the rvest steps, I rename the first column because it's blank on the page and then I merge scoring offense to two different tables -- third downs and penalties.

\begin{Shaded}
\begin{Highlighting}[]
\KeywordTok{library}\NormalTok{(tidyverse)}
\end{Highlighting}
\end{Shaded}

\begin{Shaded}
\begin{Highlighting}[]
\NormalTok{offense <-}\StringTok{ }\KeywordTok{read_csv}\NormalTok{(}\StringTok{"data/correlations.csv"}\NormalTok{)}
\end{Highlighting}
\end{Shaded}

\begin{verbatim}
## Parsed with column specification:
## cols(
##   Name = col_character(),
##   OffPoints = col_double(),
##   OffPointsG = col_double(),
##   DefPoints = col_double(),
##   DefPointsG = col_double(),
##   Pen. = col_double(),
##   Yards = col_double(),
##   `Pen./G` = col_double(),
##   OffConversions = col_double(),
##   OffConversionPct = col_double(),
##   DefConversions = col_double(),
##   DefConversionPct = col_double()
## )
\end{verbatim}

First, let's build a linear model and save it as a new dataframe called \texttt{fit}.

\begin{Shaded}
\begin{Highlighting}[]
\NormalTok{fit <-}\StringTok{ }\KeywordTok{lm}\NormalTok{(}\StringTok{`}\DataTypeTok{OffPointsG}\StringTok{`} \OperatorTok{~}\StringTok{ `}\DataTypeTok{OffConversionPct}\StringTok{`}\NormalTok{, }\DataTypeTok{data =}\NormalTok{ offense)}
\KeywordTok{summary}\NormalTok{(fit)}
\end{Highlighting}
\end{Shaded}

\begin{verbatim}
## 
## Call:
## lm(formula = OffPointsG ~ OffConversionPct, data = offense)
## 
## Residuals:
##      Min       1Q   Median       3Q      Max 
## -11.3861  -3.5411  -0.5885   2.9011  13.5188 
## 
## Coefficients:
##                  Estimate Std. Error t value Pr(>|t|)    
## (Intercept)      -4.74024    3.41041   -1.39    0.167    
## OffConversionPct  0.85625    0.08479   10.10   <2e-16 ***
## ---
## Signif. codes:  0 '***' 0.001 '**' 0.01 '*' 0.05 '.' 0.1 ' ' 1
## 
## Residual standard error: 5.111 on 128 degrees of freedom
## Multiple R-squared:  0.4434, Adjusted R-squared:  0.4391 
## F-statistic:   102 on 1 and 128 DF,  p-value: < 2.2e-16
\end{verbatim}

We've seen this output before, but let's review because if you are using scatterplots to make a point, you should do this. First, note the Min and Max residual at the top. A team has underperformed the model by 11.4 points, and a team has overperformed it by 13.5. The median residual, where half are above and half are below, is just slightly under the fit line. Close here is good.

Next: Look at the Adjusted R-squared value. What that says is that 44 percent of a team's scoring output can be predicted by their third down conversion percentage. This is just one year, so that's a little low. If we did this with more years, that would go up.

Last: Look at the p-value. We are looking for a p-value smaller than .05. At .05, we can say that our correlation didn't happen at random. And, in this case, it REALLY didn't happen at random.

What we want to do now is look at those residuals. We can add them to our dataframe like this:

\begin{Shaded}
\begin{Highlighting}[]
\NormalTok{offense}\OperatorTok{$}\NormalTok{predicted <-}\StringTok{ }\KeywordTok{predict}\NormalTok{(fit)}
\NormalTok{offense}\OperatorTok{$}\NormalTok{residuals <-}\StringTok{ }\KeywordTok{residuals}\NormalTok{(fit)}
\end{Highlighting}
\end{Shaded}

Now we can sort our data by those residuals. Sorting in descending order gives us the teams that are overperforming the model.

\begin{Shaded}
\begin{Highlighting}[]
\NormalTok{offense }\OperatorTok\StringTok{ }\KeywordTok{arrange}\NormalTok{(}\KeywordTok{desc}\NormalTok{(residuals))}
\end{Highlighting}
\end{Shaded}

\begin{verbatim}
## # A tibble: 130 x 14
##    Name  OffPoints OffPointsG DefPoints DefPointsG  Pen. Yards `Pen./G`
##    <chr>     <dbl>      <dbl>     <dbl>      <dbl> <dbl> <dbl>    <dbl>
##  1 Tole~       525       40.4       397       30.5    99   931      7.6
##  2 Utah~       618       47.5       289       22.2   101   916      7.8
##  3 Syra~       523       40.2       351       27      94   768      7.2
##  4 Okla~       677       48.4       466       33.3    86   855      6.1
##  5 Clem~       664       44.3       197       13.1    73   674      4.9
##  6 Hous~       571       43.9       483       37.2    87   683      6.7
##  7 Miss~       407       33.9       434       36.2    88   802      7.3
##  8 Neva~       404       31.1       350       26.9    77   738      5.9
##  9 Bost~       384       32         308       25.7    75   590      6.3
## 10 West~       483       40.3       326       27.2    85   800      7.1
## # ... with 120 more rows, and 6 more variables: OffConversions <dbl>,
## #   OffConversionPct <dbl>, DefConversions <dbl>, DefConversionPct <dbl>,
## #   predicted <dbl>, residuals <dbl>
\end{verbatim}

So looking at this table, what you see here are the teams who scored more than their third down conversion percentage would indicate. Some of those teams were just lucky. Some of those teams were really good at long touchdown plays that didn't need a lot of third downs to get down the field. But these are your overperformers.

But, before we can bestow any validity on it, we need to see if this linear model is appropriate. We've done that some looking at our p-values and R-squared values. But one more check is to look at the residuals themselves. We do that by plotting the residuals with the predictor. We'll get into plotting soon, but for now just seeing it is enough.

\includegraphics{SportsData_files/figure-latex/unnamed-chunk-84-1.pdf}

The lack of a shape here -- the seemingly random nature -- is a good sign that a linear model works for our data. If there was a pattern, that would indicate something else was going on in our data and we needed a different model.

Another way to view your residuals is by connecting the predicted value with the actual value.

\includegraphics{SportsData_files/figure-latex/unnamed-chunk-85-1.pdf}

The blue line here separates underperformers from overperformers.

\hypertarget{penalties}{%
\section{Penalties}\label{penalties}}

Now let's look at it where it doesn't work: Penalties.

\begin{Shaded}
\begin{Highlighting}[]
\NormalTok{penalties <-}\StringTok{ }\NormalTok{offense}
\end{Highlighting}
\end{Shaded}

\begin{Shaded}
\begin{Highlighting}[]
\NormalTok{pfit <-}\StringTok{ }\KeywordTok{lm}\NormalTok{(OffPointsG }\OperatorTok{~}\StringTok{ }\NormalTok{Yards, }\DataTypeTok{data =}\NormalTok{ penalties)}
\KeywordTok{summary}\NormalTok{(pfit)}
\end{Highlighting}
\end{Shaded}

\begin{verbatim}
## 
## Call:
## lm(formula = OffPointsG ~ Yards, data = penalties)
## 
## Residuals:
##      Min       1Q   Median       3Q      Max 
## -16.5381  -4.5779  -0.5204   4.2418  17.0543 
## 
## Coefficients:
##              Estimate Std. Error t value Pr(>|t|)    
## (Intercept) 20.897213   2.819227   7.412 1.49e-11 ***
## Yards        0.012220   0.003966   3.082  0.00252 ** 
## ---
## Signif. codes:  0 '***' 0.001 '**' 0.01 '*' 0.05 '.' 0.1 ' ' 1
## 
## Residual standard error: 6.61 on 128 degrees of freedom
## Multiple R-squared:  0.06907,    Adjusted R-squared:  0.06179 
## F-statistic: 9.496 on 1 and 128 DF,  p-value: 0.002521
\end{verbatim}

So from top to bottom:

\begin{itemize}
\tightlist
\item
  Our min and max go from -16.5 to positive 17.1
\item
  Our adjusted R-squared is \ldots{} .06. Not much at all.
\item
  Our p-value is \ldots{} .002, which is less than than .05.
\end{itemize}

So what we can say about this model is that it's statistically significant but utterly meaningless. Normally, we'd stop right here -- why bother going forward with a predictive model that isn't predictive? But let's do it anyway.

\begin{Shaded}
\begin{Highlighting}[]
\NormalTok{penalties}\OperatorTok{$}\NormalTok{predicted <-}\StringTok{ }\KeywordTok{predict}\NormalTok{(pfit)}
\NormalTok{penalties}\OperatorTok{$}\NormalTok{residuals <-}\StringTok{ }\KeywordTok{residuals}\NormalTok{(pfit)}
\end{Highlighting}
\end{Shaded}

\begin{Shaded}
\begin{Highlighting}[]
\NormalTok{penalties }\OperatorTok\StringTok{ }\KeywordTok{arrange}\NormalTok{(}\KeywordTok{desc}\NormalTok{(residuals))}
\end{Highlighting}
\end{Shaded}

\begin{verbatim}
## # A tibble: 130 x 14
##    Name  OffPoints OffPointsG DefPoints DefPointsG  Pen. Yards `Pen./G`
##    <chr>     <dbl>      <dbl>     <dbl>      <dbl> <dbl> <dbl>    <dbl>
##  1 Okla~       677       48.4       466       33.3    86   855      6.1
##  2 Utah~       618       47.5       289       22.2   101   916      7.8
##  3 Clem~       664       44.3       197       13.1    73   674      4.9
##  4 Alab~       684       45.6       271       18.1    87   796      5.8
##  5 Hous~       571       43.9       483       37.2    87   683      6.7
##  6 UCF         562       43.2       295       22.7    97   848      7.5
##  7 Memp~       601       42.9       447       31.9   101   833      7.2
##  8 Ohio        521       40.1       320       24.6    64   652      4.9
##  9 Syra~       523       40.2       351       27      94   768      7.2
## 10 West~       483       40.3       326       27.2    85   800      7.1
## # ... with 120 more rows, and 6 more variables: OffConversions <dbl>,
## #   OffConversionPct <dbl>, DefConversions <dbl>, DefConversionPct <dbl>,
## #   predicted <dbl>, residuals <dbl>
\end{verbatim}

So our model says Oklahoma \emph{should} only be scoring 31.3 points per game given how many penalty yards per game, but they're really scoring 48.4. Oy. What happens if we plot those residuals?

\includegraphics{SportsData_files/figure-latex/unnamed-chunk-90-1.pdf}

Well \ldots{} it actually says that a linear model is appropriate. Which an important lesson -- just because your residual plot says a linear model works here, that doesn't say your linear model is good. There are other measures for that, and you need to use them.

Here's the segment plot of residuals -- you'll see some really long lines. That's a bad sign.

\includegraphics{SportsData_files/figure-latex/unnamed-chunk-91-1.pdf}

\hypertarget{z-scores}{%
\chapter{Z scores}\label{z-scores}}

Z scores are a handy way to standardize scores so you can compare things across groupings. In our case, we may want to compare teams by year, or era. We can use z scores to answer questions like who was the greatest X of all time, because a Z score can put them in context to their era.

We can also use z scores to ask how much better is team A from team B.

So let's use Nebraska basketball, which if you haven't been reading lately is at a bit of a crossroads.

A Z score is a measure of how far a number is from the population mean of that number. An easier way to say that -- how different is my grade from the average grade in the class. The formula for calculating a Z score is \texttt{(MyScore\ -\ AverageScore)/Standard\ Deviation\ of\ Scores}. The standard deviation is a number calculated to show the amount of variation in a set of data. In a normal distribution, 68 percent of all scores will be within 1 standard deviation, 95 percent will be within 2 and 99 within 3.

\hypertarget{calculating-a-z-score-in-r}{%
\section{Calculating a Z score in R}\label{calculating-a-z-score-in-r}}

\begin{Shaded}
\begin{Highlighting}[]
\KeywordTok{library}\NormalTok{(tidyverse)}
\end{Highlighting}
\end{Shaded}

Let's look at the current state of Nebraska basketball using the \href{https://unl.box.com/s/wnlh0u9low1yh56enion8zjmu8r7dc8p}{same logs data we've been using, but for this season so far}.

\begin{Shaded}
\begin{Highlighting}[]
\NormalTok{gamelogs <-}\StringTok{ }\KeywordTok{read_csv}\NormalTok{(}\StringTok{"data/logs20.csv"}\NormalTok{)}
\end{Highlighting}
\end{Shaded}

\begin{verbatim}
## Parsed with column specification:
## cols(
##   .default = col_double(),
##   Date = col_date(format = ""),
##   HomeAway = col_character(),
##   Opponent = col_character(),
##   W_L = col_character(),
##   Blank = col_logical(),
##   Team = col_character(),
##   Conference = col_character(),
##   season = col_character()
## )
\end{verbatim}

\begin{verbatim}
## See spec(...) for full column specifications.
\end{verbatim}

The first thing we need to do is select some fields we think represent team quality:

\begin{Shaded}
\begin{Highlighting}[]
\NormalTok{teamquality <-}\StringTok{ }\NormalTok{gamelogs }\OperatorTok\StringTok{ }
\StringTok{  }\KeywordTok{select}\NormalTok{(Conference, Team, TeamFGPCT, TeamTotalRebounds, OpponentFGPCT, OpponentTotalRebounds)}
\end{Highlighting}
\end{Shaded}

And since we have individual game data, we need to collapse this into one record for each team. We do that with \ldots{} group by.

\begin{Shaded}
\begin{Highlighting}[]
\NormalTok{teamtotals <-}\StringTok{ }\NormalTok{teamquality }\OperatorTok\StringTok{ }
\StringTok{  }\KeywordTok{group_by}\NormalTok{(Conference, Team) }\OperatorTok\StringTok{ }
\StringTok{  }\KeywordTok{summarise}\NormalTok{(}
    \DataTypeTok{FGAvg =} \KeywordTok{mean}\NormalTok{(TeamFGPCT), }
    \DataTypeTok{ReboundAvg =} \KeywordTok{mean}\NormalTok{(TeamTotalRebounds), }
    \DataTypeTok{OppFGAvg =} \KeywordTok{mean}\NormalTok{(OpponentFGPCT),}
    \DataTypeTok{OffRebAvg =} \KeywordTok{mean}\NormalTok{(OpponentTotalRebounds)}
\NormalTok{    )}
\end{Highlighting}
\end{Shaded}

To calculate a Z score in R, the easiest way is to use the scale function in base R. To use it, you use \texttt{scale(FieldName,\ center=TRUE,\ scale=TRUE)}. The center and scale indicate if you want to subtract from the mean and if you want to divide by the standard deviation, respectively. We do.

When we have multiple Z Scores, it's pretty standard practice to add them together into a composite score. That's what we're doing at the end here with \texttt{TotalZscore}. Note: We have to invert OppZscore and OppRebZScore by multiplying it by a negative 1 because the lower someone's opponent number is, the better.

\begin{Shaded}
\begin{Highlighting}[]
\NormalTok{teamzscore <-}\StringTok{ }\NormalTok{teamtotals }\OperatorTok\StringTok{ }
\StringTok{  }\KeywordTok{mutate}\NormalTok{(}
    \DataTypeTok{FGzscore =} \KeywordTok{as.numeric}\NormalTok{(}\KeywordTok{scale}\NormalTok{(FGAvg, }\DataTypeTok{center =} \OtherTok{TRUE}\NormalTok{, }\DataTypeTok{scale =} \OtherTok{TRUE}\NormalTok{)),}
    \DataTypeTok{RebZscore =} \KeywordTok{as.numeric}\NormalTok{(}\KeywordTok{scale}\NormalTok{(ReboundAvg, }\DataTypeTok{center =} \OtherTok{TRUE}\NormalTok{, }\DataTypeTok{scale =} \OtherTok{TRUE}\NormalTok{)),}
    \DataTypeTok{OppZscore =} \KeywordTok{as.numeric}\NormalTok{(}\KeywordTok{scale}\NormalTok{(OppFGAvg, }\DataTypeTok{center =} \OtherTok{TRUE}\NormalTok{, }\DataTypeTok{scale =} \OtherTok{TRUE}\NormalTok{)) }\OperatorTok{*}\StringTok{ }\DecValTok{-1}\NormalTok{,}
    \DataTypeTok{OppRebZScore =} \KeywordTok{as.numeric}\NormalTok{(}\KeywordTok{scale}\NormalTok{(OffRebAvg, }\DataTypeTok{center =} \OtherTok{TRUE}\NormalTok{, }\DataTypeTok{scale =} \OtherTok{TRUE}\NormalTok{)) }\OperatorTok{*}\StringTok{ }\DecValTok{-1}\NormalTok{,}
    \DataTypeTok{TotalZscore =}\NormalTok{ FGzscore }\OperatorTok{+}\StringTok{ }\NormalTok{RebZscore }\OperatorTok{+}\StringTok{ }\NormalTok{OppZscore }\OperatorTok{+}\StringTok{ }\NormalTok{OppRebZScore}
\NormalTok{  )  }
\end{Highlighting}
\end{Shaded}

So now we have a dataframe called \texttt{teamzscore} that has 353 basketball teams with Z scores. What does it look like?

\begin{Shaded}
\begin{Highlighting}[]
\KeywordTok{head}\NormalTok{(teamzscore)}
\end{Highlighting}
\end{Shaded}

\begin{verbatim}
## # A tibble: 6 x 11
## # Groups:   Conference [1]
##   Conference Team  FGAvg ReboundAvg OppFGAvg OffRebAvg FGzscore RebZscore
##   <chr>      <chr> <dbl>      <dbl>    <dbl>     <dbl>    <dbl>     <dbl>
## 1 A-10       Davi~ 0.454       31.1    0.437      30.4    0.507   -0.623 
## 2 A-10       Dayt~ 0.525       32.5    0.413      29.0    2.56     0.0380
## 3 A-10       Duqu~ 0.444       32.4    0.427      32.4    0.222   -0.0145
## 4 A-10       Ford~ 0.380       30.0    0.403      34.1   -1.61    -1.14  
## 5 A-10       Geor~ 0.422       33.5    0.441      30.7   -0.398    0.480 
## 6 A-10       Geor~ 0.424       30.5    0.451      32.7   -0.341   -0.904 
## # ... with 3 more variables: OppZscore <dbl>, OppRebZScore <dbl>,
## #   TotalZscore <dbl>
\end{verbatim}

A way to read this -- a team at zero is precisely average. The larger the positive number, the more exceptional they are. The larger the negative number, the more truly terrible they are.

So who are the best teams in the country?

\begin{Shaded}
\begin{Highlighting}[]
\NormalTok{teamzscore }\OperatorTok\StringTok{ }\KeywordTok{arrange}\NormalTok{(}\KeywordTok{desc}\NormalTok{(TotalZscore))}
\end{Highlighting}
\end{Shaded}

\begin{verbatim}
## # A tibble: 353 x 11
## # Groups:   Conference [32]
##    Conference Team  FGAvg ReboundAvg OppFGAvg OffRebAvg FGzscore RebZscore
##    <chr>      <chr> <dbl>      <dbl>    <dbl>     <dbl>    <dbl>     <dbl>
##  1 Big West   UC-I~ 0.473       36.6    0.390      27.1    1.60     2.23  
##  2 Big 12     Kans~ 0.482       35.9    0.378      29.0    2.38     1.12  
##  3 WCC        Gonz~ 0.517       37.6    0.422      28.4    1.65     1.94  
##  4 Big Ten    Mich~ 0.460       37.8    0.379      29.7    1.41     1.59  
##  5 Southland  Step~ 0.490       34.2    0.427      26.6    1.71     1.02  
##  6 OVC        Murr~ 0.477       35.3    0.401      29.2    1.31     1.36  
##  7 Summit     Sout~ 0.492       35.5    0.423      31.3    1.51     1.60  
##  8 A-10       Sain~ 0.457       37.4    0.403      30.5    0.598    2.23  
##  9 A-10       Dayt~ 0.525       32.5    0.413      29.0    2.56     0.0380
## 10 Horizon    Wrig~ 0.463       37.1    0.416      33.5    1.53     2.28  
## # ... with 343 more rows, and 3 more variables: OppZscore <dbl>,
## #   OppRebZScore <dbl>, TotalZscore <dbl>
\end{verbatim}

Don't sleep on the Anteaters come tournament time!

But closer to home, how is Nebraska doing.

\begin{Shaded}
\begin{Highlighting}[]
\NormalTok{teamzscore }\OperatorTok\StringTok{ }
\StringTok{  }\KeywordTok{filter}\NormalTok{(Conference }\OperatorTok{==}\StringTok{ "Big Ten"}\NormalTok{) }\OperatorTok\StringTok{ }
\StringTok{  }\KeywordTok{arrange}\NormalTok{(}\KeywordTok{desc}\NormalTok{(TotalZscore))}
\end{Highlighting}
\end{Shaded}

\begin{verbatim}
## # A tibble: 14 x 11
## # Groups:   Conference [1]
##    Conference Team  FGAvg ReboundAvg OppFGAvg OffRebAvg FGzscore RebZscore
##    <chr>      <chr> <dbl>      <dbl>    <dbl>     <dbl>    <dbl>     <dbl>
##  1 Big Ten    Mich~ 0.460       37.8    0.379      29.7    1.41     1.59  
##  2 Big Ten    Rutg~ 0.449       37      0.385      31.1    0.789    1.20  
##  3 Big Ten    Ohio~ 0.446       33.8    0.397      28.2    0.639   -0.307 
##  4 Big Ten    Illi~ 0.442       36.4    0.416      29.1    0.402    0.895 
##  5 Big Ten    Indi~ 0.442       34.7    0.423      29.3    0.377    0.0934
##  6 Big Ten    Mich~ 0.463       33.4    0.423      32.0    1.60    -0.513 
##  7 Big Ten    Mary~ 0.415       36.4    0.399      32.3   -1.20     0.895 
##  8 Big Ten    Penn~ 0.432       35.6    0.411      34.2   -0.195    0.522 
##  9 Big Ten    Iowa~ 0.451       34.3    0.428      32.6    0.892   -0.0698
## 10 Big Ten    Purd~ 0.418       33.8    0.410      29.3   -0.994   -0.304 
## 11 Big Ten    Minn~ 0.422       35.1    0.412      33.4   -0.758    0.312 
## 12 Big Ten    Wisc~ 0.426       31.3    0.410      32.0   -0.552   -1.53  
## 13 Big Ten    Nort~ 0.417       30.6    0.423      34.6   -1.08    -1.84  
## 14 Big Ten    Nebr~ 0.412       32.5    0.446      42     -1.34    -0.939 
## # ... with 3 more variables: OppZscore <dbl>, OppRebZScore <dbl>,
## #   TotalZscore <dbl>
\end{verbatim}

So, as we can see, with our composite Z Score, Nebraska is \ldots{} not good. Not good at all.

\hypertarget{intro-to-ggplot}{%
\chapter{Intro to ggplot}\label{intro-to-ggplot}}

With \texttt{ggplot2}, we dive into the world of programmatic data visualization. The \texttt{ggplot2} library implements something called the grammar of graphics. The main concepts are:

\begin{itemize}
\tightlist
\item
  aesthetics - which in this case means the data which we are going to plot
\item
  geometries - which means the shape the data is going to take
\item
  scales - which means any transformations we might make on the data
\item
  facets - which means how we might graph many elements of the same dataset in the same space
\item
  layers - which means how we might lay multiple geometries over top of each other to reveal new information.
\end{itemize}

Hadley Wickam, who is behind all of the libraries we have used in this course to date, wrote about his layered grammar of graphics in \href{http://byrneslab.net/classes/biol607/readings/wickham_layered-grammar.pdf}{this 2009 paper that is worth your time to read}.

Here are some \texttt{ggplot2} resources you'll want to keep handy:

\begin{itemize}
\tightlist
\item
  \href{http://ggplot2.tidyverse.org/reference/index.html}{The ggplot documentation}.
\item
  \href{http://www.cookbook-r.com/Graphs/}{The ggplot cookbook}
\end{itemize}

Let's dive in using data we've already seen before -- football attendance. This workflow will represent a clear picture of what your work in this class will be like for much of the rest of the semester. One way to think of this workflow is that your R Notebook is now your digital sketchbook, where you will try different types of visualizations to find ones that work. Then, you will export your work into a program like Illustrator to finish the work.

To begin, we'll import the \texttt{ggplot2} and \texttt{dplyr} libraries. We'll read in the data, then create a new dataframe that represents \href{https://unl.box.com/s/etqna5bfvf3b0gsnw0kcjjn1rxx9335s}{our attendance data}, similar to what we've done before.

\begin{Shaded}
\begin{Highlighting}[]
\KeywordTok{library}\NormalTok{(tidyverse)}
\end{Highlighting}
\end{Shaded}

\begin{Shaded}
\begin{Highlighting}[]
\NormalTok{attendance <-}\StringTok{ }\KeywordTok{read_csv}\NormalTok{(}\StringTok{'data/attendance.csv'}\NormalTok{)}
\end{Highlighting}
\end{Shaded}

\begin{verbatim}
## Parsed with column specification:
## cols(
##   Institution = col_character(),
##   Conference = col_character(),
##   `2013` = col_double(),
##   `2014` = col_double(),
##   `2015` = col_double(),
##   `2016` = col_double(),
##   `2017` = col_double(),
##   `2018` = col_double()
## )
\end{verbatim}

First, let's get a top 10 list by announced attendance this last season. We'll use the same tricks we used in the filtering assignment.

\begin{Shaded}
\begin{Highlighting}[]
\NormalTok{attendance }\OperatorTok\StringTok{ }
\StringTok{  }\KeywordTok{arrange}\NormalTok{(}\KeywordTok{desc}\NormalTok{(}\StringTok{`}\DataTypeTok{2018}\StringTok{`}\NormalTok{)) }\OperatorTok\StringTok{ }
\StringTok{  }\KeywordTok{top_n}\NormalTok{(}\DecValTok{10}\NormalTok{) }\OperatorTok\StringTok{ }
\StringTok{  }\KeywordTok{select}\NormalTok{(Institution, }\StringTok{`}\DataTypeTok{2018}\StringTok{`}\NormalTok{)}
\end{Highlighting}
\end{Shaded}

\begin{verbatim}
## Selecting by 2018
\end{verbatim}

\begin{verbatim}
## # A tibble: 10 x 2
##    Institution `2018`
##    <chr>        <dbl>
##  1 Michigan    775156
##  2 Penn St.    738396
##  3 Ohio St.    713630
##  4 Alabama     710931
##  5 LSU         705733
##  6 Texas A&M   698908
##  7 Tennessee   650887
##  8 Georgia     649222
##  9 Nebraska    623240
## 10 Oklahoma    607146
\end{verbatim}

That looks good, so let's save it to a new data frame and use that data frame instead going forward.

\begin{Shaded}
\begin{Highlighting}[]
\NormalTok{top10 <-}\StringTok{ }\NormalTok{attendance }\OperatorTok
\StringTok{  }\KeywordTok{arrange}\NormalTok{(}\KeywordTok{desc}\NormalTok{(}\StringTok{`}\DataTypeTok{2018}\StringTok{`}\NormalTok{)) }\OperatorTok\StringTok{ }
\StringTok{  }\KeywordTok{top_n}\NormalTok{(}\DecValTok{10}\NormalTok{) }\OperatorTok\StringTok{ }
\StringTok{  }\KeywordTok{select}\NormalTok{(Institution, }\StringTok{`}\DataTypeTok{2018}\StringTok{`}\NormalTok{)}
\end{Highlighting}
\end{Shaded}

\begin{verbatim}
## Selecting by 2018
\end{verbatim}

\hypertarget{the-bar-chart}{%
\section{The bar chart}\label{the-bar-chart}}

The easiest thing we can do is create a simple bar chart of our data. \textbf{Bar charts show magnitude. They invite you to compare how much more or less one thing is compared to others.}

We could, for instance, create a bar chart of the total attendance. To do that, we simply tell \texttt{ggplot2} what our dataset is, what element of the data we want to make the bar chart out of (which is the aesthetic), and the geometry type (which is the geom). It looks like this:

\texttt{ggplot(top10,\ aes(x=Institution))\ +\ geom\_bar()}

Note: attendance is our data, \texttt{aes} means aesthetics, \texttt{x=Institution} explicitly tells \texttt{ggplot2} that our x value -- our horizontal value -- is the Institution field from the data, and then we add on the \texttt{geom\_bar()} as the geometry. And what do we get when we run that?

\begin{Shaded}
\begin{Highlighting}[]
\KeywordTok{ggplot}\NormalTok{(top10, }\KeywordTok{aes}\NormalTok{(}\DataTypeTok{x=}\NormalTok{Institution)) }\OperatorTok{+}\StringTok{ }\KeywordTok{geom_bar}\NormalTok{()}
\end{Highlighting}
\end{Shaded}

\includegraphics{SportsData_files/figure-latex/unnamed-chunk-104-1.pdf}

We get \ldots{} weirdness. We expected to see bars of different sizes, but we get all with a count of 1. What gives? Well, this is the default behavior. What we have here is something called a histogram, where \texttt{ggplot2} helpfully counted up the number of times the Institution appears and counted them up. Since we only have one record per Institution, the count is always 1. How do we fix this? By adding \texttt{weight} to our aesthetic.

\begin{Shaded}
\begin{Highlighting}[]
\KeywordTok{ggplot}\NormalTok{(top10, }\KeywordTok{aes}\NormalTok{(}\DataTypeTok{x=}\NormalTok{Institution, }\DataTypeTok{weight=}\StringTok{`}\DataTypeTok{2018}\StringTok{`}\NormalTok{)) }\OperatorTok{+}\StringTok{ }
\StringTok{  }\KeywordTok{geom_bar}\NormalTok{()}
\end{Highlighting}
\end{Shaded}

\includegraphics{SportsData_files/figure-latex/unnamed-chunk-105-1.pdf}

Closer. But \ldots{} what order is that in? And what happened to our count numbers on the left? Why are they in scientific notation?

Let's deal with the ordering first. \texttt{ggplot2}'s default behavior is to sort the data by the x axis variable. So it's in alphabetical order. To change that, we have to \texttt{reorder} it. With \texttt{reorder}, we first have to tell \texttt{ggplot} what we are reordering, and then we have to tell it HOW we are reordering it. So it's reorder(FIELD, SORTFIELD).

\begin{Shaded}
\begin{Highlighting}[]
\KeywordTok{ggplot}\NormalTok{(top10, }\KeywordTok{aes}\NormalTok{(}\DataTypeTok{x=}\KeywordTok{reorder}\NormalTok{(Institution, }\StringTok{`}\DataTypeTok{2018}\StringTok{`}\NormalTok{), }\DataTypeTok{weight=}\StringTok{`}\DataTypeTok{2018}\StringTok{`}\NormalTok{)) }\OperatorTok{+}\StringTok{ }\KeywordTok{geom_bar}\NormalTok{()}
\end{Highlighting}
\end{Shaded}

\includegraphics{SportsData_files/figure-latex/unnamed-chunk-106-1.pdf}

Better. We can argue about if the right order is smallest to largest or largest to smallest. But this gets us close. By the way, to sort it largest to smallest, put a negative sign in front of the sort field.

\begin{Shaded}
\begin{Highlighting}[]
\KeywordTok{ggplot}\NormalTok{(top10, }\KeywordTok{aes}\NormalTok{(}\DataTypeTok{x=}\KeywordTok{reorder}\NormalTok{(Institution, }\OperatorTok{-}\StringTok{`}\DataTypeTok{2018}\StringTok{`}\NormalTok{), }\DataTypeTok{weight=}\StringTok{`}\DataTypeTok{2018}\StringTok{`}\NormalTok{)) }\OperatorTok{+}\StringTok{ }\KeywordTok{geom_bar}\NormalTok{()}
\end{Highlighting}
\end{Shaded}

\includegraphics{SportsData_files/figure-latex/unnamed-chunk-107-1.pdf}

\hypertarget{scales}{%
\section{Scales}\label{scales}}

To fix the axis labels, we need try one of the other main elements of the \texttt{ggplot2} library, which is transform a scale. More often that not, that means doing something like putting it on a logarithmic scale or some other kind of transformation. In this case, we're just changing how it's represented. The default in \texttt{ggplot2} for large values is to express them as scientific notation. Rarely ever is that useful in our line of work. So we have to transform them into human readable numbers.

The easiest way to do this is to use a library called \texttt{scales} and it's already installed.

\begin{Shaded}
\begin{Highlighting}[]
\KeywordTok{library}\NormalTok{(scales)}
\end{Highlighting}
\end{Shaded}

To alter the scale, we add a piece to our plot with \texttt{+} and we tell it which scale is getting altered and what kind of data it is. In our case, our Y axis is what is needing to be altered, and it's continuous data (meaning it can be any number between x and y, vs discrete data which are categorical). So we need to add \texttt{scale\_y\_continuous} and the information we want to pass it is to alter the labels with a function called \texttt{comma}.

\begin{Shaded}
\begin{Highlighting}[]
\KeywordTok{ggplot}\NormalTok{(top10, }\KeywordTok{aes}\NormalTok{(}\DataTypeTok{x=}\KeywordTok{reorder}\NormalTok{(Institution, }\OperatorTok{-}\StringTok{`}\DataTypeTok{2018}\StringTok{`}\NormalTok{), }\DataTypeTok{weight=}\StringTok{`}\DataTypeTok{2018}\StringTok{`}\NormalTok{)) }\OperatorTok{+}\StringTok{ }
\StringTok{  }\KeywordTok{geom_bar}\NormalTok{() }\OperatorTok{+}\StringTok{ }
\StringTok{  }\KeywordTok{scale_y_continuous}\NormalTok{(}\DataTypeTok{labels=}\NormalTok{comma)}
\end{Highlighting}
\end{Shaded}

\includegraphics{SportsData_files/figure-latex/unnamed-chunk-109-1.pdf}

Better.

\hypertarget{styling}{%
\section{Styling}\label{styling}}

We are going to spend a lot more time on styling, but let's add some simple labels to this with a new bit called \texttt{labs} which is short for labels.

\begin{Shaded}
\begin{Highlighting}[]
\KeywordTok{ggplot}\NormalTok{(top10, }\KeywordTok{aes}\NormalTok{(}\DataTypeTok{x=}\KeywordTok{reorder}\NormalTok{(Institution, }\OperatorTok{-}\StringTok{`}\DataTypeTok{2018}\StringTok{`}\NormalTok{), }\DataTypeTok{weight=}\StringTok{`}\DataTypeTok{2018}\StringTok{`}\NormalTok{)) }\OperatorTok{+}\StringTok{ }
\StringTok{  }\KeywordTok{geom_bar}\NormalTok{() }\OperatorTok{+}\StringTok{ }
\StringTok{  }\KeywordTok{scale_y_continuous}\NormalTok{(}\DataTypeTok{labels=}\NormalTok{comma) }\OperatorTok{+}\StringTok{ }
\StringTok{  }\KeywordTok{labs}\NormalTok{(}\DataTypeTok{title=}\StringTok{"Top 10 Football Programs By Attendance"}\NormalTok{, }\DataTypeTok{x=}\StringTok{"School"}\NormalTok{, }\DataTypeTok{y=}\StringTok{"Attendance"}\NormalTok{)}
\end{Highlighting}
\end{Shaded}

\includegraphics{SportsData_files/figure-latex/unnamed-chunk-110-1.pdf}

The library has lots and lots of ways to alter the styling -- we can programmatically control nearly every part of the look and feel of the chart. One simple way is to apply themes in the library already. We do that the same way we've done other things -- we add them. Here's the light theme.

\begin{Shaded}
\begin{Highlighting}[]
\KeywordTok{ggplot}\NormalTok{(top10, }\KeywordTok{aes}\NormalTok{(}\DataTypeTok{x=}\KeywordTok{reorder}\NormalTok{(Institution, }\OperatorTok{-}\StringTok{`}\DataTypeTok{2018}\StringTok{`}\NormalTok{), }\DataTypeTok{weight=}\StringTok{`}\DataTypeTok{2018}\StringTok{`}\NormalTok{)) }\OperatorTok{+}\StringTok{ }
\StringTok{  }\KeywordTok{geom_bar}\NormalTok{() }\OperatorTok{+}\StringTok{ }
\StringTok{  }\KeywordTok{scale_y_continuous}\NormalTok{(}\DataTypeTok{labels=}\NormalTok{comma) }\OperatorTok{+}\StringTok{ }
\StringTok{  }\KeywordTok{labs}\NormalTok{(}\DataTypeTok{title=}\StringTok{"Top 10 Football Programs By Attendance"}\NormalTok{, }\DataTypeTok{x=}\StringTok{"School"}\NormalTok{, }\DataTypeTok{y=}\StringTok{"Attendance"}\NormalTok{) }\OperatorTok{+}\StringTok{ }
\StringTok{  }\KeywordTok{theme_light}\NormalTok{()}
\end{Highlighting}
\end{Shaded}

\includegraphics{SportsData_files/figure-latex/unnamed-chunk-111-1.pdf}

Or the minimal theme:

\begin{Shaded}
\begin{Highlighting}[]
\KeywordTok{ggplot}\NormalTok{(top10, }\KeywordTok{aes}\NormalTok{(}\DataTypeTok{x=}\KeywordTok{reorder}\NormalTok{(Institution, }\OperatorTok{-}\StringTok{`}\DataTypeTok{2018}\StringTok{`}\NormalTok{), }\DataTypeTok{weight=}\StringTok{`}\DataTypeTok{2018}\StringTok{`}\NormalTok{)) }\OperatorTok{+}\StringTok{ }
\StringTok{  }\KeywordTok{geom_bar}\NormalTok{() }\OperatorTok{+}\StringTok{ }
\StringTok{  }\KeywordTok{scale_y_continuous}\NormalTok{(}\DataTypeTok{labels=}\NormalTok{comma) }\OperatorTok{+}\StringTok{ }
\StringTok{  }\KeywordTok{labs}\NormalTok{(}\DataTypeTok{title=}\StringTok{"Top 10 Football Programs By Attendance"}\NormalTok{, }\DataTypeTok{x=}\StringTok{"School"}\NormalTok{, }\DataTypeTok{y=}\StringTok{"Attendance"}\NormalTok{) }\OperatorTok{+}\StringTok{ }
\StringTok{  }\KeywordTok{theme_minimal}\NormalTok{()}
\end{Highlighting}
\end{Shaded}

\includegraphics{SportsData_files/figure-latex/unnamed-chunk-112-1.pdf}

Later on, we'll write our own themes. For now, the built in ones will get us closer to something that looks good.

\hypertarget{one-last-trick-coord-flip}{%
\section{One last trick: coord flip}\label{one-last-trick-coord-flip}}

Sometimes, we don't want vertical bars. Maybe we think this would look better horizontal. How do we do that? By adding \texttt{coord\_flip()} to our code. It does what it says -- it inverts the coordinates of the figures.

\begin{Shaded}
\begin{Highlighting}[]
\KeywordTok{ggplot}\NormalTok{(top10, }\KeywordTok{aes}\NormalTok{(}\DataTypeTok{x=}\KeywordTok{reorder}\NormalTok{(Institution, }\OperatorTok{-}\StringTok{`}\DataTypeTok{2018}\StringTok{`}\NormalTok{), }\DataTypeTok{weight=}\StringTok{`}\DataTypeTok{2018}\StringTok{`}\NormalTok{)) }\OperatorTok{+}\StringTok{ }
\StringTok{  }\KeywordTok{geom_bar}\NormalTok{() }\OperatorTok{+}\StringTok{ }
\StringTok{  }\KeywordTok{scale_y_continuous}\NormalTok{(}\DataTypeTok{labels=}\NormalTok{comma) }\OperatorTok{+}\StringTok{ }
\StringTok{  }\KeywordTok{labs}\NormalTok{(}\DataTypeTok{title=}\StringTok{"Top 10 Football Programs By Attendance"}\NormalTok{, }\DataTypeTok{x=}\StringTok{"School"}\NormalTok{, }\DataTypeTok{y=}\StringTok{"Attendance"}\NormalTok{) }\OperatorTok{+}\StringTok{ }
\StringTok{  }\KeywordTok{theme_minimal}\NormalTok{() }\OperatorTok{+}\StringTok{ }
\StringTok{  }\KeywordTok{coord_flip}\NormalTok{()}
\end{Highlighting}
\end{Shaded}

\includegraphics{SportsData_files/figure-latex/unnamed-chunk-113-1.pdf}

\hypertarget{stacked-bar-charts}{%
\chapter{Stacked bar charts}\label{stacked-bar-charts}}

One of the elements of data visualization excellence is \textbf{inviting comparison}. Often that comes in showing \textbf{what proportion a thing is in relation to the whole thing}. With bar charts, we're showing magnitude of the whole thing. If we have information about the parts of the whole, \textbf{we can stack them on top of each other to compare them, showing both the whole and the components}. And it's a simple change to what we've already done.

\begin{Shaded}
\begin{Highlighting}[]
\KeywordTok{library}\NormalTok{(tidyverse)}
\end{Highlighting}
\end{Shaded}

We're going to use a dataset of graduation rates by gender by school in the NCAA. \href{https://unl.box.com/s/3nw1eokvs9zfdjyzvjaj3xdq01rm8sym}{You can get it here}.

\begin{Shaded}
\begin{Highlighting}[]
\NormalTok{grads <-}\StringTok{ }\KeywordTok{read_csv}\NormalTok{(}\StringTok{'data/grads.csv'}\NormalTok{)}
\end{Highlighting}
\end{Shaded}

\begin{verbatim}
## Parsed with column specification:
## cols(
##   `Institution name` = col_character(),
##   `Primary Conference in Actual Year` = col_character(),
##   `Cohort year` = col_double(),
##   Gender = col_character(),
##   `Number of completers` = col_double(),
##   Total = col_double()
## )
\end{verbatim}

What we have here is the name of the school, the conference, the cohort of when they started school, the gender, the number of that gender that graduated and the total number of graduates in that cohort.

Let's pretend for a moment we're looking at the graduation rates of men and women in the Big 10 Conference and we want to chart that. First, let's work on our data. We need to filter the ``Big Ten Conference'' school, and we want the latest year, which is 2009. So we'll create a dataframe called \texttt{BIG09} and populate it.

\begin{Shaded}
\begin{Highlighting}[]
\NormalTok{BIG09 <-}\StringTok{ }\NormalTok{grads }\OperatorTok\StringTok{ }\KeywordTok{filter}\NormalTok{(}\StringTok{`}\DataTypeTok{Primary Conference in Actual Year}\StringTok{`}\OperatorTok{==}\StringTok{"Big Ten Conference"}\NormalTok{) }\OperatorTok\StringTok{ }\KeywordTok{filter}\NormalTok{(}\StringTok{`}\DataTypeTok{Cohort year}\StringTok{`} \OperatorTok{==}\StringTok{ }\DecValTok{2009}\NormalTok{)}
\end{Highlighting}
\end{Shaded}

Reminder: \texttt{head()} will give you a quick look at what your data looks like, but \textbf{it will only show you the first six rows}.

\begin{Shaded}
\begin{Highlighting}[]
\KeywordTok{head}\NormalTok{(BIG09)}
\end{Highlighting}
\end{Shaded}

\begin{verbatim}
## # A tibble: 6 x 6
##   `Institution nam~ `Primary Confer~ `Cohort year` Gender `Number of comp~ Total
##   <chr>             <chr>                    <dbl> <chr>             <dbl> <dbl>
## 1 University of Il~ Big Ten Confere~          2009 Men                2973  5940
## 2 University of Il~ Big Ten Confere~          2009 Women              2967  5940
## 3 Northwestern Uni~ Big Ten Confere~          2009 Men                 963  1974
## 4 Northwestern Uni~ Big Ten Confere~          2009 Women              1011  1974
## 5 Indiana Universi~ Big Ten Confere~          2009 Men                2667  5626
## 6 Indiana Universi~ Big Ten Confere~          2009 Women              2959  5626
\end{verbatim}

Building on what we learned in the last chapter, we know we can turn this into a bar chart with an x value, a weight and a geom\_bar. What're going to add is a \texttt{fill}. The \texttt{fill} will stack bars on each other based on which element it is. In this case, we can fill the bar by Gender, which means it will stack the number of male graduates on top of the number of female graduates and we can see how they compare.

\begin{Shaded}
\begin{Highlighting}[]
\KeywordTok{ggplot}\NormalTok{(BIG09, }\KeywordTok{aes}\NormalTok{(}\DataTypeTok{x=}\KeywordTok{reorder}\NormalTok{(}\StringTok{`}\DataTypeTok{Institution name}\StringTok{`}\NormalTok{, }\OperatorTok{-}\NormalTok{Total), }\DataTypeTok{weight=}\StringTok{`}\DataTypeTok{Number of completers}\StringTok{`}\NormalTok{, }\DataTypeTok{fill=}\NormalTok{Gender)) }\OperatorTok{+}\StringTok{ }\KeywordTok{geom_bar}\NormalTok{() }\OperatorTok{+}\StringTok{ }\KeywordTok{coord_flip}\NormalTok{()}
\end{Highlighting}
\end{Shaded}

\includegraphics{SportsData_files/figure-latex/unnamed-chunk-118-1.pdf}

What's the problem with this chart?

Let me ask a different question -- which schools have larger differences in male and female graduation rates? Can you compare Illnois to Northwestern? Not really. We've charted the total numbers. We need the percentage of the whole.

\begin{quote}
\textbf{YOUR TURN}: Using what you know -- hint: mutate -- how could you chart this using percents of the whole instead of counts?
\end{quote}

\hypertarget{waffle-charts}{%
\chapter{Waffle charts}\label{waffle-charts}}

Pie charts are the devil. They should be an instant F in any data visualization class. I'll give you an example of why.

What's the racial breakdown of journalism majors at UNL?

Here it is in a pie chart:

\begin{Shaded}
\begin{Highlighting}[]
\KeywordTok{library}\NormalTok{(tidyverse)}

\NormalTok{enrollment <-}\StringTok{ }\KeywordTok{read.csv}\NormalTok{(}\StringTok{"~/Box/Courses/JOUR407-Data-Visualization/Data/collegeenrollment.csv"}\NormalTok{)}

\NormalTok{jour <-}\StringTok{ }\KeywordTok{filter}\NormalTok{(enrollment, MajorName }\OperatorTok{==}\StringTok{ "Journalism"}\NormalTok{)}

\NormalTok{jdf <-}\StringTok{ }\NormalTok{jour }\OperatorTok\StringTok{ }
\KeywordTok{group_by}\NormalTok{(Race) }\OperatorTok
\KeywordTok{summarise}\NormalTok{(}
       \DataTypeTok{total=}\KeywordTok{sum}\NormalTok{(Count)) }\OperatorTok
\KeywordTok{select}\NormalTok{(Race, total) }\OperatorTok\StringTok{ }
\KeywordTok{filter}\NormalTok{(total }\OperatorTok{!=}\StringTok{ }\DecValTok{0}\NormalTok{)}

\KeywordTok{ggplot}\NormalTok{(jdf, }\KeywordTok{aes}\NormalTok{(}\DataTypeTok{x=}\StringTok{""}\NormalTok{, }\DataTypeTok{y=}\NormalTok{total, }\DataTypeTok{fill=}\NormalTok{Race)) }\OperatorTok{+}\StringTok{ }\KeywordTok{geom_bar}\NormalTok{(}\DataTypeTok{width =} \DecValTok{1}\NormalTok{, }\DataTypeTok{stat =} \StringTok{"identity"}\NormalTok{) }\OperatorTok{+}\StringTok{ }\KeywordTok{coord_polar}\NormalTok{(}\StringTok{"y"}\NormalTok{, }\DataTypeTok{start=}\DecValTok{0}\NormalTok{)}
\end{Highlighting}
\end{Shaded}

\includegraphics{SportsData_files/figure-latex/unnamed-chunk-119-1.pdf}

You can see, it's pretty white. But \ldots{} what about beyond that? How carefully can you evaluate angles and area?

Not well.

So let's introduce a better way: The Waffle Chart. Some call it a square pie chart. I personally hate that. Waffles it is.

\textbf{A waffle chart is designed to show you parts of the whole -- proportionality}. How many yards on offense come from rushing or passing. How many singles, doubles, triples and home runs make up a teams hits. How many shots a basketball team takes are two pointers versus three pointers.

First, install the library in the console:

\texttt{install.packages(\textquotesingle{}waffle\textquotesingle{})}

Now load it:

\begin{Shaded}
\begin{Highlighting}[]
\KeywordTok{library}\NormalTok{(waffle)}
\end{Highlighting}
\end{Shaded}

Let's look at the debacle that was Nebraska vs.~Ohio State this fall in Football. \href{https://www.espn.com/college-football/matchup?gameId=401112241}{Here's the box score}, which we'll use for this walkthrough.

The easiest way to do waffle charts is to make vectors of your data and plug them in. To make a vector, we use the \texttt{c} or concatenate function, something we've done before.

So let's look at offense. Rushing vs passing.

\begin{Shaded}
\begin{Highlighting}[]
\NormalTok{nu <-}\StringTok{ }\KeywordTok{c}\NormalTok{(}\StringTok{"Rushing"}\NormalTok{=}\DecValTok{184}\NormalTok{, }\StringTok{"Passing"}\NormalTok{=}\DecValTok{47}\NormalTok{)}
\NormalTok{oh <-}\StringTok{ }\KeywordTok{c}\NormalTok{(}\StringTok{"Rushing"}\NormalTok{=}\DecValTok{368}\NormalTok{, }\StringTok{"Passing"}\NormalTok{=}\DecValTok{212}\NormalTok{)}
\end{Highlighting}
\end{Shaded}

So what does the breakdown of the night look like?

The waffle library can break this down in a way that's easier on the eyes than a pie chart. We call the library, add the data, specify the number of rows, give it a title and an x value label, and to clean up a quirk of the library, we've got to specify colors.

\begin{Shaded}
\begin{Highlighting}[]
\KeywordTok{waffle}\NormalTok{(nu, }\DataTypeTok{rows =} \DecValTok{10}\NormalTok{, }\DataTypeTok{title=}\StringTok{"Nebraska's offense"}\NormalTok{, }\DataTypeTok{xlab=}\StringTok{"1 square = 1 yard"}\NormalTok{, }\DataTypeTok{colors =} \KeywordTok{c}\NormalTok{(}\StringTok{"black"}\NormalTok{, }\StringTok{"red"}\NormalTok{))}
\end{Highlighting}
\end{Shaded}

\includegraphics{SportsData_files/figure-latex/unnamed-chunk-122-1.pdf}

Or, we could make this two teams in the same chart.

\begin{Shaded}
\begin{Highlighting}[]
\NormalTok{passing <-}\StringTok{ }\KeywordTok{c}\NormalTok{(}\StringTok{"Nebraska"}\NormalTok{=}\DecValTok{47}\NormalTok{, }\StringTok{"Ohio State"}\NormalTok{=}\DecValTok{212}\NormalTok{)}
\end{Highlighting}
\end{Shaded}

\begin{Shaded}
\begin{Highlighting}[]
\KeywordTok{waffle}\NormalTok{(passing, }\DataTypeTok{rows =} \DecValTok{10}\NormalTok{, }\DataTypeTok{title=}\StringTok{"Nebraska vs Ohio State: passing"}\NormalTok{, }\DataTypeTok{xlab=}\StringTok{"1 square = 1 yard"}\NormalTok{, }\DataTypeTok{colors =} \KeywordTok{c}\NormalTok{(}\StringTok{"red"}\NormalTok{, }\StringTok{"black"}\NormalTok{))}
\end{Highlighting}
\end{Shaded}

\includegraphics{SportsData_files/figure-latex/unnamed-chunk-124-1.pdf}

\hypertarget{waffle-irons}{%
\section{Waffle Irons}\label{waffle-irons}}

So what does it look like if we compare the two teams using the two vectors in the same chart? To do that -- and I am not making this up -- you have to create a waffle iron. Get it? Waffle charts? Iron?

\begin{Shaded}
\begin{Highlighting}[]
\KeywordTok{iron}\NormalTok{(}
 \KeywordTok{waffle}\NormalTok{(nu, }\DataTypeTok{rows =} \DecValTok{10}\NormalTok{, }\DataTypeTok{title=}\StringTok{"Nebraska's offense"}\NormalTok{, }\DataTypeTok{xlab=}\StringTok{"1 square = 1 yard"}\NormalTok{, }\DataTypeTok{colors =} \KeywordTok{c}\NormalTok{(}\StringTok{"black"}\NormalTok{, }\StringTok{"red"}\NormalTok{)),}
 \KeywordTok{waffle}\NormalTok{(oh, }\DataTypeTok{rows =} \DecValTok{10}\NormalTok{, }\DataTypeTok{title=}\StringTok{"Ohio State's offense"}\NormalTok{, }\DataTypeTok{xlab=}\StringTok{"1 square = 1 yard"}\NormalTok{, }\DataTypeTok{colors =} \KeywordTok{c}\NormalTok{(}\StringTok{"black"}\NormalTok{, }\StringTok{"red"}\NormalTok{))}
\NormalTok{)}
\end{Highlighting}
\end{Shaded}

\includegraphics{SportsData_files/figure-latex/unnamed-chunk-125-1.pdf}

What do you notice about this chart? Notice how the squares aren't the same size? Well, Ohio State outgained Nebraska by a long way. So the squares aren't the same size because the numbers aren't the same. We can fix that by adding an unnamed padding number so the number of shots add up to the same thing. Let's make the total for everyone be 580, Ohio State's total yards of offense. So to do that, we need to add a padding of 349 to Nebraska. REMEMBER: Don't name it or it'll show up in the legend.

\begin{Shaded}
\begin{Highlighting}[]
\NormalTok{nu <-}\StringTok{ }\KeywordTok{c}\NormalTok{(}\StringTok{"Rushing"}\NormalTok{=}\DecValTok{184}\NormalTok{, }\StringTok{"Passing"}\NormalTok{=}\DecValTok{47}\NormalTok{, }\DecValTok{349}\NormalTok{)}
\NormalTok{oh <-}\StringTok{ }\KeywordTok{c}\NormalTok{(}\StringTok{"Rushing"}\NormalTok{=}\DecValTok{368}\NormalTok{, }\StringTok{"Passing"}\NormalTok{=}\DecValTok{212}\NormalTok{, }\DecValTok{0}\NormalTok{)}
\end{Highlighting}
\end{Shaded}

Now, in our waffle iron, if we don't give that padding a color, we'll get an error. So we need to make it white. Which, given our white background, means it will disappear.

\begin{Shaded}
\begin{Highlighting}[]
\KeywordTok{iron}\NormalTok{(}
 \KeywordTok{waffle}\NormalTok{(nu, }\DataTypeTok{rows =} \DecValTok{10}\NormalTok{, }\DataTypeTok{title=}\StringTok{"Nebraska's offense"}\NormalTok{, }\DataTypeTok{xlab=}\StringTok{"1 square = 1 yard"}\NormalTok{, }\DataTypeTok{colors =} \KeywordTok{c}\NormalTok{(}\StringTok{"black"}\NormalTok{, }\StringTok{"red"}\NormalTok{, }\StringTok{"white"}\NormalTok{)),}
 \KeywordTok{waffle}\NormalTok{(oh, }\DataTypeTok{rows =} \DecValTok{10}\NormalTok{, }\DataTypeTok{title=}\StringTok{"Ohio State's offense"}\NormalTok{, }\DataTypeTok{xlab=}\StringTok{"1 square = 1 yard"}\NormalTok{, }\DataTypeTok{colors =} \KeywordTok{c}\NormalTok{(}\StringTok{"black"}\NormalTok{, }\StringTok{"red"}\NormalTok{, }\StringTok{"white"}\NormalTok{))}
\NormalTok{)}
\end{Highlighting}
\end{Shaded}

\includegraphics{SportsData_files/figure-latex/unnamed-chunk-127-1.pdf}

One last thing we can do is change the 1 square = 1 yard bit -- which makes the squares really small in this case -- by dividing our vector. Remember what you learned in Swirl about math on vectors?

\begin{Shaded}
\begin{Highlighting}[]
\KeywordTok{iron}\NormalTok{(}
 \KeywordTok{waffle}\NormalTok{(nu}\OperatorTok{/}\DecValTok{2}\NormalTok{, }\DataTypeTok{rows =} \DecValTok{10}\NormalTok{, }\DataTypeTok{title=}\StringTok{"Nebraska's offense"}\NormalTok{, }\DataTypeTok{xlab=}\StringTok{"1 square = 2 yards"}\NormalTok{, }\DataTypeTok{colors =} \KeywordTok{c}\NormalTok{(}\StringTok{"black"}\NormalTok{, }\StringTok{"red"}\NormalTok{, }\StringTok{"white"}\NormalTok{)),}
 \KeywordTok{waffle}\NormalTok{(oh}\OperatorTok{/}\DecValTok{2}\NormalTok{, }\DataTypeTok{rows =} \DecValTok{10}\NormalTok{, }\DataTypeTok{title=}\StringTok{"Ohio State's offense"}\NormalTok{, }\DataTypeTok{xlab=}\StringTok{"1 square = 2 yards"}\NormalTok{, }\DataTypeTok{colors =} \KeywordTok{c}\NormalTok{(}\StringTok{"black"}\NormalTok{, }\StringTok{"red"}\NormalTok{, }\StringTok{"white"}\NormalTok{))}
\NormalTok{)}
\end{Highlighting}
\end{Shaded}

\includegraphics{SportsData_files/figure-latex/unnamed-chunk-128-1.pdf}

News flash: Ohio State crushed Nebraska.

\hypertarget{line-charts}{%
\chapter{Line charts}\label{line-charts}}

So far, we've talked about bar charts -- stacked or otherwise -- are good for showing relative size of a thing compared to another thing. Stacked Bars and Waffle charts are good at showing proportions of a whole.

\textbf{Line charts are good for showing change over time.}

Let's look at how we can answer this question: Why was Nebraska terrible at basketball last season?

Let's start getting all that we need. We can use the tidyverse shortcut.

\begin{Shaded}
\begin{Highlighting}[]
\KeywordTok{library}\NormalTok{(tidyverse)}
\end{Highlighting}
\end{Shaded}

Now we'll \href{https://unl.box.com/s/a8m91bro10t89watsyo13yjegb1fy009}{import the data you need}. Mine looks like this:

\begin{Shaded}
\begin{Highlighting}[]
\NormalTok{logs <-}\StringTok{ }\KeywordTok{read_csv}\NormalTok{(}\StringTok{"data/logs19.csv"}\NormalTok{)}
\end{Highlighting}
\end{Shaded}

\begin{verbatim}
## Warning: Missing column names filled in: 'X1' [1]
\end{verbatim}

\begin{verbatim}
## Parsed with column specification:
## cols(
##   .default = col_double(),
##   Date = col_date(format = ""),
##   HomeAway = col_character(),
##   Opponent = col_character(),
##   W_L = col_character(),
##   Blank = col_logical(),
##   Team = col_character(),
##   Conference = col_character(),
##   season = col_character()
## )
\end{verbatim}

\begin{verbatim}
## See spec(...) for full column specifications.
\end{verbatim}

This data has every game from every team in it, so we need to use filtering to limit it, because we just want to look at Nebraska. If you don't remember, flip back to chapter 5.

\begin{Shaded}
\begin{Highlighting}[]
\NormalTok{nu <-}\StringTok{ }\NormalTok{logs }\OperatorTok\StringTok{ }\KeywordTok{filter}\NormalTok{(Team }\OperatorTok{==}\StringTok{ "Nebraska Cornhuskers"}\NormalTok{)}
\end{Highlighting}
\end{Shaded}

Because this data has just Nebraska data in it, the dates are formatted correctly, and the data is long data (instead of wide), we have what we need to make line charts.

Line charts, unlike bar charts, do have a y-axis. So in our ggplot step, we have to define what our x and y axes are. In this case, the x axis is our Date -- the most common x axis in line charts is going to be a date of some variety -- and y in this case is up to us. We've seen from previous walkthroughs that how well a team shoots the ball has a lot to do with how well a team does in a season, so let's chart that.

\begin{Shaded}
\begin{Highlighting}[]
\KeywordTok{ggplot}\NormalTok{(nu, }\KeywordTok{aes}\NormalTok{(}\DataTypeTok{x=}\NormalTok{Date, }\DataTypeTok{y=}\NormalTok{TeamFGPCT)) }\OperatorTok{+}\StringTok{ }\KeywordTok{geom_line}\NormalTok{()}
\end{Highlighting}
\end{Shaded}

\includegraphics{SportsData_files/figure-latex/unnamed-chunk-132-1.pdf}

See a problem here? Note the Y axis doesn't start with zero. That makes this look worse than it is (and that February swoon is pretty bad). To make the axis what you want, you can use \texttt{scale\_x\_continuous} or \texttt{scale\_y\_continuous} and pass in a list with the bottom and top value you want. You do that like this:

\begin{Shaded}
\begin{Highlighting}[]
\KeywordTok{ggplot}\NormalTok{(nu, }\KeywordTok{aes}\NormalTok{(}\DataTypeTok{x=}\NormalTok{Date, }\DataTypeTok{y=}\NormalTok{TeamFGPCT)) }\OperatorTok{+}\StringTok{ }\KeywordTok{geom_line}\NormalTok{() }\OperatorTok{+}\StringTok{ }\KeywordTok{scale_y_continuous}\NormalTok{(}\DataTypeTok{limits =} \KeywordTok{c}\NormalTok{(}\DecValTok{0}\NormalTok{, }\FloatTok{.6}\NormalTok{))}
\end{Highlighting}
\end{Shaded}

\includegraphics{SportsData_files/figure-latex/unnamed-chunk-133-1.pdf}

Note also that our X axis labels are automated. It knows it's a date and it just labels it by month.

\hypertarget{this-is-too-simple.}{%
\section{This is too simple.}\label{this-is-too-simple.}}

With datasets, we want to invite comparison. So let's answer the question visually. Let's put two lines on the same chart. How does Nebraska compare to Michigan State and Purdue, the eventual regular season co-champions?

\begin{Shaded}
\begin{Highlighting}[]
\NormalTok{msu <-}\StringTok{ }\NormalTok{logs }\OperatorTok\StringTok{ }\KeywordTok{filter}\NormalTok{(Team }\OperatorTok{==}\StringTok{ "Michigan State Spartans"}\NormalTok{)}
\end{Highlighting}
\end{Shaded}

In this case, because we have two different datasets, we're going to put everything in the geom instead of the ggplot step. We also have to explicitly state what dataset we're using by saying \texttt{data=} in the geom step.

First, let's chart Nebraska. Read carefully. First we set the data. Then we set our aesthetic. Unlike bars, we need an X and a Y variable. In this case, our X is the date of the game, Y is the thing we want the lines to move with. In this case, the Team Field Goal Percentage -- TeamFGPCT.

\begin{Shaded}
\begin{Highlighting}[]
\KeywordTok{ggplot}\NormalTok{() }\OperatorTok{+}\StringTok{ }\KeywordTok{geom_line}\NormalTok{(}\DataTypeTok{data=}\NormalTok{nu, }\KeywordTok{aes}\NormalTok{(}\DataTypeTok{x=}\NormalTok{Date, }\DataTypeTok{y=}\NormalTok{TeamFGPCT), }\DataTypeTok{color=}\StringTok{"red"}\NormalTok{)}
\end{Highlighting}
\end{Shaded}

\includegraphics{SportsData_files/figure-latex/unnamed-chunk-135-1.pdf}

Now, by using +, we can add Michigan State to it. REMEMBER COPY AND PASTE IS A THING. Nothing changes except what data you are using.

\begin{Shaded}
\begin{Highlighting}[]
\KeywordTok{ggplot}\NormalTok{() }\OperatorTok{+}\StringTok{ }\KeywordTok{geom_line}\NormalTok{(}\DataTypeTok{data=}\NormalTok{nu, }\KeywordTok{aes}\NormalTok{(}\DataTypeTok{x=}\NormalTok{Date, }\DataTypeTok{y=}\NormalTok{TeamFGPCT), }\DataTypeTok{color=}\StringTok{"red"}\NormalTok{) }\OperatorTok{+}\StringTok{ }\KeywordTok{geom_line}\NormalTok{(}\DataTypeTok{data=}\NormalTok{msu, }\KeywordTok{aes}\NormalTok{(}\DataTypeTok{x=}\NormalTok{Date, }\DataTypeTok{y=}\NormalTok{TeamFGPCT), }\DataTypeTok{color=}\StringTok{"green"}\NormalTok{)}
\end{Highlighting}
\end{Shaded}

\includegraphics{SportsData_files/figure-latex/unnamed-chunk-136-1.pdf}

Let's flatten our lines out by zeroing the Y axis.

\begin{Shaded}
\begin{Highlighting}[]
\KeywordTok{ggplot}\NormalTok{() }\OperatorTok{+}\StringTok{ }\KeywordTok{geom_line}\NormalTok{(}\DataTypeTok{data=}\NormalTok{nu, }\KeywordTok{aes}\NormalTok{(}\DataTypeTok{x=}\NormalTok{Date, }\DataTypeTok{y=}\NormalTok{TeamFGPCT), }\DataTypeTok{color=}\StringTok{"red"}\NormalTok{) }\OperatorTok{+}\StringTok{ }\KeywordTok{geom_line}\NormalTok{(}\DataTypeTok{data=}\NormalTok{msu, }\KeywordTok{aes}\NormalTok{(}\DataTypeTok{x=}\NormalTok{Date, }\DataTypeTok{y=}\NormalTok{TeamFGPCT), }\DataTypeTok{color=}\StringTok{"green"}\NormalTok{) }\OperatorTok{+}\StringTok{ }\KeywordTok{scale_y_continuous}\NormalTok{(}\DataTypeTok{limits =} \KeywordTok{c}\NormalTok{(}\DecValTok{0}\NormalTok{, }\FloatTok{.65}\NormalTok{))}
\end{Highlighting}
\end{Shaded}

\includegraphics{SportsData_files/figure-latex/unnamed-chunk-137-1.pdf}

So visually speaking, the difference between Nebraska and Michigan State's season is that Michigan State stayed mostly on an even keel, and Nebraska went on a two month swoon.

\hypertarget{but-what-if-i-wanted-to-add-a-lot-of-lines.}{%
\section{But what if I wanted to add a lot of lines.}\label{but-what-if-i-wanted-to-add-a-lot-of-lines.}}

Fine. How about all Power Five Schools? This data for example purposes. You don't have to do it.

\begin{Shaded}
\begin{Highlighting}[]
\NormalTok{powerfive <-}\StringTok{ }\KeywordTok{c}\NormalTok{(}\StringTok{"SEC"}\NormalTok{, }\StringTok{"Big Ten"}\NormalTok{, }\StringTok{"Pac-12"}\NormalTok{, }\StringTok{"Big 12"}\NormalTok{, }\StringTok{"ACC"}\NormalTok{)}

\NormalTok{p5conf <-}\StringTok{ }\NormalTok{logs }\OperatorTok\StringTok{ }\KeywordTok{filter}\NormalTok{(Conference }\OperatorTok\StringTok{ }\NormalTok{powerfive)}
\end{Highlighting}
\end{Shaded}

I can keep layering on layers all day if I want. And if my dataset has more than one team in it, I need to use the \texttt{group} command. And, the layering comes in order -- so if you're going to layer a bunch of lines with a smaller group of lines, you want the bunch on the bottom. So to do that, your code stacks from the bottom. The first geom in the code gets rendered first. The second gets layered on top of that. The third gets layered on that and so on.

\begin{Shaded}
\begin{Highlighting}[]
\KeywordTok{ggplot}\NormalTok{() }\OperatorTok{+}\StringTok{ }\KeywordTok{geom_line}\NormalTok{(}\DataTypeTok{data=}\NormalTok{p5conf, }\KeywordTok{aes}\NormalTok{(}\DataTypeTok{x=}\NormalTok{Date, }\DataTypeTok{y=}\NormalTok{TeamFGPCT, }\DataTypeTok{group=}\NormalTok{Team), }\DataTypeTok{color=}\StringTok{"grey"}\NormalTok{) }\OperatorTok{+}\StringTok{ }\KeywordTok{geom_line}\NormalTok{(}\DataTypeTok{data=}\NormalTok{nu, }\KeywordTok{aes}\NormalTok{(}\DataTypeTok{x=}\NormalTok{Date, }\DataTypeTok{y=}\NormalTok{TeamFGPCT), }\DataTypeTok{color=}\StringTok{"red"}\NormalTok{) }\OperatorTok{+}\StringTok{ }\KeywordTok{geom_line}\NormalTok{(}\DataTypeTok{data=}\NormalTok{msu, }\KeywordTok{aes}\NormalTok{(}\DataTypeTok{x=}\NormalTok{Date, }\DataTypeTok{y=}\NormalTok{TeamFGPCT), }\DataTypeTok{color=}\StringTok{"green"}\NormalTok{) }\OperatorTok{+}\StringTok{ }\KeywordTok{scale_y_continuous}\NormalTok{(}\DataTypeTok{limits =} \KeywordTok{c}\NormalTok{(}\DecValTok{0}\NormalTok{, }\FloatTok{.65}\NormalTok{))}
\end{Highlighting}
\end{Shaded}

\includegraphics{SportsData_files/figure-latex/unnamed-chunk-139-1.pdf}

What do we see here? How has Nebraska and Michigan State's season evolved against all the rest of the teams in college basketball?

But how does that compare to the average? We can add that pretty easily by creating a new dataframe with it and add another geom\_line.

\begin{Shaded}
\begin{Highlighting}[]
\NormalTok{average <-}\StringTok{ }\NormalTok{logs }\OperatorTok\StringTok{ }\KeywordTok{group_by}\NormalTok{(Date) }\OperatorTok\StringTok{ }\KeywordTok{summarise}\NormalTok{(}\DataTypeTok{mean_shooting=}\KeywordTok{mean}\NormalTok{(TeamFGPCT))}
\end{Highlighting}
\end{Shaded}

\begin{Shaded}
\begin{Highlighting}[]
\KeywordTok{ggplot}\NormalTok{() }\OperatorTok{+}\StringTok{ }\KeywordTok{geom_line}\NormalTok{(}\DataTypeTok{data=}\NormalTok{p5conf, }\KeywordTok{aes}\NormalTok{(}\DataTypeTok{x=}\NormalTok{Date, }\DataTypeTok{y=}\NormalTok{TeamFGPCT, }\DataTypeTok{group=}\NormalTok{Team), }\DataTypeTok{color=}\StringTok{"grey"}\NormalTok{) }\OperatorTok{+}\StringTok{ }\KeywordTok{geom_line}\NormalTok{(}\DataTypeTok{data=}\NormalTok{nu, }\KeywordTok{aes}\NormalTok{(}\DataTypeTok{x=}\NormalTok{Date, }\DataTypeTok{y=}\NormalTok{TeamFGPCT), }\DataTypeTok{color=}\StringTok{"red"}\NormalTok{) }\OperatorTok{+}\StringTok{ }\KeywordTok{geom_line}\NormalTok{(}\DataTypeTok{data=}\NormalTok{msu, }\KeywordTok{aes}\NormalTok{(}\DataTypeTok{x=}\NormalTok{Date, }\DataTypeTok{y=}\NormalTok{TeamFGPCT), }\DataTypeTok{color=}\StringTok{"green"}\NormalTok{) }\OperatorTok{+}\StringTok{ }\KeywordTok{geom_line}\NormalTok{(}\DataTypeTok{data=}\NormalTok{average, }\KeywordTok{aes}\NormalTok{(}\DataTypeTok{x=}\NormalTok{Date, }\DataTypeTok{y=}\NormalTok{mean_shooting), }\DataTypeTok{color=}\StringTok{"black"}\NormalTok{) }\OperatorTok{+}\StringTok{ }\KeywordTok{scale_y_continuous}\NormalTok{(}\DataTypeTok{limits =} \KeywordTok{c}\NormalTok{(}\DecValTok{0}\NormalTok{, }\FloatTok{.65}\NormalTok{))}
\end{Highlighting}
\end{Shaded}

\includegraphics{SportsData_files/figure-latex/unnamed-chunk-141-1.pdf}

\hypertarget{step-charts}{%
\chapter{Step charts}\label{step-charts}}

Step charts are \textbf{a method of showing progress} toward something. They combine showing change over time -- \textbf{cumulative change over time} -- with magnitude. They're good at inviting comparison.

There's great examples out there. First is the Washignton Post looking at \href{https://www.washingtonpost.com/graphics/sports/lebron-james-michael-jordan-nba-scoring-list/?utm_term=.481074150849}{Lebron passing Jordan's career point total}. Another is John Burn-Murdoch's work at the Financial Times (which is paywalled) about soccer stars. \href{http://johnburnmurdoch.github.io/projects/goal-lines/CL/}{Here's an example of his work outside the paywall}.

To replicate this, we need cumulative data -- data that is the running total of data at a given point. So think of it this way -- Nebraska scores 50 points in a basketball game and then 50 more the next, their cumulative total at two games is 100 points.

Step charts can be used for all kinds of things -- showing how a player's career has evolved over time, how a team fares over a season, or franchise history. Let's walk through an example.

\begin{Shaded}
\begin{Highlighting}[]
\KeywordTok{library}\NormalTok{(tidyverse)}
\end{Highlighting}
\end{Shaded}

And we'll use \href{https://unl.box.com/s/a8m91bro10t89watsyo13yjegb1fy009}{our basketball log data}.

\begin{Shaded}
\begin{Highlighting}[]
\NormalTok{logs <-}\StringTok{ }\KeywordTok{read_csv}\NormalTok{(}\StringTok{"data/logs19.csv"}\NormalTok{)}
\end{Highlighting}
\end{Shaded}

\begin{verbatim}
## Warning: Missing column names filled in: 'X1' [1]
\end{verbatim}

\begin{verbatim}
## Parsed with column specification:
## cols(
##   .default = col_double(),
##   Date = col_date(format = ""),
##   HomeAway = col_character(),
##   Opponent = col_character(),
##   W_L = col_character(),
##   Blank = col_logical(),
##   Team = col_character(),
##   Conference = col_character(),
##   season = col_character()
## )
\end{verbatim}

\begin{verbatim}
## See spec(...) for full column specifications.
\end{verbatim}

Here we're going to look at the scoring differential of teams. If you score more than your opponent, you win. So it stands to reason that if you score a lot more than your opponent over the course of a season, you should be very good, right? Let's see.

The first thing we're going to do is calculate that differential. Then, we'll group it by the team. After that, we're going to summarize using a new function called \texttt{cumsum} or cumulative sum -- the sum for each game as we go forward. So game 1's cumsum is the differential of that game. Game 2's cumsum is Game 1 + Game 2. Game 3 is Game 1 + 2 + 3 and so on.

\begin{Shaded}
\begin{Highlighting}[]
\NormalTok{difflogs <-}\StringTok{ }\NormalTok{logs }\OperatorTok\StringTok{ }
\StringTok{  }\KeywordTok{mutate}\NormalTok{(}\DataTypeTok{Differential =}\NormalTok{ TeamScore }\OperatorTok{-}\StringTok{ }\NormalTok{OpponentScore) }\OperatorTok\StringTok{ }
\StringTok{  }\KeywordTok{group_by}\NormalTok{(Team) }\OperatorTok\StringTok{ }
\StringTok{  }\KeywordTok{mutate}\NormalTok{(}\DataTypeTok{CumDiff =} \KeywordTok{cumsum}\NormalTok{(Differential))}
\end{Highlighting}
\end{Shaded}

Now that we have the cumulative sum for each, let's filter it down to just Big Ten teams.

\begin{Shaded}
\begin{Highlighting}[]
\NormalTok{bigdiff <-}\StringTok{ }\NormalTok{difflogs }\OperatorTok\StringTok{ }\KeywordTok{filter}\NormalTok{(Conference }\OperatorTok{==}\StringTok{ "Big Ten"}\NormalTok{)}
\end{Highlighting}
\end{Shaded}

The step chart is it's own geom, so we can employ it just like we have the others. It works almost exactly the same as a line chart, but it uses the cumulative sum instead of a regular value and, as the name implies, creates a step like shape to the line instead of a curve.

\begin{Shaded}
\begin{Highlighting}[]
\KeywordTok{ggplot}\NormalTok{() }\OperatorTok{+}\StringTok{ }\KeywordTok{geom_step}\NormalTok{(}\DataTypeTok{data=}\NormalTok{bigdiff, }\KeywordTok{aes}\NormalTok{(}\DataTypeTok{x=}\NormalTok{Date, }\DataTypeTok{y=}\NormalTok{CumDiff, }\DataTypeTok{group=}\NormalTok{Team))}
\end{Highlighting}
\end{Shaded}

\includegraphics{SportsData_files/figure-latex/unnamed-chunk-146-1.pdf}

Let's try a different element of the aesthetic: color, but this time inside the aesthetic. Last time, we did the color outside. When you put it inside, you pass it a column name and ggplot will color each line based on what thing that is, and it will create a legend that labels each line that thing.

\begin{Shaded}
\begin{Highlighting}[]
\KeywordTok{ggplot}\NormalTok{() }\OperatorTok{+}\StringTok{ }\KeywordTok{geom_step}\NormalTok{(}\DataTypeTok{data=}\NormalTok{bigdiff, }\KeywordTok{aes}\NormalTok{(}\DataTypeTok{x=}\NormalTok{Date, }\DataTypeTok{y=}\NormalTok{CumDiff, }\DataTypeTok{group=}\NormalTok{Team, }\DataTypeTok{color=}\NormalTok{Team))}
\end{Highlighting}
\end{Shaded}

\includegraphics{SportsData_files/figure-latex/unnamed-chunk-147-1.pdf}

From this, we can see two teams in the Big Ten had negative point differentials last season -- Illinois and Rutgers.

Let's look at those top teams plus Nebraska.

\begin{Shaded}
\begin{Highlighting}[]
\NormalTok{nu <-}\StringTok{ }\NormalTok{bigdiff }\OperatorTok\StringTok{ }\KeywordTok{filter}\NormalTok{(Team }\OperatorTok{==}\StringTok{ "Nebraska Cornhuskers"}\NormalTok{)}
\NormalTok{mi <-}\StringTok{ }\NormalTok{bigdiff }\OperatorTok\StringTok{ }\KeywordTok{filter}\NormalTok{(Team }\OperatorTok{==}\StringTok{ "Michigan Wolverines"}\NormalTok{)}
\NormalTok{ms <-}\StringTok{ }\NormalTok{bigdiff }\OperatorTok\StringTok{ }\KeywordTok{filter}\NormalTok{(Team }\OperatorTok{==}\StringTok{ "Michigan State Spartans"}\NormalTok{)}
\end{Highlighting}
\end{Shaded}

Let's introduce a couple of new things here. First, note when I take the color OUT of the aesthetic, the legend disappears.

The second thing I'm going to add is the annotation layer. In this case, I am adding a text annotation layer, and I can specify where by adding in a x and a y value where I want to put it. This takes some finesse. After that, I'm going to add labels and a theme.

\begin{Shaded}
\begin{Highlighting}[]
\KeywordTok{ggplot}\NormalTok{() }\OperatorTok{+}\StringTok{ }
\StringTok{  }\KeywordTok{geom_step}\NormalTok{(}\DataTypeTok{data=}\NormalTok{bigdiff, }\KeywordTok{aes}\NormalTok{(}\DataTypeTok{x=}\NormalTok{Date, }\DataTypeTok{y=}\NormalTok{CumDiff, }\DataTypeTok{group=}\NormalTok{Team), }\DataTypeTok{color=}\StringTok{"light grey"}\NormalTok{) }\OperatorTok{+}
\StringTok{  }\KeywordTok{geom_step}\NormalTok{(}\DataTypeTok{data=}\NormalTok{nu, }\KeywordTok{aes}\NormalTok{(}\DataTypeTok{x=}\NormalTok{Date, }\DataTypeTok{y=}\NormalTok{CumDiff, }\DataTypeTok{group=}\NormalTok{Team), }\DataTypeTok{color=}\StringTok{"red"}\NormalTok{) }\OperatorTok{+}\StringTok{ }
\StringTok{  }\KeywordTok{geom_step}\NormalTok{(}\DataTypeTok{data=}\NormalTok{mi, }\KeywordTok{aes}\NormalTok{(}\DataTypeTok{x=}\NormalTok{Date, }\DataTypeTok{y=}\NormalTok{CumDiff, }\DataTypeTok{group=}\NormalTok{Team), }\DataTypeTok{color=}\StringTok{"blue"}\NormalTok{) }\OperatorTok{+}\StringTok{ }
\StringTok{  }\KeywordTok{geom_step}\NormalTok{(}\DataTypeTok{data=}\NormalTok{ms, }\KeywordTok{aes}\NormalTok{(}\DataTypeTok{x=}\NormalTok{Date, }\DataTypeTok{y=}\NormalTok{CumDiff, }\DataTypeTok{group=}\NormalTok{Team), }\DataTypeTok{color=}\StringTok{"green"}\NormalTok{) }\OperatorTok{+}
\StringTok{  }\KeywordTok{annotate}\NormalTok{(}\StringTok{"text"}\NormalTok{, }\DataTypeTok{x=}\NormalTok{(}\KeywordTok{as.Date}\NormalTok{(}\StringTok{"2018-12-10"}\NormalTok{)), }\DataTypeTok{y=}\DecValTok{220}\NormalTok{, }\DataTypeTok{label=}\StringTok{"Nebraska"}\NormalTok{) }\OperatorTok{+}
\StringTok{  }\KeywordTok{labs}\NormalTok{(}\DataTypeTok{x=}\StringTok{"Date"}\NormalTok{, }\DataTypeTok{y=}\StringTok{"Cumulative Point Differential"}\NormalTok{, }\DataTypeTok{title=}\StringTok{"Nebraska was among the league's most dominant"}\NormalTok{, }\DataTypeTok{subtitle=}\StringTok{"Before the misseason skid, Nebraska was at the top of the Big Ten in point differential"}\NormalTok{, }\DataTypeTok{caption=}\StringTok{"Source: Sports-Reference.com | By Matt Waite"}\NormalTok{) }\OperatorTok{+}
\StringTok{  }\KeywordTok{theme_minimal}\NormalTok{()}
\end{Highlighting}
\end{Shaded}

\includegraphics{SportsData_files/figure-latex/unnamed-chunk-149-1.pdf}

\hypertarget{ridge-charts}{%
\chapter{Ridge charts}\label{ridge-charts}}

Ridgeplots are useful for when you want to show how different groupings compare with a large number of datapoints. So let's look at how we do this, and in the process, we learn about ggplot extensions. The extensions add functionality to ggplot, which doesn't out of the box have ridgeplots (sometimes called joyplots).

In the console, run this: \texttt{install.packages("ggridges")}

Now we can add those libraries.

\begin{Shaded}
\begin{Highlighting}[]
\KeywordTok{library}\NormalTok{(tidyverse)}
\KeywordTok{library}\NormalTok{(ggridges)}
\end{Highlighting}
\end{Shaded}

So for this, let's look at every basketball game played since the 2014-15 season. That's more than 28,000 basketball games. \href{https://unl.box.com/s/u9407jj007fxtnu1vbkybdawaqg6j3fw}{Download that data here}.

\begin{Shaded}
\begin{Highlighting}[]
\NormalTok{logs <-}\StringTok{ }\KeywordTok{read_csv}\NormalTok{(}\StringTok{"data/logs1519.csv"}\NormalTok{)}
\end{Highlighting}
\end{Shaded}

\begin{verbatim}
## Warning: Missing column names filled in: 'X1' [1]
\end{verbatim}

\begin{verbatim}
## Parsed with column specification:
## cols(
##   .default = col_double(),
##   Date = col_date(format = ""),
##   HomeAway = col_character(),
##   Opponent = col_character(),
##   W_L = col_character(),
##   Blank = col_logical(),
##   Team = col_character(),
##   Conference = col_character(),
##   season = col_character()
## )
\end{verbatim}

\begin{verbatim}
## See spec(...) for full column specifications.
\end{verbatim}

So I want to group teams by wins. Wins are the only thing that matter -- ask Tim Miles. So our data has a column called W\_L that lists if the team won or lost. The problem is it doesn't just say W or L. If the game went to overtime, it lists that. That complicates counting wins. And, with ridgeplots, I want to be be able to separate EVERY GAME by how many wins the team had over a SEASON. So I've got some work to do.

First, here's a trick to find a string of text and make that. It's called \texttt{grepl} and the basic syntax is grepl for this string in this field and then do what I tell you. In this case, we're going to create a new field called winloss look for W or L (and ignore any OT notation) and give wins a 1 and losses a 0.

\begin{Shaded}
\begin{Highlighting}[]
\NormalTok{winlosslogs <-}\StringTok{ }\NormalTok{logs }\OperatorTok\StringTok{ }\KeywordTok{mutate}\NormalTok{(}\DataTypeTok{winloss =} \KeywordTok{case_when}\NormalTok{(}
  \KeywordTok{grepl}\NormalTok{(}\StringTok{"W"}\NormalTok{, W_L) }\OperatorTok{~}\StringTok{ }\DecValTok{1}\NormalTok{, }
  \KeywordTok{grepl}\NormalTok{(}\StringTok{"L"}\NormalTok{, W_L) }\OperatorTok{~}\StringTok{ }\DecValTok{0}\NormalTok{)}
\NormalTok{)}
\end{Highlighting}
\end{Shaded}

Now I'm going to add up all the winlosses for each team, which should give me the number of wins for each team.

\begin{Shaded}
\begin{Highlighting}[]
\NormalTok{winlosslogs }\OperatorTok\StringTok{ }\KeywordTok{group_by}\NormalTok{(Team, Conference, season) }\OperatorTok\StringTok{ }\KeywordTok{summarise}\NormalTok{(}\DataTypeTok{TeamWins =} \KeywordTok{sum}\NormalTok{(winloss)) ->}\StringTok{ }\NormalTok{teamseasonwins}
\end{Highlighting}
\end{Shaded}

Now that I have season win totals, I can join that data back to my log data so each game has the total number of wins in each season.

\begin{Shaded}
\begin{Highlighting}[]
\NormalTok{winlosslogs }\OperatorTok\StringTok{ }\KeywordTok{left_join}\NormalTok{(teamseasonwins, }\DataTypeTok{by=}\KeywordTok{c}\NormalTok{(}\StringTok{"Team"}\NormalTok{, }\StringTok{"Conference"}\NormalTok{, }\StringTok{"season"}\NormalTok{)) ->}\StringTok{ }\NormalTok{wintotallogs}
\end{Highlighting}
\end{Shaded}

Now I can use that same \texttt{case\_when} logic to create some groupings. So I want to group teams together by how many wins they had over the season. For no good reason, I started with more than 25 wins, then did groups of 5 down to 10 wins. If you had fewer than 10 wins, God help your program.

The way to create a new field based on groupings like that is to use \texttt{case\_when}, which says, basically, when This Thing Is True, Do This. So in our case, we're going to pass a couple of logical statements that when they are both true, our data gets labeled how we want to label it. So we \texttt{mutate} a field called grouping and then use \texttt{case\_when}.

\begin{Shaded}
\begin{Highlighting}[]
\NormalTok{wintotallogs }\OperatorTok\StringTok{ }\KeywordTok{mutate}\NormalTok{(}\DataTypeTok{grouping =} \KeywordTok{case_when}\NormalTok{(}
\NormalTok{  TeamWins }\OperatorTok{>}\StringTok{ }\DecValTok{25} \OperatorTok{~}\StringTok{ "More than 25 wins"}\NormalTok{,}
\NormalTok{  TeamWins }\OperatorTok{>=}\StringTok{ }\DecValTok{20} \OperatorTok{&}\StringTok{ }\NormalTok{TeamWins }\OperatorTok{<=}\DecValTok{25} \OperatorTok{~}\StringTok{ "20-25 wins"}\NormalTok{,}
\NormalTok{  TeamWins }\OperatorTok{>=}\StringTok{ }\DecValTok{15} \OperatorTok{&}\StringTok{ }\NormalTok{TeamWins }\OperatorTok{<=}\DecValTok{19} \OperatorTok{~}\StringTok{ "15-19 wins"}\NormalTok{,}
\NormalTok{  TeamWins }\OperatorTok{>=}\StringTok{ }\DecValTok{10} \OperatorTok{&}\StringTok{ }\NormalTok{TeamWins }\OperatorTok{<=}\DecValTok{14} \OperatorTok{~}\StringTok{ "10-14 wins"}\NormalTok{,}
\NormalTok{  TeamWins }\OperatorTok{<}\StringTok{ }\DecValTok{10} \OperatorTok{~}\StringTok{ "Less than 10 wins"}\NormalTok{)}
\NormalTok{) ->}\StringTok{ }\NormalTok{wintotalgroupinglogs}
\end{Highlighting}
\end{Shaded}

So my \texttt{wintotalgroupinglogs} table has each game, with a field that gives the total number of wins each team had that season and labeling each game with one of five groupings. I could use \texttt{dplyr} to do group\_by on those five groups and find some things out about them, but ridgeplots do that visually.

Let's look at the differences in rebounds by those five groups. Do teams that win more than 25 games rebound better than teams that win fewer games?

The answer might surprise you.

\begin{Shaded}
\begin{Highlighting}[]
\KeywordTok{ggplot}\NormalTok{(wintotalgroupinglogs, }\KeywordTok{aes}\NormalTok{(}\DataTypeTok{x =}\NormalTok{ TeamTotalRebounds, }\DataTypeTok{y =}\NormalTok{ grouping, }\DataTypeTok{fill =}\NormalTok{ grouping)) }\OperatorTok{+}
\StringTok{  }\KeywordTok{geom_density_ridges}\NormalTok{() }\OperatorTok{+}
\StringTok{  }\KeywordTok{theme_ridges}\NormalTok{() }\OperatorTok{+}\StringTok{ }
\StringTok{  }\KeywordTok{theme}\NormalTok{(}\DataTypeTok{legend.position =} \StringTok{"none"}\NormalTok{)}
\end{Highlighting}
\end{Shaded}

\begin{verbatim}
## Picking joint bandwidth of 0.88
\end{verbatim}

\begin{verbatim}
## Warning: Removed 2 rows containing non-finite values (stat_density_ridges).
\end{verbatim}

\includegraphics{SportsData_files/figure-latex/unnamed-chunk-156-1.pdf}

Answer? Not really. Game to game, maybe. Over five seasons? The differences are indistinguishable.

How about assists?

\begin{Shaded}
\begin{Highlighting}[]
\KeywordTok{ggplot}\NormalTok{(wintotalgroupinglogs, }\KeywordTok{aes}\NormalTok{(}\DataTypeTok{x =}\NormalTok{ TeamAssists, }\DataTypeTok{y =}\NormalTok{ grouping, }\DataTypeTok{fill =}\NormalTok{ grouping)) }\OperatorTok{+}
\StringTok{  }\KeywordTok{geom_density_ridges}\NormalTok{() }\OperatorTok{+}
\StringTok{  }\KeywordTok{theme_ridges}\NormalTok{() }\OperatorTok{+}\StringTok{ }
\StringTok{  }\KeywordTok{theme}\NormalTok{(}\DataTypeTok{legend.position =} \StringTok{"none"}\NormalTok{)}
\end{Highlighting}
\end{Shaded}

\begin{verbatim}
## Picking joint bandwidth of 0.601
\end{verbatim}

\begin{verbatim}
## Warning: Removed 2 rows containing non-finite values (stat_density_ridges).
\end{verbatim}

\includegraphics{SportsData_files/figure-latex/unnamed-chunk-157-1.pdf}

There's a little better, especially between top and bottom.

\begin{Shaded}
\begin{Highlighting}[]
\KeywordTok{ggplot}\NormalTok{(wintotalgroupinglogs, }\KeywordTok{aes}\NormalTok{(}\DataTypeTok{x =}\NormalTok{ Team3PPCT, }\DataTypeTok{y =}\NormalTok{ grouping, }\DataTypeTok{fill =}\NormalTok{ grouping)) }\OperatorTok{+}
\StringTok{  }\KeywordTok{geom_density_ridges}\NormalTok{() }\OperatorTok{+}
\StringTok{  }\KeywordTok{theme_ridges}\NormalTok{() }\OperatorTok{+}\StringTok{ }
\StringTok{  }\KeywordTok{theme}\NormalTok{(}\DataTypeTok{legend.position =} \StringTok{"none"}\NormalTok{)}
\end{Highlighting}
\end{Shaded}

\begin{verbatim}
## Picking joint bandwidth of 0.0156
\end{verbatim}

\begin{verbatim}
## Warning: Removed 2 rows containing non-finite values (stat_density_ridges).
\end{verbatim}

\includegraphics{SportsData_files/figure-latex/unnamed-chunk-158-1.pdf}

If you've been paying attention this semester, you know what's coming next.

\begin{Shaded}
\begin{Highlighting}[]
\KeywordTok{ggplot}\NormalTok{(wintotalgroupinglogs, }\KeywordTok{aes}\NormalTok{(}\DataTypeTok{x =}\NormalTok{ TeamFGPCT, }\DataTypeTok{y =}\NormalTok{ grouping, }\DataTypeTok{fill =}\NormalTok{ grouping)) }\OperatorTok{+}
\StringTok{  }\KeywordTok{geom_density_ridges}\NormalTok{() }\OperatorTok{+}
\StringTok{  }\KeywordTok{theme_ridges}\NormalTok{() }\OperatorTok{+}\StringTok{ }
\StringTok{  }\KeywordTok{theme}\NormalTok{(}\DataTypeTok{legend.position =} \StringTok{"none"}\NormalTok{)}
\end{Highlighting}
\end{Shaded}

\begin{verbatim}
## Picking joint bandwidth of 0.0102
\end{verbatim}

\begin{verbatim}
## Warning: Removed 2 rows containing non-finite values (stat_density_ridges).
\end{verbatim}

\includegraphics{SportsData_files/figure-latex/unnamed-chunk-159-1.pdf}

\hypertarget{lollipop-charts}{%
\chapter{Lollipop charts}\label{lollipop-charts}}

Second to my love of waffle charts because I'm always hungry, lollipop charts are an excellently named way of showing the difference between two things on a number line -- a start and a finish, for instance. Or the difference between two related things. Say, turnovers and assists. They aren't a geom, specifically, but you can assemble them out of points and segments, which are geoms.

\begin{Shaded}
\begin{Highlighting}[]
\KeywordTok{library}\NormalTok{(tidyverse)}
\end{Highlighting}
\end{Shaded}

\begin{Shaded}
\begin{Highlighting}[]
\NormalTok{logs <-}\StringTok{ }\KeywordTok{read_csv}\NormalTok{(}\StringTok{"data/logs19.csv"}\NormalTok{)}
\end{Highlighting}
\end{Shaded}

\begin{verbatim}
## Warning: Missing column names filled in: 'X1' [1]
\end{verbatim}

\begin{verbatim}
## Parsed with column specification:
## cols(
##   .default = col_double(),
##   Date = col_date(format = ""),
##   HomeAway = col_character(),
##   Opponent = col_character(),
##   W_L = col_character(),
##   Blank = col_logical(),
##   Team = col_character(),
##   Conference = col_character(),
##   season = col_character()
## )
\end{verbatim}

\begin{verbatim}
## See spec(...) for full column specifications.
\end{verbatim}

For the first example, let's look at the difference between a team's shooting performance and the conference's shooting performance as a whole. To get this, we're going to add up all the shots made by the conference, all the attempts taken by the conference, and then mutate a percentage based on that.

\begin{Shaded}
\begin{Highlighting}[]
\NormalTok{conferenceshooting <-}\StringTok{ }\NormalTok{logs }\OperatorTok
\StringTok{  }\KeywordTok{group_by}\NormalTok{(Conference) }\OperatorTok\StringTok{ }
\StringTok{  }\KeywordTok{summarise}\NormalTok{(}\DataTypeTok{totalshots =} \KeywordTok{sum}\NormalTok{(TeamFG), }\DataTypeTok{totalattempts =} \KeywordTok{sum}\NormalTok{(TeamFGA)) }\OperatorTok
\StringTok{  }\KeywordTok{mutate}\NormalTok{(}\DataTypeTok{conferenceshootingpct =}\NormalTok{ totalshots}\OperatorTok{/}\NormalTok{totalattempts)}
\end{Highlighting}
\end{Shaded}

Now I'm going to do the same with teams.

\begin{Shaded}
\begin{Highlighting}[]
\NormalTok{teamshooting <-}\StringTok{ }\NormalTok{logs }\OperatorTok
\StringTok{  }\KeywordTok{group_by}\NormalTok{(Team, Conference) }\OperatorTok\StringTok{ }
\StringTok{  }\KeywordTok{summarise}\NormalTok{(}\DataTypeTok{totalshots =} \KeywordTok{sum}\NormalTok{(TeamFG), }\DataTypeTok{totalattempts =} \KeywordTok{sum}\NormalTok{(TeamFGA)) }\OperatorTok
\StringTok{  }\KeywordTok{mutate}\NormalTok{(}\DataTypeTok{teamshootingpct =}\NormalTok{ totalshots}\OperatorTok{/}\NormalTok{totalattempts)}
\end{Highlighting}
\end{Shaded}

The last thing I need to do is join them together. So each team will have the conference shooting percentage as well as their own.

\begin{Shaded}
\begin{Highlighting}[]
\NormalTok{shooting <-}\StringTok{ }\NormalTok{teamshooting }\OperatorTok\StringTok{ }\KeywordTok{left_join}\NormalTok{(conferenceshooting, }\DataTypeTok{by=}\StringTok{"Conference"}\NormalTok{)}
\end{Highlighting}
\end{Shaded}

I have every team in college basketball, but that's insane.

\begin{Shaded}
\begin{Highlighting}[]
\NormalTok{big10 <-}\StringTok{ }\NormalTok{shooting }\OperatorTok\StringTok{ }\KeywordTok{filter}\NormalTok{(Conference }\OperatorTok{==}\StringTok{ "Big Ten"}\NormalTok{)}
\end{Highlighting}
\end{Shaded}

So this takes a little doing, but the logic is pretty clear in the end.

A lollipop chart is made up of two things -- a line between two points, and two points. So we need a geom\_segment and two geom\_points. And because they get layered starting at the bottom, our segment is first. A geom segment is made up of two things -- an x and a y value, and an x and y end. In this case, our x and xend are the same -- the Team -- and our y and yend are our two stats. For our points, both x values are the Team and the y is the different stats. What that does is put each point on the same line.

\begin{Shaded}
\begin{Highlighting}[]
\KeywordTok{ggplot}\NormalTok{(big10) }\OperatorTok{+}
\StringTok{  }\KeywordTok{geom_segment}\NormalTok{(}\KeywordTok{aes}\NormalTok{(}\DataTypeTok{x=}\NormalTok{Team, }\DataTypeTok{xend=}\NormalTok{Team, }\DataTypeTok{y=}\NormalTok{teamshootingpct, }\DataTypeTok{yend=}\NormalTok{conferenceshootingpct), }\DataTypeTok{color=}\StringTok{"grey"}\NormalTok{) }\OperatorTok{+}\StringTok{ }
\StringTok{  }\KeywordTok{geom_point}\NormalTok{(}\KeywordTok{aes}\NormalTok{(}\DataTypeTok{x=}\NormalTok{Team, }\DataTypeTok{y=}\NormalTok{teamshootingpct), }\DataTypeTok{color=}\StringTok{"red"}\NormalTok{) }\OperatorTok{+}\StringTok{ }
\StringTok{  }\KeywordTok{geom_point}\NormalTok{(}\KeywordTok{aes}\NormalTok{(}\DataTypeTok{x=}\NormalTok{Team, }\DataTypeTok{y=}\NormalTok{conferenceshootingpct), }\DataTypeTok{color=}\StringTok{"blue"}\NormalTok{) }\OperatorTok{+}
\StringTok{  }\KeywordTok{coord_flip}\NormalTok{()}
\end{Highlighting}
\end{Shaded}

\includegraphics{SportsData_files/figure-latex/unnamed-chunk-166-1.pdf}

We can do better by changing the order of the teams by their shooting performance and giving it some theme love.

\begin{Shaded}
\begin{Highlighting}[]
\KeywordTok{ggplot}\NormalTok{(big10) }\OperatorTok{+}
\StringTok{  }\KeywordTok{geom_segment}\NormalTok{(}\KeywordTok{aes}\NormalTok{(}\DataTypeTok{x=}\KeywordTok{reorder}\NormalTok{(Team, teamshootingpct), }\DataTypeTok{xend=}\NormalTok{Team, }\DataTypeTok{y=}\NormalTok{teamshootingpct, }\DataTypeTok{yend=}\NormalTok{conferenceshootingpct), }\DataTypeTok{color=}\StringTok{"grey"}\NormalTok{) }\OperatorTok{+}\StringTok{ }
\StringTok{  }\KeywordTok{geom_point}\NormalTok{(}\KeywordTok{aes}\NormalTok{(}\DataTypeTok{x=}\KeywordTok{reorder}\NormalTok{(Team, teamshootingpct), }\DataTypeTok{y=}\NormalTok{teamshootingpct), }\DataTypeTok{color=}\StringTok{"red"}\NormalTok{) }\OperatorTok{+}\StringTok{ }
\StringTok{  }\KeywordTok{geom_point}\NormalTok{(}\KeywordTok{aes}\NormalTok{(}\DataTypeTok{x=}\KeywordTok{reorder}\NormalTok{(Team, teamshootingpct), }\DataTypeTok{y=}\NormalTok{conferenceshootingpct), }\DataTypeTok{color=}\StringTok{"blue"}\NormalTok{) }\OperatorTok{+}
\StringTok{  }\KeywordTok{coord_flip}\NormalTok{() }\OperatorTok{+}
\StringTok{   }\KeywordTok{labs}\NormalTok{(}\DataTypeTok{x=}\StringTok{""}\NormalTok{, }\DataTypeTok{y=}\StringTok{"Shooting percentage vs league average"}\NormalTok{, }\DataTypeTok{title=}\StringTok{"Except Purdue, shooting predicted Big Ten success"}\NormalTok{, }\DataTypeTok{subtitle=}\StringTok{"The Boilermakers were average shooters, went deep in the NCAA tournament"}\NormalTok{, }\DataTypeTok{caption=}\StringTok{"Source: sports-reference.com | By Matt Waite"}\NormalTok{) }\OperatorTok{+}
\StringTok{  }\KeywordTok{theme_minimal}\NormalTok{() }\OperatorTok{+}\StringTok{ }
\StringTok{  }\KeywordTok{theme}\NormalTok{(}
    \DataTypeTok{plot.title =} \KeywordTok{element_text}\NormalTok{(}\DataTypeTok{size =} \DecValTok{16}\NormalTok{, }\DataTypeTok{face =} \StringTok{"bold"}\NormalTok{, }\DataTypeTok{hjust =} \DecValTok{1}\NormalTok{),}
    \DataTypeTok{plot.subtitle =} \KeywordTok{element_text}\NormalTok{(}\DataTypeTok{hjust =} \FloatTok{1.3}\NormalTok{),}
    \DataTypeTok{axis.title =} \KeywordTok{element_text}\NormalTok{(}\DataTypeTok{size =} \DecValTok{10}\NormalTok{),}
    \DataTypeTok{axis.title.y =} \KeywordTok{element_blank}\NormalTok{(),}
    \DataTypeTok{axis.text =} \KeywordTok{element_text}\NormalTok{(}\DataTypeTok{size =} \DecValTok{7}\NormalTok{),}
    \DataTypeTok{axis.ticks =} \KeywordTok{element_blank}\NormalTok{()}
\NormalTok{  )}
\end{Highlighting}
\end{Shaded}

\includegraphics{SportsData_files/figure-latex/unnamed-chunk-167-1.pdf}

What if we wanted to order them by wins? Our data has a column called W\_L that lists if the team won or lost. The problem is it doens't just say W or L. If the game went to overtime, it lists that. That complicates counting wins. Here's a trick to find a string of text and make that. It's called \texttt{grepl} and the basic syntax is grepl for this string in this field and then do what I tell you. In this case, we're going to create a new field called winloss look for W or L (and ignore any OT notation) and give wins a 1 and losses a 0.

\begin{Shaded}
\begin{Highlighting}[]
\NormalTok{winlosslogs <-}\StringTok{ }\NormalTok{logs }\OperatorTok\StringTok{ }\KeywordTok{mutate}\NormalTok{(}\DataTypeTok{winloss =} \KeywordTok{case_when}\NormalTok{(}
  \KeywordTok{grepl}\NormalTok{(}\StringTok{"W"}\NormalTok{, W_L) }\OperatorTok{~}\StringTok{ }\DecValTok{1}\NormalTok{, }
  \KeywordTok{grepl}\NormalTok{(}\StringTok{"L"}\NormalTok{, W_L) }\OperatorTok{~}\StringTok{ }\DecValTok{0}\NormalTok{)}
\NormalTok{)}
\end{Highlighting}
\end{Shaded}

So let's look at turnovers and assists. We'll call it give and take. Does the difference between those two things indicate something when we sort them by wins?

\begin{Shaded}
\begin{Highlighting}[]
\NormalTok{giveandtake <-}\StringTok{ }\NormalTok{winlosslogs }\OperatorTok\StringTok{ }\KeywordTok{group_by}\NormalTok{(Conference, Team) }\OperatorTok\StringTok{ }\KeywordTok{summarise}\NormalTok{(}\DataTypeTok{turnovers =} \KeywordTok{sum}\NormalTok{(TeamTurnovers), }\DataTypeTok{assists =} \KeywordTok{sum}\NormalTok{(TeamAssists), }\DataTypeTok{wins=}\KeywordTok{sum}\NormalTok{(winloss)) }
\end{Highlighting}
\end{Shaded}

\begin{Shaded}
\begin{Highlighting}[]
\NormalTok{big10gt <-}\StringTok{ }\NormalTok{giveandtake }\OperatorTok\StringTok{ }\KeywordTok{filter}\NormalTok{(Conference }\OperatorTok{==}\StringTok{ "Big Ten"}\NormalTok{)}
\end{Highlighting}
\end{Shaded}

\begin{Shaded}
\begin{Highlighting}[]
\KeywordTok{ggplot}\NormalTok{(big10gt) }\OperatorTok{+}
\StringTok{  }\KeywordTok{geom_segment}\NormalTok{(}\KeywordTok{aes}\NormalTok{(}\DataTypeTok{x=}\KeywordTok{reorder}\NormalTok{(Team, wins), }\DataTypeTok{xend=}\NormalTok{Team, }\DataTypeTok{y=}\NormalTok{turnovers, }\DataTypeTok{yend=}\NormalTok{assists), }\DataTypeTok{color=}\StringTok{"grey"}\NormalTok{) }\OperatorTok{+}\StringTok{ }
\StringTok{  }\KeywordTok{geom_point}\NormalTok{(}\KeywordTok{aes}\NormalTok{(}\DataTypeTok{x=}\KeywordTok{reorder}\NormalTok{(Team, wins), }\DataTypeTok{y=}\NormalTok{turnovers), }\DataTypeTok{color=}\StringTok{"red"}\NormalTok{) }\OperatorTok{+}\StringTok{ }
\StringTok{  }\KeywordTok{geom_point}\NormalTok{(}\KeywordTok{aes}\NormalTok{(}\DataTypeTok{x=}\KeywordTok{reorder}\NormalTok{(Team, wins), }\DataTypeTok{y=}\NormalTok{assists), }\DataTypeTok{color=}\StringTok{"blue"}\NormalTok{) }\OperatorTok{+}
\StringTok{  }\KeywordTok{coord_flip}\NormalTok{()}
\end{Highlighting}
\end{Shaded}

\includegraphics{SportsData_files/figure-latex/unnamed-chunk-171-1.pdf}

Short answer: Not really. Something you need to get used to in data visualization -- not everything works. Sometimes you find things, sometimes you don't. Don't publish a graphic that doesn't find anything. Let it stay in your notebook as an idea that didn't pan out.

\hypertarget{scatterplots}{%
\chapter{Scatterplots}\label{scatterplots}}

In several chapters of this book, we've been fixated on the Nebraska basketball team's shooting percentage, which took a nose dive during the season and ultimately doomed Tim Miles job. The question is \ldots{} does it matter?

This is what we're going to start to answer today. And we'll do it with scatterplots and correlations.

First, we need libraries and \href{https://unl.box.com/s/a8m91bro10t89watsyo13yjegb1fy009}{data}.

\begin{Shaded}
\begin{Highlighting}[]
\KeywordTok{library}\NormalTok{(tidyverse)}
\end{Highlighting}
\end{Shaded}

\begin{Shaded}
\begin{Highlighting}[]
\NormalTok{logs <-}\StringTok{ }\KeywordTok{read_csv}\NormalTok{(}\StringTok{"data/logs19.csv"}\NormalTok{)}
\end{Highlighting}
\end{Shaded}

\begin{verbatim}
## Warning: Missing column names filled in: 'X1' [1]
\end{verbatim}

\begin{verbatim}
## Parsed with column specification:
## cols(
##   .default = col_double(),
##   Date = col_date(format = ""),
##   HomeAway = col_character(),
##   Opponent = col_character(),
##   W_L = col_character(),
##   Blank = col_logical(),
##   Team = col_character(),
##   Conference = col_character(),
##   season = col_character()
## )
\end{verbatim}

\begin{verbatim}
## See spec(...) for full column specifications.
\end{verbatim}

To do this, we need all teams and their season stats. How much, over the course of a season, does a thing matter? That's the question you're going to answer.

In our case, we want to know how much does shooting percentage influence wins? How much difference can we explain in wins with shooting percentage? We're going to total up the number of wins each team has and their season shooting percentage in one swoop.

Let's borrow from our ridgecharts work to get the correct wins and losses totals for each team.

\begin{Shaded}
\begin{Highlighting}[]
\NormalTok{winlosslogs <-}\StringTok{ }\NormalTok{logs }\OperatorTok\StringTok{ }\KeywordTok{mutate}\NormalTok{(}\DataTypeTok{winloss =} \KeywordTok{case_when}\NormalTok{(}
  \KeywordTok{grepl}\NormalTok{(}\StringTok{"W"}\NormalTok{, W_L) }\OperatorTok{~}\StringTok{ }\DecValTok{1}\NormalTok{, }
  \KeywordTok{grepl}\NormalTok{(}\StringTok{"L"}\NormalTok{, W_L) }\OperatorTok{~}\StringTok{ }\DecValTok{0}\NormalTok{)}
\NormalTok{)}
\end{Highlighting}
\end{Shaded}

Now we can get a dataframe together that gives us the total wins for each team, and the total shots taken and made, which let's us calculate a season shooting percentage.

\begin{Shaded}
\begin{Highlighting}[]
\NormalTok{winlosslogs }\OperatorTok\StringTok{ }
\StringTok{  }\KeywordTok{group_by}\NormalTok{(Team) }\OperatorTok
\StringTok{  }\KeywordTok{summarise}\NormalTok{(}
    \DataTypeTok{wins =} \KeywordTok{sum}\NormalTok{(winloss),}
    \DataTypeTok{totalFGAttempts =} \KeywordTok{sum}\NormalTok{(TeamFGA),}
    \DataTypeTok{totalFG =} \KeywordTok{sum}\NormalTok{(TeamFG)}
\NormalTok{  ) }\OperatorTok
\StringTok{  }\KeywordTok{mutate}\NormalTok{(}\DataTypeTok{fgpct =}\NormalTok{ totalFG}\OperatorTok{/}\NormalTok{totalFGAttempts) ->}\StringTok{ }\NormalTok{fgmodel}
\end{Highlighting}
\end{Shaded}

Now let's look at the scatterplot. With a scatterplot, we put what predicts the thing on the X axis, and the thing being predicted on the Y axis. In this case, X is our shooting percentage, y is our wins.

\begin{Shaded}
\begin{Highlighting}[]
\KeywordTok{ggplot}\NormalTok{(fgmodel, }\KeywordTok{aes}\NormalTok{(}\DataTypeTok{x=}\NormalTok{fgpct, }\DataTypeTok{y=}\NormalTok{wins)) }\OperatorTok{+}\StringTok{ }\KeywordTok{geom_point}\NormalTok{()}
\end{Highlighting}
\end{Shaded}

\includegraphics{SportsData_files/figure-latex/unnamed-chunk-176-1.pdf}

Let's talk about this. It seems that the data slopes up to the right. That would indicate a positive correlation between shooting percentage and wins. And that makes sense, no? You'd expect teams that shoot the ball well to win. But can we get a better sense of this? Yes, by adding another geom -- \texttt{geom\_smooth}.

\begin{Shaded}
\begin{Highlighting}[]
\KeywordTok{ggplot}\NormalTok{(fgmodel, }\KeywordTok{aes}\NormalTok{(}\DataTypeTok{x=}\NormalTok{fgpct, }\DataTypeTok{y=}\NormalTok{wins)) }\OperatorTok{+}\StringTok{ }\KeywordTok{geom_point}\NormalTok{() }\OperatorTok{+}\StringTok{ }\KeywordTok{geom_smooth}\NormalTok{(}\DataTypeTok{method=}\NormalTok{lm, }\DataTypeTok{se=}\OtherTok{TRUE}\NormalTok{)}
\end{Highlighting}
\end{Shaded}

\includegraphics{SportsData_files/figure-latex/unnamed-chunk-177-1.pdf}

But \ldots{} how strong a relationship is this? How much can shooting percentage explain wins? Can we put some numbers to this?

Of course we can. We can apply a linear model to this -- remember Chapter 9? We're going to create an object called fit, and then we're going to put into that object a linear model -- \texttt{lm} -- and the way to read this is ``wins are predicted by field goal percentage''. Then we just want the summary of that model.

\begin{Shaded}
\begin{Highlighting}[]
\NormalTok{fit <-}\StringTok{ }\KeywordTok{lm}\NormalTok{(wins }\OperatorTok{~}\StringTok{ }\NormalTok{fgpct, }\DataTypeTok{data =}\NormalTok{ fgmodel)}
\KeywordTok{summary}\NormalTok{(fit)}
\end{Highlighting}
\end{Shaded}

\begin{verbatim}
## 
## Call:
## lm(formula = wins ~ fgpct, data = fgmodel)
## 
## Residuals:
##      Min       1Q   Median       3Q      Max 
## -14.3536  -3.4523  -0.1125   3.3834  15.4318 
## 
## Coefficients:
##             Estimate Std. Error t value Pr(>|t|)    
## (Intercept)  -49.217      4.845  -10.16   <2e-16 ***
## fgpct        149.416     10.915   13.69   <2e-16 ***
## ---
## Signif. codes:  0 '***' 0.001 '**' 0.01 '*' 0.05 '.' 0.1 ' ' 1
## 
## Residual standard error: 5.035 on 351 degrees of freedom
## Multiple R-squared:  0.348,  Adjusted R-squared:  0.3462 
## F-statistic: 187.4 on 1 and 351 DF,  p-value: < 2.2e-16
\end{verbatim}

Remember from Chapter 9: There's just a few things you really need.

The first thing: R-squared. In this case, the Adjusted R-squared value is .3462, which we can interpret as shooting percentage predicts about 35 percent of the variance in wins. Which sounds not great, but in social science, that's huge. That's great. A psychology major would murder for that R-squared.

Second: The P-value. We want anything less than .05. If it's above .05, the change between them is not statistically significant -- it's probably explained by random chance. In our case, we have 2.2e-16, which is to say 2.2 with 16 zeros in front of it, or .000000000000000022. Is that less than .05? Yes. Yes it is. So this is not random. Again, we would expect this, so it's a good logic test.

Third: The coefficient. In this case, the coefficient for fgpct is 149.416. Since I didn't convert percentages to decimals, what this says is that for every percentage point improvement in shooting percentage, we can expect the team to win 1.49 more games plus or minus some error.

And we can use this to predict a team's wins: remember your algebra and y = mx + b. In this case, y is the wins, m is the coefficient, x is the shooting percentage and b is the intercept.

So can plug these together: Expected wins = 149.416 * shooting percentage - 49.217

Let's use Nebraska as an example. They shot about .43 on the season (.4294421 to be exact).

y = 149.416 * .4294421 - 49.217 or 14.95 wins. How many wins did Nebraska have? 19.

What does that mean? It means that as disappointing a season as it was, Nebraska actually OVERPERFORMED it's season shooting percentage. They shouldn't have won as many games as they did, according to our model.

\hypertarget{facet-wraps}{%
\chapter{Facet wraps}\label{facet-wraps}}

Sometimes the easiest way to spot a trend is to chart a bunch of small things side by side. Edward Tufte, one of the most well known data visualization thinkers on the planet, calls this ``small multiples'' where ggplot calls this a facet wrap or a facet grid, depending.

One thing we noticed earlier in the semester -- it seems that a lot of teams shoot worse as the season goes on. Do they? We could answer this a number of ways, but the best way to show people would be visually. Let's use Small Multiples.

As always, we start with libraries.

\begin{Shaded}
\begin{Highlighting}[]
\KeywordTok{library}\NormalTok{(tidyverse)}
\end{Highlighting}
\end{Shaded}

Now data.

\begin{Shaded}
\begin{Highlighting}[]
\NormalTok{logs <-}\StringTok{ }\KeywordTok{read_csv}\NormalTok{(}\StringTok{"data/logs19.csv"}\NormalTok{)}
\end{Highlighting}
\end{Shaded}

\begin{verbatim}
## Warning: Missing column names filled in: 'X1' [1]
\end{verbatim}

\begin{verbatim}
## Parsed with column specification:
## cols(
##   .default = col_double(),
##   Date = col_date(format = ""),
##   HomeAway = col_character(),
##   Opponent = col_character(),
##   W_L = col_character(),
##   Blank = col_logical(),
##   Team = col_character(),
##   Conference = col_character(),
##   season = col_character()
## )
\end{verbatim}

\begin{verbatim}
## See spec(...) for full column specifications.
\end{verbatim}

Let's narrow our pile and look just at the Big Ten.

\begin{Shaded}
\begin{Highlighting}[]
\NormalTok{big10 <-}\StringTok{ }\NormalTok{logs }\OperatorTok\StringTok{ }\KeywordTok{filter}\NormalTok{(Conference }\OperatorTok{==}\StringTok{ "Big Ten"}\NormalTok{)}
\end{Highlighting}
\end{Shaded}

The first thing we can do is look at a line chart, like we have done in previous chapters.

\begin{Shaded}
\begin{Highlighting}[]
\KeywordTok{ggplot}\NormalTok{() }\OperatorTok{+}\StringTok{ }\KeywordTok{geom_line}\NormalTok{(}\DataTypeTok{data=}\NormalTok{big10, }\KeywordTok{aes}\NormalTok{(}\DataTypeTok{x=}\NormalTok{Date, }\DataTypeTok{y=}\NormalTok{TeamFGPCT, }\DataTypeTok{group=}\NormalTok{Team)) }\OperatorTok{+}\StringTok{ }\KeywordTok{scale_y_continuous}\NormalTok{(}\DataTypeTok{limits =} \KeywordTok{c}\NormalTok{(}\DecValTok{0}\NormalTok{, }\FloatTok{.7}\NormalTok{))}
\end{Highlighting}
\end{Shaded}

\includegraphics{SportsData_files/figure-latex/unnamed-chunk-182-1.pdf}

And, not surprisingly, we get a hairball. We could color certain lines, but that would limit us to focus on one team. What if we did all of them at once? We do that with a \texttt{facet\_wrap}. The only thing we MUST pass into a \texttt{facet\_wrap} is what thing we're going to separate them out by. In this case, we precede that field with a tilde, so in our case we want the Team field. It looks like this:

\begin{Shaded}
\begin{Highlighting}[]
\KeywordTok{ggplot}\NormalTok{() }\OperatorTok{+}\StringTok{ }\KeywordTok{geom_line}\NormalTok{(}\DataTypeTok{data=}\NormalTok{big10, }\KeywordTok{aes}\NormalTok{(}\DataTypeTok{x=}\NormalTok{Date, }\DataTypeTok{y=}\NormalTok{TeamFGPCT, }\DataTypeTok{group=}\NormalTok{Team)) }\OperatorTok{+}\StringTok{ }\KeywordTok{scale_y_continuous}\NormalTok{(}\DataTypeTok{limits =} \KeywordTok{c}\NormalTok{(}\DecValTok{0}\NormalTok{, }\FloatTok{.7}\NormalTok{)) }\OperatorTok{+}\StringTok{ }\KeywordTok{facet_wrap}\NormalTok{(}\OperatorTok{~}\NormalTok{Team)}
\end{Highlighting}
\end{Shaded}

\includegraphics{SportsData_files/figure-latex/unnamed-chunk-183-1.pdf}

Answer: Not immediately clear, but we can look at this and analyze it. We could add a piece of annotation to help us out.

\begin{Shaded}
\begin{Highlighting}[]
\NormalTok{big10 }\OperatorTok\StringTok{ }\KeywordTok{summarise}\NormalTok{(}\KeywordTok{mean}\NormalTok{(TeamFGPCT))}
\end{Highlighting}
\end{Shaded}

\begin{verbatim}
## # A tibble: 1 x 1
##   `mean(TeamFGPCT)`
##               <dbl>
## 1             0.442
\end{verbatim}

\begin{Shaded}
\begin{Highlighting}[]
\KeywordTok{ggplot}\NormalTok{() }\OperatorTok{+}\StringTok{ }\KeywordTok{geom_hline}\NormalTok{(}\DataTypeTok{yintercept=}\NormalTok{.}\DecValTok{4447}\NormalTok{, }\DataTypeTok{color=}\StringTok{"blue"}\NormalTok{) }\OperatorTok{+}\StringTok{ }\KeywordTok{geom_line}\NormalTok{(}\DataTypeTok{data=}\NormalTok{big10, }\KeywordTok{aes}\NormalTok{(}\DataTypeTok{x=}\NormalTok{Date, }\DataTypeTok{y=}\NormalTok{TeamFGPCT, }\DataTypeTok{group=}\NormalTok{Team)) }\OperatorTok{+}\StringTok{ }\KeywordTok{scale_y_continuous}\NormalTok{(}\DataTypeTok{limits =} \KeywordTok{c}\NormalTok{(}\DecValTok{0}\NormalTok{, }\FloatTok{.7}\NormalTok{)) }\OperatorTok{+}\StringTok{ }\KeywordTok{facet_wrap}\NormalTok{(}\OperatorTok{~}\NormalTok{Team)}
\end{Highlighting}
\end{Shaded}

\includegraphics{SportsData_files/figure-latex/unnamed-chunk-185-1.pdf}

What do you see here? How do teams compare? How do they change over time? I'm not asking you these questions because they're an assignment -- I'm asking because that's exactly what this chart helps answer. Your brain will immediately start making those connections.

\hypertarget{facet-grid-vs-facet-wraps}{%
\section{Facet grid vs facet wraps}\label{facet-grid-vs-facet-wraps}}

Facet grids allow us to put teams on the same plane, versus just repeating them. And we can specify that plane as vertical or horizontal. For example, here's our chart from above, but using facet\_grid to stack them.

\begin{Shaded}
\begin{Highlighting}[]
\KeywordTok{ggplot}\NormalTok{() }\OperatorTok{+}\StringTok{ }\KeywordTok{geom_hline}\NormalTok{(}\DataTypeTok{yintercept=}\NormalTok{.}\DecValTok{4447}\NormalTok{, }\DataTypeTok{color=}\StringTok{"blue"}\NormalTok{) }\OperatorTok{+}\StringTok{ }\KeywordTok{geom_line}\NormalTok{(}\DataTypeTok{data=}\NormalTok{big10, }\KeywordTok{aes}\NormalTok{(}\DataTypeTok{x=}\NormalTok{Date, }\DataTypeTok{y=}\NormalTok{TeamFGPCT, }\DataTypeTok{group=}\NormalTok{Team)) }\OperatorTok{+}\StringTok{ }\KeywordTok{scale_y_continuous}\NormalTok{(}\DataTypeTok{limits =} \KeywordTok{c}\NormalTok{(}\DecValTok{0}\NormalTok{, }\FloatTok{.7}\NormalTok{)) }\OperatorTok{+}\StringTok{ }\KeywordTok{facet_grid}\NormalTok{(Team }\OperatorTok{~}\StringTok{ }\NormalTok{.)}
\end{Highlighting}
\end{Shaded}

\includegraphics{SportsData_files/figure-latex/unnamed-chunk-186-1.pdf}

And here they are next to each other:

\begin{Shaded}
\begin{Highlighting}[]
\KeywordTok{ggplot}\NormalTok{() }\OperatorTok{+}\StringTok{ }\KeywordTok{geom_hline}\NormalTok{(}\DataTypeTok{yintercept=}\NormalTok{.}\DecValTok{4447}\NormalTok{, }\DataTypeTok{color=}\StringTok{"blue"}\NormalTok{) }\OperatorTok{+}\StringTok{ }\KeywordTok{geom_line}\NormalTok{(}\DataTypeTok{data=}\NormalTok{big10, }\KeywordTok{aes}\NormalTok{(}\DataTypeTok{x=}\NormalTok{Date, }\DataTypeTok{y=}\NormalTok{TeamFGPCT, }\DataTypeTok{group=}\NormalTok{Team)) }\OperatorTok{+}\StringTok{ }\KeywordTok{scale_y_continuous}\NormalTok{(}\DataTypeTok{limits =} \KeywordTok{c}\NormalTok{(}\DecValTok{0}\NormalTok{, }\FloatTok{.7}\NormalTok{)) }\OperatorTok{+}\StringTok{ }\KeywordTok{facet_grid}\NormalTok{(. }\OperatorTok{~}\StringTok{ }\NormalTok{Team)}
\end{Highlighting}
\end{Shaded}

\includegraphics{SportsData_files/figure-latex/unnamed-chunk-187-1.pdf}

Note: We'd have some work to do with the labeling on this -- we'll get to that -- but you can see where this is valuable comparing a group of things. One warning: Don't go too crazy with this or it loses it's visual power.

\hypertarget{other-types}{%
\section{Other types}\label{other-types}}

Line charts aren't the only things we can do. We can do any kind of chart in ggplot. Staying with shooting, where are team's winning and losing performances coming from when we talk about team shooting and opponent shooting?

\begin{Shaded}
\begin{Highlighting}[]
\KeywordTok{ggplot}\NormalTok{() }\OperatorTok{+}\StringTok{ }\KeywordTok{geom_point}\NormalTok{(}\DataTypeTok{data=}\NormalTok{big10, }\KeywordTok{aes}\NormalTok{(}\DataTypeTok{x=}\NormalTok{TeamFGPCT, }\DataTypeTok{y=}\NormalTok{OpponentFGPCT, }\DataTypeTok{color=}\NormalTok{W_L)) }\OperatorTok{+}\StringTok{ }\KeywordTok{scale_y_continuous}\NormalTok{(}\DataTypeTok{limits =} \KeywordTok{c}\NormalTok{(}\DecValTok{0}\NormalTok{, }\FloatTok{.7}\NormalTok{)) }\OperatorTok{+}\StringTok{ }\KeywordTok{scale_x_continuous}\NormalTok{(}\DataTypeTok{limits =} \KeywordTok{c}\NormalTok{(}\DecValTok{0}\NormalTok{, }\FloatTok{.7}\NormalTok{)) }\OperatorTok{+}\StringTok{ }\KeywordTok{facet_wrap}\NormalTok{(}\OperatorTok{~}\NormalTok{Team)}
\end{Highlighting}
\end{Shaded}

\includegraphics{SportsData_files/figure-latex/unnamed-chunk-188-1.pdf}

\hypertarget{tables}{%
\chapter{Tables}\label{tables}}

But not a table. A table with features.

Sometimes, the best way to show your data is with a table -- simple rows and columns. It allows a reader to compare whatever they want to compare a little easier than a graph where you've chosen what to highlight. R has a neat package called \texttt{formattable} and you'll install it like anything else with \texttt{install.packages(\textquotesingle{}formattable\textquotesingle{})}.

So what does it do? Let's gather our libraries and \href{https://unl.box.com/s/g3eeuogx8bog72ig28enuakhpdlbn394}{get some data}.

\begin{Shaded}
\begin{Highlighting}[]
\KeywordTok{library}\NormalTok{(tidyverse)}
\KeywordTok{library}\NormalTok{(formattable)}
\end{Highlighting}
\end{Shaded}

\begin{Shaded}
\begin{Highlighting}[]
\NormalTok{offense <-}\StringTok{ }\KeywordTok{read_csv}\NormalTok{(}\StringTok{"data/offensechange.csv"}\NormalTok{)}
\end{Highlighting}
\end{Shaded}

\begin{verbatim}
## Parsed with column specification:
## cols(
##   Year = col_double(),
##   Name = col_character(),
##   G = col_double(),
##   `Rush Yards` = col_double(),
##   `Pass Yards` = col_double(),
##   Plays = col_double(),
##   `Total Yards` = col_double(),
##   `Yards/Play` = col_double(),
##   `Yards/G` = col_double()
## )
\end{verbatim}

Let's ask this question: Which college football team saw the greatest improvement in yards per game last regular season? The simplest way to calculate that is by percent change.

\begin{Shaded}
\begin{Highlighting}[]
\NormalTok{changeTotalOffense <-}\StringTok{ }\NormalTok{offense }\OperatorTok
\StringTok{  }\KeywordTok{select}\NormalTok{(Name, Year, }\StringTok{`}\DataTypeTok{Yards/G}\StringTok{`}\NormalTok{) }\OperatorTok\StringTok{ }
\StringTok{  }\KeywordTok{spread}\NormalTok{(Year, }\StringTok{`}\DataTypeTok{Yards/G}\StringTok{`}\NormalTok{) }\OperatorTok\StringTok{ }
\StringTok{  }\KeywordTok{mutate}\NormalTok{(}\DataTypeTok{Change=}\NormalTok{(}\StringTok{`}\DataTypeTok{2019}\StringTok{`}\OperatorTok{-}\StringTok{`}\DataTypeTok{2018}\StringTok{`}\NormalTok{)}\OperatorTok{/}\StringTok{`}\DataTypeTok{2018}\StringTok{`}\NormalTok{) }\OperatorTok\StringTok{ }
\StringTok{  }\KeywordTok{arrange}\NormalTok{(}\KeywordTok{desc}\NormalTok{(Change)) }\OperatorTok\StringTok{ }
\StringTok{  }\KeywordTok{top_n}\NormalTok{(}\DecValTok{20}\NormalTok{)}
\end{Highlighting}
\end{Shaded}

\begin{verbatim}
## Selecting by Change
\end{verbatim}

We've output tables to the screen a thousand times in this class with \texttt{head}, but formattable makes them look good with very little code.

\begin{Shaded}
\begin{Highlighting}[]
\KeywordTok{formattable}\NormalTok{(changeTotalOffense)}
\end{Highlighting}
\end{Shaded}

Name

2018

2019

Change

Central Michigan

254.7

433.6

0.7023950

LSU

402.1

564.1

0.4028849

UTSA

247.1

344.9

0.3957912

San Jose State

323.7

427.4

0.3203584

Navy

349.3

455.8

0.3048955

Louisville

352.6

447.3

0.2685763

SMU

387.2

489.8

0.2649793

BYU

364.9

443.8

0.2162236

New Mexico

330.0

400.3

0.2130303

Charlotte

343.1

411.8

0.2002332

Iowa State

371.0

444.3

0.1975741

USC

382.6

454.0

0.1866179

Troy

389.4

456.3

0.1718028

Louisiana-Lafayette

424.3

494.1

0.1645062

Georgia State

378.2

439.8

0.1628768

Louisiana Tech

379.3

436.8

0.1515950

Minnesota

379.6

432.0

0.1380400

Ball State

408.6

463.0

0.1331375

Texas

411.6

465.8

0.1316812

Florida State

361.2

408.3

0.1303987

So there you have it. Central Michigan improved the most (but look at who came in second!). First thing I don't like about formattable tables -- the right alignment. Let's fix that.

\begin{Shaded}
\begin{Highlighting}[]
\KeywordTok{formattable}\NormalTok{(changeTotalOffense, }\DataTypeTok{align=}\StringTok{"l"}\NormalTok{)}
\end{Highlighting}
\end{Shaded}

Name

2018

2019

Change

Central Michigan

254.7

433.6

0.7023950

LSU

402.1

564.1

0.4028849

UTSA

247.1

344.9

0.3957912

San Jose State

323.7

427.4

0.3203584

Navy

349.3

455.8

0.3048955

Louisville

352.6

447.3

0.2685763

SMU

387.2

489.8

0.2649793

BYU

364.9

443.8

0.2162236

New Mexico

330.0

400.3

0.2130303

Charlotte

343.1

411.8

0.2002332

Iowa State

371.0

444.3

0.1975741

USC

382.6

454.0

0.1866179

Troy

389.4

456.3

0.1718028

Louisiana-Lafayette

424.3

494.1

0.1645062

Georgia State

378.2

439.8

0.1628768

Louisiana Tech

379.3

436.8

0.1515950

Minnesota

379.6

432.0

0.1380400

Ball State

408.6

463.0

0.1331375

Texas

411.6

465.8

0.1316812

Florida State

361.2

408.3

0.1303987

Next? I forgot to multiply by 100. No matter. Formattable can fix that for us.

\begin{Shaded}
\begin{Highlighting}[]
\KeywordTok{formattable}\NormalTok{(}
\NormalTok{  changeTotalOffense, }
  \DataTypeTok{align=}\StringTok{"l"}\NormalTok{,}
  \KeywordTok{list}\NormalTok{(}
    \StringTok{`}\DataTypeTok{Change}\StringTok{`}\NormalTok{ =}\StringTok{ }\NormalTok{percent)}
\NormalTok{  )}
\end{Highlighting}
\end{Shaded}

Name

2018

2019

Change

Central Michigan

254.7

433.6

70.24\%

LSU

402.1

564.1

40.29\%

UTSA

247.1

344.9

39.58\%

San Jose State

323.7

427.4

32.04\%

Navy

349.3

455.8

30.49\%

Louisville

352.6

447.3

26.86\%

SMU

387.2

489.8

26.50\%

BYU

364.9

443.8

21.62\%

New Mexico

330.0

400.3

21.30\%

Charlotte

343.1

411.8

20.02\%

Iowa State

371.0

444.3

19.76\%

USC

382.6

454.0

18.66\%

Troy

389.4

456.3

17.18\%

Louisiana-Lafayette

424.3

494.1

16.45\%

Georgia State

378.2

439.8

16.29\%

Louisiana Tech

379.3

436.8

15.16\%

Minnesota

379.6

432.0

13.80\%

Ball State

408.6

463.0

13.31\%

Texas

411.6

465.8

13.17\%

Florida State

361.2

408.3

13.04\%

Something else not great? I can't really see the magnitude of the 2019 column. A team could improve a lot, but still not gain that many yards. Formattable has embeddable bar charts in the table. They look like this.

\begin{Shaded}
\begin{Highlighting}[]
\KeywordTok{formattable}\NormalTok{(}
\NormalTok{  changeTotalOffense, }
  \DataTypeTok{align=}\StringTok{"l"}\NormalTok{,}
  \KeywordTok{list}\NormalTok{(}
    \StringTok{`}\DataTypeTok{2019}\StringTok{`}\NormalTok{ =}\StringTok{ }\KeywordTok{color_bar}\NormalTok{(}\StringTok{"#FA614B"}\NormalTok{), }
    \StringTok{`}\DataTypeTok{Change}\StringTok{`}\NormalTok{ =}\StringTok{ }\NormalTok{percent)}
\NormalTok{  )}
\end{Highlighting}
\end{Shaded}

Name

2018

2019

Change

Central Michigan

254.7

{433.6}

70.24\%

LSU

402.1

{564.1}

40.29\%

UTSA

247.1

{344.9}

39.58\%

San Jose State

323.7

{427.4}

32.04\%

Navy

349.3

{455.8}

30.49\%

Louisville

352.6

{447.3}

26.86\%

SMU

387.2

{489.8}

26.50\%

BYU

364.9

{443.8}

21.62\%

New Mexico

330.0

{400.3}

21.30\%

Charlotte

343.1

{411.8}

20.02\%

Iowa State

371.0

{444.3}

19.76\%

USC

382.6

{454.0}

18.66\%

Troy

389.4

{456.3}

17.18\%

Louisiana-Lafayette

424.3

{494.1}

16.45\%

Georgia State

378.2

{439.8}

16.29\%

Louisiana Tech

379.3

{436.8}

15.16\%

Minnesota

379.6

{432.0}

13.80\%

Ball State

408.6

{463.0}

13.31\%

Texas

411.6

{465.8}

13.17\%

Florida State

361.2

{408.3}

13.04\%

That gives me some more to mess with.

One thing you can do is set the bar widths to make them relative to each other, using a different function called a \texttt{normalize\_bar}.

\begin{Shaded}
\begin{Highlighting}[]
\KeywordTok{formattable}\NormalTok{(}
\NormalTok{  changeTotalOffense, }
  \DataTypeTok{align=}\StringTok{"r"}\NormalTok{,}
  \KeywordTok{list}\NormalTok{(}
    \StringTok{`}\DataTypeTok{2019}\StringTok{`}\NormalTok{ =}\StringTok{ }\KeywordTok{normalize_bar}\NormalTok{(}\StringTok{"#FA614B"}\NormalTok{), }
    \StringTok{`}\DataTypeTok{2018}\StringTok{`}\NormalTok{ =}\StringTok{ }\KeywordTok{normalize_bar}\NormalTok{(}\StringTok{"#FA614B"}\NormalTok{), }
    \StringTok{`}\DataTypeTok{Change}\StringTok{`}\NormalTok{ =}\StringTok{ }\NormalTok{percent)}
\NormalTok{  )}
\end{Highlighting}
\end{Shaded}

Name

2018

2019

Change

Central Michigan

{254.7}

{433.6}

70.24\%

LSU

{402.1}

{564.1}

40.29\%

UTSA

{247.1}

{344.9}

39.58\%

San Jose State

{323.7}

{427.4}

32.04\%

Navy

{349.3}

{455.8}

30.49\%

Louisville

{352.6}

{447.3}

26.86\%

SMU

{387.2}

{489.8}

26.50\%

BYU

{364.9}

{443.8}

21.62\%

New Mexico

{330.0}

{400.3}

21.30\%

Charlotte

{343.1}

{411.8}

20.02\%

Iowa State

{371.0}

{444.3}

19.76\%

USC

{382.6}

{454.0}

18.66\%

Troy

{389.4}

{456.3}

17.18\%

Louisiana-Lafayette

{424.3}

{494.1}

16.45\%

Georgia State

{378.2}

{439.8}

16.29\%

Louisiana Tech

{379.3}

{436.8}

15.16\%

Minnesota

{379.6}

{432.0}

13.80\%

Ball State

{408.6}

{463.0}

13.31\%

Texas

{411.6}

{465.8}

13.17\%

Florida State

{361.2}

{408.3}

13.04\%

Note: bookdown is formatting this weird. Your numbers won't look like this.

Another way to deal with this -- color tiles. Change the rectangle that houses the data to a color indicating the intensity of it.

\begin{Shaded}
\begin{Highlighting}[]
\KeywordTok{formattable}\NormalTok{(}
\NormalTok{  changeTotalOffense, }
  \DataTypeTok{align=}\StringTok{"r"}\NormalTok{,}
  \KeywordTok{list}\NormalTok{(}
     \KeywordTok{area}\NormalTok{(}\DataTypeTok{col =} \DecValTok{2}\OperatorTok{:}\DecValTok{3}\NormalTok{) }\OperatorTok{~}\StringTok{ }\KeywordTok{color_tile}\NormalTok{(}\StringTok{"#FFF6F4"}\NormalTok{, }\StringTok{"#FA614B"}\NormalTok{),}
    \StringTok{`}\DataTypeTok{Change}\StringTok{`}\NormalTok{ =}\StringTok{ }\NormalTok{percent)}
\NormalTok{  )}
\end{Highlighting}
\end{Shaded}

Name

2018

2019

Change

Central Michigan

{254.7}

{433.6}

70.24\%

LSU

{402.1}

{564.1}

40.29\%

UTSA

{247.1}

{344.9}

39.58\%

San Jose State

{323.7}

{427.4}

32.04\%

Navy

{349.3}

{455.8}

30.49\%

Louisville

{352.6}

{447.3}

26.86\%

SMU

{387.2}

{489.8}

26.50\%

BYU

{364.9}

{443.8}

21.62\%

New Mexico

{330.0}

{400.3}

21.30\%

Charlotte

{343.1}

{411.8}

20.02\%

Iowa State

{371.0}

{444.3}

19.76\%

USC

{382.6}

{454.0}

18.66\%

Troy

{389.4}

{456.3}

17.18\%

Louisiana-Lafayette

{424.3}

{494.1}

16.45\%

Georgia State

{378.2}

{439.8}

16.29\%

Louisiana Tech

{379.3}

{436.8}

15.16\%

Minnesota

{379.6}

{432.0}

13.80\%

Ball State

{408.6}

{463.0}

13.31\%

Texas

{411.6}

{465.8}

13.17\%

Florida State

{361.2}

{408.3}

13.04\%

\hypertarget{exporting-tables}{%
\subsection{Exporting tables}\label{exporting-tables}}

The first thing you need to do is install some libraries -- do this in the console, not in an R Studio code block because htmltools get's a little weird.

\begin{verbatim}
install.packages("htmltools")
install.packages("webshot")

webshot::install_phantomjs()
\end{verbatim}

Now, copy, paste and run this code block entirely. Don't change anything.

\begin{Shaded}
\begin{Highlighting}[]
\KeywordTok{library}\NormalTok{(}\StringTok{"htmltools"}\NormalTok{)}
\KeywordTok{library}\NormalTok{(}\StringTok{"webshot"}\NormalTok{)    }

\NormalTok{export_formattable <-}\StringTok{ }\ControlFlowTok{function}\NormalTok{(f, file, }\DataTypeTok{width =} \StringTok{"100%"}\NormalTok{, }\DataTypeTok{height =} \OtherTok{NULL}\NormalTok{, }
                               \DataTypeTok{background =} \StringTok{"white"}\NormalTok{, }\DataTypeTok{delay =} \FloatTok{0.2}\NormalTok{)}
\NormalTok{    \{}
\NormalTok{      w <-}\StringTok{ }\KeywordTok{as.htmlwidget}\NormalTok{(f, }\DataTypeTok{width =}\NormalTok{ width, }\DataTypeTok{height =}\NormalTok{ height)}
\NormalTok{      path <-}\StringTok{ }\KeywordTok{html_print}\NormalTok{(w, }\DataTypeTok{background =}\NormalTok{ background, }\DataTypeTok{viewer =} \OtherTok{NULL}\NormalTok{)}
\NormalTok{      url <-}\StringTok{ }\KeywordTok{paste0}\NormalTok{(}\StringTok{"file:///"}\NormalTok{, }\KeywordTok{gsub}\NormalTok{(}\StringTok{"}\CharTok{\textbackslash{}\textbackslash{}\textbackslash{}\textbackslash{}}\StringTok{"}\NormalTok{, }\StringTok{"/"}\NormalTok{, }\KeywordTok{normalizePath}\NormalTok{(path)))}
      \KeywordTok{webshot}\NormalTok{(url,}
              \DataTypeTok{file =}\NormalTok{ file,}
              \DataTypeTok{selector =} \StringTok{".formattable_widget"}\NormalTok{,}
              \DataTypeTok{delay =}\NormalTok{ delay)}
\NormalTok{    \}}
\end{Highlighting}
\end{Shaded}

Now, save your formattable table to an object using the \texttt{\textless{}-} assignment operator.

After you've done that, you can call the function you ran in the previous block to export as a png file. In my case, I created an object called table, which is populated with my formattable table. Then, in export\_formattable, I pass in that \texttt{table} object and give it a name.

\begin{Shaded}
\begin{Highlighting}[]
\NormalTok{table <-}\StringTok{ }\KeywordTok{formattable}\NormalTok{(}
\NormalTok{  changeTotalOffense, }
  \DataTypeTok{align=}\StringTok{"r"}\NormalTok{,}
  \KeywordTok{list}\NormalTok{(}
     \KeywordTok{area}\NormalTok{(}\DataTypeTok{col =} \DecValTok{2}\OperatorTok{:}\DecValTok{3}\NormalTok{) }\OperatorTok{~}\StringTok{ }\KeywordTok{color_tile}\NormalTok{(}\StringTok{"#FFF6F4"}\NormalTok{, }\StringTok{"#FA614B"}\NormalTok{),}
    \StringTok{`}\DataTypeTok{Change}\StringTok{`}\NormalTok{ =}\StringTok{ }\NormalTok{percent)}
\NormalTok{  )}

\KeywordTok{export_formattable}\NormalTok{(table,}\StringTok{"table.png"}\NormalTok{)}
\end{Highlighting}
\end{Shaded}

\includegraphics{SportsData_files/figure-latex/unnamed-chunk-199-1.png}

For now, pngs are what you need to export. There is a way to export PDFs, but they lose all the formatting when you do that, which is kind of pointless.

\hypertarget{bubble-charts}{%
\chapter{Bubble charts}\label{bubble-charts}}

Here is the real talk: Bubble charts are hard. The reason they are hard is not because of the code, or the complexity or anything like that. They're a scatterplot with magnitude added -- the size of the dot in the scatterplot has meaning. The hard part is seeing when a bubble chart works and when it doesn't.

If you want to see it work spectacularly well, \href{https://www.youtube.com/watch?v=hVimVzgtD6w}{watch a semi-famous Ted Talk} by Hans Rosling from 2006 where bubble charts were the centerpiece. It's worth watching. It'll change your perspective on the world. No seriously. It will.

And since then, people have wanted bubble charts. And we're back to the original problem: They're hard. There's a finite set of circumstances where they work.

First, I'm going to show you an example of them not working to illustrate the point.

I'm going to load up my libraries: tidyverse per usual, rvest to get some data (we'll discuss rvest in greater detail in an upcoming chapter) and ggrepel because I end up using it every time I do a scatterplot.

\begin{Shaded}
\begin{Highlighting}[]
\KeywordTok{library}\NormalTok{(tidyverse)}
\KeywordTok{library}\NormalTok{(rvest)}
\KeywordTok{library}\NormalTok{(ggrepel)}
\end{Highlighting}
\end{Shaded}

So for this example, I want to look at Nebraska's offense in the 2019 season. It \ldots{} hasn't gone well. And typical of Nebraska teams for the last decade, they're turning the ball over a lot. So given the number of turnovers, how does Nebraska compare to other teams in the FBS?

I'm going to create a scatterplot yards per game on the X axis and points per game on the Y. They're pretty highly correlated with each other. And then I'm going to make the dot the size of the turnovers -- the bubble in my bubble charts.

Using Rvest, I'm going to grab total offense rankings, scoring offense rankings and turnover rankings and then merge them together with just the fields I need. This will all get explained more thoroughly coming up, but that's what this block of code does. When it's done, I'll have a dataframe called \texttt{offense} which I'll use to build my bubble chart.

\begin{Shaded}
\begin{Highlighting}[]
\NormalTok{yardsurl <-}\StringTok{ "http://cfbstats.com/2019/leader/national/team/offense/split01/category10/sort01.html"}

\NormalTok{yards19 <-}\StringTok{ }\NormalTok{yardsurl }\OperatorTok
\StringTok{  }\KeywordTok{read_html}\NormalTok{() }\OperatorTok
\StringTok{  }\KeywordTok{html_nodes}\NormalTok{(}\DataTypeTok{xpath =} \StringTok{'//*[@id="content"]/div[2]/table'}\NormalTok{) }\OperatorTok
\StringTok{  }\KeywordTok{html_table}\NormalTok{()}

\NormalTok{yards19 <-}\StringTok{ }\NormalTok{yards19[[}\DecValTok{1}\NormalTok{]] }\OperatorTok\StringTok{ }\KeywordTok{select}\NormalTok{(Name, }\StringTok{`}\DataTypeTok{Yards/G}\StringTok{`}\NormalTok{)}

\NormalTok{pointsurl <-}\StringTok{ "http://cfbstats.com/2019/leader/national/team/offense/split01/category09/sort01.html"}

\NormalTok{points19 <-}\StringTok{ }\NormalTok{pointsurl }\OperatorTok
\StringTok{  }\KeywordTok{read_html}\NormalTok{() }\OperatorTok
\StringTok{  }\KeywordTok{html_nodes}\NormalTok{(}\DataTypeTok{xpath =} \StringTok{'//*[@id="content"]/div[2]/table'}\NormalTok{) }\OperatorTok
\StringTok{  }\KeywordTok{html_table}\NormalTok{()}

\NormalTok{points19 <-}\StringTok{ }\NormalTok{points19[[}\DecValTok{1}\NormalTok{]] }\OperatorTok\StringTok{ }\KeywordTok{select}\NormalTok{(Name, }\StringTok{`}\DataTypeTok{Points/G}\StringTok{`}\NormalTok{)}

\NormalTok{turnoversurl <-}\StringTok{ "http://cfbstats.com/2019/leader/national/team/offense/split01/category12/sort01.html"}

\NormalTok{turnovers19 <-}\StringTok{ }\NormalTok{turnoversurl }\OperatorTok
\StringTok{  }\KeywordTok{read_html}\NormalTok{() }\OperatorTok
\StringTok{  }\KeywordTok{html_nodes}\NormalTok{(}\DataTypeTok{xpath =} \StringTok{'//*[@id="content"]/div[2]/table'}\NormalTok{) }\OperatorTok
\StringTok{  }\KeywordTok{html_table}\NormalTok{()}

\NormalTok{turnovers19 <-}\StringTok{ }\NormalTok{turnovers19[[}\DecValTok{1}\NormalTok{]] }\OperatorTok\StringTok{ }\KeywordTok{select}\NormalTok{(Name, }\StringTok{`}\DataTypeTok{Total Lost}\StringTok{`}\NormalTok{)}


\NormalTok{offense <-}\StringTok{ }\NormalTok{yards19 }\OperatorTok
\StringTok{  }\KeywordTok{left_join}\NormalTok{(points19, }\DataTypeTok{by=}\KeywordTok{c}\NormalTok{(}\StringTok{"Name"}\NormalTok{)) }\OperatorTok
\StringTok{  }\KeywordTok{left_join}\NormalTok{(turnovers19, }\DataTypeTok{by=}\KeywordTok{c}\NormalTok{(}\StringTok{"Name"}\NormalTok{))}
\end{Highlighting}
\end{Shaded}

A bubble chart is just a scatterplot with one additional element in the aesthetic -- a size. Here's the scatterplot version.

\begin{Shaded}
\begin{Highlighting}[]
\KeywordTok{ggplot}\NormalTok{() }\OperatorTok{+}\StringTok{ }\KeywordTok{geom_point}\NormalTok{(}\DataTypeTok{data=}\NormalTok{offense, }\KeywordTok{aes}\NormalTok{(}\DataTypeTok{x=}\StringTok{`}\DataTypeTok{Yards/G}\StringTok{`}\NormalTok{, }\DataTypeTok{y=}\StringTok{`}\DataTypeTok{Points/G}\StringTok{`}\NormalTok{))}
\end{Highlighting}
\end{Shaded}

\includegraphics{SportsData_files/figure-latex/unnamed-chunk-202-1.pdf}

As expected, yards per game pretty tightly predicts points per game, but you could have guessed that without a chart. So let's add the size element.

\begin{Shaded}
\begin{Highlighting}[]
\KeywordTok{ggplot}\NormalTok{() }\OperatorTok{+}\StringTok{ }\KeywordTok{geom_point}\NormalTok{(}\DataTypeTok{data=}\NormalTok{offense, }\KeywordTok{aes}\NormalTok{(}\DataTypeTok{x=}\StringTok{`}\DataTypeTok{Yards/G}\StringTok{`}\NormalTok{, }\DataTypeTok{y=}\StringTok{`}\DataTypeTok{Points/G}\StringTok{`}\NormalTok{, }\DataTypeTok{size=}\StringTok{`}\DataTypeTok{Total Lost}\StringTok{`}\NormalTok{)) }
\end{Highlighting}
\end{Shaded}

\includegraphics{SportsData_files/figure-latex/unnamed-chunk-203-1.pdf}

Eh. What does this chart tell you? Trick question, there's not much new here. The dots are too big. Also, we can't see when they overlap. We can fix that by adding an alpha element outside the aesthetic -- alpha in this case is transparency -- and we can manually change the size of the dots by adding \texttt{scale\_size} and a \texttt{range}.

\begin{Shaded}
\begin{Highlighting}[]
\KeywordTok{ggplot}\NormalTok{() }\OperatorTok{+}\StringTok{ }\KeywordTok{geom_point}\NormalTok{(}\DataTypeTok{data=}\NormalTok{offense, }\KeywordTok{aes}\NormalTok{(}\DataTypeTok{x=}\StringTok{`}\DataTypeTok{Yards/G}\StringTok{`}\NormalTok{, }\DataTypeTok{y=}\StringTok{`}\DataTypeTok{Points/G}\StringTok{`}\NormalTok{, }\DataTypeTok{size=}\StringTok{`}\DataTypeTok{Total Lost}\StringTok{`}\NormalTok{), }\DataTypeTok{alpha=}\NormalTok{.}\DecValTok{2}\NormalTok{) }\OperatorTok{+}\StringTok{ }\KeywordTok{scale_size}\NormalTok{(}\DataTypeTok{range =} \KeywordTok{c}\NormalTok{(.}\DecValTok{1}\NormalTok{, }\DecValTok{15}\NormalTok{), }\DataTypeTok{name=}\StringTok{"Turnovers"}\NormalTok{)}
\end{Highlighting}
\end{Shaded}

\includegraphics{SportsData_files/figure-latex/unnamed-chunk-204-1.pdf}

Before we do any more work, let's return to the earlier question: What story does this tell? Can you discern a story from the bubbles? Are teams with lots of turnovers doing poorly and teams with few turnovers doing well? The problem is, you can't really tell. So this is a dead end for a bubble chart. If you get a big mess, it's a dead giveaway that you probably don't have a bubble chart.

So let's look at something else. Let's look at something that isn't directly correlated -- we'll look at offensive points per game vs defensive points per game.

I'm going to edit the same rvest code to grab those points per game stats and merge it all together. When it's done, I'll have a dataframe called \texttt{football} and we can look at where good teams fall on the chart with turnover margin as a scaled dot.

\begin{Shaded}
\begin{Highlighting}[]
\NormalTok{ourl <-}\StringTok{ "http://cfbstats.com/2019/leader/national/team/offense/split01/category09/sort01.html"}

\NormalTok{o19 <-}\StringTok{ }\NormalTok{ourl }\OperatorTok
\StringTok{  }\KeywordTok{read_html}\NormalTok{() }\OperatorTok
\StringTok{  }\KeywordTok{html_nodes}\NormalTok{(}\DataTypeTok{xpath =} \StringTok{'//*[@id="content"]/div[2]/table'}\NormalTok{) }\OperatorTok
\StringTok{  }\KeywordTok{html_table}\NormalTok{()}

\NormalTok{o19 <-}\StringTok{ }\NormalTok{o19[[}\DecValTok{1}\NormalTok{]] }\OperatorTok\StringTok{ }\KeywordTok{select}\NormalTok{(Name, }\StringTok{`}\DataTypeTok{Points/G}\StringTok{`}\NormalTok{)}

\NormalTok{durl <-}\StringTok{ "http://cfbstats.com/2019/leader/national/team/defense/split01/category09/sort01.html"}

\NormalTok{d19 <-}\StringTok{ }\NormalTok{durl }\OperatorTok
\StringTok{  }\KeywordTok{read_html}\NormalTok{() }\OperatorTok
\StringTok{  }\KeywordTok{html_nodes}\NormalTok{(}\DataTypeTok{xpath =} \StringTok{'//*[@id="content"]/div[2]/table'}\NormalTok{) }\OperatorTok
\StringTok{  }\KeywordTok{html_table}\NormalTok{()}

\NormalTok{d19 <-}\StringTok{ }\NormalTok{d19[[}\DecValTok{1}\NormalTok{]] }\OperatorTok\StringTok{ }\KeywordTok{select}\NormalTok{(Name, }\StringTok{`}\DataTypeTok{Points/G}\StringTok{`}\NormalTok{)}

\NormalTok{turnoversurl <-}\StringTok{ "http://cfbstats.com/2019/leader/national/team/offense/split01/category12/sort01.html"}

\NormalTok{turnovers19 <-}\StringTok{ }\NormalTok{turnoversurl }\OperatorTok
\StringTok{  }\KeywordTok{read_html}\NormalTok{() }\OperatorTok
\StringTok{  }\KeywordTok{html_nodes}\NormalTok{(}\DataTypeTok{xpath =} \StringTok{'//*[@id="content"]/div[2]/table'}\NormalTok{) }\OperatorTok
\StringTok{  }\KeywordTok{html_table}\NormalTok{()}

\NormalTok{turnovers19 <-}\StringTok{ }\NormalTok{turnovers19[[}\DecValTok{1}\NormalTok{]] }\OperatorTok\StringTok{ }\KeywordTok{select}\NormalTok{(Name, Margin)}

\NormalTok{football <-}\StringTok{ }\NormalTok{o19 }\OperatorTok\StringTok{ }
\StringTok{  }\KeywordTok{left_join}\NormalTok{(d19, }\DataTypeTok{by=}\KeywordTok{c}\NormalTok{(}\StringTok{"Name"}\NormalTok{)) }\OperatorTok\StringTok{ }
\StringTok{  }\KeywordTok{left_join}\NormalTok{(turnovers19, }\DataTypeTok{by=}\KeywordTok{c}\NormalTok{(}\StringTok{"Name"}\NormalTok{)) }\OperatorTok\StringTok{ }
\StringTok{  }\KeywordTok{rename}\NormalTok{(}\StringTok{`}\DataTypeTok{Offensive Points Per Game}\StringTok{`}\NormalTok{ =}\StringTok{ `}\DataTypeTok{Points/G.x}\StringTok{`}\NormalTok{, }\StringTok{`}\DataTypeTok{Defensive Points Per Game}\StringTok{`}\NormalTok{=}\StringTok{`}\DataTypeTok{Points/G.y}\StringTok{`}\NormalTok{)}
\end{Highlighting}
\end{Shaded}

Now we can do the bubble chart.

\begin{Shaded}
\begin{Highlighting}[]
\KeywordTok{ggplot}\NormalTok{() }\OperatorTok{+}\StringTok{ }\KeywordTok{geom_point}\NormalTok{(}\DataTypeTok{data=}\NormalTok{football, }\KeywordTok{aes}\NormalTok{(}\DataTypeTok{x=}\StringTok{`}\DataTypeTok{Offensive Points Per Game}\StringTok{`}\NormalTok{, }\DataTypeTok{y=}\StringTok{`}\DataTypeTok{Defensive Points Per Game}\StringTok{`}\NormalTok{, }\DataTypeTok{size=}\NormalTok{Margin), }\DataTypeTok{alpha=}\NormalTok{.}\DecValTok{2}\NormalTok{) }\OperatorTok{+}\StringTok{ }\KeywordTok{scale_size}\NormalTok{(}\DataTypeTok{range =} \KeywordTok{c}\NormalTok{(}\DecValTok{0}\NormalTok{, }\DecValTok{6}\NormalTok{), }\DataTypeTok{name=}\StringTok{"Turnovers"}\NormalTok{)}
\end{Highlighting}
\end{Shaded}

\includegraphics{SportsData_files/figure-latex/unnamed-chunk-206-1.pdf}

Better! Teams are spread out a little more. Bottom right quadrant -- the good defense, good offense quadrant -- have some large dots. The upper left quadrant -- bad defense, bad offense -- have some very small dots, meaning they have a really terrible turnover margin.

But what would make this chart better -- and what you saw in the Rosling video -- is color. What if we colored the dots by if they were above or below zero? Meaning, do they have a positive or negative turnover margin? We can do that with a quick mutate and a case\_when statement.

\begin{Shaded}
\begin{Highlighting}[]
\NormalTok{football <-}\StringTok{ }\NormalTok{football }\OperatorTok\StringTok{ }\KeywordTok{mutate}\NormalTok{(}\DataTypeTok{PositiveNegative =} \KeywordTok{case_when}\NormalTok{(}
\NormalTok{  Margin }\OperatorTok{>}\StringTok{ }\DecValTok{0} \OperatorTok{~}\StringTok{ "Positive"}\NormalTok{,}
\NormalTok{  Margin }\OperatorTok{<}\StringTok{ }\DecValTok{0} \OperatorTok{~}\StringTok{ "Negative"}\NormalTok{,}
\NormalTok{  Margin }\OperatorTok{==}\StringTok{ }\DecValTok{0} \OperatorTok{~}\StringTok{ "Even"}
\NormalTok{))}
\end{Highlighting}
\end{Shaded}

Now we can add \texttt{color=PositiveNegative} to the aesthetic and our dots will be colored by if they are positive, negative or zero.

\begin{Shaded}
\begin{Highlighting}[]
\KeywordTok{ggplot}\NormalTok{() }\OperatorTok{+}\StringTok{ }\KeywordTok{geom_point}\NormalTok{(}\DataTypeTok{data=}\NormalTok{football, }\KeywordTok{aes}\NormalTok{(}\DataTypeTok{x=}\StringTok{`}\DataTypeTok{Offensive Points Per Game}\StringTok{`}\NormalTok{, }\DataTypeTok{y=}\StringTok{`}\DataTypeTok{Defensive Points Per Game}\StringTok{`}\NormalTok{, }\DataTypeTok{size=}\NormalTok{Margin, }\DataTypeTok{color=}\NormalTok{PositiveNegative), }\DataTypeTok{alpha=}\NormalTok{.}\DecValTok{2}\NormalTok{) }\OperatorTok{+}\StringTok{ }\KeywordTok{scale_size}\NormalTok{(}\DataTypeTok{range =} \KeywordTok{c}\NormalTok{(}\DecValTok{0}\NormalTok{, }\DecValTok{6}\NormalTok{), }\DataTypeTok{name=}\StringTok{"Turnovers"}\NormalTok{)}
\end{Highlighting}
\end{Shaded}

\includegraphics{SportsData_files/figure-latex/unnamed-chunk-208-1.pdf}

Now we're getting somewhere. What's the story that this chart tells? Blue dots -- positive turnover margins -- are all drifting toward that good offense, good defense quadrant. Green dots -- negative turnover margins -- are drifting toward that bad defense, bad offense quadrant.

Let's add some annotations. Let's look at the top two turnover margin teams and where they come out.

\begin{Shaded}
\begin{Highlighting}[]
\NormalTok{topteams <-}\StringTok{ }\NormalTok{football }\OperatorTok\StringTok{ }\KeywordTok{filter}\NormalTok{(Name }\OperatorTok{==}\StringTok{ "LSU"} \OperatorTok{|}\StringTok{ }\NormalTok{Name }\OperatorTok{==}\StringTok{ "Alabama"}\NormalTok{)}
\end{Highlighting}
\end{Shaded}

\begin{Shaded}
\begin{Highlighting}[]
\KeywordTok{ggplot}\NormalTok{() }\OperatorTok{+}
\StringTok{  }\KeywordTok{geom_point}\NormalTok{(}\DataTypeTok{data=}\NormalTok{football, }\KeywordTok{aes}\NormalTok{(}\DataTypeTok{x=}\StringTok{`}\DataTypeTok{Offensive Points Per Game}\StringTok{`}\NormalTok{, }\DataTypeTok{y=}\StringTok{`}\DataTypeTok{Defensive Points Per Game}\StringTok{`}\NormalTok{, }\DataTypeTok{size=}\NormalTok{Margin, }\DataTypeTok{color=}\NormalTok{PositiveNegative), }\DataTypeTok{alpha=}\NormalTok{.}\DecValTok{2}\NormalTok{) }\OperatorTok{+}
\StringTok{  }\KeywordTok{geom_text_repel}\NormalTok{(}\DataTypeTok{data=}\NormalTok{topteams, }\KeywordTok{aes}\NormalTok{(}\DataTypeTok{x=}\StringTok{`}\DataTypeTok{Offensive Points Per Game}\StringTok{`}\NormalTok{, }\DataTypeTok{y=}\StringTok{`}\DataTypeTok{Defensive Points Per Game}\StringTok{`}\NormalTok{, }\DataTypeTok{label=}\NormalTok{Name)) }\OperatorTok{+}\StringTok{ }
\StringTok{  }\KeywordTok{scale_size}\NormalTok{(}\DataTypeTok{range =} \KeywordTok{c}\NormalTok{(}\DecValTok{0}\NormalTok{, }\DecValTok{6}\NormalTok{), }\DataTypeTok{name=}\StringTok{"Turnovers"}\NormalTok{)}
\end{Highlighting}
\end{Shaded}

\includegraphics{SportsData_files/figure-latex/unnamed-chunk-210-1.pdf}

No surprise there. What about Nebraska?

\begin{Shaded}
\begin{Highlighting}[]
\NormalTok{nu <-}\StringTok{ }\NormalTok{football }\OperatorTok\StringTok{ }\KeywordTok{filter}\NormalTok{(Name}\OperatorTok{==}\StringTok{"Nebraska"}\NormalTok{)}
\end{Highlighting}
\end{Shaded}

\begin{Shaded}
\begin{Highlighting}[]
\KeywordTok{ggplot}\NormalTok{() }\OperatorTok{+}
\StringTok{  }\KeywordTok{geom_point}\NormalTok{(}\DataTypeTok{data=}\NormalTok{football, }\KeywordTok{aes}\NormalTok{(}\DataTypeTok{x=}\StringTok{`}\DataTypeTok{Offensive Points Per Game}\StringTok{`}\NormalTok{, }\DataTypeTok{y=}\StringTok{`}\DataTypeTok{Defensive Points Per Game}\StringTok{`}\NormalTok{, }\DataTypeTok{size=}\NormalTok{Margin, }\DataTypeTok{color=}\NormalTok{PositiveNegative), }\DataTypeTok{alpha=}\NormalTok{.}\DecValTok{2}\NormalTok{) }\OperatorTok{+}
\StringTok{  }\KeywordTok{geom_text_repel}\NormalTok{(}\DataTypeTok{data=}\NormalTok{topteams, }\KeywordTok{aes}\NormalTok{(}\DataTypeTok{x=}\StringTok{`}\DataTypeTok{Offensive Points Per Game}\StringTok{`}\NormalTok{, }\DataTypeTok{y=}\StringTok{`}\DataTypeTok{Defensive Points Per Game}\StringTok{`}\NormalTok{, }\DataTypeTok{label=}\NormalTok{Name)) }\OperatorTok{+}\StringTok{ }
\StringTok{  }\KeywordTok{geom_point}\NormalTok{(}\DataTypeTok{data=}\NormalTok{nu, }\KeywordTok{aes}\NormalTok{(}\DataTypeTok{x=}\StringTok{`}\DataTypeTok{Offensive Points Per Game}\StringTok{`}\NormalTok{, }\DataTypeTok{y=}\StringTok{`}\DataTypeTok{Defensive Points Per Game}\StringTok{`}\NormalTok{, }\DataTypeTok{size=}\NormalTok{Margin), }\DataTypeTok{alpha=}\NormalTok{.}\DecValTok{2}\NormalTok{) }\OperatorTok{+}
\StringTok{  }\KeywordTok{geom_text_repel}\NormalTok{(}\DataTypeTok{data=}\NormalTok{nu, }\KeywordTok{aes}\NormalTok{(}\DataTypeTok{x=}\StringTok{`}\DataTypeTok{Offensive Points Per Game}\StringTok{`}\NormalTok{, }\DataTypeTok{y=}\StringTok{`}\DataTypeTok{Defensive Points Per Game}\StringTok{`}\NormalTok{, }\DataTypeTok{label=}\NormalTok{Name)) }\OperatorTok{+}\StringTok{ }
\StringTok{  }\KeywordTok{scale_size}\NormalTok{(}\DataTypeTok{range =} \KeywordTok{c}\NormalTok{(}\DecValTok{0}\NormalTok{, }\DecValTok{6}\NormalTok{), }\DataTypeTok{name=}\StringTok{"Turnovers"}\NormalTok{)}
\end{Highlighting}
\end{Shaded}

\includegraphics{SportsData_files/figure-latex/unnamed-chunk-212-1.pdf}

Sadly, no surprise there either.

The last things we need to do? Add some labels, apply our finishing touches.

\begin{Shaded}
\begin{Highlighting}[]
\KeywordTok{ggplot}\NormalTok{() }\OperatorTok{+}
\StringTok{  }\KeywordTok{geom_point}\NormalTok{(}\DataTypeTok{data=}\NormalTok{football, }\KeywordTok{aes}\NormalTok{(}\DataTypeTok{x=}\StringTok{`}\DataTypeTok{Offensive Points Per Game}\StringTok{`}\NormalTok{, }\DataTypeTok{y=}\StringTok{`}\DataTypeTok{Defensive Points Per Game}\StringTok{`}\NormalTok{, }\DataTypeTok{size=}\NormalTok{Margin, }\DataTypeTok{color=}\NormalTok{PositiveNegative), }\DataTypeTok{alpha=}\NormalTok{.}\DecValTok{2}\NormalTok{) }\OperatorTok{+}
\StringTok{  }\KeywordTok{geom_text_repel}\NormalTok{(}\DataTypeTok{data=}\NormalTok{topteams, }\KeywordTok{aes}\NormalTok{(}\DataTypeTok{x=}\StringTok{`}\DataTypeTok{Offensive Points Per Game}\StringTok{`}\NormalTok{, }\DataTypeTok{y=}\StringTok{`}\DataTypeTok{Defensive Points Per Game}\StringTok{`}\NormalTok{, }\DataTypeTok{label=}\NormalTok{Name)) }\OperatorTok{+}\StringTok{ }
\StringTok{  }\KeywordTok{geom_point}\NormalTok{(}\DataTypeTok{data=}\NormalTok{nu, }\KeywordTok{aes}\NormalTok{(}\DataTypeTok{x=}\StringTok{`}\DataTypeTok{Offensive Points Per Game}\StringTok{`}\NormalTok{, }\DataTypeTok{y=}\StringTok{`}\DataTypeTok{Defensive Points Per Game}\StringTok{`}\NormalTok{, }\DataTypeTok{size=}\NormalTok{Margin), }\DataTypeTok{alpha=}\NormalTok{.}\DecValTok{2}\NormalTok{) }\OperatorTok{+}
\StringTok{  }\KeywordTok{geom_text_repel}\NormalTok{(}\DataTypeTok{data=}\NormalTok{nu, }\KeywordTok{aes}\NormalTok{(}\DataTypeTok{x=}\StringTok{`}\DataTypeTok{Offensive Points Per Game}\StringTok{`}\NormalTok{, }\DataTypeTok{y=}\StringTok{`}\DataTypeTok{Defensive Points Per Game}\StringTok{`}\NormalTok{, }\DataTypeTok{label=}\NormalTok{Name)) }\OperatorTok{+}
\StringTok{  }\KeywordTok{scale_size}\NormalTok{(}\DataTypeTok{range =} \KeywordTok{c}\NormalTok{(}\DecValTok{0}\NormalTok{, }\DecValTok{6}\NormalTok{), }\DataTypeTok{name=}\StringTok{"Turnovers"}\NormalTok{) }\OperatorTok{+}
\StringTok{  }\KeywordTok{labs}\NormalTok{(}\DataTypeTok{title=}\StringTok{"Same song, different year"}\NormalTok{, }\DataTypeTok{subtitle=}\StringTok{"The Husker's turnover margin is zero, putting them miles from the top."}\NormalTok{, }\DataTypeTok{caption=}\StringTok{"Source: NCAA | By Matt Waite"}\NormalTok{)  }\OperatorTok{+}\StringTok{ }\KeywordTok{theme_minimal}\NormalTok{() }\OperatorTok{+}\StringTok{ }
\StringTok{  }\KeywordTok{theme}\NormalTok{(}
    \DataTypeTok{plot.title =} \KeywordTok{element_text}\NormalTok{(}\DataTypeTok{size =} \DecValTok{16}\NormalTok{, }\DataTypeTok{face =} \StringTok{"bold"}\NormalTok{),}
    \DataTypeTok{axis.title =} \KeywordTok{element_text}\NormalTok{(}\DataTypeTok{size =} \DecValTok{8}\NormalTok{), }
    \DataTypeTok{plot.subtitle =} \KeywordTok{element_text}\NormalTok{(}\DataTypeTok{size=}\DecValTok{10}\NormalTok{), }
    \DataTypeTok{panel.grid.minor =} \KeywordTok{element_blank}\NormalTok{()}
\NormalTok{    )}
\end{Highlighting}
\end{Shaded}

\includegraphics{SportsData_files/figure-latex/unnamed-chunk-213-1.pdf}

\hypertarget{circular-bar-plots}{%
\chapter{Circular bar plots}\label{circular-bar-plots}}

At the 27:36 mark in the \href{https://www.omaha.com/sports/podcasts/half-court-press/half-court-press-creighton-cruises-in-opener-nebraska-stunned-in/article_67081a35-3a8f-5e9e-ae67-e88fcacbb362.html}{Half Court Podcast}, Omaha World Herald Writer Chris Heady said ``November basketball doesn't matter, but it shows you where you are.''

It's a tempting phrase to believe, especially a day after Nebraska lost the first game of the Fred Hoiberg era at home to a baseball school, UC Riverside. And it wasn't close. The Huskers, because of a total roster turnover, were a complete mystery before the game. And what happened during it wasn't pretty, so there was a little soul searching going on in Lincoln.

But does November basketball really not matter?

Let's look, using a new form of chart called a circular bar plot. It's a chart type that combines several forms we've used before: bar charts to show magnitude, stacked bar charts to show proportion, but we're going to add bending the chart around a circle to add some visual interstingness to it. We're also going to use time as an x-axis value to make a not subtle circle of time reference -- a common technique with circular bar charts.

First we need some libraries.

\begin{Shaded}
\begin{Highlighting}[]
\KeywordTok{library}\NormalTok{(tidyverse)}
\KeywordTok{library}\NormalTok{(lubridate)}
\end{Highlighting}
\end{Shaded}

Let's import \href{https://unl.box.com/s/a8m91bro10t89watsyo13yjegb1fy009}{every basketball game from last year}.

\begin{Shaded}
\begin{Highlighting}[]
\NormalTok{logs <-}\StringTok{ }\KeywordTok{read_csv}\NormalTok{(}\StringTok{"data/logs19.csv"}\NormalTok{)}
\end{Highlighting}
\end{Shaded}

\begin{verbatim}
## Warning: Missing column names filled in: 'X1' [1]
\end{verbatim}

\begin{verbatim}
## Parsed with column specification:
## cols(
##   .default = col_double(),
##   Date = col_date(format = ""),
##   HomeAway = col_character(),
##   Opponent = col_character(),
##   W_L = col_character(),
##   Blank = col_logical(),
##   Team = col_character(),
##   Conference = col_character(),
##   season = col_character()
## )
\end{verbatim}

\begin{verbatim}
## See spec(...) for full column specifications.
\end{verbatim}

So let's test the notion of November Basketball Doesn't Matter. What matters in basketball? Let's start simple: Wins.

Sports Reference's win columns are weird, so we need to scan through them and find W and L and we'll give them numbers using \texttt{case\_when}. I'm also going to filter out tournament basketball.

\begin{Shaded}
\begin{Highlighting}[]
\NormalTok{winlosslogs <-}\StringTok{ }\NormalTok{logs }\OperatorTok\StringTok{ }\KeywordTok{mutate}\NormalTok{(}\DataTypeTok{winloss =} \KeywordTok{case_when}\NormalTok{(}
  \KeywordTok{grepl}\NormalTok{(}\StringTok{"W"}\NormalTok{, W_L) }\OperatorTok{~}\StringTok{ }\DecValTok{1}\NormalTok{, }
  \KeywordTok{grepl}\NormalTok{(}\StringTok{"L"}\NormalTok{, W_L) }\OperatorTok{~}\StringTok{ }\DecValTok{0}\NormalTok{)}
\NormalTok{) }\OperatorTok\StringTok{ }\KeywordTok{filter}\NormalTok{(Date }\OperatorTok{<}\StringTok{ "2019-03-19"}\NormalTok{)}
\end{Highlighting}
\end{Shaded}

Now we can group by date and conference and sum up the wins. How many wins by day does each conference get?

\begin{Shaded}
\begin{Highlighting}[]
\NormalTok{dates <-}\StringTok{ }\NormalTok{winlosslogs }\OperatorTok\StringTok{ }\KeywordTok{group_by}\NormalTok{(Date, Conference) }\OperatorTok\StringTok{ }\KeywordTok{summarise}\NormalTok{(}\DataTypeTok{wins =} \KeywordTok{sum}\NormalTok{(winloss))}
\end{Highlighting}
\end{Shaded}

Earlier, we did stacked bar charts. We have what we need to do that now.

\begin{Shaded}
\begin{Highlighting}[]
\KeywordTok{ggplot}\NormalTok{() }\OperatorTok{+}\StringTok{ }\KeywordTok{geom_bar}\NormalTok{(}\DataTypeTok{data=}\NormalTok{dates, }\KeywordTok{aes}\NormalTok{(}\DataTypeTok{x=}\NormalTok{Date, }\DataTypeTok{weight=}\NormalTok{wins, }\DataTypeTok{fill=}\NormalTok{Conference)) }\OperatorTok{+}\StringTok{ }\KeywordTok{theme_minimal}\NormalTok{()}
\end{Highlighting}
\end{Shaded}

\includegraphics{SportsData_files/figure-latex/unnamed-chunk-218-1.pdf}

Eeek. This is already looking not great. But to make it a circular bar chart, we add \texttt{coord\_polar()} to our chart.

\begin{Shaded}
\begin{Highlighting}[]
\KeywordTok{ggplot}\NormalTok{() }\OperatorTok{+}\StringTok{ }\KeywordTok{geom_bar}\NormalTok{(}\DataTypeTok{data=}\NormalTok{dates, }\KeywordTok{aes}\NormalTok{(}\DataTypeTok{x=}\NormalTok{Date, }\DataTypeTok{weight=}\NormalTok{wins, }\DataTypeTok{fill=}\NormalTok{Conference)) }\OperatorTok{+}\StringTok{ }\KeywordTok{theme_minimal}\NormalTok{() }\OperatorTok{+}\StringTok{ }\KeywordTok{coord_polar}\NormalTok{()}
\end{Highlighting}
\end{Shaded}

\includegraphics{SportsData_files/figure-latex/unnamed-chunk-219-1.pdf}

Based on that, the day is probably too thin a slice, and there's way too many conferences in college basketball. Let's group this by months and filter out all but the power five conferences.

\begin{Shaded}
\begin{Highlighting}[]
\NormalTok{p5 <-}\StringTok{ }\KeywordTok{c}\NormalTok{(}\StringTok{"SEC"}\NormalTok{, }\StringTok{"Big Ten"}\NormalTok{, }\StringTok{"Pac-12"}\NormalTok{, }\StringTok{"Big 12"}\NormalTok{, }\StringTok{"ACC"}\NormalTok{)}
\end{Highlighting}
\end{Shaded}

To get months, we're going to use a function in the library \texttt{lubridate} called \texttt{floor\_date}, which combine with mutate will give us a field of just months.

\begin{Shaded}
\begin{Highlighting}[]
\NormalTok{wins <-}\StringTok{ }\NormalTok{winlosslogs }\OperatorTok\StringTok{ }\KeywordTok{mutate}\NormalTok{(}\DataTypeTok{month =} \KeywordTok{floor_date}\NormalTok{(Date, }\DataTypeTok{unit=}\StringTok{"months"}\NormalTok{)) }\OperatorTok\StringTok{ }\KeywordTok{group_by}\NormalTok{(month, Conference) }\OperatorTok\StringTok{ }\KeywordTok{summarise}\NormalTok{(}\DataTypeTok{wins=}\KeywordTok{sum}\NormalTok{(winloss)) }\OperatorTok\StringTok{ }\KeywordTok{filter}\NormalTok{(Conference }\OperatorTok\StringTok{ }\NormalTok{p5) }
\end{Highlighting}
\end{Shaded}

Now we can use wins to make our circular bar chart of wins by month in the Power Five.

\begin{Shaded}
\begin{Highlighting}[]
\KeywordTok{ggplot}\NormalTok{() }\OperatorTok{+}\StringTok{ }\KeywordTok{geom_bar}\NormalTok{(}\DataTypeTok{data=}\NormalTok{wins, }\KeywordTok{aes}\NormalTok{(}\DataTypeTok{x=}\NormalTok{month, }\DataTypeTok{weight=}\NormalTok{wins, }\DataTypeTok{fill=}\NormalTok{Conference)) }\OperatorTok{+}\StringTok{ }\KeywordTok{theme_minimal}\NormalTok{() }\OperatorTok{+}\StringTok{ }\KeywordTok{coord_polar}\NormalTok{()}
\end{Highlighting}
\end{Shaded}

\includegraphics{SportsData_files/figure-latex/unnamed-chunk-222-1.pdf}

Yikes. That looks a lot like a broken pie chart. So months are too thick of a slice. Let's use weeks in our floor date to see what that gives us.

\begin{Shaded}
\begin{Highlighting}[]
\NormalTok{wins <-}\StringTok{ }\NormalTok{winlosslogs }\OperatorTok\StringTok{ }\KeywordTok{mutate}\NormalTok{(}\DataTypeTok{week =} \KeywordTok{floor_date}\NormalTok{(Date, }\DataTypeTok{unit=}\StringTok{"weeks"}\NormalTok{)) }\OperatorTok\StringTok{ }\KeywordTok{group_by}\NormalTok{(week, Conference) }\OperatorTok\StringTok{ }\KeywordTok{summarise}\NormalTok{(}\DataTypeTok{wins=}\KeywordTok{sum}\NormalTok{(winloss)) }\OperatorTok\StringTok{ }\KeywordTok{filter}\NormalTok{(Conference }\OperatorTok\StringTok{ }\NormalTok{p5) }
\end{Highlighting}
\end{Shaded}

\begin{Shaded}
\begin{Highlighting}[]
\KeywordTok{ggplot}\NormalTok{() }\OperatorTok{+}\StringTok{ }\KeywordTok{geom_bar}\NormalTok{(}\DataTypeTok{data=}\NormalTok{wins, }\KeywordTok{aes}\NormalTok{(}\DataTypeTok{x=}\NormalTok{week, }\DataTypeTok{weight=}\NormalTok{wins, }\DataTypeTok{fill=}\NormalTok{Conference)) }\OperatorTok{+}\StringTok{ }\KeywordTok{theme_minimal}\NormalTok{() }\OperatorTok{+}\StringTok{ }\KeywordTok{coord_polar}\NormalTok{()}
\end{Highlighting}
\end{Shaded}

\includegraphics{SportsData_files/figure-latex/unnamed-chunk-224-1.pdf}

That looks better. But what does it say? Does November basketball matter? What this is saying is \ldots{} yeah, it kinda does. The reason? Lots of wins get piled up in November and December, during non-conference play. So if you are a team with NCAA tournament dreams, you need to win games in November to make sure your tournament resume is where it needs to be come March. Does an individual win or loss matter? Probably not. But your record in November does.

\hypertarget{intro-to-rvest}{%
\chapter{Intro to rvest}\label{intro-to-rvest}}

All the way back in Chapter 2, we used Google Sheets and importHTML to get our own data out of a website. For me, that's a lot of pointing and clicking and copying and pasting. R has a library that can automate the harvesting of data from HTML on the internet. It's called \texttt{rvest}.

Let's grab \href{http://www.cfbstats.com/2019/leader/national/team/offense/split01/category09/sort01.html}{a simple, basic HTML table from College Football Stats}. This is scoring offense for 2019. There's nothing particularly strange about this table -- it's simply formatted and easy to scrape.

First we'll need some libraries. We're going to use a library called \texttt{rvest}, which you can get by running \texttt{install.packages(\textquotesingle{}rvest\textquotesingle{})} in the console.

\begin{Shaded}
\begin{Highlighting}[]
\KeywordTok{library}\NormalTok{(rvest)}
\KeywordTok{library}\NormalTok{(tidyverse)}
\end{Highlighting}
\end{Shaded}

The rvest package has functions that make fetching, reading and parsing HTML simple. The first thing we need to do is specify a url that we're going to scrape.

\begin{Shaded}
\begin{Highlighting}[]
\NormalTok{scoringoffenseurl <-}\StringTok{ "http://www.cfbstats.com/2019/leader/national/team/offense/split01/category09/sort01.html"}
\end{Highlighting}
\end{Shaded}

Now, the most difficult part of scraping data from any website is knowing what exact HTML tag you need to grab. In this case, we want a \texttt{\textless{}table\textgreater{}} tag that has all of our data table in it. But how do you tell R which one that is? Well, it's easy, once you know what to do. But it's not simple. So I've made a short video to show you how to find it.

When you have simple tables, the code is very simple. You create a variable to receive the data, then pass it the url, read the html that was fetched, find the node you need using your XPath value you just copied and you tell rvest that it's a table.

\begin{Shaded}
\begin{Highlighting}[]
\NormalTok{scoringoffense <-}\StringTok{ }\NormalTok{scoringoffenseurl }\OperatorTok
\StringTok{  }\KeywordTok{read_html}\NormalTok{() }\OperatorTok
\StringTok{  }\KeywordTok{html_nodes}\NormalTok{(}\DataTypeTok{xpath =} \StringTok{'//*[@id="content"]/div[2]/table'}\NormalTok{) }\OperatorTok
\StringTok{  }\KeywordTok{html_table}\NormalTok{()}
\end{Highlighting}
\end{Shaded}

What we get from this is \ldots{} not a dataframe. It's a list with one element in it, which just so happens to be our dataframe. When you get this, the solution is simple: just overwrite the variable you created with the first list element.

\begin{Shaded}
\begin{Highlighting}[]
\NormalTok{scoringoffense <-}\StringTok{ }\NormalTok{scoringoffense[[}\DecValTok{1}\NormalTok{]]}
\end{Highlighting}
\end{Shaded}

And what do we have?

\begin{Shaded}
\begin{Highlighting}[]
\KeywordTok{head}\NormalTok{(scoringoffense)}
\end{Highlighting}
\end{Shaded}

\begin{verbatim}
##           Name  G TD FG 1XP 2XP Safety Points Points/G
## 1 1        LSU 15 95 21  89   1      1    726     48.4
## 2 2    Alabama 13 83 12  80   0      0    614     47.2
## 3 3 Ohio State 14 88 13  87   0      1    656     46.9
## 4 4    Clemson 15 88 14  85   2      0    659     43.9
## 5 5        UCF 13 74 15  71   1      1    564     43.4
## 6 6   Oklahoma 14 76 19  75   1      0    590     42.1
\end{verbatim}

We have data, ready for analysis.

\hypertarget{a-slightly-more-complicated-example}{%
\section{A slightly more complicated example}\label{a-slightly-more-complicated-example}}

What if we want more than one year in our dataframe?

This is a common problem. What if we want to look at every scoring offense going back several years? The website has them going back to 2009. How can we combine them?

First, we should note, that the data does not have anything in it to indicate what year it comes from. So we're going to have to add that. And we're going to have to figure out a way to stack two dataframes on top of each other.

So let's grab 2018.

\begin{Shaded}
\begin{Highlighting}[]
\NormalTok{scoringoffenseurl18 <-}\StringTok{ "http://www.cfbstats.com/2018/leader/national/team/offense/split01/category09/sort01.html"}

\NormalTok{scoringoffense18 <-}\StringTok{ }\NormalTok{scoringoffenseurl18 }\OperatorTok
\StringTok{  }\KeywordTok{read_html}\NormalTok{() }\OperatorTok
\StringTok{  }\KeywordTok{html_nodes}\NormalTok{(}\DataTypeTok{xpath =} \StringTok{'//*[@id="content"]/div[2]/table'}\NormalTok{) }\OperatorTok
\StringTok{  }\KeywordTok{html_table}\NormalTok{()}

\NormalTok{scoringoffense18 <-}\StringTok{ }\NormalTok{scoringoffense18[[}\DecValTok{1}\NormalTok{]]}
\end{Highlighting}
\end{Shaded}

First, how are we going to know, in the data, which year our data is from? We can use mutate.

\begin{Shaded}
\begin{Highlighting}[]
\NormalTok{scoringoffense19 <-}\StringTok{ }\NormalTok{scoringoffense }\OperatorTok\StringTok{ }\KeywordTok{mutate}\NormalTok{(}\DataTypeTok{YEAR =} \DecValTok{2019}\NormalTok{)}
\end{Highlighting}
\end{Shaded}

\begin{verbatim}
## Error: Column 1 must be named.
## Use .name_repair to specify repair.
\end{verbatim}

Uh oh. Error. What does it say? Column 1 must be named. If you look at our data in the environment tab in the upper right corner, you'll see that indeed, the first column has no name. It's the FBS rank of each team. So we can fix that and mutate in the same step. We'll do that using \texttt{rename} and since the field doesn't have a name to rename it, we'll use a position argument. We'll say rename column 1 as Rank.

\begin{Shaded}
\begin{Highlighting}[]
\NormalTok{scoringoffense19 <-}\StringTok{ }\NormalTok{scoringoffense }\OperatorTok\StringTok{ }\KeywordTok{rename}\NormalTok{(}\DataTypeTok{Rank =} \DecValTok{1}\NormalTok{) }\OperatorTok\StringTok{ }\KeywordTok{mutate}\NormalTok{(}\DataTypeTok{YEAR =} \DecValTok{2019}\NormalTok{)}
\NormalTok{scoringoffense18 <-}\StringTok{ }\NormalTok{scoringoffense18 }\OperatorTok\StringTok{ }\KeywordTok{rename}\NormalTok{(}\DataTypeTok{Rank =} \DecValTok{1}\NormalTok{) }\OperatorTok\StringTok{ }\KeywordTok{mutate}\NormalTok{(}\DataTypeTok{YEAR =} \DecValTok{2018}\NormalTok{)}
\end{Highlighting}
\end{Shaded}

And now, to combine the two tables together length-wise -- we need to make long data -- we'll use a base R function called \texttt{rbind}. The good thing is rbind is simple. The bad part is it can only do two tables at a time, so if you have more than that, you'll need to do it in steps.

\begin{Shaded}
\begin{Highlighting}[]
\NormalTok{combined <-}\StringTok{ }\KeywordTok{rbind}\NormalTok{(scoringoffense19, scoringoffense18)}
\end{Highlighting}
\end{Shaded}

Note in the environment tab we now have a data frame called combined that has 260 observations -- which just so happens to be what 130 from 2019 and 130 from 2018 add up to.

\begin{Shaded}
\begin{Highlighting}[]
\KeywordTok{head}\NormalTok{(combined)}
\end{Highlighting}
\end{Shaded}

\begin{verbatim}
##   Rank       Name  G TD FG 1XP 2XP Safety Points Points/G YEAR
## 1    1        LSU 15 95 21  89   1      1    726     48.4 2019
## 2    2    Alabama 13 83 12  80   0      0    614     47.2 2019
## 3    3 Ohio State 14 88 13  87   0      1    656     46.9 2019
## 4    4    Clemson 15 88 14  85   2      0    659     43.9 2019
## 5    5        UCF 13 74 15  71   1      1    564     43.4 2019
## 6    6   Oklahoma 14 76 19  75   1      0    590     42.1 2019
\end{verbatim}

\hypertarget{an-even-more-complicated-example}{%
\section{An even more complicated example}\label{an-even-more-complicated-example}}

What do you do when the table has non-standard headers?

Unfortunately, non-standard means there's no one way to do it -- it's going to depend on the table and the headers. But here's one idea: Don't try to make it work.

I'll explain.

Let's try to get \href{https://www.sports-reference.com/cbb/seasons/2019-school-stats.html}{season team stats from Sports Reference}. If you look at that page, you'll see the problem right away -- the headers span two rows, and they repeat. That's going to be all kinds of no good. You can't import that. Dataframes must have names all in one row. If you have two-line headers, you have a problem you have to fix before you can do anything else with it.

First we'll grab the page.

\begin{Shaded}
\begin{Highlighting}[]
\NormalTok{url <-}\StringTok{ "https://www.sports-reference.com/cbb/seasons/2019-school-stats.html"}
\end{Highlighting}
\end{Shaded}

Now, similar to our example above, we'll read the html, use XPath to find the table, and then read that table with a directive passed to it setting the header to FALSE. That tells rvest that there isn't a header row. Just import it as data.

\begin{Shaded}
\begin{Highlighting}[]
\NormalTok{stats <-}\StringTok{ }\NormalTok{url }\OperatorTok
\StringTok{  }\KeywordTok{read_html}\NormalTok{() }\OperatorTok
\StringTok{  }\KeywordTok{html_nodes}\NormalTok{(}\DataTypeTok{xpath =} \StringTok{'//*[@id="basic_school_stats"]'}\NormalTok{) }\OperatorTok
\StringTok{  }\KeywordTok{html_table}\NormalTok{(}\DataTypeTok{header=}\OtherTok{FALSE}\NormalTok{)}
\end{Highlighting}
\end{Shaded}

What we get back is a list of one element (similar to above). So let's pop it out into a data frame.

\begin{Shaded}
\begin{Highlighting}[]
\NormalTok{stats <-}\StringTok{ }\NormalTok{stats[[}\DecValTok{1}\NormalTok{]]}
\end{Highlighting}
\end{Shaded}

And we'll take a look at what we have.

\begin{Shaded}
\begin{Highlighting}[]
\KeywordTok{head}\NormalTok{(stats)}
\end{Highlighting}
\end{Shaded}

\begin{verbatim}
##   X1                     X2      X3      X4      X5      X6      X7      X8
## 1                           Overall Overall Overall Overall Overall Overall
## 2 Rk                 School       G       W       L    W-L%     SRS     SOS
## 3  1 Abilene Christian NCAA      34      27       7    .794   -1.91   -7.34
## 4  2              Air Force      32      14      18    .438   -4.28    0.24
## 5  3                  Akron      33      17      16    .515    4.86    1.09
## 6  4            Alabama A&M      32       5      27    .156  -19.23   -8.38
##      X9   X10  X11  X12  X13  X14    X15    X16 X17           X18           X19
## 1 Conf. Conf. Home Home Away Away Points Points  NA School Totals School Totals
## 2     W     L    W    L    W    L    Tm.   Opp.  NA            MP            FG
## 3    14     4   13    2   10    4   2502   2161  NA          1370           897
## 4     8    10    9    6    3    9   2179   2294  NA          1300           802
## 5     8    10   14    3    1   10   2271   2107  NA          1325           797
## 6     4    14    4    7    0   18   1938   2285  NA          1295           736
##             X20           X21           X22           X23           X24
## 1 School Totals School Totals School Totals School Totals School Totals
## 2           FGA           FG%            3P           3PA           3P%
## 3          1911          .469           251           660          .380
## 4          1776          .452           234           711          .329
## 5          1948          .409           297           929          .320
## 6          1809          .407           182           578          .315
##             X25           X26           X27           X28           X29
## 1 School Totals School Totals School Totals School Totals School Totals
## 2            FT           FTA           FT%           ORB           TRB
## 3           457           642          .712           325          1110
## 4           341           503          .678           253          1077
## 5           380           539          .705           312          1204
## 6           284           453          .627           314          1032
##             X30           X31           X32           X33           X34
## 1 School Totals School Totals School Totals School Totals School Totals
## 2           AST           STL           BLK           TOV            PF
## 3           525           297            93           407           635
## 4           434           154            57           423           543
## 5           399           185           106           388           569
## 6           385           234            50           487           587
\end{verbatim}

So, that's not ideal. We have headers and data mixed together, and our columns are named X1 to X34. Also note: They're all character fields. Because the headers are interspersed with data, it all gets called character data. So we've got to first rename each field.

\begin{Shaded}
\begin{Highlighting}[]
\NormalTok{stats <-}\StringTok{ }\NormalTok{stats }\OperatorTok\StringTok{ }\KeywordTok{rename}\NormalTok{(}\DataTypeTok{Rank=}\NormalTok{X1, }\DataTypeTok{School=}\NormalTok{X2, }\DataTypeTok{Games=}\NormalTok{X3, }\DataTypeTok{OverallWins=}\NormalTok{X4, }\DataTypeTok{OverallLosses=}\NormalTok{X5, }\DataTypeTok{WinPct=}\NormalTok{X6, }\DataTypeTok{OverallSRS=}\NormalTok{X7, }\DataTypeTok{OverallSOS=}\NormalTok{X8, }\DataTypeTok{ConferenceWins=}\NormalTok{X9, }\DataTypeTok{ConferenceLosses=}\NormalTok{X10, }\DataTypeTok{HomeWins=}\NormalTok{X11, }\DataTypeTok{HomeLosses=}\NormalTok{X12, }\DataTypeTok{AwayWins=}\NormalTok{X13, }\DataTypeTok{AwayLosses=}\NormalTok{X14, }\DataTypeTok{ForPoints=}\NormalTok{X15, }\DataTypeTok{OppPoints=}\NormalTok{X16, }\DataTypeTok{Blank=}\NormalTok{X17, }\DataTypeTok{Minutes=}\NormalTok{X18, }\DataTypeTok{FieldGoalsMade=}\NormalTok{X19, }\DataTypeTok{FieldGoalsAttempted=}\NormalTok{X20, }\DataTypeTok{FieldGoalPCT=}\NormalTok{X21, }\DataTypeTok{ThreePointMade=}\NormalTok{X22, }\DataTypeTok{ThreePointAttempts=}\NormalTok{X23, }\DataTypeTok{ThreePointPct=}\NormalTok{X24, }\DataTypeTok{FreeThrowsMade=}\NormalTok{X25, }\DataTypeTok{FreeThrowsAttempted=}\NormalTok{X26, }\DataTypeTok{FreeThrowPCT=}\NormalTok{X27, }\DataTypeTok{OffensiveRebounds=}\NormalTok{X28, }\DataTypeTok{TotalRebounds=}\NormalTok{X29, }\DataTypeTok{Assists=}\NormalTok{X30, }\DataTypeTok{Steals=}\NormalTok{X31, }\DataTypeTok{Blocks=}\NormalTok{X32, }\DataTypeTok{Turnovers=}\NormalTok{X33, }\DataTypeTok{PersonalFouls=}\NormalTok{X34)}
\end{Highlighting}
\end{Shaded}

Now we have to get rid of those headers interspersed in the data. We can do that with filter that say keep all the stuff that isn't this.

\begin{Shaded}
\begin{Highlighting}[]
\NormalTok{stats <-}\StringTok{ }\NormalTok{stats }\OperatorTok\StringTok{ }\KeywordTok{filter}\NormalTok{(Rank }\OperatorTok{!=}\StringTok{ "Rk"} \OperatorTok{&}\StringTok{ }\NormalTok{Games }\OperatorTok{!=}\StringTok{ "Overall"}\NormalTok{) }
\end{Highlighting}
\end{Shaded}

And finally, we need to change the file type of all the fields that need it. We're going to use a clever little trick, which goes like this: We're going to use \texttt{mutate\_at}, which means mutate these fields. The pattern for \texttt{mutate\_at} is \texttt{mutate\_at} these variables and do this thing to them. But instead of specifying which of 33 variables we're going to mutate, we're going to specify the one we don't want to change, which is the name of the school. And we just want to convert them to numeric. Here's what it looks like:

\begin{Shaded}
\begin{Highlighting}[]
\NormalTok{stats }\OperatorTok\StringTok{ }\KeywordTok{mutate_at}\NormalTok{(}\KeywordTok{vars}\NormalTok{(}\OperatorTok{-}\NormalTok{School), as.numeric)}
\end{Highlighting}
\end{Shaded}

\begin{verbatim}
##     Rank                      School Games OverallWins OverallLosses WinPct
## 1      1      Abilene Christian NCAA    34          27             7  0.794
## 2      2                   Air Force    32          14            18  0.438
## 3      3                       Akron    33          17            16  0.515
## 4      4                 Alabama A&M    32           5            27  0.156
## 5      5          Alabama-Birmingham    35          20            15  0.571
## 6      6               Alabama State    31          12            19  0.387
## 7      7                     Alabama    34          18            16  0.529
## 8      8                 Albany (NY)    32          12            20  0.375
## 9      9                Alcorn State    31          10            21  0.323
## 10    10                    American    30          15            15  0.500
## 11    11           Appalachian State    32          11            21  0.344
## 12    12          Arizona State NCAA    34          23            11  0.676
## 13    13                     Arizona    32          17            15  0.531
## 14    14                 Little Rock    31          10            21  0.323
## 15    15         Arkansas-Pine Bluff    32          13            19  0.406
## 16    16              Arkansas State    32          13            19  0.406
## 17    17                    Arkansas    34          18            16  0.529
## 18    18                        Army    32          13            19  0.406
## 19    19                 Auburn NCAA    40          30            10  0.750
## 20    20                 Austin Peay    33          22            11  0.667
## 21    21                  Ball State    33          16            17  0.485
## 22    22                 Baylor NCAA    34          20            14  0.588
## 23    23                Belmont NCAA    33          27             6  0.818
## 24    24             Bethune-Cookman    31          14            17  0.452
## 25    25                  Binghamton    33          10            23  0.303
## 26    26                 Boise State    33          13            20  0.394
## 27    27              Boston College    31          14            17  0.452
## 28    28           Boston University    33          15            18  0.455
## 29    29         Bowling Green State    34          22            12  0.647
## 30    30                Bradley NCAA    35          20            15  0.571
## 31    31               Brigham Young    32          19            13  0.594
## 32    32                       Brown    32          20            12  0.625
## 33    33                      Bryant    30          10            20  0.333
## 34    34                    Bucknell    33          21            12  0.636
## 35    35                Buffalo NCAA    36          32             4  0.889
## 36    36                      Butler    33          16            17  0.485
## 37    37                    Cal Poly    29           6            23  0.207
## 38    38       Cal State Bakersfield    34          18            16  0.529
## 39    39         Cal State Fullerton    34          16            18  0.471
## 40    40        Cal State Northridge    34          13            21  0.382
## 41    41          California Baptist    31          16            15  0.516
## 42    42                    UC-Davis    31          11            20  0.355
## 43    43              UC-Irvine NCAA    37          31             6  0.838
## 44    44                UC-Riverside    33          10            23  0.303
## 45    45            UC-Santa Barbara    32          22            10  0.688
## 46    46    University of California    31           8            23  0.258
## 47    47                    Campbell    33          20            13  0.606
## 48    48                    Canisius    32          15            17  0.469
## 49    49            Central Arkansas    33          14            19  0.424
## 50    50   Central Connecticut State    31          11            20  0.355
## 51    51        Central Florida NCAA    33          24             9  0.727
## 52    52            Central Michigan    35          23            12  0.657
## 53    53         Charleston Southern    34          18            16  0.529
## 54    54                   Charlotte    29           8            21  0.276
## 55    55                 Chattanooga    32          12            20  0.375
## 56    56               Chicago State    32           3            29  0.094
## 57    57             Cincinnati NCAA    35          28             7  0.800
## 58    58                     Citadel    30          12            18  0.400
## 59    59                     Clemson    34          20            14  0.588
## 60    60             Cleveland State    31          10            21  0.323
## 61    61            Coastal Carolina    34          17            17  0.500
## 62    62                Colgate NCAA    35          24            11  0.686
## 63    63       College of Charleston    33          24             9  0.727
## 64    64              Colorado State    32          12            20  0.375
## 65    65                    Colorado    36          23            13  0.639
## 66    66                    Columbia    28          10            18  0.357
## 67    67                 Connecticut    33          16            17  0.485
## 68    68                Coppin State    33           8            25  0.242
## 69    69                     Cornell    31          15            16  0.484
## 70    70                   Creighton    35          20            15  0.571
## 71    71                   Dartmouth    30          11            19  0.367
## 72    72                    Davidson    34          24            10  0.706
## 73    73                      Dayton    33          21            12  0.636
## 74    74              Delaware State    31           6            25  0.194
## 75    75                    Delaware    33          17            16  0.515
## 76    76                      Denver    30           8            22  0.267
## 77    77                      DePaul    36          19            17  0.528
## 78    78               Detroit Mercy    31          11            20  0.355
## 79    79                       Drake    34          24            10  0.706
## 80    80                      Drexel    32          13            19  0.406
## 81    81                   Duke NCAA    38          32             6  0.842
## 82    82                    Duquesne    32          19            13  0.594
## 83    83               East Carolina    31          10            21  0.323
## 84    84        East Tennessee State    34          24            10  0.706
## 85    85            Eastern Illinois    32          14            18  0.438
## 86    86            Eastern Kentucky    31          13            18  0.419
## 87    87            Eastern Michigan    32          15            17  0.469
## 88    88          Eastern Washington    34          16            18  0.471
## 89    89                        Elon    32          11            21  0.344
## 90    90                  Evansville    32          11            21  0.344
## 91    91                   Fairfield    31           9            22  0.290
## 92    92    Fairleigh Dickinson NCAA    35          21            14  0.600
## 93    93                 Florida A&M    31          12            19  0.387
## 94    94            Florida Atlantic    33          17            16  0.515
## 95    95          Florida Gulf Coast    32          14            18  0.438
## 96    96       Florida International    34          20            14  0.588
## 97    97          Florida State NCAA    37          29             8  0.784
## 98    98                Florida NCAA    36          20            16  0.556
## 99    99                     Fordham    32          12            20  0.375
## 100  100                Fresno State    32          23             9  0.719
## 101  101                      Furman    33          25             8  0.758
## 102  102           Gardner-Webb NCAA    35          23            12  0.657
## 103  103                George Mason    33          18            15  0.545
## 104  104           George Washington    33           9            24  0.273
## 105  105                  Georgetown    33          19            14  0.576
## 106  106            Georgia Southern    33          21            12  0.636
## 107  107          Georgia State NCAA    34          24            10  0.706
## 108  108                Georgia Tech    32          14            18  0.438
## 109  109                     Georgia    32          11            21  0.344
## 110  110                Gonzaga NCAA    37          33             4  0.892
## 111  111                   Grambling    34          17            17  0.500
## 112  112                Grand Canyon    34          20            14  0.588
## 113  113                   Green Bay    38          21            17  0.553
## 114  114                     Hampton    35          18            17  0.514
## 115  115                    Hartford    33          18            15  0.545
## 116  116                     Harvard    31          19            12  0.613
## 117  117                      Hawaii    31          18            13  0.581
## 118  118                  High Point    31          16            15  0.516
## 119  119                     Hofstra    35          27             8  0.771
## 120  120                  Holy Cross    33          16            17  0.485
## 121  121             Houston Baptist    30          12            18  0.400
## 122  122                Houston NCAA    37          33             4  0.892
## 123  123                      Howard    34          17            17  0.500
## 124  124                 Idaho State    30          11            19  0.367
## 125  125                       Idaho    32           5            27  0.156
## 126  126            Illinois-Chicago    32          16            16  0.500
## 127  127              Illinois State    33          17            16  0.515
## 128  128                    Illinois    33          12            21  0.364
## 129  129              Incarnate Word    31           6            25  0.194
## 130  130               Indiana State    31          15            16  0.484
## 131  131                     Indiana    35          19            16  0.543
## 132  132                   Iona NCAA    33          17            16  0.515
## 133  133             Iowa State NCAA    35          23            12  0.657
## 134  134                   Iowa NCAA    35          23            12  0.657
## 135  135           Purdue-Fort Wayne    33          18            15  0.545
## 136  136                       IUPUI    33          16            17  0.485
## 137  137               Jackson State    32          13            19  0.406
## 138  138          Jacksonville State    33          24             9  0.727
## 139  139                Jacksonville    32          12            20  0.375
## 140  140               James Madison    33          14            19  0.424
## 141  141           Kansas State NCAA    34          25             9  0.735
## 142  142                 Kansas NCAA    36          26            10  0.722
## 143  143              Kennesaw State    32           6            26  0.188
## 144  144                  Kent State    33          22            11  0.667
## 145  145               Kentucky NCAA    37          30             7  0.811
## 146  146                    La Salle    31          10            21  0.323
## 147  147                   Lafayette    30          10            20  0.333
## 148  148                       Lamar    33          20            13  0.606
## 149  149                      Lehigh    31          20            11  0.645
## 150  150                Liberty NCAA    36          29             7  0.806
## 151  151                    Lipscomb    37          29             8  0.784
## 152  152        Cal State Long Beach    34          15            19  0.441
## 153  153      Long Island University    32          16            16  0.500
## 154  154                    Longwood    34          16            18  0.471
## 155  155                   Louisiana    32          19            13  0.594
## 156  156            Louisiana-Monroe    35          19            16  0.543
## 157  157        Louisiana State NCAA    35          28             7  0.800
## 158  158              Louisiana Tech    33          20            13  0.606
## 159  159             Louisville NCAA    34          20            14  0.588
## 160  160                 Loyola (IL)    34          20            14  0.588
## 161  161            Loyola Marymount    34          22            12  0.647
## 162  162                 Loyola (MD)    32          11            21  0.344
## 163  163                       Maine    32           5            27  0.156
## 164  164                   Manhattan    32          11            21  0.344
## 165  165                      Marist    31          12            19  0.387
## 166  166              Marquette NCAA    34          24            10  0.706
## 167  167                    Marshall    37          23            14  0.622
## 168  168   Maryland-Baltimore County    34          21            13  0.618
## 169  169      Maryland-Eastern Shore    32           7            25  0.219
## 170  170               Maryland NCAA    34          23            11  0.676
## 171  171        Massachusetts-Lowell    32          15            17  0.469
## 172  172               Massachusetts    32          11            21  0.344
## 173  173               McNeese State    31           9            22  0.290
## 174  174                     Memphis    36          22            14  0.611
## 175  175                      Mercer    31          11            20  0.355
## 176  176                  Miami (FL)    32          14            18  0.438
## 177  177                  Miami (OH)    32          15            17  0.469
## 178  178         Michigan State NCAA    39          32             7  0.821
## 179  179               Michigan NCAA    37          30             7  0.811
## 180  180            Middle Tennessee    32          11            21  0.344
## 181  181                   Milwaukee    31           9            22  0.290
## 182  182              Minnesota NCAA    36          22            14  0.611
## 183  183      Mississippi State NCAA    34          23            11  0.676
## 184  184    Mississippi Valley State    32           6            26  0.188
## 185  185            Mississippi NCAA    33          20            13  0.606
## 186  186        Missouri-Kansas City    32          11            21  0.344
## 187  187              Missouri State    32          16            16  0.500
## 188  188                    Missouri    32          15            17  0.469
## 189  189                    Monmouth    35          14            21  0.400
## 190  190               Montana State    32          15            17  0.469
## 191  191                Montana NCAA    35          26             9  0.743
## 192  192              Morehead State    33          13            20  0.394
## 193  193                Morgan State    30           9            21  0.300
## 194  194            Mount St. Mary's    31           9            22  0.290
## 195  195           Murray State NCAA    33          28             5  0.848
## 196  196                        Navy    31          12            19  0.387
## 197  197                       Omaha    32          21            11  0.656
## 198  198                    Nebraska    36          19            17  0.528
## 199  199            Nevada-Las Vegas    31          17            14  0.548
## 200  200                 Nevada NCAA    34          29             5  0.853
## 201  201               New Hampshire    29           5            24  0.172
## 202  202       New Mexico State NCAA    35          30             5  0.857
## 203  203                  New Mexico    32          14            18  0.438
## 204  204                 New Orleans    33          19            14  0.576
## 205  205                     Niagara    32          13            19  0.406
## 206  206              Nicholls State    31          14            17  0.452
## 207  207                        NJIT    35          22            13  0.629
## 208  208               Norfolk State    36          22            14  0.611
## 209  209               North Alabama    32          10            22  0.313
## 210  210    North Carolina-Asheville    31           4            27  0.129
## 211  211          North Carolina A&T    32          19            13  0.594
## 212  212 North Carolina Central NCAA    34          18            16  0.529
## 213  213   North Carolina-Greensboro    36          29             7  0.806
## 214  214        North Carolina State    36          24            12  0.667
## 215  215   North Carolina-Wilmington    33          10            23  0.303
## 216  216         North Carolina NCAA    36          29             7  0.806
## 217  217     North Dakota State NCAA    35          19            16  0.543
## 218  218                North Dakota    30          12            18  0.400
## 219  219               North Florida    33          16            17  0.485
## 220  220                 North Texas    33          21            12  0.636
## 221  221           Northeastern NCAA    34          23            11  0.676
## 222  222            Northern Arizona    31          10            21  0.323
## 223  223           Northern Colorado    32          21            11  0.656
## 224  224           Northern Illinois    34          17            17  0.500
## 225  225               Northern Iowa    34          16            18  0.471
## 226  226      Northern Kentucky NCAA    35          26             9  0.743
## 227  227          Northwestern State    31          11            20  0.355
## 228  228                Northwestern    32          13            19  0.406
## 229  229                  Notre Dame    33          14            19  0.424
## 230  230                     Oakland    33          16            17  0.485
## 231  231             Ohio State NCAA    35          20            15  0.571
## 232  232                        Ohio    31          14            17  0.452
## 233  233              Oklahoma State    32          12            20  0.375
## 234  234               Oklahoma NCAA    34          20            14  0.588
## 235  235           Old Dominion NCAA    35          26             9  0.743
## 236  236                Oral Roberts    32          11            21  0.344
## 237  237                Oregon State    31          18            13  0.581
## 238  238                 Oregon NCAA    38          25            13  0.658
## 239  239                     Pacific    32          14            18  0.438
## 240  240                  Penn State    32          14            18  0.438
## 241  241                Pennsylvania    31          19            12  0.613
## 242  242                  Pepperdine    34          16            18  0.471
## 243  243                  Pittsburgh    33          14            19  0.424
## 244  244              Portland State    32          16            16  0.500
## 245  245                    Portland    32           7            25  0.219
## 246  246           Prairie View NCAA    35          22            13  0.629
## 247  247                Presbyterian    36          20            16  0.556
## 248  248                   Princeton    28          16            12  0.571
## 249  249                  Providence    34          18            16  0.529
## 250  250                 Purdue NCAA    36          26            10  0.722
## 251  251                  Quinnipiac    31          16            15  0.516
## 252  252                     Radford    33          22            11  0.667
## 253  253                Rhode Island    33          18            15  0.545
## 254  254                        Rice    32          13            19  0.406
## 255  255                    Richmond    33          13            20  0.394
## 256  256                       Rider    31          16            15  0.516
## 257  257               Robert Morris    35          18            17  0.514
## 258  258                     Rutgers    31          14            17  0.452
## 259  259            Sacramento State    31          15            16  0.484
## 260  260                Sacred Heart    32          15            17  0.469
## 261  261          Saint Francis (PA)    33          18            15  0.545
## 262  262              Saint Joseph's    33          14            19  0.424
## 263  263            Saint Louis NCAA    36          23            13  0.639
## 264  264      Saint Mary's (CA) NCAA    34          22            12  0.647
## 265  265               Saint Peter's    32          10            22  0.313
## 266  266           Sam Houston State    33          21            12  0.636
## 267  267                     Samford    33          17            16  0.515
## 268  268             San Diego State    34          21            13  0.618
## 269  269                   San Diego    36          21            15  0.583
## 270  270               San Francisco    31          21            10  0.677
## 271  271              San Jose State    31           4            27  0.129
## 272  272                 Santa Clara    31          16            15  0.516
## 273  273              Savannah State    31          11            20  0.355
## 274  274                     Seattle    33          18            15  0.545
## 275  275             Seton Hall NCAA    34          20            14  0.588
## 276  276                       Siena    33          17            16  0.515
## 277  277               South Alabama    34          17            17  0.500
## 278  278        South Carolina State    34           8            26  0.235
## 279  279      South Carolina Upstate    32           6            26  0.188
## 280  280              South Carolina    32          16            16  0.500
## 281  281          South Dakota State    33          24             9  0.727
## 282  282                South Dakota    30          13            17  0.433
## 283  283               South Florida    38          24            14  0.632
## 284  284    Southeast Missouri State    31          10            21  0.323
## 285  285      Southeastern Louisiana    33          17            16  0.515
## 286  286         Southern California    33          16            17  0.485
## 287  287            SIU Edwardsville    31          10            21  0.323
## 288  288           Southern Illinois    32          17            15  0.531
## 289  289          Southern Methodist    32          15            17  0.469
## 290  290        Southern Mississippi    33          20            13  0.606
## 291  291               Southern Utah    34          17            17  0.500
## 292  292                    Southern    32           7            25  0.219
## 293  293             St. Bonaventure    34          18            16  0.529
## 294  294            St. Francis (NY)    33          17            16  0.515
## 295  295        St. John's (NY) NCAA    34          21            13  0.618
## 296  296                    Stanford    31          15            16  0.484
## 297  297           Stephen F. Austin    30          14            16  0.467
## 298  298                     Stetson    31           7            24  0.226
## 299  299                 Stony Brook    33          24             9  0.727
## 300  300               Syracuse NCAA    34          20            14  0.588
## 301  301                 Temple NCAA    33          23            10  0.697
## 302  302            Tennessee-Martin    31          12            19  0.387
## 303  303             Tennessee State    30           9            21  0.300
## 304  304              Tennessee Tech    31           8            23  0.258
## 305  305              Tennessee NCAA    37          31             6  0.838
## 306  306    Texas A&M-Corpus Christi    32          14            18  0.438
## 307  307                   Texas A&M    32          14            18  0.438
## 308  308             Texas-Arlington    33          17            16  0.515
## 309  309             Texas Christian    37          23            14  0.622
## 310  310               Texas-El Paso    29           8            21  0.276
## 311  311     Texas-Rio Grande Valley    37          20            17  0.541
## 312  312           Texas-San Antonio    32          17            15  0.531
## 313  313              Texas Southern    38          24            14  0.632
## 314  314                 Texas State    34          24            10  0.706
## 315  315             Texas Tech NCAA    38          31             7  0.816
## 316  316                       Texas    37          21            16  0.568
## 317  317                      Toledo    33          25             8  0.758
## 318  318                      Towson    32          10            22  0.313
## 319  319                        Troy    30          12            18  0.400
## 320  320                      Tulane    31           4            27  0.129
## 321  321                       Tulsa    32          18            14  0.563
## 322  322                        UCLA    33          17            16  0.515
## 323  323             Utah State NCAA    35          28             7  0.800
## 324  324                 Utah Valley    35          25            10  0.714
## 325  325                        Utah    31          17            14  0.548
## 326  326                  Valparaiso    33          15            18  0.455
## 327  327                  Vanderbilt    32           9            23  0.281
## 328  328                Vermont NCAA    34          27             7  0.794
## 329  329              Villanova NCAA    36          26            10  0.722
## 330  330  Virginia Commonwealth NCAA    33          25             8  0.758
## 331  331                         VMI    32          11            21  0.344
## 332  332          Virginia Tech NCAA    35          26             9  0.743
## 333  333               Virginia NCAA    38          35             3  0.921
## 334  334                      Wagner    30          13            17  0.433
## 335  335                 Wake Forest    31          11            20  0.355
## 336  336            Washington State    32          11            21  0.344
## 337  337             Washington NCAA    36          27             9  0.750
## 338  338                 Weber State    33          18            15  0.545
## 339  339               West Virginia    36          15            21  0.417
## 340  340            Western Carolina    32           7            25  0.219
## 341  341            Western Illinois    31          10            21  0.323
## 342  342            Western Kentucky    34          20            14  0.588
## 343  343            Western Michigan    32           8            24  0.250
## 344  344               Wichita State    37          22            15  0.595
## 345  345              William & Mary    31          14            17  0.452
## 346  346                    Winthrop    30          18            12  0.600
## 347  347              Wisconsin NCAA    34          23            11  0.676
## 348  348                Wofford NCAA    35          30             5  0.857
## 349  349                Wright State    35          21            14  0.600
## 350  350                     Wyoming    32           8            24  0.250
## 351  351                      Xavier    35          19            16  0.543
## 352  352                   Yale NCAA    30          22             8  0.733
## 353  353            Youngstown State    32          12            20  0.375
##     OverallSRS OverallSOS ConferenceWins ConferenceLosses HomeWins HomeLosses
## 1        -1.91      -7.34             14                4       13          2
## 2        -4.28       0.24              8               10        9          6
## 3         4.86       1.09              8               10       14          3
## 4       -19.23      -8.38              4               14        4          7
## 5         0.36      -1.52             10                8       11          5
## 6       -15.60      -7.84              9                9        8          3
## 7         9.45       9.01              8               10       10          6
## 8        -9.38      -6.70              7                9        6          8
## 9       -22.08      -8.97              6               12       10          3
## 10       -4.19      -7.23              9                9        8          7
## 11       -3.73       0.10              6               12        9          5
## 12       10.28       6.04             12                6       13          3
## 13        8.32       6.32              8               10       12          5
## 14       -4.87      -2.07              5               13        7          9
## 15      -14.43      -8.18             10                8        8          2
## 16       -7.10      -1.23              7               11       10          4
## 17       11.75       8.78              8               10       12          6
## 18       -7.57      -4.73              8               10       10          4
## 19       20.84      10.92             11                7       15          2
## 20        0.59      -4.41             13                5       10          2
## 21        3.39       1.21              6               12        7          7
## 22       13.38       9.26             10                8       13          5
## 23        9.12      -2.60             16                2       13          1
## 24      -11.98      -9.74              9                7       11          4
## 25      -13.92      -4.69              5               11        5         11
## 26        3.61       1.08              7               11        8          7
## 27        5.83       7.76              5               13       10          8
## 28       -6.61      -5.39              7               11        7          7
## 29        4.24       0.86             12                6       14          2
## 30       -0.08      -0.90              9                9       10          6
## 31        6.15       3.31             11                5       13          3
## 32       -0.62      -3.29              7                7       13          3
## 33      -15.19      -7.66              7               11        8          6
## 34        0.59      -2.93             13                5       12          3
## 35       15.56       2.62             16                2       14          0
## 36        9.22       8.10              7               11       12          4
## 37      -13.95      -3.54              2               14        4          8
## 38       -5.12      -2.63              7                9        9          4
## 39       -3.29      -1.14             10                6        9          4
## 40       -6.54      -3.39              7                9        7          9
## 41       -3.89      -4.12              7                9        8          7
## 42       -6.26      -1.50              7                9        7          6
## 43        5.70      -2.30             15                1       12          2
## 44      -11.12      -3.19              4               12        7          7
## 45       -1.62      -5.55             10                6       12          3
## 46       -3.16       5.42              3               15        7          9
## 47       -3.62      -4.39             12                4       12          4
## 48       -8.79      -4.85             11                7        5          8
## 49      -11.37      -4.81              8               10        8          5
## 50      -14.02      -6.71              5               13        5          7
## 51       13.37       5.58             13                5       15          2
## 52        2.79      -0.34             10                8       13          4
## 53       -2.43      -4.19              9                7       12          4
## 54       -8.34      -0.89              5               13        5          9
## 55       -7.87      -0.76              7               11        8          6
## 56      -24.83       0.67              0               16        3          8
## 57       14.53       5.50             14                4       16          2
## 58       -7.94       0.03              4               14        8          7
## 59       13.85       8.99              9                9       14          5
## 60       -7.96      -1.52              5               13        8          9
## 61       -0.50      -1.94              9                9       10          4
## 62        1.23      -3.83             13                5       15          1
## 63        2.36      -2.95             12                6       13          2
## 64       -0.11       1.41              7               11        8          9
## 65        9.68       3.60             10                8       15          2
## 66       -5.18      -2.14              5                9        5          7
## 67        6.81       4.32              6               12       13          5
## 68      -18.90      -7.11              7                9        3          7
## 69       -6.12      -2.02              7                7        9          4
## 70       12.00       8.59              9                9       13          6
## 71       -5.75      -3.00              2               12        8          6
## 72        6.38       1.96             14                4       14          2
## 73        9.67       2.91             13                5       13          4
## 74      -26.82     -10.13              2               14        3          9
## 75       -7.91      -4.82              8               10        9          7
## 76      -11.84      -2.97              3               13        6          7
## 77        6.00       4.05              7               11       16          7
## 78       -6.33      -0.36              8               10        6          6
## 79        2.52      -0.66             12                6       13          2
## 80       -7.02      -2.93              7               11        9          7
## 81       26.90      11.98             14                4       15          2
## 82        0.53      -0.63             10                8       14          4
## 83       -5.49       1.31              3               15        8          9
## 84        5.21      -1.79             13                5       13          3
## 85      -11.81      -5.01              7               11        7          6
## 86       -7.40      -2.47              6               12        9          5
## 87        0.40       4.43              9                9       11          7
## 88       -6.77      -4.17             12                8        9          4
## 89      -11.53      -3.56              7               11        5         11
## 90       -3.84       0.13              5               13        9          7
## 91      -10.20      -7.14              6               12        5          7
## 92       -6.09      -7.24             12                6       13          4
## 93      -13.57      -8.09              9                7        6          3
## 94       -1.42      -1.76              8               10        9          5
## 95       -5.11      -1.48              9                7       10          4
## 96       -4.55      -1.45             10                8       12          4
## 97       17.99      10.26             13                5       15          1
## 98       15.42      11.22              9                9        9          6
## 99       -5.02      -2.40              3               15        9         10
## 100       8.99       0.67             13                5       13          4
## 101       7.46      -1.50             13                5       13          3
## 102      -2.61      -4.43             11                6       13          0
## 103       0.84       0.08             11                7       11          6
## 104      -6.65       1.20              4               14        6         11
## 105       6.80       5.31              9                9       13          6
## 106       3.75       0.13             12                6       10          4
## 107       2.44       0.65             13                5       13          1
## 108       6.93       8.15              6               12       11          7
## 109       5.31       8.12              2               16        8          9
## 110      27.79       5.01             16                0       17          0
## 111      -9.73      -9.34             10                8       11          3
## 112       3.85      -1.49             10                6       12          2
## 113      -2.24      -0.36             10                8       15          3
## 114      -3.34      -5.06              9                7       12          3
## 115      -3.95      -4.89             10                6       10          4
## 116       1.86       0.66             10                4        9          2
## 117      -1.30      -3.61              9                7       12          5
## 118      -5.63      -4.39              9                7        9          4
## 119       4.68      -4.53             15                3       15          1
## 120      -6.43      -4.01              6               12        8          6
## 121      -9.36      -5.53              8               10        9          4
## 122      18.91       4.61             16                2       19          1
## 123     -12.30      -9.52             10                6        6          8
## 124     -13.17      -4.81              7               13        6          7
## 125     -18.74      -6.19              2               18        4         11
## 126      -3.15      -1.99             10                8       12          4
## 127      -2.06       0.25              9                9       12          4
## 128       8.95      11.53              7               13        9          6
## 129     -18.93      -5.11              1               17        5          9
## 130      -2.72       0.62              7               11        9          5
## 131      13.82      10.10              8               12       15          6
## 132      -4.78      -5.50             12                6        8          3
## 133      18.07       9.30              9                9       12          4
## 134      14.27       9.84             10               10       14          4
## 135      -3.47      -3.60              9                7       11          5
## 136      -2.72      -3.04              8               10       11          4
## 137     -14.60      -9.33             10                8        9          4
## 138       1.59      -5.47             15                3       11          1
## 139      -8.27      -4.74              5               11        5          8
## 140      -8.52      -4.07              6               12        8          6
## 141      15.39       9.18             14                4       13          2
## 142      18.35      12.79             12                6       16          0
## 143     -16.17      -1.04              3               13        6          8
## 144       1.26       0.61             11                7       14          3
## 145      21.43      10.29             15                3       17          1
## 146      -3.61       1.29              8               10        5          9
## 147     -11.17      -4.93              7               11        4         11
## 148      -5.94      -7.22             12                6       13          2
## 149      -2.34      -4.46             12                6       11          3
## 150       5.27      -3.88             14                2       16          1
## 151       9.21      -1.20             14                2       14          3
## 152      -4.57      -0.44              8                8        9          5
## 153      -8.99      -9.08              9                9        8          5
## 154      -8.43      -6.62              5               11       10          5
## 155      -2.07      -1.51             10                8       10          4
## 156       0.81      -1.96              9                9       14          3
## 157      16.50       9.16             16                2       15          2
## 158       0.95      -2.31              9                9       15          1
## 159      17.28      11.04             10                8       14          4
## 160       3.99      -0.16             12                6       13          4
## 161       2.93       0.90              8                8       12          4
## 162      -9.13      -4.60              7               11        7          5
## 163     -15.11      -4.21              3               13        3          9
## 164     -12.89      -7.32              8               10        4          9
## 165      -9.49      -7.53              7               11        4          7
## 166      14.78       6.96             12                6       16          3
## 167      -0.64      -0.61             11                7       16          3
## 168      -5.85      -5.85             11                5       13          4
## 169     -24.21      -7.91              5               11        5          7
## 170      16.01      10.09             13                7       15          3
## 171      -8.28      -6.95              7                9        8          5
## 172      -3.02      -0.18              4               14        9          8
## 173     -14.41      -6.15              5               13        7          8
## 174      10.70       5.09             11                7       17          2
## 175      -2.86      -0.24              6               12        9          6
## 176       9.39       8.70              5               13       11          5
## 177       0.77       2.31              7               11       10          5
## 178      24.93      12.34             16                4       15          1
## 179      21.82      10.55             15                5       17          1
## 180      -6.09       1.44              8               10        9          5
## 181      -8.90      -2.50              4               14        6          8
## 182      12.52      11.27              9               11       14          3
## 183      15.96       9.07             10                8       14          3
## 184     -22.47      -6.99              4               14        6          6
## 185      12.32       8.13             10                8       11          5
## 186      -6.61      -1.55              6               10        8          5
## 187      -1.20      -1.03             10                8       11          4
## 188       8.60       9.16              5               13        9          7
## 189     -10.66      -4.83             10                8        6          6
## 190      -7.91      -5.81             11                9        9          4
## 191       1.50      -5.05             16                4       12          2
## 192      -7.73      -1.76              8               10        7          7
## 193     -15.98     -10.70              4               12        6          7
## 194     -14.57      -6.02              6               12        4          9
## 195       8.96      -3.11             16                2       15          1
## 196     -10.09      -4.00              8               10        8          6
## 197      -2.32      -3.39             13                3       10          2
## 198      14.85      11.31              6               14       13          5
## 199       1.58       0.48             11                7       10          6
## 200      16.00       2.68             15                3       15          0
## 201     -18.66      -6.03              3               13        4         10
## 202      10.05      -2.38             15                1       16          1
## 203      -0.55       1.61              7               11        9          7
## 204      -8.88      -6.47             12                6       12          4
## 205     -11.33      -7.77              6               12        8          8
## 206     -11.87      -6.98              7               11        9          4
## 207      -3.63      -4.74              8                8       11          5
## 208      -8.04      -8.74             14                2       11          2
## 209     -11.00      -1.69              7                9        8          4
## 210     -19.81      -2.35              2               14        3         10
## 211     -11.24      -9.88             13                3       11          2
## 212     -11.53     -11.24             10                6       10          2
## 213       4.08      -0.90             15                3       15          2
## 214      14.94       6.22              9                9       17          5
## 215      -8.10      -2.04              5               13        5          9
## 216      23.94      11.35             16                2       14          2
## 217      -4.01      -2.07              9                7       10          3
## 218      -8.78      -3.55              6               10        8          6
## 219      -3.47      -0.70              9                7       10          3
## 220      -0.25      -3.48              8               10       12          4
## 221       3.95      -0.70             14                4       11          2
## 222     -10.75      -6.23              8               12        5          7
## 223      -3.92      -6.76             15                5       10          3
## 224       2.58       1.11              8               10       10          6
## 225      -1.54       0.62              9                9        9          5
## 226       4.67      -2.39             13                5       17          1
## 227     -17.06      -5.99              6               12        8          7
## 228       9.97       9.16              4               16       10          8
## 229       7.97       8.28              3               15       11          8
## 230      -2.27      -1.48             11                7       10          6
## 231      13.89      11.00              8               12       12          6
## 232      -1.84       3.16              6               12       11          5
## 233       8.09      11.53              5               13        8          7
## 234      15.30      12.21              7               11       11          4
## 235       3.74      -1.18             13                5       14          2
## 236      -9.34      -1.91              7                9        7          6
## 237       7.84       4.29             10                8       10          5
## 238      13.95       6.13             10                8       13          4
## 239      -2.46       1.97              4               12        9          7
## 240      12.55      11.51              7               13        9          6
## 241       2.04      -0.76              7                7       10          4
## 242       0.90       0.63              6               10        9          5
## 243       7.88       6.69              3               15       11          7
## 244      -8.95      -5.78             11                9       12          4
## 245     -10.80       0.51              0               16        6         11
## 246      -7.29      -9.17             17                1       11          0
## 247      -3.56      -5.35              9                7       12          3
## 248      -3.43      -0.89              8                6        7          5
## 249       8.29       6.71              7               11       11          7
## 250      21.40      11.99             16                4       15          0
## 251      -6.78      -8.03             11                7        7          7
## 252       1.28      -1.95             12                4       11          3
## 253       2.83       1.19              9                9        9          5
## 254      -6.56      -2.06              8               10        9          7
## 255      -1.37      -0.79              6               12        7         10
## 256      -4.96      -6.51             11                7        9          3
## 257      -9.41      -7.62             11                7       13          4
## 258       8.89       9.76              7               13       10          7
## 259      -8.78      -6.78              8               12        9          5
## 260      -7.88      -8.11             11                7       10          3
## 261      -8.74      -6.16             12                6       12          4
## 262      -0.43       1.66              6               12       10          5
## 263       5.06       2.23             10                8       15          2
## 264      12.58       4.40             11                5       14          3
## 265     -12.15      -6.40              6               12        7          7
## 266      -3.17      -5.84             16                2       12          2
## 267       0.69      -2.03              6               12       11          6
## 268       5.29       2.47             11                7       14          3
## 269       6.12       3.61              7                9       12          4
## 270       8.56       1.13              9                7       13          3
## 271     -16.37       0.59              1               17        4         11
## 272      -1.91       0.92              8                8       11          6
## 273     -20.51      -9.04              8                8        6          5
## 274      -1.47      -3.85              6               10       12          6
## 275      10.25       8.34              9                9       11          4
## 276      -7.90      -6.47             11                7        7          7
## 277      -4.46      -2.71              8               10       13          6
## 278     -16.86      -8.30              5               11        5          6
## 279     -15.30      -4.27              1               16        5          9
## 280       8.34       9.63             11                7       11          6
## 281       5.92      -4.74             14                2       14          1
## 282      -5.52      -3.06              7                9        8          6
## 283       5.96       1.69              8               10       18          5
## 284     -11.32      -4.39              5               13        6          8
## 285      -7.59      -6.30             12                6        9          5
## 286       8.23       4.96              8               10       12          5
## 287     -14.47      -5.06              6               12        7          8
## 288       1.39      -0.17             10                8        9          6
## 289       5.81       4.12              6               12       10          8
## 290       2.58      -0.83             11                7       11          3
## 291      -8.91      -6.22              9               11       11          4
## 292     -16.79      -7.37              6               12        6          5
## 293       3.34       0.25             12                6        9          5
## 294     -10.19      -8.45              9                9       11          3
## 295       7.88       5.50              8               10       11          4
## 296       6.03       5.13              8               10       10          4
## 297     -10.99      -5.92              7               11       10          6
## 298     -15.01      -3.46              3               13        7          8
## 299      -2.10      -6.48             12                4       10          4
## 300      13.73      10.09             10                8       13          6
## 301       8.17       4.93             13                5       13          2
## 302     -10.94      -4.32              6               12       10          4
## 303     -11.62      -3.15              6               12        6          7
## 304     -14.76      -3.00              4               14        5         10
## 305      21.55      10.16             15                3       18          0
## 306      -9.11      -5.68              9                9        9          7
## 307       7.21       8.59              6               12        9          8
## 308      -0.50       0.84             12                6        9          5
## 309      13.81       9.60              7               11       15          5
## 310      -8.23      -1.71              3               15        8          8
## 311      -2.04      -1.61              9                7       11          9
## 312       0.69       0.05             11                7       11          4
## 313      -5.55      -6.99             14                4       10          2
## 314       1.54      -3.00             12                6       11          4
## 315      22.79       9.53             14                4       17          1
## 316      16.06      11.46              8               10       15          6
## 317       8.21       0.71             13                5       14          2
## 318      -8.44      -3.57              6               12        5          8
## 319      -6.12      -1.01              5               13        8          8
## 320      -6.72       3.63              0               18        3         12
## 321       4.65       3.86              8               10       14          3
## 322       7.17       6.78              9                9       13          5
## 323      11.98       0.89             15                3       14          1
## 324       3.00      -2.30             12                4       14          1
## 325       6.13       5.10             11                7       10          5
## 326      -3.38      -1.00              7               11        8          7
## 327       3.57       7.79              0               18        8         10
## 328       4.88      -4.18             14                2       16          2
## 329      14.33       7.97             13                5       13          2
## 330      11.96       2.87             16                2       16          1
## 331     -10.50       0.09              4               14        8          7
## 332      19.28       7.79             12                6       14          2
## 333      25.46      10.15             16                2       15          1
## 334     -12.99      -7.63              8               10        7          7
## 335       1.21       8.47              4               14        8          8
## 336      -1.67       2.20              4               14        9          7
## 337      12.01       7.01             15                3       15          1
## 338      -4.37      -6.12             11                9       10          5
## 339       6.94      10.57              4               14       11          7
## 340      -9.28       0.34              4               14        4          8
## 341     -10.04      -5.18              4               12        7          6
## 342       3.14       0.96             11                7       10          4
## 343      -6.26       1.48              2               16        5          9
## 344       7.72       5.91             10                8       10          4
## 345      -4.78      -2.08             10                8        9          5
## 346      -2.89      -3.74             10                6       11          3
## 347      17.90      11.01             14                6       12          3
## 348      13.92       0.80             18                0       15          1
## 349       3.29      -0.89             13                5       15          2
## 350      -9.75       0.19              4               14        6         10
## 351       9.61       8.06              9                9       13          5
## 352       5.52      -1.24             10                4       11          2
## 353      -7.78      -2.05              8               10        6          7
##     AwayWins AwayLosses ForPoints OppPoints Blank Minutes FieldGoalsMade
## 1         10          4      2502      2161    NA    1370            897
## 2          3          9      2179      2294    NA    1300            802
## 3          1         10      2271      2107    NA    1325            797
## 4          0         18      1938      2285    NA    1295            736
## 5          6          6      2470      2370    NA    1410            906
## 6          3         13      2086      2235    NA    1250            712
## 7          4          8      2448      2433    NA    1365            856
## 8          6         10      2150      2216    NA    1300            728
## 9          0         17      1995      2194    NA    1245            709
## 10         6          8      2161      2070    NA    1220            780
## 11         2         12      2557      2539    NA    1290            880
## 12         6          5      2638      2494    NA    1380            899
## 13         4          7      2269      2205    NA    1290            794
## 14         3         12      2301      2353    NA    1250            817
## 15         4         17      2108      2269    NA    1300            719
## 16         2         13      2359      2475    NA    1310            796
## 17         5          8      2559      2458    NA    1370            893
## 18         3         13      2268      2305    NA    1280            837
## 19         4          6      3188      2750    NA    1615           1097
## 20         8          7      2712      2413    NA    1335            973
## 21         8          7      2478      2362    NA    1340            889
## 22         5          6      2442      2302    NA    1365            869
## 23        12          3      2868      2439    NA    1330           1042
## 24         3         12      2290      2245    NA    1250            833
## 25         5         12      2152      2376    NA    1320            793
## 26         3         10      2364      2263    NA    1330            845
## 27         2          8      2196      2256    NA    1260            768
## 28         8         10      2385      2387    NA    1325            922
## 29         5          8      2668      2496    NA    1370            945
## 30         5          8      2329      2285    NA    1405            812
## 31         5          8      2527      2436    NA    1295            878
## 32         7          9      2355      2204    NA    1290            805
## 33         2         13      2088      2314    NA    1200            709
## 34         7          8      2534      2418    NA    1325            867
## 35        12          3      3037      2550    NA    1445           1083
## 36         2         10      2374      2337    NA    1330            861
## 37         1         14      1930      2183    NA    1185            706
## 38         7         10      2412      2412    NA    1375            887
## 39         4         11      2437      2414    NA    1385            868
## 40         4         11      2601      2701    NA    1375            968
## 41         8          6      2403      2241    NA    1260            831
## 42         3         12      2038      2128    NA    1260            735
## 43        13          2      2675      2353    NA    1500            987
## 44         2         14      2153      2279    NA    1325            792
## 45         7          6      2350      2114    NA    1290            825
## 46         1         10      2121      2387    NA    1245            739
## 47         5          8      2495      2328    NA    1325            867
## 48         9          5      2250      2376    NA    1295            798
## 49         5         13      2383      2498    NA    1330            810
## 50         5         12      2205      2367    NA    1260            754
## 51         5          5      2385      2128    NA    1325            819
## 52         7          6      2896      2690    NA    1425            991
## 53         4         11      2574      2369    NA    1370            926
## 54         3          9      1776      1992    NA    1160            598
## 55         4         11      2268      2376    NA    1285            833
## 56         0         19      1970      2710    NA    1280            718
## 57         7          4      2510      2194    NA    1410            872
## 58         4         10      2553      2584    NA    1210            899
## 59         4          6      2339      2174    NA    1360            833
## 60         2         12      2300      2437    NA    1250            799
## 61         6         11      2641      2520    NA    1370            899
## 62         9          9      2648      2458    NA    1415            955
## 63         8          5      2453      2265    NA    1330            878
## 64         3          8      2393      2406    NA    1285            866
## 65         5          9      2648      2429    NA    1445            924
## 66         4          9      2059      2052    NA    1160            778
## 67         1          8      2436      2354    NA    1325            874
## 68         4         16      2118      2507    NA    1330            710
## 69         6         12      2149      2211    NA    1260            754
## 70         4          8      2727      2561    NA    1420            980
## 71         2         12      2160      2116    NA    1210            791
## 72         7          5      2409      2229    NA    1365            852
## 73         7          4      2406      2183    NA    1335            902
## 74         2         15      1956      2371    NA    1240            695
## 75         7          8      2352      2380    NA    1360            830
## 76         1         14      2086      2319    NA    1205            755
## 77         3          9      2821      2751    NA    1450           1009
## 78         4         14      2279      2464    NA    1240            784
## 79         7          6      2563      2387    NA    1385            908
## 80         3         11      2425      2475    NA    1280            874
## 81         7          2      3143      2576    NA    1525           1157
## 82         4          6      2351      2314    NA    1295            810
## 83         1         11      2088      2299    NA    1255            749
## 84         8          6      2707      2371    NA    1375           1004
## 85         5         10      2299      2457    NA    1310            826
## 86         3         11      2562      2577    NA    1265            902
## 87         4         10      2203      2213    NA    1295            818
## 88         4         13      2460      2527    NA    1380            876
## 89         6          7      2266      2445    NA    1290            812
## 90         2         13      2234      2317    NA    1295            747
## 91         3         13      2072      2167    NA    1240            761
## 92         7          9      2620      2515    NA    1420            921
## 93         6         14      1927      2057    NA    1245            719
## 94         6          9      2322      2243    NA    1335            809
## 95         4         10      2280      2360    NA    1280            809
## 96         6          9      2798      2734    NA    1360            986
## 97         6          4      2771      2485    NA    1500            960
## 98         5          6      2440      2289    NA    1455            857
## 99         3          8      2108      2141    NA    1295            759
## 100        7          3      2448      2162    NA    1285            847
## 101       11          4      2562      2182    NA    1340            930
## 102        8          9      2718      2468    NA    1430            955
## 103        6          6      2314      2289    NA    1330            825
## 104        2         10      2097      2356    NA    1340            745
## 105        5          6      2625      2576    NA    1355            894
## 106        7          7      2726      2489    NA    1330           1012
## 107        7          7      2598      2491    NA    1360            903
## 108        3          9      2091      2130    NA    1285            755
## 109        2          9      2274      2364    NA    1280            776
## 110        9          1      3243      2400    NA    1480           1177
## 111        6         12      2426      2336    NA    1370            834
## 112        5          8      2560      2353    NA    1370            907
## 113        5         13      3090      3036    NA    1535           1106
## 114        6         12      2847      2669    NA    1430            966
## 115        8          9      2477      2407    NA    1345            848
## 116        8          9      2222      2174    NA    1265            787
## 117        4          5      2241      2131    NA    1250            788
## 118        5          8      2115      2105    NA    1255            763
## 119       10          6      2919      2553    NA    1440            998
## 120        6         11      2216      2296    NA    1335            848
## 121        3         13      2465      2484    NA    1225            865
## 122       10          1      2787      2258    NA    1480            982
## 123        9          8      2684      2661    NA    1370            937
## 124        5         11      2198      2369    NA    1200            779
## 125        1         13      2174      2460    NA    1285            778
## 126        4         12      2391      2385    NA    1295            865
## 127        2         10      2257      2314    NA    1330            813
## 128        1          9      2398      2483    NA    1335            862
## 129        0         14      2044      2355    NA    1255            711
## 130        4          9      2146      2218    NA    1250            757
## 131        3          9      2504      2374    NA    1420            919
## 132        5          9      2532      2508    NA    1320            857
## 133        5          6      2692      2385    NA    1400            974
## 134        4          6      2740      2585    NA    1415            914
## 135        6          9      2728      2556    NA    1330            991
## 136        3         13      2507      2429    NA    1320            891
## 137        3         14      1974      2107    NA    1285            717
## 138       10          7      2574      2283    NA    1345            953
## 139        6         11      2385      2402    NA    1285            889
## 140        4         11      2322      2409    NA    1350            821
## 141        7          5      2236      2025    NA    1360            806
## 142        3          8      2725      2525    NA    1455            989
## 143        0         15      2030      2455    NA    1285            742
## 144        8          7      2481      2417    NA    1330            878
## 145        8          2      2806      2394    NA    1490            978
## 146        3          8      2091      2243    NA    1240            719
## 147        6          9      2209      2349    NA    1220            802
## 148        6         10      2597      2322    NA    1340            902
## 149        9          8      2479      2413    NA    1250            856
## 150       11          3      2654      2209    NA    1440            963
## 151       14          4      3076      2597    NA    1480           1071
## 152        4         12      2545      2584    NA    1375            877
## 153        7          9      2373      2337    NA    1285            822
## 154        5         12      2407      2408    NA    1375            803
## 155        7          7      2612      2575    NA    1290            901
## 156        4         12      2773      2637    NA    1425            922
## 157        9          1      2815      2558    NA    1435            988
## 158        4         10      2386      2235    NA    1335            840
## 159        5          6      2536      2324    NA    1380            854
## 160        5          7      2231      2066    NA    1360            831
## 161        8          7      2278      2125    NA    1360            817
## 162        3         16      2332      2477    NA    1295            842
## 163        2         18      1988      2271    NA    1310            747
## 164        4         11      1833      2011    NA    1285            651
## 165        6         11      2086      2147    NA    1250            739
## 166        6          4      2629      2363    NA    1375            893
## 167        6         10      2978      2979    NA    1495           1047
## 168        6          8      2296      2177    NA    1400            820
## 169        2         17      1833      2291    NA    1285            666
## 170        6          5      2429      2228    NA    1360            858
## 171        5         12      2456      2420    NA    1290            875
## 172        1         11      2263      2354    NA    1290            808
## 173        2         14      2149      2303    NA    1245            760
## 174        3          8      2882      2680    NA    1450           1013
## 175        2         12      2248      2246    NA    1245            787
## 176        0         10      2297      2275    NA    1285            805
## 177        4         10      2267      2233    NA    1285            777
## 178        8          4      3025      2534    NA    1570           1071
## 179        7          4      2575      2158    NA    1480            941
## 180        2         11      2141      2313    NA    1280            778
## 181        3         12      2146      2315    NA    1245            765
## 182        2          9      2543      2498    NA    1445            887
## 183        5          5      2628      2394    NA    1370            936
## 184        0         20      2058      2498    NA    1285            730
## 185        6          5      2485      2347    NA    1325            876
## 186        3         13      2220      2341    NA    1290            774
## 187        5          9      2206      2157    NA    1285            774
## 188        2          8      2150      2168    NA    1290            758
## 189        5         11      2286      2490    NA    1420            794
## 190        4         12      2502      2552    NA    1280            858
## 191        9          5      2665      2397    NA    1405            975
## 192        4         11      2431      2543    NA    1335            855
## 193        2         13      2161      2304    NA    1210            758
## 194        5         13      2092      2286    NA    1240            733
## 195       10          3      2727      2255    NA    1320            988
## 196        4         13      2061      2250    NA    1240            732
## 197        9          8      2523      2454    NA    1290            950
## 198        2         10      2586      2421    NA    1445            916
## 199        6          5      2276      2242    NA    1250            791
## 200        9          3      2723      2270    NA    1360            924
## 201        1         14      1742      1990    NA    1160            623
## 202       10          1      2734      2259    NA    1405            962
## 203        4          9      2443      2454    NA    1280            814
## 204        5          9      2384      2325    NA    1345            862
## 205        5         10      2318      2432    NA    1285            795
## 206        3         13      2260      2302    NA    1250            807
## 207       11          8      2472      2401    NA    1410            849
## 208        9          8      2652      2503    NA    1460            900
## 209        2         18      2145      2358    NA    1285            744
## 210        1         15      1857      2313    NA    1245            655
## 211        7         10      2241      2209    NA    1290            825
## 212        4         12      2429      2266    NA    1365            874
## 213       11          4      2738      2443    NA    1445            997
## 214        4          6      2882      2568    NA    1450           1061
## 215        2         12      2515      2670    NA    1335            880
## 216       11          1      3089      2636    NA    1445           1118
## 217        4          9      2556      2541    NA    1410            874
## 218        4         11      2220      2196    NA    1200            822
## 219        5         13      2485      2531    NA    1325            902
## 220        6          7      2302      2077    NA    1325            831
## 221        8          6      2564      2406    NA    1385            880
## 222        5         13      2262      2402    NA    1250            785
## 223       11          5      2450      2229    NA    1295            849
## 224        5          9      2506      2413    NA    1375            940
## 225        3          9      2215      2230    NA    1360            769
## 226        7          7      2747      2410    NA    1410            979
## 227        3         13      2051      2280    NA    1250            725
## 228        1          9      2108      2082    NA    1290            735
## 229        1          9      2266      2276    NA    1320            776
## 230        6         10      2534      2515    NA    1325            864
## 231        4          7      2419      2318    NA    1405            839
## 232        3         10      2174      2288    NA    1255            803
## 233        2         10      2178      2288    NA    1285            760
## 234        5          7      2423      2318    NA    1365            883
## 235        8          4      2300      2128    NA    1400            823
## 236        3         13      2314      2483    NA    1285            854
## 237        6          5      2275      2165    NA    1250            800
## 238        5          7      2661      2364    NA    1530            958
## 239        5         10      2138      2241    NA    1290            710
## 240        3          9      2229      2196    NA    1295            791
## 241        8          5      2250      2117    NA    1265            821
## 242        2         11      2572      2529    NA    1370            865
## 243        0         11      2306      2267    NA    1330            769
## 244        3         10      2475      2404    NA    1290            856
## 245        1         12      2078      2389    NA    1290            707
## 246        9         12      2627      2547    NA    1400            903
## 247        8         12      2817      2610    NA    1445            966
## 248        8          6      1948      1951    NA    1135            685
## 249        5          6      2428      2374    NA    1385            835
## 250        6          6      2760      2421    NA    1460            967
## 251        9          6      2294      2255    NA    1270            739
## 252        9          7      2440      2247    NA    1330            893
## 253        5          7      2293      2239    NA    1335            821
## 254        4         11      2367      2481    NA    1305            795
## 255        4          7      2314      2333    NA    1320            849
## 256        6         10      2381      2333    NA    1245            847
## 257        5         13      2448      2403    NA    1420            861
## 258        4          9      2105      2132    NA    1250            771
## 259        4         10      2197      2155    NA    1250            806
## 260        4         13      2558      2509    NA    1285            885
## 261        6         11      2517      2510    NA    1320            885
## 262        2         11      2328      2397    NA    1325            793
## 263        4          8      2399      2297    NA    1440            855
## 264        5          5      2463      2185    NA    1365            899
## 265        2         14      2012      2196    NA    1305            707
## 266        8          9      2476      2307    NA    1345            889
## 267        5          9      2583      2487    NA    1355            921
## 268        4          7      2437      2291    NA    1360            845
## 269        5          9      2591      2455    NA    1460            914
## 270        6          5      2358      2101    NA    1250            864
## 271        0         12      2042      2534    NA    1250            713
## 272        4          6      2115      2172    NA    1255            735
## 273        4         14      2360      2694    NA    1245            825
## 274        4          8      2420      2283    NA    1330            852
## 275        3          8      2508      2443    NA    1375            888
## 276        8          7      2131      2178    NA    1345            772
## 277        3         10      2478      2465    NA    1375            867
## 278        2         19      2388      2593    NA    1375            811
## 279        1         15      2169      2395    NA    1290            775
## 280        4          8      2327      2316    NA    1285            806
## 281        7          6      2789      2425    NA    1320            984
## 282        4          8      2118      2136    NA    1205            729
## 283        5          7      2752      2553    NA    1555            911
## 284        2         13      2216      2378    NA    1265            796
## 285        7          9      2224      2228    NA    1320            765
## 286        2          9      2518      2410    NA    1340            937
## 287        2         11      2265      2520    NA    1280            794
## 288        7          7      2194      2144    NA    1280            802
## 289        3          7      2303      2249    NA    1280            833
## 290        7          7      2392      2160    NA    1330            899
## 291        4         12      2573      2566    NA    1380            881
## 292        1         18      2077      2360    NA    1285            752
## 293        6          7      2250      2145    NA    1385            821
## 294        6         12      2345      2320    NA    1325            841
## 295        4          7      2623      2542    NA    1375            944
## 296        4          9      2255      2227    NA    1245            808
## 297        3          9      2105      2161    NA    1210            745
## 298        0         16      2151      2429    NA    1245            794
## 299       13          4      2364      2188    NA    1335            815
## 300        6          4      2370      2246    NA    1365            808
## 301        8          5      2465      2358    NA    1340            873
## 302        1         14      2356      2486    NA    1250            854
## 303        3         14      2182      2332    NA    1210            751
## 304        3         13      2104      2350    NA    1250            742
## 305        7          3      3035      2580    NA    1505           1106
## 306        5          9      2126      2124    NA    1290            774
## 307        3          7      2253      2297    NA    1280            806
## 308        7         10      2284      2303    NA    1345            781
## 309        3          7      2730      2574    NA    1500            990
## 310        0         13      1859      2001    NA    1170            639
## 311        8          7      2634      2623    NA    1490            895
## 312        5          7      2481      2407    NA    1295            896
## 313       13         11      3102      3008    NA    1550           1083
## 314       10          5      2470      2204    NA    1370            876
## 315        6          3      2765      2261    NA    1530            990
## 316        2          8      2628      2458    NA    1495            929
## 317        8          5      2537      2256    NA    1330            901
## 318        4          8      2170      2294    NA    1310            810
## 319        4         10      2179      2242    NA    1210            773
## 320        0         11      2082      2403    NA    1240            733
## 321        3          8      2296      2271    NA    1290            776
## 322        3          7      2580      2567    NA    1335            919
## 323        9          4      2753      2349    NA    1410            959
## 324        8          8      2697      2474    NA    1400            930
## 325        6          5      2343      2311    NA    1245            805
## 326        5          8      2205      2187    NA    1335            802
## 327        1         10      2180      2315    NA    1285            736
## 328       11          4      2508      2143    NA    1370            864
## 329        5          6      2654      2425    NA    1455            884
## 330        8          4      2344      2044    NA    1330            820
## 331        2         13      2433      2617    NA    1290            843
## 332        5          5      2574      2172    NA    1410            893
## 333       10          1      2714      2132    NA    1535            974
## 334        6         10      1974      2044    NA    1205            655
## 335        1         10      2119      2344    NA    1250            714
## 336        2          8      2393      2517    NA    1280            833
## 337        7          4      2511      2331    NA    1445            883
## 338        5          7      2617      2454    NA    1330            920
## 339        0         10      2655      2786    NA    1460            901
## 340        3         14      2304      2563    NA    1295            792
## 341        2         14      2198      2272    NA    1245            818
## 342        6          8      2429      2355    NA    1380            853
## 343        2         14      2246      2439    NA    1300            774
## 344        7          7      2609      2542    NA    1485            929
## 345        5         11      2321      2385    NA    1265            841
## 346        7          8      2499      2358    NA    1210            855
## 347        8          5      2333      2099    NA    1385            874
## 348       11          3      2879      2295    NA    1410           1044
## 349        5          8      2561      2365    NA    1405            885
## 350        1         11      2105      2411    NA    1285            690
## 351        4          8      2526      2472    NA    1420            922
## 352        8          5      2427      2202    NA    1215            893
## 353        5         11      2415      2526    NA    1305            879
##     FieldGoalsAttempted FieldGoalPCT ThreePointMade ThreePointAttempts
## 1                  1911        0.469            251                660
## 2                  1776        0.452            234                711
## 3                  1948        0.409            297                929
## 4                  1809        0.407            182                578
## 5                  2003        0.452            234                694
## 6                  1764        0.404            216                673
## 7                  1945        0.440            244                718
## 8                  1750        0.416            274                790
## 9                  1721        0.412            211                646
## 10                 1645        0.474            190                584
## 11                 1938        0.454            292                814
## 12                 2012        0.447            240                714
## 13                 1861        0.427            235                699
## 14                 1687        0.484            206                582
## 15                 1685        0.427            194                574
## 16                 1904        0.418            237                709
## 17                 2004        0.446            259                751
## 18                 1956        0.428            272                863
## 19                 2439        0.450            454               1204
## 20                 2060        0.472            268                702
## 21                 1892        0.470            202                621
## 22                 1966        0.442            274                803
## 23                 2094        0.498            343                922
## 24                 1899        0.439            204                635
## 25                 1842        0.431            275                788
## 26                 1801        0.469            252                715
## 27                 1808        0.425            218                687
## 28                 1922        0.480            221                639
## 29                 2134        0.443            272                762
## 30                 1874        0.433            240                653
## 31                 1878        0.468            226                684
## 32                 1818        0.443            255                745
## 33                 1680        0.422            230                708
## 34                 1928        0.450            311                885
## 35                 2344        0.462            344               1022
## 36                 1934        0.445            292                827
## 37                 1715        0.412            225                687
## 38                 2092        0.424            221                683
## 39                 1911        0.454            187                589
## 40                 2103        0.460            234                648
## 41                 1859        0.447            287                800
## 42                 1673        0.439            203                619
## 43                 2159        0.457            252                701
## 44                 1787        0.443            279                743
## 45                 1819        0.454            204                594
## 46                 1735        0.426            207                592
## 47                 1906        0.455            313                912
## 48                 1834        0.435            256                791
## 49                 1903        0.426            276                784
## 50                 1821        0.414            201                612
## 51                 1763        0.465            229                628
## 52                 2177        0.455            305                822
## 53                 2156        0.429            324                945
## 54                 1442        0.415            191                615
## 55                 1881        0.443            290                795
## 56                 1752        0.410            141                471
## 57                 2018        0.432            232                673
## 58                 2011        0.447            370               1076
## 59                 1857        0.449            217                661
## 60                 1834        0.436            303                820
## 61                 1986        0.453            284                766
## 62                 1995        0.479            320                815
## 63                 1818        0.483            227                667
## 64                 1818        0.476            259                722
## 65                 2036        0.454            230                711
## 66                 1720        0.452            237                655
## 67                 1959        0.446            252                732
## 68                 1827        0.389            240                841
## 69                 1705        0.442            243                731
## 70                 2044        0.479            372                961
## 71                 1750        0.452            264                721
## 72                 1893        0.450            312                882
## 73                 1789        0.504            205                617
## 74                 1963        0.354            234                789
## 75                 1808        0.459            282                746
## 76                 1723        0.438            227                622
## 77                 2143        0.471            241                703
## 78                 1874        0.418            294                825
## 79                 1943        0.467            279                773
## 80                 1900        0.460            237                669
## 81                 2418        0.478            278                903
## 82                 1894        0.428            271                841
## 83                 1791        0.418            155                545
## 84                 2067        0.486            274                758
## 85                 1912        0.432            272                735
## 86                 2125        0.424            272                832
## 87                 1823        0.449            187                634
## 88                 2040        0.429            310                886
## 89                 1862        0.436            335                943
## 90                 1776        0.421            264                776
## 91                 1755        0.434            259                740
## 92                 1939        0.475            269                672
## 93                 1640        0.438            180                489
## 94                 1971        0.410            290                895
## 95                 1787        0.453            269                724
## 96                 2239        0.440            279                923
## 97                 2171        0.442            272                819
## 98                 2015        0.425            291                872
## 99                 1890        0.402            276                831
## 100                1866        0.454            342                897
## 101                1962        0.474            338                935
## 102                1962        0.487            281                719
## 103                1855        0.445            210                637
## 104                1842        0.404            196                631
## 105                2017        0.443            269                757
## 106                2028        0.499            204                632
## 107                1968        0.459            330                859
## 108                1714        0.440            176                573
## 109                1761        0.441            210                653
## 110                2239        0.526            287                790
## 111                1846        0.452            223                553
## 112                2006        0.452            257                777
## 113                2413        0.458            303                879
## 114                2236        0.432            291                844
## 115                1862        0.455            322                846
## 116                1711        0.460            260                722
## 117                1746        0.451            266                751
## 118                1719        0.444            180                583
## 119                2055        0.486            308                800
## 120                1834        0.462            237                670
## 121                1896        0.456            200                588
## 122                2203        0.446            333                939
## 123                2163        0.433            244                669
## 124                1742        0.447            274                752
## 125                1764        0.441            251                678
## 126                1883        0.459            303                856
## 127                1887        0.431            244                734
## 128                2001        0.431            259                751
## 129                1527        0.466            174                508
## 130                1719        0.440            178                491
## 131                2009        0.457            211                676
## 132                1897        0.452            297                845
## 133                2047        0.476            294                811
## 134                2006        0.456            285                782
## 135                2059        0.481            360                949
## 136                1969        0.453            234                695
## 137                1774        0.404            155                554
## 138                2063        0.462            200                634
## 139                1953        0.455            213                645
## 140                1859        0.442            243                670
## 141                1878        0.429            241                721
## 142                2128        0.465            260                743
## 143                1945        0.381            136                438
## 144                2032        0.432            261                775
## 145                2050        0.477            215                607
## 146                1798        0.400            259                783
## 147                1786        0.449            280                736
## 148                1976        0.456            228                664
## 149                1775        0.482            287                679
## 150                1977        0.487            322                873
## 151                2230        0.480            327                876
## 152                2031        0.432            217                630
## 153                1880        0.437            277                813
## 154                1875        0.428            333                939
## 155                2026        0.445            283                821
## 156                2029        0.454            353                891
## 157                2162        0.457            236                740
## 158                1928        0.436            253                756
## 159                1967        0.434            294                860
## 160                1687        0.493            206                563
## 161                1800        0.454            163                510
## 162                1836        0.459            210                616
## 163                1728        0.432            201                647
## 164                1608        0.405            219                661
## 165                1647        0.449            245                675
## 166                1969        0.454            319                822
## 167                2367        0.442            362               1058
## 168                1893        0.433            261                807
## 169                1705        0.391            172                610
## 170                1909        0.449            247                707
## 171                1852        0.472            246                721
## 172                1831        0.441            256                732
## 173                1676        0.453            168                548
## 174                2238        0.453            260                807
## 175                1746        0.451            228                663
## 176                1851        0.435            268                805
## 177                1847        0.421            269                832
## 178                2230        0.480            319                844
## 179                2102        0.448            287                839
## 180                1858        0.419            212                650
## 181                1820        0.420            237                692
## 182                2039        0.435            191                603
## 183                1982        0.472            292                774
## 184                1923        0.380            172                554
## 185                1906        0.460            272                760
## 186                1794        0.431            259                728
## 187                1714        0.452            224                649
## 188                1764        0.430            264                727
## 189                1915        0.415            168                565
## 190                1882        0.456            306                818
## 191                1983        0.492            287                763
## 192                1993        0.429            260                762
## 193                1920        0.395            168                521
## 194                1755        0.418            223                723
## 195                2007        0.492            258                731
## 196                1785        0.410            208                656
## 197                1998        0.475            282                719
## 198                2133        0.429            270                800
## 199                1852        0.427            251                745
## 200                1999        0.462            297                855
## 201                1656        0.376            261                787
## 202                2092        0.460            326                977
## 203                1927        0.422            292                844
## 204                1941        0.444            189                566
## 205                1887        0.421            265                738
## 206                1914        0.422            324                883
## 207                1913        0.444            246                713
## 208                2079        0.433            282                770
## 209                1914        0.389            240                805
## 210                1627        0.403            223                687
## 211                1797        0.459            227                649
## 212                1922        0.455            221                693
## 213                2184        0.457            280                818
## 214                2319        0.458            292                830
## 215                1986        0.443            260                744
## 216                2410        0.464            312                862
## 217                1926        0.454            332                910
## 218                1771        0.464            241                647
## 219                2014        0.448            308                897
## 220                1910        0.435            284                838
## 221                1848        0.476            328                858
## 222                1819        0.432            259                749
## 223                1837        0.462            307                849
## 224                1995        0.471            229                642
## 225                1847        0.416            276                792
## 226                2047        0.478            306                844
## 227                1790        0.405            190                646
## 228                1827        0.402            233                744
## 229                1973        0.393            272                863
## 230                1859        0.465            314                816
## 231                1931        0.434            264                774
## 232                1838        0.437            187                633
## 233                1796        0.423            280                753
## 234                1975        0.447            226                654
## 235                2026        0.406            264                756
## 236                1874        0.456            266                734
## 237                1728        0.463            203                633
## 238                2126        0.451            295                840
## 239                1681        0.422            198                566
## 240                1898        0.417            222                693
## 241                1810        0.454            282                803
## 242                1949        0.444            313                819
## 243                1841        0.418            225                680
## 244                2017        0.424            238                769
## 245                1708        0.414            216                664
## 246                2061        0.438            225                703
## 247                2150        0.449            394               1034
## 248                1666        0.411            207                679
## 249                1973        0.423            225                690
## 250                2145        0.451            365                977
## 251                1730        0.427            348                923
## 252                1938        0.461            285                751
## 253                1914        0.429            172                615
## 254                1867        0.426            277                797
## 255                1800        0.472            251                723
## 256                1893        0.447            222                651
## 257                1991        0.432            280                789
## 258                1842        0.419            191                612
## 259                1800        0.448            175                527
## 260                1926        0.460            243                681
## 261                2035        0.435            254                724
## 262                1943        0.408            280                869
## 263                2052        0.417            205                675
## 264                1905        0.472            253                670
## 265                1639        0.431            184                567
## 266                2011        0.442            308                840
## 267                1951        0.472            254                687
## 268                1940        0.436            261                721
## 269                2002        0.457            284                808
## 270                1851        0.467            258                727
## 271                1765        0.404            210                641
## 272                1653        0.445            239                663
## 273                2080        0.397            351               1199
## 274                1945        0.438            238                642
## 275                2021        0.439            240                741
## 276                1779        0.434            288                849
## 277                1883        0.460            285                761
## 278                1891        0.429            193                582
## 279                1872        0.414            269                817
## 280                1907        0.423            242                662
## 281                1965        0.501            327                802
## 282                1664        0.438            227                666
## 283                2120        0.430            272                819
## 284                1809        0.440            266                719
## 285                1735        0.441            235                696
## 286                2039        0.460            281                735
## 287                1898        0.418            201                665
## 288                1720        0.466            213                570
## 289                1884        0.442            263                772
## 290                1929        0.466            275                712
## 291                1997        0.441            272                799
## 292                1710        0.440            180                544
## 293                1914        0.429            208                635
## 294                1989        0.423            270                806
## 295                2100        0.450            289                809
## 296                1783        0.453            207                653
## 297                1696        0.439            201                588
## 298                1962        0.405            215                684
## 299                1937        0.421            233                715
## 300                1907        0.424            274                824
## 301                1998        0.437            247                748
## 302                1915        0.446            250                726
## 303                1739        0.432            240                716
## 304                1784        0.416            213                620
## 305                2231        0.496            262                714
## 306                1803        0.429            202                621
## 307                1842        0.438            204                663
## 308                1908        0.409            234                773
## 309                2168        0.457            281                813
## 310                1592        0.401            199                620
## 311                2139        0.418            203                636
## 312                2068        0.433            294                855
## 313                2379        0.455            276                852
## 314                1971        0.444            244                744
## 315                2110        0.469            277                759
## 316                2146        0.433            325                934
## 317                1990        0.453            325                859
## 318                1860        0.435            180                550
## 319                1719        0.450            236                704
## 320                1783        0.411            213                670
## 321                1732        0.448            222                652
## 322                2012        0.457            256                721
## 323                2038        0.471            274                772
## 324                1929        0.482            283                740
## 325                1735        0.464            298                806
## 326                1783        0.450            185                585
## 327                1750        0.421            230                739
## 328                1887        0.458            273                761
## 329                2019        0.438            380               1081
## 330                1872        0.438            235                771
## 331                1974        0.427            323                899
## 332                1900        0.470            327                831
## 333                2056        0.474            321                813
## 334                1696        0.386            236                773
## 335                1812        0.394            197                640
## 336                1861        0.448            301                828
## 337                1956        0.451            273                780
## 338                1934        0.476            249                684
## 339                2181        0.413            269                852
## 340                1744        0.454            296                800
## 341                1860        0.440            241                626
## 342                1910        0.447            199                608
## 343                1822        0.425            228                723
## 344                2277        0.408            276                890
## 345                1773        0.474            254                735
## 346                1878        0.455            372                997
## 347                1945        0.449            241                672
## 348                2132        0.490            385                929
## 349                2028        0.436            281                818
## 350                1656        0.417            248                720
## 351                1977        0.466            245                740
## 352                1812        0.493            230                634
## 353                2057        0.427            303                891
##     ThreePointPct FreeThrowsMade FreeThrowsAttempted FreeThrowPCT
## 1           0.380            457                 642        0.712
## 2           0.329            341                 503        0.678
## 3           0.320            380                 539        0.705
## 4           0.315            284                 453        0.627
## 5           0.337            424                 630        0.673
## 6           0.321            446                 684        0.652
## 7           0.340            492                 739        0.666
## 8           0.347            420                 564        0.745
## 9           0.327            366                 543        0.674
## 10          0.325            411                 589        0.698
## 11          0.359            505                 706        0.715
## 12          0.336            600                 882        0.680
## 13          0.336            446                 620        0.719
## 14          0.354            461                 703        0.656
## 15          0.338            476                 702        0.678
## 16          0.334            530                 721        0.735
## 17          0.345            514                 769        0.668
## 18          0.315            322                 471        0.684
## 19          0.377            540                 760        0.711
## 20          0.382            498                 705        0.706
## 21          0.325            498                 709        0.702
## 22          0.341            430                 635        0.677
## 23          0.372            441                 598        0.737
## 24          0.321            420                 689        0.610
## 25          0.349            291                 452        0.644
## 26          0.352            422                 584        0.723
## 27          0.317            442                 633        0.698
## 28          0.346            320                 468        0.684
## 29          0.357            506                 755        0.670
## 30          0.368            465                 672        0.692
## 31          0.330            545                 750        0.727
## 32          0.342            490                 685        0.715
## 33          0.325            440                 618        0.712
## 34          0.351            489                 653        0.749
## 35          0.337            527                 767        0.687
## 36          0.353            360                 488        0.738
## 37          0.328            293                 452        0.648
## 38          0.324            417                 620        0.673
## 39          0.317            514                 726        0.708
## 40          0.361            431                 676        0.638
## 41          0.359            454                 576        0.788
## 42          0.328            365                 515        0.709
## 43          0.359            449                 640        0.702
## 44          0.376            290                 424        0.684
## 45          0.343            496                 693        0.716
## 46          0.350            436                 603        0.723
## 47          0.343            448                 590        0.759
## 48          0.324            398                 575        0.692
## 49          0.352            487                 667        0.730
## 50          0.328            496                 634        0.782
## 51          0.365            518                 798        0.649
## 52          0.371            609                 908        0.671
## 53          0.343            398                 588        0.677
## 54          0.311            389                 524        0.742
## 55          0.365            312                 457        0.683
## 56          0.299            393                 581        0.676
## 57          0.345            534                 758        0.704
## 58          0.344            385                 523        0.736
## 59          0.328            456                 623        0.732
## 60          0.370            399                 585        0.682
## 61          0.371            559                 778        0.719
## 62          0.393            418                 563        0.742
## 63          0.340            470                 615        0.764
## 64          0.359            402                 579        0.694
## 65          0.323            570                 757        0.753
## 66          0.362            266                 377        0.706
## 67          0.344            436                 641        0.680
## 68          0.285            458                 689        0.665
## 69          0.332            398                 551        0.722
## 70          0.387            395                 582        0.679
## 71          0.366            314                 436        0.720
## 72          0.354            393                 525        0.749
## 73          0.332            397                 575        0.690
## 74          0.297            332                 489        0.679
## 75          0.378            410                 564        0.727
## 76          0.365            349                 460        0.759
## 77          0.343            562                 773        0.727
## 78          0.356            417                 581        0.718
## 79          0.361            468                 618        0.757
## 80          0.354            440                 584        0.753
## 81          0.308            551                 803        0.686
## 82          0.322            460                 663        0.694
## 83          0.284            435                 638        0.682
## 84          0.361            425                 634        0.670
## 85          0.370            375                 545        0.688
## 86          0.327            486                 683        0.712
## 87          0.295            380                 604        0.629
## 88          0.350            398                 546        0.729
## 89          0.355            307                 443        0.693
## 90          0.340            476                 660        0.721
## 91          0.350            291                 420        0.693
## 92          0.400            509                 702        0.725
## 93          0.368            309                 503        0.614
## 94          0.324            414                 551        0.751
## 95          0.372            393                 588        0.668
## 96          0.302            547                 855        0.640
## 97          0.332            579                 778        0.744
## 98          0.334            435                 603        0.721
## 99          0.332            314                 462        0.680
## 100         0.381            412                 572        0.720
## 101         0.361            364                 508        0.717
## 102         0.391            527                 741        0.711
## 103         0.330            454                 634        0.716
## 104         0.311            411                 600        0.685
## 105         0.355            568                 771        0.737
## 106         0.323            498                 720        0.692
## 107         0.384            462                 694        0.666
## 108         0.307            405                 588        0.689
## 109         0.322            512                 726        0.705
## 110         0.363            602                 791        0.761
## 111         0.403            535                 781        0.685
## 112         0.331            487                 666        0.731
## 113         0.345            575                 818        0.703
## 114         0.345            622                 790        0.787
## 115         0.381            459                 619        0.742
## 116         0.360            388                 538        0.721
## 117         0.354            399                 576        0.693
## 118         0.309            409                 605        0.676
## 119         0.385            615                 767        0.802
## 120         0.354            283                 424        0.667
## 121         0.340            535                 793        0.675
## 122         0.355            490                 697        0.703
## 123         0.365            566                 780        0.726
## 124         0.364            366                 498        0.735
## 125         0.370            367                 507        0.724
## 126         0.354            358                 525        0.682
## 127         0.332            387                 540        0.717
## 128         0.345            415                 591        0.702
## 129         0.343            448                 553        0.810
## 130         0.363            454                 621        0.731
## 131         0.312            455                 695        0.655
## 132         0.351            521                 699        0.745
## 133         0.363            450                 615        0.732
## 134         0.364            627                 849        0.739
## 135         0.379            386                 557        0.693
## 136         0.337            491                 699        0.702
## 137         0.280            385                 607        0.634
## 138         0.315            468                 630        0.743
## 139         0.330            394                 582        0.677
## 140         0.363            437                 607        0.720
## 141         0.334            383                 574        0.667
## 142         0.350            487                 691        0.705
## 143         0.311            410                 590        0.695
## 144         0.337            464                 641        0.724
## 145         0.354            635                 859        0.739
## 146         0.331            394                 512        0.770
## 147         0.380            325                 437        0.744
## 148         0.343            565                 823        0.687
## 149         0.423            480                 620        0.774
## 150         0.369            406                 524        0.775
## 151         0.373            607                 801        0.758
## 152         0.344            574                 831        0.691
## 153         0.341            452                 643        0.703
## 154         0.355            468                 639        0.732
## 155         0.345            527                 710        0.742
## 156         0.396            576                 737        0.782
## 157         0.319            603                 802        0.752
## 158         0.335            453                 690        0.657
## 159         0.342            534                 687        0.777
## 160         0.366            363                 545        0.666
## 161         0.320            481                 645        0.746
## 162         0.341            438                 617        0.710
## 163         0.311            293                 452        0.648
## 164         0.331            312                 534        0.584
## 165         0.363            363                 530        0.685
## 166         0.388            524                 692        0.757
## 167         0.342            522                 723        0.722
## 168         0.323            395                 573        0.689
## 169         0.282            329                 489        0.673
## 170         0.349            466                 627        0.743
## 171         0.341            460                 639        0.720
## 172         0.350            391                 574        0.681
## 173         0.307            461                 647        0.713
## 174         0.322            596                 835        0.714
## 175         0.344            446                 648        0.688
## 176         0.333            419                 570        0.735
## 177         0.323            444                 626        0.709
## 178         0.378            564                 749        0.753
## 179         0.342            406                 579        0.701
## 180         0.326            373                 551        0.677
## 181         0.342            379                 508        0.746
## 182         0.317            578                 848        0.682
## 183         0.377            464                 647        0.717
## 184         0.310            426                 647        0.658
## 185         0.358            461                 589        0.783
## 186         0.356            413                 590        0.700
## 187         0.345            434                 627        0.692
## 188         0.363            370                 526        0.703
## 189         0.297            530                 773        0.686
## 190         0.374            480                 651        0.737
## 191         0.376            428                 621        0.689
## 192         0.341            461                 654        0.705
## 193         0.322            477                 669        0.713
## 194         0.308            403                 594        0.678
## 195         0.353            493                 673        0.733
## 196         0.317            389                 544        0.715
## 197         0.392            341                 473        0.721
## 198         0.338            484                 694        0.697
## 199         0.337            443                 647        0.685
## 200         0.347            578                 816        0.708
## 201         0.332            235                 385        0.610
## 202         0.334            484                 714        0.678
## 203         0.346            523                 745        0.702
## 204         0.334            471                 681        0.692
## 205         0.359            463                 629        0.736
## 206         0.367            322                 418        0.770
## 207         0.345            528                 732        0.721
## 208         0.366            570                 803        0.710
## 209         0.298            417                 619        0.674
## 210         0.325            324                 437        0.741
## 211         0.350            364                 548        0.664
## 212         0.319            460                 641        0.718
## 213         0.342            464                 666        0.697
## 214         0.352            468                 661        0.708
## 215         0.349            493                 673        0.733
## 216         0.362            541                 728        0.743
## 217         0.365            476                 613        0.777
## 218         0.372            335                 482        0.695
## 219         0.343            373                 539        0.692
## 220         0.339            356                 532        0.669
## 221         0.382            476                 634        0.751
## 222         0.346            433                 614        0.705
## 223         0.362            445                 633        0.703
## 224         0.357            397                 543        0.731
## 225         0.348            401                 537        0.747
## 226         0.363            483                 725        0.666
## 227         0.294            411                 604        0.680
## 228         0.313            405                 551        0.735
## 229         0.315            442                 594        0.744
## 230         0.385            492                 663        0.742
## 231         0.341            477                 650        0.734
## 232         0.295            381                 602        0.633
## 233         0.372            378                 549        0.689
## 234         0.346            431                 618        0.697
## 235         0.349            388                 585        0.663
## 236         0.362            340                 492        0.691
## 237         0.321            472                 636        0.742
## 238         0.351            450                 624        0.721
## 239         0.350            520                 698        0.745
## 240         0.320            425                 613        0.693
## 241         0.351            326                 511        0.638
## 242         0.382            529                 696        0.760
## 243         0.331            543                 779        0.697
## 244         0.309            525                 748        0.702
## 245         0.325            448                 668        0.671
## 246         0.320            596                 875        0.681
## 247         0.381            491                 674        0.728
## 248         0.305            371                 502        0.739
## 249         0.326            533                 771        0.691
## 250         0.374            461                 641        0.719
## 251         0.377            468                 646        0.724
## 252         0.379            369                 515        0.717
## 253         0.280            479                 704        0.680
## 254         0.348            500                 729        0.686
## 255         0.347            365                 552        0.661
## 256         0.341            465                 755        0.616
## 257         0.355            446                 638        0.699
## 258         0.312            372                 584        0.637
## 259         0.332            410                 592        0.693
## 260         0.357            545                 722        0.755
## 261         0.351            493                 698        0.706
## 262         0.322            462                 615        0.751
## 263         0.304            484                 809        0.598
## 264         0.378            412                 555        0.742
## 265         0.325            414                 570        0.726
## 266         0.367            390                 518        0.753
## 267         0.370            487                 666        0.731
## 268         0.362            486                 679        0.716
## 269         0.351            479                 641        0.747
## 270         0.355            372                 568        0.655
## 271         0.328            406                 620        0.655
## 272         0.360            406                 567        0.716
## 273         0.293            359                 531        0.676
## 274         0.371            478                 668        0.716
## 275         0.324            492                 697        0.706
## 276         0.339            299                 440        0.680
## 277         0.375            459                 669        0.686
## 278         0.332            573                 772        0.742
## 279         0.329            350                 492        0.711
## 280         0.366            473                 694        0.682
## 281         0.408            494                 635        0.778
## 282         0.341            433                 585        0.740
## 283         0.332            658                1018        0.646
## 284         0.370            358                 519        0.690
## 285         0.338            459                 640        0.717
## 286         0.382            363                 563        0.645
## 287         0.302            476                 677        0.703
## 288         0.374            377                 550        0.685
## 289         0.341            374                 523        0.715
## 290         0.386            321                 489        0.656
## 291         0.340            539                 757        0.712
## 292         0.331            393                 588        0.668
## 293         0.328            400                 539        0.742
## 294         0.335            393                 580        0.678
## 295         0.357            446                 620        0.719
## 296         0.317            432                 643        0.672
## 297         0.342            414                 607        0.682
## 298         0.314            348                 503        0.692
## 299         0.326            501                 695        0.721
## 300         0.333            480                 701        0.685
## 301         0.330            472                 649        0.727
## 302         0.344            398                 528        0.754
## 303         0.335            440                 628        0.701
## 304         0.344            407                 645        0.631
## 305         0.367            561                 744        0.754
## 306         0.325            376                 541        0.695
## 307         0.308            437                 631        0.693
## 308         0.303            488                 659        0.741
## 309         0.346            469                 688        0.682
## 310         0.321            382                 578        0.661
## 311         0.319            641                 923        0.694
## 312         0.344            395                 536        0.737
## 313         0.324            660                 988        0.668
## 314         0.328            474                 694        0.683
## 315         0.365            508                 694        0.732
## 316         0.348            445                 637        0.699
## 317         0.378            410                 531        0.772
## 318         0.327            370                 532        0.695
## 319         0.335            397                 550        0.722
## 320         0.318            403                 585        0.689
## 321         0.340            522                 751        0.695
## 322         0.355            486                 768        0.633
## 323         0.355            559                 747        0.748
## 324         0.382            554                 764        0.725
## 325         0.370            435                 616        0.706
## 326         0.316            416                 595        0.699
## 327         0.311            478                 710        0.673
## 328         0.359            507                 678        0.748
## 329         0.352            506                 695        0.728
## 330         0.305            469                 669        0.701
## 331         0.359            424                 581        0.730
## 332         0.394            461                 606        0.761
## 333         0.395            445                 598        0.744
## 334         0.305            428                 575        0.744
## 335         0.308            494                 681        0.725
## 336         0.364            426                 564        0.755
## 337         0.350            472                 679        0.695
## 338         0.364            528                 708        0.746
## 339         0.316            584                 849        0.688
## 340         0.370            424                 628        0.675
## 341         0.385            321                 440        0.730
## 342         0.327            524                 724        0.724
## 343         0.315            470                 680        0.691
## 344         0.310            475                 655        0.725
## 345         0.346            385                 576        0.668
## 346         0.373            417                 565        0.738
## 347         0.359            344                 531        0.648
## 348         0.414            406                 577        0.704
## 349         0.344            510                 692        0.737
## 350         0.344            477                 660        0.723
## 351         0.331            437                 644        0.679
## 352         0.363            411                 557        0.738
## 353         0.340            354                 505        0.701
##     OffensiveRebounds TotalRebounds Assists Steals Blocks Turnovers
## 1                 325          1110     525    297     93       407
## 2                 253          1077     434    154     57       423
## 3                 312          1204     399    185    106       388
## 4                 314          1032     385    234     50       487
## 5                 367          1279     401    218     82       399
## 6                 365          1094     313    203    102       451
## 7                 384          1285     418    157    160       465
## 8                 294          1081     402    195     78       454
## 9                 334          1079     391    229    107       492
## 10                251          1014     402    221    131       386
## 11                310          1127     406    171     90       418
## 12                399          1351     459    213    109       466
## 13                324          1115     398    165     73       370
## 14                244          1060     443    169     87       503
## 15                317          1063     365    216    109       478
## 16                403          1211     363    183    139       448
## 17                345          1152     543    276    172       450
## 18                259          1125     499    198     50       376
## 19                457          1369     572    369    190       466
## 20                397          1215     454    252    110       392
## 21                308          1243     426    217    118       480
## 22                450          1281     473    209    159       446
## 23                286          1275     645    220    125       376
## 24                415          1274     444    234    108       501
## 25                244          1060     324    183    152       409
## 26                242          1016     398    200     68       379
## 27                331          1139     382    159    103       378
## 28                311          1091     447    199     73       425
## 29                408          1396     427    227     90       412
## 30                313          1195     423    186    132       440
## 31                263          1118     482    203    107       354
## 32                261          1157     418    241    123       458
## 33                285           969     319    158     79       383
## 34                286          1154     519    173    116       409
## 35                452          1470     599    263    140       433
## 36                288          1058     434    193     62       359
## 37                284           975     301    147     99       334
## 38                486          1272     379    218    105       439
## 39                277          1224     407    200    118       450
## 40                352          1214     512    225    154       410
## 41                306          1171     345    148     85       379
## 42                218           929     375    196     71       412
## 43                429          1471     479    210    153       441
## 44                286          1051     425    164     60       425
## 45                379          1204     431    159    106       355
## 46                249           900     338    227     99       352
## 47                266          1050     482    218    104       337
## 48                300          1026     466    204    108       412
## 49                299          1164     463    202    120       478
## 50                332          1107     353    172    110       427
## 51                315          1215     439    189    150       393
## 52                419          1321     467    267    107       425
## 53                407          1277     487    263    110       441
## 54                199           892     288    138    101       418
## 55                278          1095     418    139     79       409
## 56                252           995     376    193     72       512
## 57                442          1264     466    214    155       365
## 58                322          1107     462    201    110       386
## 59                309          1204     410    223    154       450
## 60                303          1093     460    174     97       412
## 61                376          1256     451    214     78       472
## 62                322          1240     541    213    121       456
## 63                260          1060     387    216     76       347
## 64                256          1080     472    199     76       372
## 65                363          1352     478    181    107       479
## 66                255           985     443    218     78       366
## 67                380          1167     422    200    145       427
## 68                314          1218     326    169    115       555
## 69                191          1015     453    181    107       414
## 70                288          1196     561    237     92       464
## 71                246          1020     426    158    100       361
## 72                264          1185     499    175     82       378
## 73                261          1145     541    164     81       414
## 74                378          1144     333    214     66       435
## 75                254          1083     425    149     78       378
## 76                275          1041     354    142     78       408
## 77                396          1347     513    227    117       488
## 78                336          1019     319    238     77       355
## 79                279          1214     518    180    130       449
## 80                312          1126     464    144     62       377
## 81                495          1567     606    346    257       488
## 82                379          1133     442    250    151       451
## 83                317          1061     382    224    110       416
## 84                442          1351     534    252    135       483
## 85                347          1147     434    200    120       427
## 86                362          1167     458    317    101       447
## 87                400          1155     369    237    143       442
## 88                299          1192     467    200    101       401
## 89                265          1082     468    135     62       413
## 90                215          1095     394    185     89       416
## 91                273           968     401    202     63       436
## 92                323          1155     486    261    132       477
## 93                288           991     355    226    111       499
## 94                366          1296     419    184     93       485
## 95                272          1064     449    219    133       479
## 96                367          1217     491    359    153       477
## 97                418          1391     475    266    161       492
## 98                375          1201     437    257    131       420
## 99                305          1082     360    188    108       345
## 100               300          1137     457    228    111       400
## 101               326          1171     524    288    137       407
## 102               271          1190     502    238    101       407
## 103               323          1166     399    204     97       428
## 104               274          1116     375    170     96       413
## 105               350          1299     542    198    131       443
## 106               320          1190     403    255    132       451
## 107               273          1110     415    269    158       391
## 108               262          1074     433    221    166       459
## 109               357          1215     423    185    163       506
## 110               354          1443     668    278    204       394
## 111               307          1287     420    193    147       536
## 112               350          1238     478    207     75       411
## 113               376          1406     578    291    154       500
## 114               363          1370     487    215    121       379
## 115               278          1050     462    228     69       401
## 116               288          1119     369    174    123       491
## 117               278          1054     478    156     66       368
## 118               363          1183     404    170     91       406
## 119               305          1170     492    228    122       342
## 120               220           916     513    241    127       350
## 121               409          1142     449    234     54       442
## 122               448          1502     548    240    161       410
## 123               396          1253     463    198    144       427
## 124               275           975     383    154     88       360
## 125               258          1061     399    130     53       433
## 126               271          1106     478    210    124       457
## 127               298          1139     421    162    106       417
## 128               351          1091     446    245     86       435
## 129               204           845     354    184     53       504
## 130               250          1007     342    207     73       402
## 131               341          1272     469    226    154       433
## 132               283          1152     432    223     90       406
## 133               327          1230     527    247    160       379
## 134               357          1246     542    211    112       427
## 135               271          1140     525    247    158       395
## 136               403          1233     471    228    103       462
## 137               331          1125     359    209    124       436
## 138               400          1310     451    245    138       440
## 139               330          1163     457    213    166       463
## 140               345          1123     403    211    101       420
## 141               325          1137     464    256     76       385
## 142               374          1376     478    243    139       472
## 143               372          1163     362    185    106       437
## 144               412          1185     430    226    119       378
## 145               427          1428     501    220    176       466
## 146               328          1060     391    230     98       435
## 147               233          1004     419    158    126       402
## 148               446          1282     481    283     97       473
## 149               232          1105     471    171     71       418
## 150               270          1174     532    223     96       395
## 151               329          1378     654    268    108       489
## 152               377          1263     399    219    105       479
## 153               321          1196     430    218    157       471
## 154               284          1194     401    210     71       477
## 155               364          1142     487    226    125       423
## 156               317          1176     498    206     78       398
## 157               460          1355     452    308    148       452
## 158               368          1204     440    190    115       386
## 159               345          1297     458    149     99       412
## 160               190          1013     468    220     78       407
## 161               314          1098     460    250    102       425
## 162               292           997     448    278     73       444
## 163               280          1015     471    239     94       460
## 164               326          1026     360    221     78       487
## 165               241           967     377    172     53       387
## 166               331          1292     456    168    145       467
## 167               319          1260     527    334    170       453
## 168               296          1145     453    268     83       428
## 169               264           989     361    186     70       431
## 170               378          1336     447    148    163       437
## 171               293          1115     500    181     85       447
## 172               342          1123     468    154     93       445
## 173               310          1054     429    147     97       450
## 174               436          1358     553    288    144       538
## 175               307          1057     393    207     80       407
## 176               291          1061     413    212    100       372
## 177               353          1190     349    194     87       413
## 178               414          1579     715    201    205       493
## 179               299          1307     512    225    149       334
## 180               382          1228     352    190     85       512
## 181               288          1094     356    145     94       382
## 182               402          1314     523    175    140       418
## 183               393          1215     481    274    172       451
## 184               378          1165     361    176     70       415
## 185               322          1128     475    235    118       416
## 186               261           966     351    233     90       391
## 187               270          1000     354    221    127       385
## 188               345          1130     354    149     53       448
## 189               360          1224     379    237    101       502
## 190               292          1083     491    169     46       394
## 191               293          1176     510    229    106       416
## 192               357          1177     440    186    137       402
## 193               397          1078     362    230     48       401
## 194               293          1036     354    146    107       410
## 195               351          1247     591    249    149       401
## 196               359          1160     399    167     58       437
## 197               273          1047     409    180     61       304
## 198               379          1278     466    260    151       345
## 199               408          1204     401    184    115       401
## 200               325          1274     498    211    134       352
## 201               260           986     325    116     53       356
## 202               445          1343     514    197     91       419
## 203               351          1199     446    220    107       455
## 204               393          1166     497    285    134       517
## 205               315          1143     343    128    127       390
## 206               280          1041     428    208    128       386
## 207               250          1178     376    223     94       421
## 208               366          1318     457    219    154       523
## 209               351          1170     334    209     74       435
## 210               298           931     331    196     86       456
## 211               291          1059     452    215    108       417
## 212               399          1265     526    207     93       515
## 213               399          1251     453    315    140       423
## 214               486          1390     539    257    138       449
## 215               387          1221     495    150     68       436
## 216               477          1589     678    252    120       473
## 217               229          1105     402    160     87       368
## 218               233          1009     398    171     87       380
## 219               363          1252     520    212    165       536
## 220               368          1244     428    210     88       443
## 221               237          1075     480    211     74       381
## 222               290          1055     381    196     65       372
## 223               309          1128     422    176    101       398
## 224               304          1179     374    188     75       404
## 225               262          1115     370    165     75       393
## 226               364          1298     592    218    139       441
## 227               352          1103     340    192    138       496
## 228               278          1086     451    176    134       349
## 229               370          1208     423    179    148       306
## 230               275          1063     580    191     99       426
## 231               328          1215     494    206     85       438
## 232               332          1152     451    200    115       448
## 233               295          1093     422    183    139       386
## 234               305          1263     430    203    102       401
## 235               418          1365     439    197    160       408
## 236               335          1147     434    152    107       447
## 237               310          1091     465    185    164       384
## 238               361          1307     501    290    162       442
## 239               285          1039     325    199     46       423
## 240               382          1174     389    245    122       399
## 241               270          1101     465    210     76       395
## 242               290          1085     479    229     95       432
## 243               383          1211     387    227    126       448
## 244               491          1282     434    264    130       449
## 245               233          1032     369    175    123       425
## 246               392          1151     445    309     55       445
## 247               324          1248     565    214     69       358
## 248               272          1016     295    168     72       354
## 249               399          1241     491    253    136       445
## 250               421          1341     518    220    143       381
## 251               311          1111     393    159     62       415
## 252               335          1170     467    196    101       358
## 253               375          1194     401    235    129       405
## 254               316          1221     425    171    104       438
## 255               196           957     519    249    101       336
## 256               363          1144     450    282    101       415
## 257               373          1178     487    257     92       507
## 258               391          1210     390    201    124       408
## 259               308          1050     431    239    104       404
## 260               354          1231     471    170    133       465
## 261               430          1246     413    221     94       402
## 262               258          1145     397    186    110       305
## 263               492          1428     460    254    145       464
## 264               329          1180     341    201     87       355
## 265               293          1032     301    172    177       488
## 266               369          1191     555    242     94       413
## 267               316          1225     461    238    127       463
## 268               346          1228     487    197    113       424
## 269               290          1238     516    204    112       466
## 270               340          1138     439    188    105       321
## 271               323          1130     423    131     52       484
## 272               242          1026     416    146    111       433
## 273               383          1188     462    234     96       478
## 274               336          1221     386    141    118       403
## 275               361          1213     459    235    132       423
## 276               248           973     415    190     93       321
## 277               288          1094     471    201    141       456
## 278               376          1137     386    167     60       493
## 279               305          1051     434    207     94       440
## 280               383          1160     413    202    140       433
## 281               251          1259     489    202     65       388
## 282               219           942     382    165     68       324
## 283               496          1483     521    306    124       604
## 284               284          1081     402    182     75       447
## 285               346          1117     452    215     80       536
## 286               324          1188     536    202    117       392
## 287               351          1079     382    198     70       438
## 288               289          1081     438    212    129       447
## 289               365          1135     441    189    123       350
## 290               274          1178     553    253    110       345
## 291               282          1250     426    199    114       466
## 292               313           980     372    216     86       511
## 293               333          1183     417    199    164       421
## 294               346          1191     387    194    136       413
## 295               264          1120     472    302    111       357
## 296               322          1147     392    186    147       467
## 297               313          1008     348    192    113       442
## 298               359          1110     418    210     94       394
## 299               410          1348     372    209    142       478
## 300               363          1174     408    278    162       423
## 301               316          1136     476    284     73       368
## 302               351          1062     441    223     88       418
## 303               317          1032     342    219     78       497
## 304               330          1122     389    226    134       517
## 305               379          1391     660    221    199       412
## 306               349          1158     423    197     90       456
## 307               372          1195     376    215    142       444
## 308               372          1235     421    211     97       467
## 309               395          1347     598    255    162       499
## 310               285          1070     307    135     84       466
## 311               358          1215     503    345    105       471
## 312               323          1216     420    213     82       402
## 313               467          1503     591    309    114       589
## 314               409          1291     493    234     84       455
## 315               315          1297     518    278    186       457
## 316               358          1264     486    227    152       401
## 317               327          1281     538    170    152       402
## 318               389          1184     319    154     94       390
## 319               277          1021     392    171    135       408
## 320               298          1131     444    147    109       467
## 321               239          1119     430    176     76       404
## 322               380          1361     472    194    136       467
## 323               377          1405     598    215    147       449
## 324               318          1271     484    201     89       487
## 325               317          1096     440    144     80       409
## 326               294          1115     431    202    101       439
## 327               325          1139     371    154    126       454
## 328               304          1194     399    188    136       377
## 329               368          1253     503    194    106       388
## 330               358          1208     450    263    148       458
## 331               280          1110     442    217     61       390
## 332               310          1142     531    236     78       392
## 333               342          1326     544    211    149       342
## 334               378          1095     351    184    108       414
## 335               399          1174     329    150     77       417
## 336               276          1076     478    169     89       440
## 337               334          1129     419    323    206       479
## 338               220          1177     383    179    140       382
## 339               519          1425     488    223    126       557
## 340               303          1135     389    135     53       577
## 341               254          1100     319    165    132       353
## 342               338          1256     417    235    174       466
## 343               345          1197     374    147     78       490
## 344               452          1417     499    174    135       446
## 345               251          1057     529    158    146       390
## 346               330          1232     460    134     83       456
## 347               283          1197     430    177    140       327
## 348               365          1232     528    233    105       377
## 349               382          1229     484    214     72       402
## 350               167           983     331    176     88       450
## 351               371          1281     519    190    128       450
## 352               259          1157     503    177    131       392
## 353               418          1207     449    210    115       423
##     PersonalFouls
## 1             635
## 2             543
## 3             569
## 4             587
## 5             578
## 6             565
## 7             572
## 8             566
## 9             548
## 10            520
## 11            585
## 12            675
## 13            594
## 14            622
## 15            629
## 16            684
## 17            693
## 18            544
## 19            731
## 20            596
## 21            579
## 22            636
## 23            509
## 24            610
## 25            474
## 26            570
## 27            509
## 28            578
## 29            596
## 30            617
## 31            625
## 32            600
## 33            575
## 34            627
## 35            659
## 36            593
## 37            544
## 38            747
## 39            639
## 40            641
## 41            623
## 42            596
## 43            673
## 44            562
## 45            558
## 46            574
## 47            522
## 48            627
## 49            611
## 50            597
## 51            554
## 52            593
## 53            571
## 54            490
## 55            553
## 56            625
## 57            563
## 58            528
## 59            552
## 60            551
## 61            691
## 62            547
## 63            486
## 64            522
## 65            616
## 66            539
## 67            643
## 68            620
## 69            492
## 70            580
## 71            468
## 72            539
## 73            463
## 74            589
## 75            523
## 76            506
## 77            630
## 78            528
## 79            568
## 80            542
## 81            595
## 82            580
## 83            622
## 84            603
## 85            569
## 86            661
## 87            529
## 88            632
## 89            539
## 90            602
## 91            612
## 92            578
## 93            580
## 94            575
## 95            514
## 96            655
## 97            705
## 98            614
## 99            559
## 100           592
## 101           481
## 102           561
## 103           563
## 104           526
## 105           581
## 106           590
## 107           583
## 108           560
## 109           565
## 110           603
## 111           699
## 112           598
## 113           730
## 114           649
## 115           556
## 116           554
## 117           556
## 118           502
## 119           526
## 120           482
## 121           593
## 122           705
## 123           652
## 124           497
## 125           582
## 126           684
## 127           577
## 128           700
## 129           592
## 130           586
## 131           596
## 132           564
## 133           517
## 134           565
## 135           495
## 136           578
## 137           643
## 138           553
## 139           523
## 140           612
## 141           550
## 142           610
## 143           525
## 144           601
## 145           600
## 146           630
## 147           515
## 148           622
## 149           571
## 150           543
## 151           633
## 152           620
## 153           610
## 154           624
## 155           612
## 156           545
## 157           634
## 158           535
## 159           591
## 160           471
## 161           602
## 162           573
## 163           566
## 164           672
## 165           583
## 166           635
## 167           664
## 168           578
## 169           575
## 170           526
## 171           589
## 172           643
## 173           467
## 174           739
## 175           557
## 176           465
## 177           581
## 178           647
## 179           514
## 180           583
## 181           579
## 182           578
## 183           586
## 184           570
## 185           603
## 186           647
## 187           540
## 188           625
## 189           698
## 190           567
## 191           655
## 192           561
## 193           614
## 194           576
## 195           527
## 196           534
## 197           486
## 198           580
## 199           556
## 200           584
## 201           533
## 202           632
## 203           627
## 204           631
## 205           511
## 206           587
## 207           582
## 208           763
## 209           614
## 210           533
## 211           514
## 212           610
## 213           631
## 214           691
## 215           585
## 216           611
## 217           523
## 218           485
## 219           573
## 220           589
## 221           539
## 222           537
## 223           538
## 224           653
## 225           554
## 226           664
## 227           578
## 228           583
## 229           455
## 230           534
## 231           623
## 232           504
## 233           516
## 234           527
## 235           580
## 236           575
## 237           536
## 238           665
## 239           630
## 240           597
## 241           500
## 242           639
## 243           611
## 244           681
## 245           588
## 246           758
## 247           617
## 248           459
## 249           600
## 250           625
## 251           524
## 252           528
## 253           582
## 254           568
## 255           488
## 256           591
## 257           683
## 258           566
## 259           566
## 260           572
## 261           598
## 262           497
## 263           634
## 264           572
## 265           614
## 266           659
## 267           560
## 268           638
## 269           667
## 270           543
## 271           539
## 272           597
## 273           564
## 274           657
## 275           638
## 276           462
## 277           649
## 278           670
## 279           621
## 280           623
## 281           557
## 282           527
## 283           790
## 284           615
## 285           602
## 286           590
## 287           612
## 288           474
## 289           515
## 290           507
## 291           696
## 292           716
## 293           553
## 294           635
## 295           650
## 296           586
## 297           579
## 298           536
## 299           631
## 300           588
## 301           581
## 302           490
## 303           735
## 304           554
## 305           654
## 306           610
## 307           521
## 308           630
## 309           586
## 310           551
## 311           763
## 312           581
## 313           679
## 314           667
## 315           663
## 316           627
## 317           517
## 318           561
## 319           523
## 320           535
## 321           529
## 322           563
## 323           649
## 324           612
## 325           548
## 326           546
## 327           568
## 328           568
## 329           580
## 330           650
## 331           541
## 332           532
## 333           542
## 334           639
## 335           555
## 336           554
## 337           653
## 338           609
## 339           707
## 340           671
## 341           487
## 342           528
## 343           564
## 344           705
## 345           516
## 346           535
## 347           511
## 348           589
## 349           545
## 350           588
## 351           550
## 352           510
## 353           656
\end{verbatim}

And just like that, we have a method for getting up to the minute season stats for every team in Division I.

\hypertarget{advanced-rvest}{%
\chapter{Advanced rvest}\label{advanced-rvest}}

With the chapter, we learned how to grab one table from one page. But what if you needed more than that? What if you needed hundreds of tables from hundreds of pages? What if you needed to combine one table on one page into a bigger table, but hundreds of times. There's a way to do this, it just takes patience, a lot of logic, a lot of debugging and, for me, a fair bit of swearing.

So what we are after are game by game stats for each college basketball team in America.

\href{https://www.sports-reference.com/cbb/seasons/2019-school-stats.html}{We can see from this page} that each team is linked. If we follow each link, we get a ton of tables. But they aren't what we need. There's a link to gamelogs underneath the team names.

So we can see from this that we've got some problems.

\begin{enumerate}
\def\labelenumi{\arabic{enumi}.}
\tightlist
\item
  The team name isn't in the table. Nor is the conference.
\item
  There's a date we'll have to deal with.
\item
  Non-standard headers and a truly huge number of fields.
\item
  And how do we get each one of those urls without having to copy them all into some horrible list?
\end{enumerate}

So let's start with that last question first and grab libraries we need.

\begin{Shaded}
\begin{Highlighting}[]
\KeywordTok{library}\NormalTok{(tidyverse)}
\KeywordTok{library}\NormalTok{(rvest)}
\KeywordTok{library}\NormalTok{(lubridate)}
\end{Highlighting}
\end{Shaded}

First things first, we need to grab the url to each team from that first link.

\begin{Shaded}
\begin{Highlighting}[]
\NormalTok{url <-}\StringTok{ "https://www.sports-reference.com/cbb/seasons/2019-school-stats.html"}
\end{Highlighting}
\end{Shaded}

But notice first, we don't want to grab the table. The table doesn't help us. We need to grab the only \emph{link} in the table. So we can do that by using the table xpath node, then grabbing the anchor tags in the table, then get only the link out of them (instead of the linked text).

\begin{Shaded}
\begin{Highlighting}[]
\NormalTok{schools <-}\StringTok{ }\NormalTok{url }\OperatorTok
\StringTok{  }\KeywordTok{read_html}\NormalTok{() }\OperatorTok
\StringTok{  }\KeywordTok{html_nodes}\NormalTok{(}\DataTypeTok{xpath =} \StringTok{'//*[@id="basic_school_stats"]'}\NormalTok{) }\OperatorTok
\StringTok{  }\KeywordTok{html_nodes}\NormalTok{(}\StringTok{"a"}\NormalTok{) }\OperatorTok
\StringTok{  }\KeywordTok{html_attr}\NormalTok{(}\StringTok{'href'}\NormalTok{)}
\end{Highlighting}
\end{Shaded}

Notice we now have a list called schools with \ldots{} 353 elements. That's the number of teams in college basketball, so we're off to a good start. Here's what the fourth element is.

\begin{Shaded}
\begin{Highlighting}[]
\NormalTok{schools[}\DecValTok{4}\NormalTok{]}
\end{Highlighting}
\end{Shaded}

\begin{verbatim}
## [1] "/cbb/schools/alabama-am/2019.html"
\end{verbatim}

So note, that's the relative path to Alabama A\&M's team page. By relative path, I mean it doesn't have the root domain. So we need to add that to each request or we'll get no where.

So that's a problem to note.

Before we solve that, let's just make sure we can get one page and get what we need.

We'll scrape Abilene Christian.

To merge all this into one big table, we need to grab the team name and their conference and merge it into the table. But those values come from somewhere else. The scraping works just about the same, but instead of html\_table you use html\_text.

So the first part of this is reading the html of the page so we don't do that for each element -- we just do it once so as to not overwhelm their servers.

The second part is we're grabbing the team name based on it's location in the page.

Third: The conference.

Fourth is the table itself, noting to ignore the headers. The last bit fixes the headers, removes the garbage header data from the table, converts the data to numbers, fixes the date and mutates a team and conference value. It looks like a lot, and it took a bit of twiddling to get it done, but it's no different from what you did for your last homework.

\begin{Shaded}
\begin{Highlighting}[]
\NormalTok{page <-}\StringTok{ }\KeywordTok{read_html}\NormalTok{(}\StringTok{"https://www.sports-reference.com/cbb/schools/abilene-christian/2019-gamelogs.html"}\NormalTok{)}
  
\NormalTok{team <-}\StringTok{ }\NormalTok{page }\OperatorTok
\StringTok{  }\KeywordTok{html_nodes}\NormalTok{(}\DataTypeTok{xpath =} \StringTok{'//*[@id="meta"]/div[2]/h1/span[2]'}\NormalTok{) }\OperatorTok
\StringTok{  }\KeywordTok{html_text}\NormalTok{()}

\NormalTok{conference <-}\StringTok{ }\NormalTok{page }\OperatorTok
\StringTok{    }\KeywordTok{html_nodes}\NormalTok{(}\DataTypeTok{xpath =} \StringTok{'//*[@id="meta"]/div[2]/p[1]/a'}\NormalTok{) }\OperatorTok
\StringTok{    }\KeywordTok{html_text}\NormalTok{()}

\NormalTok{table <-}\StringTok{ }\NormalTok{page }\OperatorTok
\StringTok{  }\KeywordTok{html_nodes}\NormalTok{(}\DataTypeTok{xpath =} \StringTok{'//*[@id="sgl-basic"]'}\NormalTok{) }\OperatorTok
\StringTok{  }\KeywordTok{html_table}\NormalTok{(}\DataTypeTok{header=}\OtherTok{FALSE}\NormalTok{)}

\NormalTok{table <-}\StringTok{ }\NormalTok{table[[}\DecValTok{1}\NormalTok{]] }\OperatorTok\StringTok{ }\KeywordTok{rename}\NormalTok{(}\DataTypeTok{Game=}\NormalTok{X1, }\DataTypeTok{Date=}\NormalTok{X2, }\DataTypeTok{HomeAway=}\NormalTok{X3, }\DataTypeTok{Opponent=}\NormalTok{X4, }\DataTypeTok{W_L=}\NormalTok{X5, }\DataTypeTok{TeamScore=}\NormalTok{X6, }\DataTypeTok{OpponentScore=}\NormalTok{X7, }\DataTypeTok{TeamFG=}\NormalTok{X8, }\DataTypeTok{TeamFGA=}\NormalTok{X9, }\DataTypeTok{TeamFGPCT=}\NormalTok{X10, }\DataTypeTok{Team3P=}\NormalTok{X11, }\DataTypeTok{Team3PA=}\NormalTok{X12, }\DataTypeTok{Team3PPCT=}\NormalTok{X13, }\DataTypeTok{TeamFT=}\NormalTok{X14, }\DataTypeTok{TeamFTA=}\NormalTok{X15, }\DataTypeTok{TeamFTPCT=}\NormalTok{X16, }\DataTypeTok{TeamOffRebounds=}\NormalTok{X17, }\DataTypeTok{TeamTotalRebounds=}\NormalTok{X18, }\DataTypeTok{TeamAssists=}\NormalTok{X19, }\DataTypeTok{TeamSteals=}\NormalTok{X20, }\DataTypeTok{TeamBlocks=}\NormalTok{X21, }\DataTypeTok{TeamTurnovers=}\NormalTok{X22, }\DataTypeTok{TeamPersonalFouls=}\NormalTok{X23, }\DataTypeTok{Blank=}\NormalTok{X24, }\DataTypeTok{OpponentFG=}\NormalTok{X25, }\DataTypeTok{OpponentFGA=}\NormalTok{X26, }\DataTypeTok{OpponentFGPCT=}\NormalTok{X27, }\DataTypeTok{Opponent3P=}\NormalTok{X28, }\DataTypeTok{Opponent3PA=}\NormalTok{X29, }\DataTypeTok{Opponent3PPCT=}\NormalTok{X30, }\DataTypeTok{OpponentFT=}\NormalTok{X31, }\DataTypeTok{OpponentFTA=}\NormalTok{X32, }\DataTypeTok{OpponentFTPCT=}\NormalTok{X33, }\DataTypeTok{OpponentOffRebounds=}\NormalTok{X34, }\DataTypeTok{OpponentTotalRebounds=}\NormalTok{X35, }\DataTypeTok{OpponentAssists=}\NormalTok{X36, }\DataTypeTok{OpponentSteals=}\NormalTok{X37, }\DataTypeTok{OpponentBlocks=}\NormalTok{X38, }\DataTypeTok{OpponentTurnovers=}\NormalTok{X39, }\DataTypeTok{OpponentPersonalFouls=}\NormalTok{X40) }\OperatorTok\StringTok{ }\KeywordTok{filter}\NormalTok{(Game }\OperatorTok{!=}\StringTok{ ""}\NormalTok{) }\OperatorTok\StringTok{ }\KeywordTok{filter}\NormalTok{(Game }\OperatorTok{!=}\StringTok{ "G"}\NormalTok{) }\OperatorTok\StringTok{ }\KeywordTok{mutate}\NormalTok{(}\DataTypeTok{Team=}\NormalTok{team) }\OperatorTok\StringTok{ }\KeywordTok{mutate_at}\NormalTok{(}\KeywordTok{vars}\NormalTok{(}\OperatorTok{-}\NormalTok{Team, }\OperatorTok{-}\NormalTok{Date, }\OperatorTok{-}\NormalTok{Opponent, }\OperatorTok{-}\NormalTok{HomeAway, }\OperatorTok{-}\NormalTok{W_L), as.numeric) }\OperatorTok\StringTok{ }\KeywordTok{mutate}\NormalTok{(}\DataTypeTok{Date =} \KeywordTok{ymd}\NormalTok{(Date)) }\OperatorTok\StringTok{ }\KeywordTok{mutate}\NormalTok{(}\DataTypeTok{Team=}\NormalTok{team, }\DataTypeTok{Conference=}\NormalTok{conference)}
\end{Highlighting}
\end{Shaded}

Now what we're left with is how do we do this for ALL the teams. We need to send 353 requests to their servers to get each page. And each url is not the one we have -- we need to alter it.

First we have to add the root domain to each request. And, each request needs to go to /2019-gamelogs.html instead of /2019.html. If you look at the urls two the page we have and the page we need, that's all that changes.

What we're going to use is what is known in programming as a loop. We can loop through a list and have it do something to each element in the loop. And once it's done, we can move on to the next thing.

Think of it like a program that will go though a list of your classmates and ask each one of them for their year in school. It will start at one end of the list and move through asking each one ``What year in school are you?'' and will move on after getting an answer.

Except we want to take a url, add something to it, alter it, then request it and grab a bunch of data from it. Once we're done doing all that, we'll take all that info and cram it into a bigger dataset and then move on to the next one. Here's what that looks like:

\begin{Shaded}
\begin{Highlighting}[]
\NormalTok{uri <-}\StringTok{ "https://www.sports-reference.com"}

\NormalTok{logs <-}\StringTok{ }\KeywordTok{tibble}\NormalTok{()}

\ControlFlowTok{for}\NormalTok{ (i }\ControlFlowTok{in}\NormalTok{ schools)\{}
\NormalTok{  log_url <-}\StringTok{ }\KeywordTok{gsub}\NormalTok{(}\StringTok{"/2019.html"}\NormalTok{,}\StringTok{"/2019-gamelogs.html"}\NormalTok{, i)}
\NormalTok{  school_url <-}\StringTok{ }\KeywordTok{paste}\NormalTok{(uri, log_url, }\DataTypeTok{sep=}\StringTok{""}\NormalTok{)  }\CommentTok{# creating the url to fetch}
  
\NormalTok{  page <-}\StringTok{ }\KeywordTok{read_html}\NormalTok{(school_url)}
  
\NormalTok{  team <-}\StringTok{ }\NormalTok{page }\OperatorTok
\StringTok{    }\KeywordTok{html_nodes}\NormalTok{(}\DataTypeTok{xpath =} \StringTok{'//*[@id="meta"]/div[2]/h1/span[2]'}\NormalTok{) }\OperatorTok
\StringTok{    }\KeywordTok{html_text}\NormalTok{()}
  
\NormalTok{  conference <-}\StringTok{ }\NormalTok{page }\OperatorTok
\StringTok{    }\KeywordTok{html_nodes}\NormalTok{(}\DataTypeTok{xpath =} \StringTok{'//*[@id="meta"]/div[2]/p[1]/a'}\NormalTok{) }\OperatorTok
\StringTok{    }\KeywordTok{html_text}\NormalTok{()}

\NormalTok{  table <-}\StringTok{ }\NormalTok{page }\OperatorTok
\StringTok{    }\KeywordTok{html_nodes}\NormalTok{(}\DataTypeTok{xpath =} \StringTok{'//*[@id="sgl-basic"]'}\NormalTok{) }\OperatorTok
\StringTok{    }\KeywordTok{html_table}\NormalTok{(}\DataTypeTok{header=}\OtherTok{FALSE}\NormalTok{)}

\NormalTok{table <-}\StringTok{ }\NormalTok{table[[}\DecValTok{1}\NormalTok{]] }\OperatorTok\StringTok{ }\KeywordTok{rename}\NormalTok{(}\DataTypeTok{Game=}\NormalTok{X1, }\DataTypeTok{Date=}\NormalTok{X2, }\DataTypeTok{HomeAway=}\NormalTok{X3, }\DataTypeTok{Opponent=}\NormalTok{X4, }\DataTypeTok{W_L=}\NormalTok{X5, }\DataTypeTok{TeamScore=}\NormalTok{X6, }\DataTypeTok{OpponentScore=}\NormalTok{X7, }\DataTypeTok{TeamFG=}\NormalTok{X8, }\DataTypeTok{TeamFGA=}\NormalTok{X9, }\DataTypeTok{TeamFGPCT=}\NormalTok{X10, }\DataTypeTok{Team3P=}\NormalTok{X11, }\DataTypeTok{Team3PA=}\NormalTok{X12, }\DataTypeTok{Team3PPCT=}\NormalTok{X13, }\DataTypeTok{TeamFT=}\NormalTok{X14, }\DataTypeTok{TeamFTA=}\NormalTok{X15, }\DataTypeTok{TeamFTPCT=}\NormalTok{X16, }\DataTypeTok{TeamOffRebounds=}\NormalTok{X17, }\DataTypeTok{TeamTotalRebounds=}\NormalTok{X18, }\DataTypeTok{TeamAssists=}\NormalTok{X19, }\DataTypeTok{TeamSteals=}\NormalTok{X20, }\DataTypeTok{TeamBlocks=}\NormalTok{X21, }\DataTypeTok{TeamTurnovers=}\NormalTok{X22, }\DataTypeTok{TeamPersonalFouls=}\NormalTok{X23, }\DataTypeTok{Blank=}\NormalTok{X24, }\DataTypeTok{OpponentFG=}\NormalTok{X25, }\DataTypeTok{OpponentFGA=}\NormalTok{X26, }\DataTypeTok{OpponentFGPCT=}\NormalTok{X27, }\DataTypeTok{Opponent3P=}\NormalTok{X28, }\DataTypeTok{Opponent3PA=}\NormalTok{X29, }\DataTypeTok{Opponent3PPCT=}\NormalTok{X30, }\DataTypeTok{OpponentFT=}\NormalTok{X31, }\DataTypeTok{OpponentFTA=}\NormalTok{X32, }\DataTypeTok{OpponentFTPCT=}\NormalTok{X33, }\DataTypeTok{OpponentOffRebounds=}\NormalTok{X34, }\DataTypeTok{OpponentTotalRebounds=}\NormalTok{X35, }\DataTypeTok{OpponentAssists=}\NormalTok{X36, }\DataTypeTok{OpponentSteals=}\NormalTok{X37, }\DataTypeTok{OpponentBlocks=}\NormalTok{X38, }\DataTypeTok{OpponentTurnovers=}\NormalTok{X39, }\DataTypeTok{OpponentPersonalFouls=}\NormalTok{X40) }\OperatorTok\StringTok{ }\KeywordTok{filter}\NormalTok{(Game }\OperatorTok{!=}\StringTok{ ""}\NormalTok{) }\OperatorTok\StringTok{ }\KeywordTok{filter}\NormalTok{(Game }\OperatorTok{!=}\StringTok{ "G"}\NormalTok{) }\OperatorTok\StringTok{ }\KeywordTok{mutate}\NormalTok{(}\DataTypeTok{Team=}\NormalTok{team) }\OperatorTok\StringTok{ }\KeywordTok{mutate_at}\NormalTok{(}\KeywordTok{vars}\NormalTok{(}\OperatorTok{-}\NormalTok{Team, }\OperatorTok{-}\NormalTok{Date, }\OperatorTok{-}\NormalTok{Opponent, }\OperatorTok{-}\NormalTok{HomeAway, }\OperatorTok{-}\NormalTok{W_L), as.numeric) }\OperatorTok\StringTok{ }\KeywordTok{mutate}\NormalTok{(}\DataTypeTok{Date =} \KeywordTok{ymd}\NormalTok{(Date)) }\OperatorTok\StringTok{ }\KeywordTok{mutate}\NormalTok{(}\DataTypeTok{Team=}\NormalTok{team, }\DataTypeTok{Conference=}\NormalTok{conference)}

\NormalTok{  logs <-}\StringTok{ }\KeywordTok{rbind}\NormalTok{(logs, table)  }\CommentTok{# binding them all together}
  \KeywordTok{Sys.sleep}\NormalTok{(}\DecValTok{5}\NormalTok{)  }\CommentTok{# Sys.sleep(3) pauses the loop for 3s so as not to overwhelm website's server}
\NormalTok{\}}
\end{Highlighting}
\end{Shaded}

The magic here is in \texttt{for\ (i\ in\ schools)\{}. What that says is for each iterator in schools -- for each school in schools -- do what follows each time. So we take the code we wrote for one school and use it for every school.

This part:

\begin{verbatim}
  log_url <- gsub("/2019.html","/2019-gamelogs.html", i)
  school_url <- paste(uri, log_url, sep="")  # creating the url to fetch
  
  page <- read_html(school_url)
\end{verbatim}

\texttt{log\_url} is what changes our school page url to our logs url, and \texttt{school\_url} is taking that log url and the root domain and merging them together to create the complete url. Then, page just reads that url we created.

What follows that is taken straight from our example of just doing one.

The last bits are using rbind to take our data and mash it into a bigger table, over and over and over again until we have them all in a single table. Then, we tell our scraper to wait a few seconds because we don't want our script to machine gun requests at their server as fast as it can go. That's a guaranteed way to get them to block scrapers, and could knock them off the internet. Aggressive scrapers aren't cool. Don't do it.

Lastly, we write it out to a csv file.

\begin{Shaded}
\begin{Highlighting}[]
\KeywordTok{write.csv}\NormalTok{(logs, }\StringTok{"logs.csv"}\NormalTok{)}
\end{Highlighting}
\end{Shaded}

So with a little programming knowhow, a little bit of problem solving and the stubbornness not to quit on it, you can get a whole lot of data scattered all over the place with not a lot of code.

\hypertarget{one-last-bit}{%
\section{One last bit}\label{one-last-bit}}

Most tables that Sports Reference sites have are in plain vanilla HTML. But some of them -- particularly player based stuff -- are hidden with a little trick. The site puts the data in a comment -- where a browser will ignore it -- and then uses javascript to interpret the commented data. To a human on the page, it looks the same. To a browswer or a scraper, it's invisible. You'll get errors. How do you get around it?

\begin{enumerate}
\def\labelenumi{\arabic{enumi}.}
\tightlist
\item
  Scrape the comments.
\item
  Turn the comment into text.
\item
  Then read that text as html.
\item
  Proceed as normal.
\end{enumerate}

\begin{Shaded}
\begin{Highlighting}[]
\NormalTok{h <-}\StringTok{ }\KeywordTok{read_html}\NormalTok{(}\StringTok{'https://www.baseball-reference.com/leagues/MLB/2017-standard-pitching.shtml'}\NormalTok{)}

\NormalTok{df <-}\StringTok{ }\NormalTok{h }\OperatorTok\StringTok{ }\KeywordTok{html_nodes}\NormalTok{(}\DataTypeTok{xpath =} \StringTok{'//comment()'}\NormalTok{) }\OperatorTok\StringTok{    }\CommentTok{# select comment nodes}
\StringTok{    }\KeywordTok{html_text}\NormalTok{() }\OperatorTok\StringTok{    }\CommentTok{# extract comment text}
\StringTok{    }\KeywordTok{paste}\NormalTok{(}\DataTypeTok{collapse =} \StringTok{''}\NormalTok{) }\OperatorTok\StringTok{    }\CommentTok{# collapse to a single string}
\StringTok{    }\KeywordTok{read_html}\NormalTok{() }\OperatorTok\StringTok{    }\CommentTok{# reparse to HTML}
\StringTok{    }\KeywordTok{html_node}\NormalTok{(}\StringTok{'table'}\NormalTok{) }\OperatorTok\StringTok{    }\CommentTok{# select the desired table}
\StringTok{    }\KeywordTok{html_table}\NormalTok{() }
\end{Highlighting}
\end{Shaded}

\hypertarget{annotations}{%
\chapter{Annotations}\label{annotations}}

Some of the best sports data visualizations start with a provocative question. At a college just under three hours from Kansas City, my classes are lousy with Chiefs fans. So the first day of classes in the spring of 2019, I asked them: Are the Chief's Screwed in the Playoffs? The answer ultimately was yes, and how I was able to make that argument before a playoff game had even been played is a good example of how labeling and annotations can make a chart much better.

Going to add a new library to the mix called \texttt{ggrepel}. You'll need to install it in the console with \texttt{install.packages("ggrepel")}.

\begin{Shaded}
\begin{Highlighting}[]
\KeywordTok{library}\NormalTok{(tidyverse)}
\KeywordTok{library}\NormalTok{(ggrepel)}
\end{Highlighting}
\end{Shaded}

Now we'll grab the data and join that data together using the Team name as the common element.

\begin{Shaded}
\begin{Highlighting}[]
\NormalTok{offense <-}\StringTok{ }\KeywordTok{read_csv}\NormalTok{(}\StringTok{"data/nfloffense.csv"}\NormalTok{)}
\end{Highlighting}
\end{Shaded}

\begin{verbatim}
## Parsed with column specification:
## cols(
##   .default = col_double(),
##   Team = col_character()
## )
\end{verbatim}

\begin{verbatim}
## See spec(...) for full column specifications.
\end{verbatim}

\begin{Shaded}
\begin{Highlighting}[]
\NormalTok{defense <-}\StringTok{ }\KeywordTok{read_csv}\NormalTok{(}\StringTok{"data/nfldefense.csv"}\NormalTok{)}
\end{Highlighting}
\end{Shaded}

\begin{verbatim}
## Parsed with column specification:
## cols(
##   .default = col_double(),
##   Team = col_character()
## )
## See spec(...) for full column specifications.
\end{verbatim}

\begin{Shaded}
\begin{Highlighting}[]
\NormalTok{total <-}\StringTok{ }\NormalTok{offense }\OperatorTok\StringTok{ }\KeywordTok{left_join}\NormalTok{(defense, }\DataTypeTok{by=}\StringTok{"Team"}\NormalTok{)}

\KeywordTok{head}\NormalTok{(total)}
\end{Highlighting}
\end{Shaded}

\begin{verbatim}
## # A tibble: 6 x 52
##   Team      G PointsFor OffYards OffPlays OffYardsPerPlay OffensiveTurnov~
##   <chr> <dbl>     <dbl>    <dbl>    <dbl>           <dbl>            <dbl>
## 1 Kans~    16       565     6810      996             6.8               18
## 2 Los ~    16       527     6738     1060             6.4               19
## 3 New ~    16       504     6067     1010             6                 16
## 4 New ~    16       436     6295     1073             5.9               18
## 5 Indi~    16       433     6179     1070             5.8               24
## 6 Pitt~    16       428     6453     1058             6.1               26
## # ... with 45 more variables: FumblesLost <dbl>, OffFirstDowns <dbl>,
## #   OffPassingComp <dbl>, OffPassingAtt <dbl>, OffPassingYards <dbl>,
## #   OffensivePassingTD <dbl>, OffPassingINT <dbl>,
## #   OffensivePassingYardsPerAtt <dbl>, OffensivePassingFirstDowns <dbl>,
## #   OffRushingAtt <dbl>, OffRushingYards <dbl>, OffRushingTD <dbl>,
## #   RushingYardsPerAtt <dbl>, RushingFirstDowns <dbl>,
## #   OffensivePenalties <dbl>, OffPenaltyYards <dbl>,
## #   OffFirstFromPenalties <dbl>, OffScoringPct <dbl>,
## #   OffensiveTurnoverPct <dbl>, OffensiveExpectedPoints <dbl>,
## #   PointsAllowed <dbl>, YdsAllowed <dbl>, PlaysFaced <dbl>,
## #   DefYardPerPlay <dbl>, Takeaways <dbl>, DefFumblesLost <dbl>,
## #   FirstDownsAllowed <dbl>, PassingCompsAllowed <dbl>, PassingAttFaced <dbl>,
## #   PassingYdsAllowed <dbl>, PassingTDAllowed <dbl>, DefPassingINT <dbl>,
## #   PassingYardsPerPlayAllowed <dbl>, PassingFirstDownsAllowed <dbl>,
## #   RushingAttFaced <dbl>, RushingYdsAllowed <dbl>, RushingTDAllowed <dbl>,
## #   RushingYardsPerAttAllowed <dbl>, RushingFirstDownsAllowed <dbl>,
## #   DefPenalties <dbl>, DefPenaltyYards <dbl>, DefFirstDownByPenalties <dbl>,
## #   OffensiveScoringPctAllowed <dbl>, DefTurnoverPercentage <dbl>,
## #   DefExpectedPoints <dbl>
\end{verbatim}

I'm going to set up a point chart that places team on two-axes -- yards per play on offense on the x axis, and yards per play on defense.

To build the annotations, I want the league average for offensive yards per play and defensive yards per play. We're going to use those as a proxy for quality. If your team averages more yards per play on offense, that's good. If they average fewer yards per play on defense, that too is good. So that sets up a situation where we have four corners, anchored by good at both and bad at both. The averages will create lines to divide those four corners up.

\begin{Shaded}
\begin{Highlighting}[]
\NormalTok{league_averages <-}\StringTok{ }\NormalTok{total }\OperatorTok\StringTok{ }\KeywordTok{summarise}\NormalTok{(}\DataTypeTok{AvgOffYardsPer =} \KeywordTok{mean}\NormalTok{(OffYardsPerPlay), }\DataTypeTok{AvgDefYardsPer =} \KeywordTok{mean}\NormalTok{(DefYardPerPlay))}

\NormalTok{league_averages}
\end{Highlighting}
\end{Shaded}

\begin{verbatim}
## # A tibble: 1 x 2
##   AvgOffYardsPer AvgDefYardsPer
##            <dbl>          <dbl>
## 1           5.59           5.59
\end{verbatim}

I also want to highlight playoff teams and, of course, the Chiefs, since that was my question. Are they screwed. First, we filter them from our total list.

\begin{Shaded}
\begin{Highlighting}[]
\NormalTok{playoff_teams <-}\StringTok{ }\KeywordTok{c}\NormalTok{(}\StringTok{"Kansas City Chiefs"}\NormalTok{, }\StringTok{"New England Patriots"}\NormalTok{, }\StringTok{"Los Angeles Chargers"}\NormalTok{, }\StringTok{"Indianapolis Colts"}\NormalTok{, }\StringTok{"New Orleans Saints"}\NormalTok{, }\StringTok{"Los Angeles Rams"}\NormalTok{, }\StringTok{"Chicago Bears"}\NormalTok{, }\StringTok{"Dallas Cowboys"}\NormalTok{, }\StringTok{"Philadelphia Eagles"}\NormalTok{)}

\NormalTok{playoffs <-}\StringTok{ }\NormalTok{total }\OperatorTok\StringTok{ }\KeywordTok{filter}\NormalTok{(Team }\OperatorTok\StringTok{ }\NormalTok{playoff_teams)}

\NormalTok{chiefs <-}\StringTok{ }\NormalTok{total }\OperatorTok\StringTok{ }\KeywordTok{filter}\NormalTok{(Team }\OperatorTok{==}\StringTok{ "Kansas City Chiefs"}\NormalTok{)}
\end{Highlighting}
\end{Shaded}

Now we create the plot. We have three geom\_points, starting with everyone, then playoff teams, then the Chiefs. I alter the colors on each to separate them. Next, I add a geom\_hline to add the horizontal line of my defensive average and a geom\_vline for my offensive average. Next, I want to add some text annotations, labeling two corners of my chart (the other two, in my opinion, become obvious). Then, I want to label all the playoff teams. I use \texttt{geom\_text\_repel} to do that -- it's using the ggrepel library to push the text away from the dots, respective of other labels and other dots. It means you don't have to move them around so you can read them, or so they don't cover up the dots.

The rest is just adding labels and messing with the theme.

\begin{Shaded}
\begin{Highlighting}[]
\KeywordTok{ggplot}\NormalTok{() }\OperatorTok{+}\StringTok{ }
\StringTok{  }\KeywordTok{geom_point}\NormalTok{(}\DataTypeTok{data=}\NormalTok{total, }\KeywordTok{aes}\NormalTok{(}\DataTypeTok{x=}\NormalTok{OffYardsPerPlay, }\DataTypeTok{y=}\NormalTok{DefYardPerPlay), }\DataTypeTok{color=}\StringTok{"light grey"}\NormalTok{) }\OperatorTok{+}
\StringTok{  }\KeywordTok{geom_point}\NormalTok{(}\DataTypeTok{data=}\NormalTok{playoffs, }\KeywordTok{aes}\NormalTok{(}\DataTypeTok{x=}\NormalTok{OffYardsPerPlay, }\DataTypeTok{y=}\NormalTok{DefYardPerPlay)) }\OperatorTok{+}
\StringTok{  }\KeywordTok{geom_point}\NormalTok{(}\DataTypeTok{data=}\NormalTok{chiefs, }\KeywordTok{aes}\NormalTok{(}\DataTypeTok{x=}\NormalTok{OffYardsPerPlay, }\DataTypeTok{y=}\NormalTok{DefYardPerPlay), }\DataTypeTok{color=}\StringTok{"red"}\NormalTok{) }\OperatorTok{+}
\StringTok{  }\KeywordTok{geom_hline}\NormalTok{(}\DataTypeTok{yintercept=}\FloatTok{5.59375}\NormalTok{, }\DataTypeTok{color=}\StringTok{"dark grey"}\NormalTok{) }\OperatorTok{+}\StringTok{ }
\StringTok{  }\KeywordTok{geom_vline}\NormalTok{(}\DataTypeTok{xintercept=}\FloatTok{5.590625}\NormalTok{, }\DataTypeTok{color=}\StringTok{"dark grey"}\NormalTok{) }\OperatorTok{+}\StringTok{ }
\StringTok{  }\KeywordTok{geom_text}\NormalTok{(}\KeywordTok{aes}\NormalTok{(}\DataTypeTok{x=}\FloatTok{6.2}\NormalTok{, }\DataTypeTok{y=}\DecValTok{5}\NormalTok{, }\DataTypeTok{label=}\StringTok{"Good Offense, Good Defense"}\NormalTok{), }\DataTypeTok{color=}\StringTok{"light blue"}\NormalTok{) }\OperatorTok{+}
\StringTok{  }\KeywordTok{geom_text}\NormalTok{(}\KeywordTok{aes}\NormalTok{(}\DataTypeTok{x=}\DecValTok{5}\NormalTok{, }\DataTypeTok{y=}\DecValTok{6}\NormalTok{, }\DataTypeTok{label=}\StringTok{"Bad Defense, Bad Offense"}\NormalTok{), }\DataTypeTok{color=}\StringTok{"light blue"}\NormalTok{) }\OperatorTok{+}
\StringTok{  }\KeywordTok{geom_text_repel}\NormalTok{(}\DataTypeTok{data=}\NormalTok{playoffs, }\KeywordTok{aes}\NormalTok{(}\DataTypeTok{x=}\NormalTok{OffYardsPerPlay, }\DataTypeTok{y=}\NormalTok{DefYardPerPlay, }\DataTypeTok{label=}\NormalTok{Team)) }\OperatorTok{+}
\StringTok{  }\KeywordTok{labs}\NormalTok{(}\DataTypeTok{x=}\StringTok{"Offensive Yards Per Play"}\NormalTok{, }\DataTypeTok{y=}\StringTok{"Defensive Points Per Play"}\NormalTok{, }\DataTypeTok{title=}\StringTok{"Are the Chiefs screwed in the playoffs?"}\NormalTok{, }\DataTypeTok{subtitle=}\StringTok{"Their offense is great. Their defense? Not so much"}\NormalTok{, }\DataTypeTok{caption=}\StringTok{"Source: Sports-Reference.com | By Matt Waite"}\NormalTok{) }\OperatorTok{+}
\StringTok{  }\KeywordTok{theme_minimal}\NormalTok{() }\OperatorTok{+}\StringTok{ }
\StringTok{  }\KeywordTok{theme}\NormalTok{(}
    \DataTypeTok{plot.title =} \KeywordTok{element_text}\NormalTok{(}\DataTypeTok{size =} \DecValTok{16}\NormalTok{, }\DataTypeTok{face =} \StringTok{"bold"}\NormalTok{),}
    \DataTypeTok{axis.title =} \KeywordTok{element_text}\NormalTok{(}\DataTypeTok{size =} \DecValTok{10}\NormalTok{),}
    \DataTypeTok{axis.text =} \KeywordTok{element_text}\NormalTok{(}\DataTypeTok{size =} \DecValTok{7}\NormalTok{),}
    \DataTypeTok{axis.ticks =} \KeywordTok{element_blank}\NormalTok{(),}
    \DataTypeTok{panel.grid.minor =} \KeywordTok{element_blank}\NormalTok{(),}
    \DataTypeTok{panel.grid.major.x =} \KeywordTok{element_blank}\NormalTok{()}
\NormalTok{  )}
\end{Highlighting}
\end{Shaded}

\includegraphics{SportsData_files/figure-latex/unnamed-chunk-254-1.pdf}

\hypertarget{finishing-touches-part-1}{%
\chapter{Finishing touches, part 1}\label{finishing-touches-part-1}}

The output from ggplot is good, but not great. We need to add some pieces to it. The elements of a good graphic are:

\begin{itemize}
\tightlist
\item
  Headline
\item
  Chatter
\item
  The main body
\item
  Annotations
\item
  Labels
\item
  Source line
\item
  Credit line
\end{itemize}

That looks like:

\includegraphics[width=12.97in]{images/chartannotated}

\hypertarget{graphics-vs-visual-stories}{%
\section{Graphics vs visual stories}\label{graphics-vs-visual-stories}}

While the elements above are nearly required in every chart, they aren't when you are making visual stories.

\begin{itemize}
\tightlist
\item
  When you have a visual story, things like credit lines can become a byline.
\item
  In visual stories, source lines are often a note at the end of the story.
\item
  Graphics don't always get headlines -- sometimes just labels, letting the visual story headline carry the load.
\end{itemize}

\href{https://www.nytimes.com/interactive/2018/02/14/business/economy/inflation-prices.html}{An example from The Upshot}. Note how the charts don't have headlines, source or credit lines.

\hypertarget{getting-ggplot-closer-to-output}{%
\section{Getting ggplot closer to output}\label{getting-ggplot-closer-to-output}}

Let's explore fixing up ggplot's output before we send it to a finishing program like Adobe Illustrator. We'll need a graphic to work with first.

\begin{Shaded}
\begin{Highlighting}[]
\KeywordTok{library}\NormalTok{(tidyverse)}
\KeywordTok{library}\NormalTok{(ggrepel)}
\end{Highlighting}
\end{Shaded}

\begin{Shaded}
\begin{Highlighting}[]
\NormalTok{scoring <-}\StringTok{ }\KeywordTok{read_csv}\NormalTok{(}\StringTok{"data/scoringoffense.csv"}\NormalTok{)}
\end{Highlighting}
\end{Shaded}

\begin{verbatim}
## Parsed with column specification:
## cols(
##   Name = col_character(),
##   G = col_double(),
##   TD = col_double(),
##   FG = col_double(),
##   `1XP` = col_double(),
##   `2XP` = col_double(),
##   Safety = col_double(),
##   Points = col_double(),
##   `Points/G` = col_double(),
##   Year = col_double()
## )
\end{verbatim}

\begin{Shaded}
\begin{Highlighting}[]
\NormalTok{total <-}\StringTok{ }\KeywordTok{read_csv}\NormalTok{(}\StringTok{"data/totaloffense.csv"}\NormalTok{)}
\end{Highlighting}
\end{Shaded}

\begin{verbatim}
## Parsed with column specification:
## cols(
##   Name = col_character(),
##   G = col_double(),
##   `Rush Yards` = col_double(),
##   `Pass Yards` = col_double(),
##   Plays = col_double(),
##   `Total Yards` = col_double(),
##   `Yards/Play` = col_double(),
##   `Yards/G` = col_double(),
##   Year = col_double()
## )
\end{verbatim}

\begin{Shaded}
\begin{Highlighting}[]
\NormalTok{offense <-}\StringTok{ }\NormalTok{total }\OperatorTok\StringTok{ }\KeywordTok{left_join}\NormalTok{(scoring, }\DataTypeTok{by=}\KeywordTok{c}\NormalTok{(}\StringTok{"Name"}\NormalTok{, }\StringTok{"Year"}\NormalTok{))}
\end{Highlighting}
\end{Shaded}

We're going to need this later, so let's grab Nebraska's 2018 stats from this dataframe.

\begin{Shaded}
\begin{Highlighting}[]
\NormalTok{nu <-}\StringTok{ }\NormalTok{offense }\OperatorTok\StringTok{ }\KeywordTok{filter}\NormalTok{(Name }\OperatorTok{==}\StringTok{ "Nebraska"}\NormalTok{) }\OperatorTok\StringTok{ }\KeywordTok{filter}\NormalTok{(Year }\OperatorTok{==}\StringTok{ }\DecValTok{2018}\NormalTok{)}
\end{Highlighting}
\end{Shaded}

We'll start with the basics.

\begin{Shaded}
\begin{Highlighting}[]
\KeywordTok{ggplot}\NormalTok{(offense, }\KeywordTok{aes}\NormalTok{(}\DataTypeTok{x=}\StringTok{`}\DataTypeTok{Yards/G}\StringTok{`}\NormalTok{, }\DataTypeTok{y=}\StringTok{`}\DataTypeTok{Points/G}\StringTok{`}\NormalTok{)) }\OperatorTok{+}\StringTok{ }
\StringTok{  }\KeywordTok{geom_point}\NormalTok{(}\DataTypeTok{color=}\StringTok{"grey"}\NormalTok{)}
\end{Highlighting}
\end{Shaded}

\includegraphics{SportsData_files/figure-latex/unnamed-chunk-259-1.pdf}

Let's take changing things one by one. The first thing we can do is change the figure size. Sometimes you don't want a square. We can use the \texttt{knitr} output settings in our chunk to do this easily in our notebooks.

\begin{Shaded}
\begin{Highlighting}[]
\KeywordTok{ggplot}\NormalTok{(offense, }\KeywordTok{aes}\NormalTok{(}\DataTypeTok{x=}\StringTok{`}\DataTypeTok{Yards/G}\StringTok{`}\NormalTok{, }\DataTypeTok{y=}\StringTok{`}\DataTypeTok{Points/G}\StringTok{`}\NormalTok{)) }\OperatorTok{+}\StringTok{ }
\StringTok{  }\KeywordTok{geom_point}\NormalTok{(}\DataTypeTok{color=}\StringTok{"grey"}\NormalTok{)}
\end{Highlighting}
\end{Shaded}

\includegraphics{SportsData_files/figure-latex/unnamed-chunk-260-1.pdf}

Now let's add a fit line.

\begin{Shaded}
\begin{Highlighting}[]
\KeywordTok{ggplot}\NormalTok{(offense, }\KeywordTok{aes}\NormalTok{(}\DataTypeTok{x=}\StringTok{`}\DataTypeTok{Yards/G}\StringTok{`}\NormalTok{, }\DataTypeTok{y=}\StringTok{`}\DataTypeTok{Points/G}\StringTok{`}\NormalTok{)) }\OperatorTok{+}\StringTok{ }
\StringTok{  }\KeywordTok{geom_point}\NormalTok{(}\DataTypeTok{color=}\StringTok{"grey"}\NormalTok{) }\OperatorTok{+}\StringTok{ }\KeywordTok{geom_smooth}\NormalTok{(}\DataTypeTok{method=}\NormalTok{lm, }\DataTypeTok{se=}\OtherTok{FALSE}\NormalTok{)}
\end{Highlighting}
\end{Shaded}

\includegraphics{SportsData_files/figure-latex/unnamed-chunk-261-1.pdf}

And now some labels.

\begin{Shaded}
\begin{Highlighting}[]
\KeywordTok{ggplot}\NormalTok{(offense, }\KeywordTok{aes}\NormalTok{(}\DataTypeTok{x=}\StringTok{`}\DataTypeTok{Yards/G}\StringTok{`}\NormalTok{, }\DataTypeTok{y=}\StringTok{`}\DataTypeTok{Points/G}\StringTok{`}\NormalTok{)) }\OperatorTok{+}\StringTok{ }
\StringTok{  }\KeywordTok{geom_point}\NormalTok{(}\DataTypeTok{color=}\StringTok{"grey"}\NormalTok{) }\OperatorTok{+}\StringTok{ }\KeywordTok{geom_smooth}\NormalTok{(}\DataTypeTok{method=}\NormalTok{lm, }\DataTypeTok{se=}\OtherTok{FALSE}\NormalTok{) }\OperatorTok{+}\StringTok{ }
\StringTok{  }\KeywordTok{labs}\NormalTok{(}\DataTypeTok{x=}\StringTok{"Total yards per game"}\NormalTok{, }\DataTypeTok{y=}\StringTok{"Points per game"}\NormalTok{, }\DataTypeTok{title=}\StringTok{"Nebraska's underperforming offense"}\NormalTok{, }\DataTypeTok{subtitle=}\StringTok{"The Husker's offense was the strength of the team. They underperformed."}\NormalTok{, }\DataTypeTok{caption=}\StringTok{"Source: NCAA | By Matt Waite"}\NormalTok{)}
\end{Highlighting}
\end{Shaded}

\includegraphics{SportsData_files/figure-latex/unnamed-chunk-262-1.pdf}

Let's get rid of the default plot look and drop the grey background.

\begin{Shaded}
\begin{Highlighting}[]
\KeywordTok{ggplot}\NormalTok{(offense, }\KeywordTok{aes}\NormalTok{(}\DataTypeTok{x=}\StringTok{`}\DataTypeTok{Yards/G}\StringTok{`}\NormalTok{, }\DataTypeTok{y=}\StringTok{`}\DataTypeTok{Points/G}\StringTok{`}\NormalTok{)) }\OperatorTok{+}\StringTok{ }
\StringTok{  }\KeywordTok{geom_point}\NormalTok{(}\DataTypeTok{color=}\StringTok{"grey"}\NormalTok{) }\OperatorTok{+}\StringTok{ }\KeywordTok{geom_smooth}\NormalTok{(}\DataTypeTok{method=}\NormalTok{lm, }\DataTypeTok{se=}\OtherTok{FALSE}\NormalTok{) }\OperatorTok{+}\StringTok{ }
\StringTok{  }\KeywordTok{labs}\NormalTok{(}\DataTypeTok{x=}\StringTok{"Total yards per game"}\NormalTok{, }\DataTypeTok{y=}\StringTok{"Points per game"}\NormalTok{, }\DataTypeTok{title=}\StringTok{"Nebraska's underperforming offense"}\NormalTok{, }\DataTypeTok{subtitle=}\StringTok{"The Husker's offense was the strength of the team. They underperformed."}\NormalTok{, }\DataTypeTok{caption=}\StringTok{"Source: NCAA | By Matt Waite"}\NormalTok{) }\OperatorTok{+}\StringTok{ }
\StringTok{  }\KeywordTok{theme_minimal}\NormalTok{()}
\end{Highlighting}
\end{Shaded}

\includegraphics{SportsData_files/figure-latex/unnamed-chunk-263-1.pdf}

Off to a good start, but our text has no real heirarchy. We'd want our headline to stand out more. So let's change that. When it comes to changing text, the place to do that is in the theme element. \href{http://ggplot2.tidyverse.org/reference/theme.html}{There are a lot of ways to modify the theme}. We'll start easy. Let's make the headline bigger and bold.

\begin{Shaded}
\begin{Highlighting}[]
\KeywordTok{ggplot}\NormalTok{(offense, }\KeywordTok{aes}\NormalTok{(}\DataTypeTok{x=}\StringTok{`}\DataTypeTok{Yards/G}\StringTok{`}\NormalTok{, }\DataTypeTok{y=}\StringTok{`}\DataTypeTok{Points/G}\StringTok{`}\NormalTok{)) }\OperatorTok{+}\StringTok{ }
\StringTok{  }\KeywordTok{geom_point}\NormalTok{(}\DataTypeTok{color=}\StringTok{"grey"}\NormalTok{) }\OperatorTok{+}\StringTok{ }\KeywordTok{geom_smooth}\NormalTok{(}\DataTypeTok{method=}\NormalTok{lm, }\DataTypeTok{se=}\OtherTok{FALSE}\NormalTok{) }\OperatorTok{+}\StringTok{ }
\StringTok{  }\KeywordTok{labs}\NormalTok{(}\DataTypeTok{x=}\StringTok{"Total yards per game"}\NormalTok{, }\DataTypeTok{y=}\StringTok{"Points per game"}\NormalTok{, }\DataTypeTok{title=}\StringTok{"Nebraska's underperforming offense"}\NormalTok{, }\DataTypeTok{subtitle=}\StringTok{"The Husker's offense was the strength of the team. They underperformed."}\NormalTok{, }\DataTypeTok{caption=}\StringTok{"Source: NCAA | By Matt Waite"}\NormalTok{) }\OperatorTok{+}\StringTok{ }
\StringTok{  }\KeywordTok{theme_minimal}\NormalTok{() }\OperatorTok{+}\StringTok{ }
\StringTok{  }\KeywordTok{theme}\NormalTok{(}
    \DataTypeTok{plot.title =} \KeywordTok{element_text}\NormalTok{(}\DataTypeTok{size =} \DecValTok{16}\NormalTok{, }\DataTypeTok{face =} \StringTok{"bold"}\NormalTok{)}
\NormalTok{    ) }
\end{Highlighting}
\end{Shaded}

\includegraphics{SportsData_files/figure-latex/unnamed-chunk-264-1.pdf}

Now let's fix a few other things -- like the axis labels being too big, the subtitle could be bigger and lets drop some grid lines.

\begin{Shaded}
\begin{Highlighting}[]
\KeywordTok{ggplot}\NormalTok{(offense, }\KeywordTok{aes}\NormalTok{(}\DataTypeTok{x=}\StringTok{`}\DataTypeTok{Yards/G}\StringTok{`}\NormalTok{, }\DataTypeTok{y=}\StringTok{`}\DataTypeTok{Points/G}\StringTok{`}\NormalTok{)) }\OperatorTok{+}\StringTok{ }
\StringTok{  }\KeywordTok{geom_point}\NormalTok{(}\DataTypeTok{color=}\StringTok{"grey"}\NormalTok{) }\OperatorTok{+}\StringTok{ }\KeywordTok{geom_smooth}\NormalTok{(}\DataTypeTok{method=}\NormalTok{lm, }\DataTypeTok{se=}\OtherTok{FALSE}\NormalTok{) }\OperatorTok{+}\StringTok{ }
\StringTok{  }\KeywordTok{labs}\NormalTok{(}\DataTypeTok{x=}\StringTok{"Total yards per game"}\NormalTok{, }\DataTypeTok{y=}\StringTok{"Points per game"}\NormalTok{, }\DataTypeTok{title=}\StringTok{"Nebraska's underperforming offense"}\NormalTok{, }\DataTypeTok{subtitle=}\StringTok{"The Husker's offense was the strength of the team. They underperformed."}\NormalTok{, }\DataTypeTok{caption=}\StringTok{"Source: NCAA | By Matt Waite"}\NormalTok{) }\OperatorTok{+}\StringTok{ }
\StringTok{  }\KeywordTok{theme_minimal}\NormalTok{() }\OperatorTok{+}\StringTok{ }
\StringTok{  }\KeywordTok{theme}\NormalTok{(}
    \DataTypeTok{plot.title =} \KeywordTok{element_text}\NormalTok{(}\DataTypeTok{size =} \DecValTok{16}\NormalTok{, }\DataTypeTok{face =} \StringTok{"bold"}\NormalTok{),}
    \DataTypeTok{axis.title =} \KeywordTok{element_text}\NormalTok{(}\DataTypeTok{size =} \DecValTok{8}\NormalTok{), }
    \DataTypeTok{plot.subtitle =} \KeywordTok{element_text}\NormalTok{(}\DataTypeTok{size=}\DecValTok{10}\NormalTok{), }
    \DataTypeTok{panel.grid.minor =} \KeywordTok{element_blank}\NormalTok{()}
\NormalTok{    ) }
\end{Highlighting}
\end{Shaded}

\includegraphics{SportsData_files/figure-latex/unnamed-chunk-265-1.pdf}

Missing from this graph is the context that the headline promises. Where is Nebraska? We haven't added it yet. So let's add a point and a label for it.

\begin{Shaded}
\begin{Highlighting}[]
\KeywordTok{ggplot}\NormalTok{(offense, }\KeywordTok{aes}\NormalTok{(}\DataTypeTok{x=}\StringTok{`}\DataTypeTok{Yards/G}\StringTok{`}\NormalTok{, }\DataTypeTok{y=}\StringTok{`}\DataTypeTok{Points/G}\StringTok{`}\NormalTok{)) }\OperatorTok{+}\StringTok{ }
\StringTok{  }\KeywordTok{geom_point}\NormalTok{(}\DataTypeTok{color=}\StringTok{"grey"}\NormalTok{) }\OperatorTok{+}\StringTok{ }\KeywordTok{geom_smooth}\NormalTok{(}\DataTypeTok{method=}\NormalTok{lm, }\DataTypeTok{se=}\OtherTok{FALSE}\NormalTok{) }\OperatorTok{+}\StringTok{ }
\StringTok{  }\KeywordTok{labs}\NormalTok{(}\DataTypeTok{x=}\StringTok{"Total yards per game"}\NormalTok{, }\DataTypeTok{y=}\StringTok{"Points per game"}\NormalTok{, }\DataTypeTok{title=}\StringTok{"Nebraska's underperforming offense"}\NormalTok{, }\DataTypeTok{subtitle=}\StringTok{"The Husker's offense was the strength of the team. They underperformed."}\NormalTok{, }\DataTypeTok{caption=}\StringTok{"Source: NCAA | By Matt Waite"}\NormalTok{) }\OperatorTok{+}\StringTok{ }
\StringTok{  }\KeywordTok{theme_minimal}\NormalTok{() }\OperatorTok{+}\StringTok{ }
\StringTok{  }\KeywordTok{theme}\NormalTok{(}
    \DataTypeTok{plot.title =} \KeywordTok{element_text}\NormalTok{(}\DataTypeTok{size =} \DecValTok{16}\NormalTok{, }\DataTypeTok{face =} \StringTok{"bold"}\NormalTok{),}
    \DataTypeTok{axis.title =} \KeywordTok{element_text}\NormalTok{(}\DataTypeTok{size =} \DecValTok{8}\NormalTok{), }
    \DataTypeTok{plot.subtitle =} \KeywordTok{element_text}\NormalTok{(}\DataTypeTok{size=}\DecValTok{10}\NormalTok{), }
    \DataTypeTok{panel.grid.minor =} \KeywordTok{element_blank}\NormalTok{()}
\NormalTok{    ) }\OperatorTok{+}
\StringTok{  }\KeywordTok{geom_point}\NormalTok{(}\DataTypeTok{data=}\NormalTok{nu, }\KeywordTok{aes}\NormalTok{(}\DataTypeTok{x=}\StringTok{`}\DataTypeTok{Yards/G}\StringTok{`}\NormalTok{, }\DataTypeTok{y=}\StringTok{`}\DataTypeTok{Points/G}\StringTok{`}\NormalTok{), }\DataTypeTok{color=}\StringTok{"red"}\NormalTok{) }\OperatorTok{+}\StringTok{ }
\StringTok{  }\KeywordTok{geom_text_repel}\NormalTok{(}\DataTypeTok{data=}\NormalTok{nu, }\KeywordTok{aes}\NormalTok{(}\DataTypeTok{x=}\StringTok{`}\DataTypeTok{Yards/G}\StringTok{`}\NormalTok{, }\DataTypeTok{y=}\StringTok{`}\DataTypeTok{Points/G}\StringTok{`}\NormalTok{, }\DataTypeTok{label=}\StringTok{"Nebraska 2018"}\NormalTok{))}
\end{Highlighting}
\end{Shaded}

\includegraphics{SportsData_files/figure-latex/unnamed-chunk-266-1.pdf}

If we're happy with this output -- if it meets all of our needs for publication -- then we can simply export it as a png file. We do that by adding \texttt{+\ ggsave("plot.png",\ width=5,\ height=2)} to the end of our code. Note the width and the height are from our knitr parameters at the top -- you have to repeat them or the graph will export at the default 7x7.

If there's more work you want to do with this graph that isn't easy or possible in R but is in Illustrator, simply change the file extension to \texttt{pdf} instead of \texttt{png}. The pdf will open as a vector file in Illustrator with everything being fully editable.

\hypertarget{finishing-touches-2}{%
\chapter{Finishing Touches 2}\label{finishing-touches-2}}

Frequently in my classes, students use the waffle charts library quite extensively to make graphics. This is a quick walkthough on how to get a waffle chart into a publication ready state.

\begin{Shaded}
\begin{Highlighting}[]
\KeywordTok{library}\NormalTok{(waffle)}
\end{Highlighting}
\end{Shaded}

Let's look at the offensive numbers from Nebraska v. Wisconsin football game. Nebraska lost 41-24, but Wisconsin gained only 15 yards more than Nebraska did. You can find the \href{https://www.ncaa.com/game/football/fbs/2018/10/06/nebraska-wisconsin/team-stats}{official stats on the NCAA's website}.

I'm going to make two vectors for each team and record rushing yards and passing yards.

\begin{Shaded}
\begin{Highlighting}[]
\NormalTok{nu <-}\StringTok{ }\KeywordTok{c}\NormalTok{(}\StringTok{"Rushing"}\NormalTok{=}\DecValTok{111}\NormalTok{, }\StringTok{"Passing"}\NormalTok{=}\DecValTok{407}\NormalTok{, }\DecValTok{15}\NormalTok{)}
\NormalTok{wi <-}\StringTok{ }\KeywordTok{c}\NormalTok{(}\StringTok{"Rushing"}\NormalTok{=}\DecValTok{370}\NormalTok{, }\StringTok{"Passing"}\NormalTok{=}\DecValTok{163}\NormalTok{, }\DecValTok{0}\NormalTok{)}
\end{Highlighting}
\end{Shaded}

So what does the breakdown of Nebraska's night look like? How balanced was the offense?

The waffle library can break this down in a way that's easier on the eyes than a pie chart. We call the library, add the data, specify the number of rows, give it a title and an x value label, and to clean up a quirk of the library, we've got to specify colors.

\textbf{ADDITIONALLY}

We can add labels and themes, but you have to be careful. The waffle library is applying it's own theme, but if we override something they are using in their theme, some things that are hidden come back and make it worse. So here is an example of how to use ggplot's \texttt{labs} and the theme to make a fully publication ready graphic.

\begin{Shaded}
\begin{Highlighting}[]
\KeywordTok{waffle}\NormalTok{(nu}\OperatorTok{/}\DecValTok{10}\NormalTok{, }\DataTypeTok{rows =} \DecValTok{5}\NormalTok{, }\DataTypeTok{xlab=}\StringTok{"1 square = 10 yards"}\NormalTok{, }\DataTypeTok{colors =} \KeywordTok{c}\NormalTok{(}\StringTok{"black"}\NormalTok{, }\StringTok{"red"}\NormalTok{, }\StringTok{"white"}\NormalTok{)) }\OperatorTok{+}\StringTok{ }\KeywordTok{labs}\NormalTok{(}\DataTypeTok{title=}\StringTok{"Nebraska vs Wisconsin on offense"}\NormalTok{, }\DataTypeTok{subtitle=}\StringTok{"The Huskers couldn't get much of a running game going."}\NormalTok{, }\DataTypeTok{caption=}\StringTok{"Source: NCAA | Graphic by Matt Waite"}\NormalTok{) }\OperatorTok{+}\StringTok{ }
\StringTok{  }\KeywordTok{theme}\NormalTok{(}
    \DataTypeTok{plot.title =} \KeywordTok{element_text}\NormalTok{(}\DataTypeTok{size =} \DecValTok{16}\NormalTok{, }\DataTypeTok{face =} \StringTok{"bold"}\NormalTok{),}
    \DataTypeTok{axis.title =} \KeywordTok{element_text}\NormalTok{(}\DataTypeTok{size =} \DecValTok{10}\NormalTok{),}
    \DataTypeTok{axis.title.y =} \KeywordTok{element_blank}\NormalTok{()}
\NormalTok{  )}
\end{Highlighting}
\end{Shaded}

\includegraphics{SportsData_files/figure-latex/unnamed-chunk-269-1.pdf}

Note: The alignment of text sucks.

How to fix that? We can use ggsave to a pdf and fix it in Illustrator.

\begin{Shaded}
\begin{Highlighting}[]
\KeywordTok{waffle}\NormalTok{(nu}\OperatorTok{/}\DecValTok{10}\NormalTok{, }\DataTypeTok{rows =} \DecValTok{5}\NormalTok{, }\DataTypeTok{xlab=}\StringTok{"1 square = 10 yards"}\NormalTok{, }\DataTypeTok{colors =} \KeywordTok{c}\NormalTok{(}\StringTok{"black"}\NormalTok{, }\StringTok{"red"}\NormalTok{)) }\OperatorTok{+}\StringTok{ }\KeywordTok{labs}\NormalTok{(}\DataTypeTok{title=}\StringTok{"Nebraska vs Wisconsin on offense"}\NormalTok{, }\DataTypeTok{subtitle=}\StringTok{"The Huskers couldn't get much of a running game going."}\NormalTok{, }\DataTypeTok{caption=}\StringTok{"Source: NCAA | Graphic by Matt Waite"}\NormalTok{) }\OperatorTok{+}\StringTok{ }
\StringTok{  }\KeywordTok{theme}\NormalTok{(}
    \DataTypeTok{plot.title =} \KeywordTok{element_text}\NormalTok{(}\DataTypeTok{size =} \DecValTok{16}\NormalTok{, }\DataTypeTok{face =} \StringTok{"bold"}\NormalTok{),}
    \DataTypeTok{axis.title =} \KeywordTok{element_text}\NormalTok{(}\DataTypeTok{size =} \DecValTok{10}\NormalTok{),}
    \DataTypeTok{axis.title.y =} \KeywordTok{element_blank}\NormalTok{()}
\NormalTok{  ) }\OperatorTok{+}\StringTok{ }\KeywordTok{ggsave}\NormalTok{(}\StringTok{"waffle.pdf"}\NormalTok{)}
\end{Highlighting}
\end{Shaded}

But what if we're using a waffle iron? And what if we want to change the output size? It gets tougher.

Truth is, I'm not sure what is going on with the sizing. You can try it and you'll find that the outputs are \ldots{} unpredictable.

Things you need to know about waffle irons:

\begin{itemize}
\tightlist
\item
  They're a convenience method, but all they're really doing is executing two waffle charts together. If you don't apply the theme to both waffle charts, it breaks.
\item
  You will have to get creative about applying headline and subtitle to the top waffle chart and the caption to the bottom.
\item
  Using ggsave doesn't work either. So you'll have to use R's pdf output.
\end{itemize}

Here is a full example. I start with my waffle iron code, but note that each waffle is pretty much a self contained thing. That's because a waffle iron isn't really a thing. It's just a way to group waffles together, so you have to make each waffle individually. My first waffle has the title and subtitle but no x axis labels and the bottom one has not title or subtitle but the axis labels and the caption.

\begin{Shaded}
\begin{Highlighting}[]
\KeywordTok{iron}\NormalTok{(}
 \KeywordTok{waffle}\NormalTok{(}
\NormalTok{   nu}\OperatorTok{/}\DecValTok{10}\NormalTok{, }
   \DataTypeTok{rows =} \DecValTok{2}\NormalTok{, }
   \DataTypeTok{colors =} \KeywordTok{c}\NormalTok{(}\StringTok{"black"}\NormalTok{, }\StringTok{"red"}\NormalTok{, }\StringTok{"white"}\NormalTok{)) }\OperatorTok{+}\StringTok{ }
\StringTok{   }\KeywordTok{labs}\NormalTok{(}\DataTypeTok{title=}\StringTok{"Nebraska vs Wisconsin: By the numbers"}\NormalTok{, }\DataTypeTok{subtitle=}\StringTok{"The Huskers couldn't run, Wisconsin could."}\NormalTok{) }\OperatorTok{+}\StringTok{ }
\StringTok{   }\KeywordTok{theme}\NormalTok{(}
    \DataTypeTok{plot.title =} \KeywordTok{element_text}\NormalTok{(}\DataTypeTok{size =} \DecValTok{16}\NormalTok{, }\DataTypeTok{face =} \StringTok{"bold"}\NormalTok{),}
    \DataTypeTok{axis.title =} \KeywordTok{element_text}\NormalTok{(}\DataTypeTok{size =} \DecValTok{10}\NormalTok{),}
    \DataTypeTok{axis.title.y =} \KeywordTok{element_blank}\NormalTok{()}
\NormalTok{  ),}
 \KeywordTok{waffle}\NormalTok{(}
\NormalTok{   wi}\OperatorTok{/}\DecValTok{10}\NormalTok{, }
   \DataTypeTok{rows =} \DecValTok{2}\NormalTok{, }
   \DataTypeTok{xlab=}\StringTok{"1 square = 10 yards"}\NormalTok{, }
   \DataTypeTok{colors =} \KeywordTok{c}\NormalTok{(}\StringTok{"black"}\NormalTok{, }\StringTok{"red"}\NormalTok{, }\StringTok{"white"}\NormalTok{)) }\OperatorTok{+}\StringTok{ }\KeywordTok{labs}\NormalTok{(}\DataTypeTok{caption=}\StringTok{"Source: NCAA | Graphic by Matt Waite"}\NormalTok{)}
\NormalTok{) }
\end{Highlighting}
\end{Shaded}

\includegraphics{SportsData_files/figure-latex/unnamed-chunk-271-1.pdf}

If you try to use ggsave on that, you'll only get the last waffle chart. Like I said, irons aren't really anything, so ggplot ignores them. So to do this, we have to use R's pdf capability.

Here's the same code, but wrapped in the R pdf functions. The first line says we're going to output this as a pdf with this name. Then my code, then \texttt{dev.off} to tell R that's what I want as a PDF. Don't forget that.

\begin{Shaded}
\begin{Highlighting}[]
\KeywordTok{pdf}\NormalTok{(}\StringTok{"waffleiron.pdf"}\NormalTok{)}
\KeywordTok{iron}\NormalTok{(}
 \KeywordTok{waffle}\NormalTok{(}
\NormalTok{   nu}\OperatorTok{/}\DecValTok{10}\NormalTok{, }
   \DataTypeTok{rows =} \DecValTok{2}\NormalTok{, }
   \DataTypeTok{colors =} \KeywordTok{c}\NormalTok{(}\StringTok{"black"}\NormalTok{, }\StringTok{"red"}\NormalTok{, }\StringTok{"white"}\NormalTok{)) }\OperatorTok{+}\StringTok{ }
\StringTok{   }\KeywordTok{labs}\NormalTok{(}\DataTypeTok{title=}\StringTok{"Nebraska vs Wisconsin: By the numbers"}\NormalTok{, }\DataTypeTok{subtitle=}\StringTok{"The Huskers couldn't run, Wisconsin could."}\NormalTok{) }\OperatorTok{+}\StringTok{ }
\StringTok{   }\KeywordTok{theme}\NormalTok{(}
    \DataTypeTok{plot.title =} \KeywordTok{element_text}\NormalTok{(}\DataTypeTok{size =} \DecValTok{16}\NormalTok{, }\DataTypeTok{face =} \StringTok{"bold"}\NormalTok{),}
    \DataTypeTok{axis.title =} \KeywordTok{element_text}\NormalTok{(}\DataTypeTok{size =} \DecValTok{10}\NormalTok{),}
    \DataTypeTok{axis.title.y =} \KeywordTok{element_blank}\NormalTok{()}
\NormalTok{  ),}
 \KeywordTok{waffle}\NormalTok{(}
\NormalTok{   wi}\OperatorTok{/}\DecValTok{10}\NormalTok{, }
   \DataTypeTok{rows =} \DecValTok{2}\NormalTok{, }
   \DataTypeTok{xlab=}\StringTok{"1 square = 10 yards"}\NormalTok{, }
   \DataTypeTok{colors =} \KeywordTok{c}\NormalTok{(}\StringTok{"black"}\NormalTok{, }\StringTok{"red"}\NormalTok{, }\StringTok{"white"}\NormalTok{)) }\OperatorTok{+}\StringTok{ }\KeywordTok{labs}\NormalTok{(}\DataTypeTok{caption=}\StringTok{"Source: NCAA | Graphic by Matt Waite"}\NormalTok{)}
\NormalTok{) }
\KeywordTok{dev.off}\NormalTok{()}
\end{Highlighting}
\end{Shaded}

It probably still needs work in Illustrator, but less than before.

\hypertarget{plotly}{%
\chapter{Plotly}\label{plotly}}

\textbf{By John Strasheim}

We've been working on making charts and graphs and outputting them as static images -- png files. Why? Because images will embed into any website, work just fine on mobile, and require no special coding. But the wonder of the internet is that it's interactive. The trouble with interactive graphics, though, is the code can be exceedingly complicated and difficult for beginners to grasp, as is the case with the javascript visualization library D3. Or the tools are ultra simple and allow for minimal customization, such as tools like Tableau. The third problem is that the accessible interactive tools -- the ones that don't require a ton of code knowledge -- cost money to publish.

The library we're going to look at in this chapter is from a company called Plotly, which sits in between all these problems. You can create interactive graphs with point and click, but you can also do it in code. You can publish graphs for free, but if you're going to do it for a company or with a large audience, you need to pay. For our purposes, you'll see the power without needing to pony up.

You will need to install plotly. Go to the console -- not in the notebook -- and run \texttt{install.packages("plotly")}.

Then we'll load it:

\begin{Shaded}
\begin{Highlighting}[]
\KeywordTok{library}\NormalTok{(tidyverse)}
\KeywordTok{library}\NormalTok{(plotly)}
\end{Highlighting}
\end{Shaded}

To start out, we'll make a graph similar to something we've already done. We're going to use a \href{https://unl.box.com/s/xg0eqvmz9ynnegvjabv21ev5qlsd7mjs}{dataset of batting stats from the 2019 season}. This dataset has players who had more than 190 plate appearances and includes their basic stats and some advanced metrics.

\begin{Shaded}
\begin{Highlighting}[]
\NormalTok{batting <-}\StringTok{ }\KeywordTok{read_csv}\NormalTok{(}\StringTok{"data/batting.csv"}\NormalTok{)}
\end{Highlighting}
\end{Shaded}

\begin{verbatim}
## Parsed with column specification:
## cols(
##   .default = col_double(),
##   Name = col_character(),
##   Tm = col_character(),
##   Division = col_character()
## )
\end{verbatim}

\begin{verbatim}
## See spec(...) for full column specifications.
\end{verbatim}

Let's look at the relationship between a player's batting average and wRC+ (\href{http://m.mlb.com/glossary/advanced-stats/weighted-runs-created-plus}{weighted runs created plus}).

\begin{Shaded}
\begin{Highlighting}[]
\KeywordTok{ggplot}\NormalTok{() }\OperatorTok{+}
\StringTok{  }\KeywordTok{geom_point}\NormalTok{(}\DataTypeTok{data=}\NormalTok{batting, }\KeywordTok{aes}\NormalTok{(}\DataTypeTok{x=}\StringTok{`}\DataTypeTok{wRC+}\StringTok{`}\NormalTok{, }\DataTypeTok{y=}\NormalTok{AVG)) }\OperatorTok{+}
\StringTok{  }\KeywordTok{geom_smooth}\NormalTok{(}\DataTypeTok{data=}\NormalTok{batting, }\KeywordTok{aes}\NormalTok{(}\DataTypeTok{x=}\StringTok{`}\DataTypeTok{wRC+}\StringTok{`}\NormalTok{, }\DataTypeTok{y=}\NormalTok{AVG), }\DataTypeTok{method=}\StringTok{'lm'}\NormalTok{, }\DataTypeTok{se=}\OtherTok{FALSE}\NormalTok{) }\OperatorTok{+}
\StringTok{  }\KeywordTok{labs}\NormalTok{(}\DataTypeTok{x=}\StringTok{'wRC+'}\NormalTok{, }\DataTypeTok{y=}\StringTok{'Batting Average'}\NormalTok{, }\DataTypeTok{title=} \StringTok{"2019 MLB Season Batting"}\NormalTok{, }\DataTypeTok{subtitle=}\StringTok{"190 PAs min to Qualify"}\NormalTok{, }\DataTypeTok{caption=}\StringTok{"Source:  FanGraphs | by John Strasheim"}\NormalTok{)}
\end{Highlighting}
\end{Shaded}

\includegraphics{SportsData_files/figure-latex/unnamed-chunk-275-1.pdf}

You can obviously add more aesthetics to make it look better, but you get the picture. If I'm a fan of sports though, obviously I want to see who the outlier points are or just scroll through and see each player at an individual location. We can always annotate data, but that process can be tedious.

Here is where plotly comes in.

Plotly will make your visualizatons interactive. Additionally, you can zoom in on certain parts of the viz too. For example, you can drag a box around players with a .300 average, and see all the guys in that specific range.

Let's start simple with just the minimum needed for a scatterplot. For that, we need to specify our data source -- \texttt{batting} -- and set an X and a Y value, just like above. One difference? We prepend a \textasciitilde{} before the field names. We'll add a color to separate out players by division too.

\begin{Shaded}
\begin{Highlighting}[]
\KeywordTok{plot_ly}\NormalTok{(}\DataTypeTok{data=}\NormalTok{batting, }\DataTypeTok{x=} \OperatorTok{~}\StringTok{`}\DataTypeTok{wRC+}\StringTok{`}\NormalTok{, }\DataTypeTok{y=} \OperatorTok{~}\NormalTok{AVG, }\DataTypeTok{color=} \OperatorTok{~}\NormalTok{Division)}
\end{Highlighting}
\end{Shaded}

\begin{verbatim}
## No trace type specified:
##   Based on info supplied, a 'scatter' trace seems appropriate.
##   Read more about this trace type -> https://plot.ly/r/reference/#scatter
\end{verbatim}

\begin{verbatim}
## No scatter mode specifed:
##   Setting the mode to markers
##   Read more about this attribute -> https://plot.ly/r/reference/#scatter-mode
\end{verbatim}

\includegraphics{SportsData_files/figure-latex/unnamed-chunk-276-1.pdf}

We get a chart, but hover over a point. Recognize those players? You can't unless you know each player's specific stats to divine who they are. That isn't very friendly, so let's add a hover element. Then we want to specify what we want our users to see when they hover over a data point, hence hoverinfo = ``text''. The next step will be to define what our text is. How that gets done is a little bit of HTML and a little bit of R. What is in quotes is what the users are going to see directly, what's after the quotes is what data is going to appear. So ``Player:'', Name translates to something like Player: Christian Yelich when the user hovers above Yelich's data point.

Then we add the HTML, \texttt{\textless{}br\textgreater{}}, or break. All this means is we're having a line break so all of our data is not on one line. Simple. Do this process for whatever variables you want to have your users see. So for mine, I wanted my users to see the Player's name, wRC+, Batting Avg, and what division they are in.

\begin{Shaded}
\begin{Highlighting}[]
\KeywordTok{plot_ly}\NormalTok{(}\DataTypeTok{data=}\NormalTok{batting, }\DataTypeTok{x=} \OperatorTok{~}\StringTok{`}\DataTypeTok{wRC+}\StringTok{`}\NormalTok{, }\DataTypeTok{y=} \OperatorTok{~}\StringTok{`}\DataTypeTok{AVG}\StringTok{`}\NormalTok{, }\DataTypeTok{color=} \OperatorTok{~}\StringTok{`}\DataTypeTok{Division}\StringTok{`}\NormalTok{,}
        \DataTypeTok{hoverinfo =} \StringTok{"text"}\NormalTok{,}
        \DataTypeTok{text =} \OperatorTok{~}\KeywordTok{paste}\NormalTok{(}\StringTok{"Player:"}\NormalTok{, Name,}
                      \StringTok{'<br>wRC+:'}\NormalTok{, }\StringTok{`}\DataTypeTok{wRC+}\StringTok{`}\NormalTok{,}
                      \StringTok{'<br>AVG:'}\NormalTok{, AVG,}
                      \StringTok{'<br>Team:'}\NormalTok{, Tm,}
                      \StringTok{'<br>Division:'}\NormalTok{, Division}
\NormalTok{                      ))}
\end{Highlighting}
\end{Shaded}

\begin{verbatim}
## No trace type specified:
##   Based on info supplied, a 'scatter' trace seems appropriate.
##   Read more about this trace type -> https://plot.ly/r/reference/#scatter
\end{verbatim}

\begin{verbatim}
## No scatter mode specifed:
##   Setting the mode to markers
##   Read more about this attribute -> https://plot.ly/r/reference/#scatter-mode
\end{verbatim}

\includegraphics{SportsData_files/figure-latex/unnamed-chunk-277-1.pdf}

Now we can see each player a little better. If you look at the players on the farthest right, you'll find Mike Trout (shocker), Yordan Alvarez and Christian Yelich.

To finish, we're going to fix the layout a bit, very similar to how we've been doing it in ggplot. We're just telling plotly what we want the layout to be of our viz, starting with the title, and then doing the x and y axis names after that.

\begin{Shaded}
\begin{Highlighting}[]
\KeywordTok{plot_ly}\NormalTok{(}\DataTypeTok{data=}\NormalTok{batting, }\DataTypeTok{x=} \OperatorTok{~}\StringTok{`}\DataTypeTok{wRC+}\StringTok{`}\NormalTok{, }\DataTypeTok{y=} \OperatorTok{~}\StringTok{`}\DataTypeTok{AVG}\StringTok{`}\NormalTok{, }\DataTypeTok{color=} \OperatorTok{~}\StringTok{`}\DataTypeTok{Division}\StringTok{`}\NormalTok{,}
        \DataTypeTok{hoverinfo =} \StringTok{"text"}\NormalTok{,}
        \DataTypeTok{text =} \OperatorTok{~}\KeywordTok{paste}\NormalTok{(}\StringTok{"Player:"}\NormalTok{, Name,}
                      \StringTok{'<br>wRC+:'}\NormalTok{, }\StringTok{`}\DataTypeTok{wRC+}\StringTok{`}\NormalTok{,}
                      \StringTok{'<br>AVG:'}\NormalTok{, AVG,}
                      \StringTok{'<br>Team:'}\NormalTok{, Tm,}
                      \StringTok{'<br>Division:'}\NormalTok{, Division}
\NormalTok{                      )) }\OperatorTok\StringTok{ }
\StringTok{  }\KeywordTok{layout}\NormalTok{(}
    \DataTypeTok{title =} \StringTok{"2019 MLB Season Batting"}\NormalTok{,}
    \DataTypeTok{xaxis =} \KeywordTok{list}\NormalTok{(}\DataTypeTok{title =} \StringTok{"wRC+"}\NormalTok{),}
    \DataTypeTok{yaxis =} \KeywordTok{list}\NormalTok{(}\DataTypeTok{title =} \StringTok{"Batting Average"}\NormalTok{)}
\NormalTok{  )}
\end{Highlighting}
\end{Shaded}

\begin{verbatim}
## No trace type specified:
##   Based on info supplied, a 'scatter' trace seems appropriate.
##   Read more about this trace type -> https://plot.ly/r/reference/#scatter
\end{verbatim}

\begin{verbatim}
## No scatter mode specifed:
##   Setting the mode to markers
##   Read more about this attribute -> https://plot.ly/r/reference/#scatter-mode
\end{verbatim}

\includegraphics{SportsData_files/figure-latex/unnamed-chunk-278-1.pdf}

\hypertarget{publishing-using-plotly}{%
\section{Publishing using Plotly}\label{publishing-using-plotly}}

We want to now export our plotly visualization. First, you'll need to sign up for a \href{https://plot.ly/}{free plotly account}. Then you'll need to register your plotly username and your API key.

More info about how to do that can be \href{https://plot.ly/r/getting-started/\#initialization-for-online-plotting}{found on Plotly's website}.

For our purposes, we need to register our username and API key this way, where you put your username and API key where prompted:

\begin{Shaded}
\begin{Highlighting}[]
\KeywordTok{Sys.setenv}\NormalTok{(}\StringTok{"plotly_username"}\NormalTok{=}\StringTok{"Enter your plotly username here"}\NormalTok{)}
\KeywordTok{Sys.setenv}\NormalTok{(}\StringTok{"plotly_api_key"}\NormalTok{=}\StringTok{"Enter your API key here"}\NormalTok{)}
\end{Highlighting}
\end{Shaded}

Now run this line of code specifiying what variable you are exporting, and what you want the file to be named on plotly's servers. From plotly's website you can then do several different things like editing it on there, embedding it on websites, or create a shareable link.

To publish our chart, we need to save it to an object similar to how we've been creating dataframes. So something like this:

\begin{Shaded}
\begin{Highlighting}[]
\NormalTok{p <-}\StringTok{ }\KeywordTok{plot_ly}\NormalTok{(}\DataTypeTok{data=}\NormalTok{batting, }\DataTypeTok{x=} \OperatorTok{~}\StringTok{`}\DataTypeTok{wRC+}\StringTok{`}\NormalTok{, }\DataTypeTok{y=} \OperatorTok{~}\StringTok{`}\DataTypeTok{AVG}\StringTok{`}\NormalTok{, }\DataTypeTok{color=} \OperatorTok{~}\StringTok{`}\DataTypeTok{Division}\StringTok{`}\NormalTok{,}
        \DataTypeTok{hoverinfo =} \StringTok{"text"}\NormalTok{,}
        \DataTypeTok{text =} \OperatorTok{~}\KeywordTok{paste}\NormalTok{(}\StringTok{"Player:"}\NormalTok{, Name,}
                      \StringTok{'<br>wRC+:'}\NormalTok{, }\StringTok{`}\DataTypeTok{wRC+}\StringTok{`}\NormalTok{,}
                      \StringTok{'<br>AVG:'}\NormalTok{, AVG,}
                      \StringTok{'<br>Team:'}\NormalTok{, Tm,}
                      \StringTok{'<br>Division:'}\NormalTok{, Division}
\NormalTok{                      )) }\OperatorTok\StringTok{ }
\StringTok{  }\KeywordTok{layout}\NormalTok{(}
    \DataTypeTok{title =} \StringTok{"2019 MLB Season Batting"}\NormalTok{,}
    \DataTypeTok{xaxis =} \KeywordTok{list}\NormalTok{(}\DataTypeTok{title =} \StringTok{"wRC+"}\NormalTok{),}
    \DataTypeTok{yaxis =} \KeywordTok{list}\NormalTok{(}\DataTypeTok{title =} \StringTok{"Batting Average"}\NormalTok{)}
\NormalTok{  )}
\end{Highlighting}
\end{Shaded}

To publish it, we simply run the following, passing in our chart value \texttt{p} for plotly and we give it a filename.

\begin{Shaded}
\begin{Highlighting}[]
\KeywordTok{api_create}\NormalTok{(p, }\DataTypeTok{filename=}\StringTok{"MLBOffense19"}\NormalTok{)}
\end{Highlighting}
\end{Shaded}

If all goes well, a browser will pop up with \href{https://plot.ly/~mattwaite/1/\#/}{your chart in it}.

\hypertarget{clustering}{%
\chapter{Clustering}\label{clustering}}

One common effort in sports is to classify teams and players -- who are this players peers? What teams are like this one? Who should we compare a player to? Truth is, most sports commentators use nothing more sophisticated that looking at a couple of stats or use the ``eye test'' to say a player is like this or that.

There's better ways.

In this chapter, we're going to use a method that sounds advanced but it really quite simple called k-means clustering. It's based on the concept of the k-nearest neighbor algorithm. You're probably already scared. Don't be.

Imagine two dots on a scatterplot. If you took a ruler out and measured the distance between those dots, you'd know how far apart they are. In math, that's called the Euclidean distance. It's just the space between them in numbers. Where k-nearest neighbor comes in, you have lots of dots and you want measure the distance between all of them. What does k-means clustering do? It lumps them into groups based on the average distance between them. Players who are good on offense but bad on defense are over here, good offense good defense are over here. And using the Eucliean distance between them, we can decide who is in and who is out of those groups.

For this exercise, I want to look at Cam Mack, Nebraska's point guard and probably the most interesting player on Fred Hoiberg's first team. This is Mack's first year in major college basketball -- he played a year at a community college -- so we don't have much to go on. But with three games in the books, who does Cam Mack compare to?

To answer this, we'll use k-means clustering.

First thing we do is load some libraries and set a seed, so if we run this repeatedly, our random numbers are generated from the same base. If you don't have the cluster library, just add it on the console with \texttt{install.packages("cluster")}

\begin{Shaded}
\begin{Highlighting}[]
\KeywordTok{library}\NormalTok{(tidyverse)}
\KeywordTok{library}\NormalTok{(cluster)}

\KeywordTok{set.seed}\NormalTok{(}\DecValTok{1234}\NormalTok{)}
\end{Highlighting}
\end{Shaded}

I've gone and scraped \href{https://unl.box.com/s/0g56ve61y6hakyxzr1u4t534721bqvg8}{stats for every player in this current season} so let's load that up.

\begin{Shaded}
\begin{Highlighting}[]
\NormalTok{players <-}\StringTok{ }\KeywordTok{read_csv}\NormalTok{(}\StringTok{"data/players20.csv"}\NormalTok{)}
\end{Highlighting}
\end{Shaded}

\begin{verbatim}
## Parsed with column specification:
## cols(
##   .default = col_double(),
##   Team = col_character(),
##   Player = col_character(),
##   Class = col_character(),
##   Pos = col_character(),
##   Height = col_character(),
##   Hometown = col_character(),
##   `High School` = col_character(),
##   Summary = col_character()
## )
\end{verbatim}

\begin{verbatim}
## See spec(...) for full column specifications.
\end{verbatim}

To cluster this data properly, we have some work to do.

First, it won't do to have players who haven't played, so we can use filter to find anyone with greater than 0 minutes played. Next, Cam Mack is a guard, so let's just look at guards. Third, we want to limit the data to things that make sense to look at for Cam Mack -- things like shooting, three point shooting, assists, turnovers and points.

\begin{Shaded}
\begin{Highlighting}[]
\NormalTok{playersselected <-}\StringTok{ }\NormalTok{players }\OperatorTok\StringTok{ }
\StringTok{  }\KeywordTok{filter}\NormalTok{(MP}\OperatorTok{>}\DecValTok{0}\NormalTok{) }\OperatorTok\StringTok{ }\KeywordTok{filter}\NormalTok{(Pos }\OperatorTok{==}\StringTok{ "G"}\NormalTok{) }\OperatorTok\StringTok{ }
\StringTok{  }\KeywordTok{select}\NormalTok{(Player, Team, Pos, MP, }\StringTok{`}\DataTypeTok{FG%}\StringTok{`}\NormalTok{, }\StringTok{`}\DataTypeTok{3P%}\StringTok{`}\NormalTok{, AST, TOV, PTS) }\OperatorTok\StringTok{ }
\StringTok{  }\KeywordTok{na.omit}\NormalTok{() }
\end{Highlighting}
\end{Shaded}

Now, k-means clustering doesn't work as well with data that can be on different scales. So comparing a percentage to a count metric -- shooting percentage to points -- would create chaos because shooting percentages are a fraction of 1 and points, depending on when they are in the season, could be quite large. So we have to scale each metric -- put them on a similar basis using the distance from the max value as our guide. Also, k-means clustering won't work with text data, so we need to create a dataframe that's just the numbers, but scaled. We can do that with another select, and using mutate\_all with the scale function. The \texttt{na.omit()} means get rid of any blanks, because they too will cause errors.

\begin{Shaded}
\begin{Highlighting}[]
\NormalTok{playersscaled <-}\StringTok{ }\NormalTok{playersselected }\OperatorTok\StringTok{ }
\StringTok{  }\KeywordTok{select}\NormalTok{(MP, }\StringTok{`}\DataTypeTok{FG%}\StringTok{`}\NormalTok{, }\StringTok{`}\DataTypeTok{3P%}\StringTok{`}\NormalTok{, AST, TOV, PTS) }\OperatorTok\StringTok{ }
\StringTok{  }\KeywordTok{mutate_all}\NormalTok{(scale) }\OperatorTok\StringTok{ }
\StringTok{  }\KeywordTok{na.omit}\NormalTok{()}
\end{Highlighting}
\end{Shaded}

With k-means clustering, we decide how many clusters we want. Most often, researchers will try a handful of different cluster numbers and see what works. But there are methods for finding the optimal number. One method is called the Elbow method. One implementation of this, \href{https://uc-r.github.io/kmeans_clustering}{borrowed from the University of Cincinnati's Business Analytics program}, does this quite nicely with a graph that will help you decide for yourself.

All you need to do in this code is change out the data frame -- \texttt{playersscaled} in this case -- and run it.

\begin{Shaded}
\begin{Highlighting}[]
\CommentTok{# function to compute total within-cluster sum of square }
\NormalTok{wss <-}\StringTok{ }\ControlFlowTok{function}\NormalTok{(k) \{}
  \KeywordTok{kmeans}\NormalTok{(playersscaled, k, }\DataTypeTok{nstart =} \DecValTok{10}\NormalTok{ )}\OperatorTok{$}\NormalTok{tot.withinss}
\NormalTok{\}}

\CommentTok{# Compute and plot wss for k = 1 to k = 15}
\NormalTok{k.values <-}\StringTok{ }\DecValTok{1}\OperatorTok{:}\DecValTok{15}

\CommentTok{# extract wss for 2-15 clusters}
\NormalTok{wss_values <-}\StringTok{ }\KeywordTok{map_dbl}\NormalTok{(k.values, wss)}
\end{Highlighting}
\end{Shaded}

\begin{verbatim}
## Warning: did not converge in 10 iterations

## Warning: did not converge in 10 iterations

## Warning: did not converge in 10 iterations

## Warning: did not converge in 10 iterations
\end{verbatim}

\begin{Shaded}
\begin{Highlighting}[]
\KeywordTok{plot}\NormalTok{(k.values, wss_values,}
       \DataTypeTok{type=}\StringTok{"b"}\NormalTok{, }\DataTypeTok{pch =} \DecValTok{19}\NormalTok{, }\DataTypeTok{frame =} \OtherTok{FALSE}\NormalTok{, }
       \DataTypeTok{xlab=}\StringTok{"Number of clusters K"}\NormalTok{,}
       \DataTypeTok{ylab=}\StringTok{"Total within-clusters sum of squares"}\NormalTok{)}
\end{Highlighting}
\end{Shaded}

\includegraphics{SportsData_files/figure-latex/unnamed-chunk-286-1.pdf}

The Elbow method -- so named because you're looking for the ``elbow'' where the line flattens out. In this case, it looks like a K of 5 is ideal. So let's try that. We're going to use the kmeans function, saving it to an object called k5. We just need to tell it our dataframe name, how many centers (k) we want, and we'll use a sensible default for how many different configurations to try.

\begin{Shaded}
\begin{Highlighting}[]
\NormalTok{k5 <-}\StringTok{ }\KeywordTok{kmeans}\NormalTok{(playersscaled, }\DataTypeTok{centers =} \DecValTok{5}\NormalTok{, }\DataTypeTok{nstart =} \DecValTok{25}\NormalTok{)}
\end{Highlighting}
\end{Shaded}

Let's look at what we get.

\begin{Shaded}
\begin{Highlighting}[]
\NormalTok{k5}
\end{Highlighting}
\end{Shaded}

\begin{verbatim}
## K-means clustering with 5 clusters of sizes 863, 989, 242, 58, 495
## 
## Cluster means:
##           MP         FG%         3P%        AST        TOV        PTS
## 1 -0.8549890  0.02411493 -0.04833668 -0.7090093 -0.7700257 -0.7982928
## 2  0.5451701  0.14083415  0.12854961  0.1392254  0.2539075  0.3248641
## 3 -1.3602270 -1.98012908 -1.65325745 -0.9392522 -1.1172744 -1.1417360
## 4 -1.3785814  3.22934178  3.92207534 -0.9251265 -1.1485250 -1.1077564
## 5  1.2279089  0.26624900  0.17613521  1.5255303  1.5159849  1.4306790
## 
## Clustering vector:
##    [1] 5 2 2 2 2 1 1 1 1 2 2 5 2 1 1 1 5 2 1 1 3 3 5 2 2 2 2 2 1 4 2 2 5 2 2 1 3
##   [38] 5 5 2 2 1 2 1 1 5 2 2 2 1 1 2 5 2 2 2 2 1 1 3 2 5 1 2 2 1 1 1 1 1 2 2 5 2
##   [75] 2 1 3 1 3 5 2 2 2 1 1 1 1 5 2 2 2 2 1 3 5 2 2 1 1 1 5 2 1 2 1 1 1 1 4 2 2
##  [112] 1 1 1 2 1 3 5 2 2 2 1 1 1 5 2 2 1 3 3 5 2 2 1 1 1 5 2 2 2 2 1 1 5 5 2 2 2
##  [149] 2 3 5 2 2 2 1 1 1 3 5 2 2 2 2 5 1 3 5 5 2 2 2 2 1 1 2 2 2 2 1 1 1 1 1 1 5
##  [186] 2 3 3 5 2 2 2 1 1 3 1 5 5 2 2 1 1 1 4 3 5 2 2 5 2 2 1 1 1 3 5 2 1 1 1 3 3
##  [223] 5 5 2 2 2 2 2 3 5 2 2 5 1 5 2 5 2 2 1 1 1 2 2 2 2 2 1 3 5 5 5 2 2 2 1 1 3
##  [260] 3 5 5 2 2 2 1 1 1 3 5 5 2 2 2 1 1 1 5 2 2 2 3 5 5 2 2 1 3 2 5 5 2 5 2 1 1
##  [297] 5 2 2 2 1 1 1 1 1 5 2 2 1 3 1 3 3 5 5 2 2 1 1 5 5 2 2 2 1 1 2 2 2 2 1 1 1
##  [334] 5 5 2 2 2 2 1 5 5 2 1 1 1 3 5 2 2 2 2 1 3 5 5 2 1 2 1 1 3 3 5 2 2 2 2 1 1
##  [371] 1 5 5 2 2 1 1 1 1 1 1 1 5 5 2 5 2 1 1 1 3 5 5 2 1 1 5 5 2 2 1 3 5 5 1 1 1
##  [408] 5 5 2 1 1 3 5 2 2 2 2 1 1 1 4 5 2 2 2 2 1 1 1 5 5 2 1 1 2 5 2 2 2 2 1 3 5
##  [445] 5 2 2 1 1 1 1 5 5 2 2 1 3 5 2 2 2 1 1 1 1 2 5 2 2 2 1 1 1 5 2 2 1 1 1 1 3
##  [482] 5 5 2 2 1 1 3 3 5 5 2 1 1 2 1 5 5 2 2 2 1 3 3 2 2 2 5 1 1 5 2 2 2 2 1 3 5
##  [519] 2 2 2 1 1 4 1 5 5 2 2 2 1 3 4 5 5 2 5 1 1 1 3 5 2 2 2 2 1 1 1 1 5 5 2 1 1
##  [556] 1 5 5 2 2 2 1 1 3 2 2 2 2 1 3 3 3 2 2 2 1 1 1 3 5 5 5 1 1 1 1 3 3 5 5 2 2
##  [593] 1 2 5 5 2 2 2 2 1 1 4 5 2 2 1 3 3 3 5 2 2 2 2 1 1 3 3 5 5 2 2 1 5 2 2 2 1
##  [630] 1 3 5 2 5 2 1 1 3 5 2 2 2 1 4 3 5 2 2 2 2 1 2 2 2 2 2 2 1 1 1 5 2 5 1 1 1
##  [667] 5 2 2 5 1 1 1 3 4 5 2 2 1 1 1 5 5 2 2 2 1 1 1 3 5 2 2 2 5 1 1 2 2 2 1 3 3
##  [704] 3 2 5 2 2 2 2 4 1 3 2 5 5 2 5 1 3 5 5 2 2 1 1 3 1 2 5 2 1 1 2 2 5 2 2 1 2
##  [741] 1 1 5 5 2 1 5 5 2 1 1 1 1 4 3 5 2 2 1 1 5 2 2 2 2 1 1 5 5 5 2 2 1 4 2 2 2
##  [778] 1 1 1 1 1 5 2 2 2 2 1 1 3 3 2 2 2 2 2 4 1 4 1 1 2 2 5 2 2 2 1 5 2 2 2 1 1
##  [815] 1 2 5 5 2 2 3 1 3 3 5 2 2 2 1 3 5 2 2 1 1 1 1 1 1 2 2 2 2 2 1 3 3 5 2 2 2
##  [852] 1 1 2 1 5 2 2 2 1 5 2 1 1 1 5 2 2 2 1 1 1 5 5 2 2 1 1 1 4 5 2 2 2 3 2 5 2
##  [889] 1 2 1 1 1 4 5 2 2 3 1 1 3 2 5 2 2 1 2 2 5 2 1 1 3 3 3 5 5 2 2 1 3 3 5 2 1
##  [926] 1 1 2 2 2 2 1 2 1 1 1 1 5 5 2 5 1 1 1 1 1 5 2 2 2 1 1 5 2 2 2 1 1 1 5 2 2
##  [963] 2 2 1 1 5 2 2 1 1 1 3 3 5 5 2 2 2 1 1 1 3 5 5 2 1 1 1 1 1 1 5 5 2 1 2 3 5
## [1000] 2 2 1 1 1 3 5 5 2 2 1 1 1 1 1 5 2 2 2 2 2 1 1 3 2 5 2 2 2 2 2 1 4 3 5 5 2
## [1037] 2 3 1 5 2 2 2 1 3 1 2 5 2 1 2 1 2 3 2 5 2 2 2 1 3 5 2 5 5 2 1 1 3 5 2 5 1
## [1074] 1 3 3 5 2 5 2 1 1 1 1 1 1 1 5 2 2 1 1 1 1 5 5 5 2 1 1 1 1 3 5 2 2 2 1 2 1
## [1111] 1 4 5 5 1 2 1 1 1 1 5 5 2 2 2 1 2 1 1 1 3 2 2 2 1 2 2 1 1 4 1 3 5 2 2 2 1
## [1148] 1 5 5 5 2 1 3 5 5 2 2 1 3 2 5 2 2 2 1 1 1 5 5 2 2 2 1 3 1 5 2 2 1 1 4 3 1
## [1185] 3 5 2 5 1 3 1 3 5 2 2 1 3 1 3 1 3 5 2 2 2 1 1 1 3 1 5 2 5 2 2 1 5 5 2 2 1
## [1222] 5 5 2 1 1 2 5 2 2 2 1 1 5 5 2 2 2 3 1 4 5 2 2 2 1 1 3 2 2 2 2 1 2 1 1 5 5
## [1259] 2 1 1 3 3 1 5 5 5 2 1 1 3 5 2 2 2 2 3 5 2 2 1 3 3 5 5 2 2 2 1 4 1 2 5 5 2
## [1296] 2 1 3 5 2 2 2 1 4 3 5 2 1 1 1 1 3 3 5 2 2 2 2 1 1 2 2 2 2 2 1 5 2 2 2 2 1
## [1333] 1 3 1 5 5 2 2 2 1 1 1 2 2 2 5 2 1 1 2 5 2 1 2 2 1 1 2 5 1 2 1 1 4 3 5 5 2
## [1370] 2 2 1 3 3 5 2 2 2 1 3 4 4 2 2 2 2 1 2 1 3 5 2 2 1 1 1 1 3 5 5 2 1 1 1 5 2
## [1407] 5 1 1 4 5 2 2 2 1 1 3 3 2 2 2 2 2 1 3 4 3 5 1 1 1 1 1 3 5 5 2 2 2 1 1 4 3
## [1444] 3 3 2 1 1 1 1 5 2 2 2 1 1 1 3 2 2 2 2 2 1 5 2 2 2 2 1 1 3 1 5 5 1 1 1 1 1
## [1481] 5 2 2 1 1 1 3 5 5 2 1 1 1 1 1 5 5 2 2 1 1 5 5 5 2 2 1 4 3 2 5 2 2 2 1 1 2
## [1518] 2 2 2 1 1 1 1 1 2 2 2 2 1 1 1 5 5 2 2 1 3 3 5 5 5 2 2 1 5 2 2 1 1 1 1 2 2
## [1555] 2 2 2 5 2 3 2 5 5 1 5 2 2 2 1 1 1 5 2 1 1 1 1 1 3 5 2 5 2 1 1 1 1 4 5 2 2
## [1592] 2 1 5 2 5 1 3 1 1 2 2 2 2 2 2 1 3 3 2 2 2 2 2 2 1 1 3 5 2 5 2 1 1 2 1 1 4
## [1629] 3 3 3 5 5 5 2 2 1 5 2 2 2 2 1 3 2 2 2 2 2 1 1 3 5 5 2 1 2 1 1 4 3 5 2 2 2
## [1666] 1 3 3 2 2 2 1 4 5 5 2 2 2 1 1 1 2 5 2 2 2 1 1 1 5 2 2 2 1 3 1 5 2 2 2 1 2
## [1703] 1 1 1 5 5 2 2 1 5 2 2 2 2 1 1 1 5 2 2 1 1 1 1 4 5 5 2 2 2 1 1 1 2 5 2 5 2
## [1740] 5 2 2 2 2 1 3 5 2 2 1 1 3 4 5 5 2 2 1 1 3 5 2 2 2 2 3 2 5 2 2 1 1 5 2 2 1
## [1777] 1 1 1 3 2 5 5 2 1 1 1 5 2 2 2 4 1 1 1 3 5 5 5 2 2 2 1 4 5 5 2 2 1 5 2 2 2
## [1814] 1 2 1 3 1 4 3 2 5 5 1 1 1 5 5 2 2 1 1 3 3 5 5 2 2 1 1 1 5 5 5 5 2 3 5 2 5
## [1851] 2 2 1 1 1 1 2 2 2 5 5 1 4 5 2 5 1 1 1 1 3 5 2 2 2 1 3 2 5 5 1 2 1 4 3 5 2
## [1888] 5 1 1 1 3 2 2 2 2 2 2 1 3 1 2 2 2 1 2 1 1 5 2 2 1 1 2 5 2 2 1 1 1 4 5 2 2
## [1925] 2 2 1 1 1 5 2 5 2 1 1 3 3 5 2 2 1 1 1 2 5 2 1 2 1 5 2 2 1 1 1 3 2 2 2 2 1
## [1962] 5 2 2 2 2 2 1 4 4 5 5 5 2 2 1 1 3 3 3 5 2 2 5 2 1 1 1 1 1 2 2 1 2 1 1 3 5
## [1999] 5 2 2 1 1 1 5 2 2 5 1 2 1 1 1 5 5 5 2 2 3 5 2 2 2 2 1 1 5 2 2 2 3 3 4 4 5
## [2036] 2 1 2 3 5 5 2 1 3 5 2 2 2 2 2 1 1 4 3 5 2 2 1 1 3 5 2 1 1 1 3 3 1 5 5 2 2
## [2073] 1 1 1 1 5 5 5 1 1 1 3 5 5 2 2 2 1 1 1 3 5 2 2 5 1 1 1 3 5 5 2 2 2 1 1 4 3
## [2110] 2 2 2 2 2 1 1 1 5 5 5 2 1 1 1 1 2 2 1 1 1 3 2 5 2 2 2 2 1 1 5 2 5 2 2 1 3
## [2147] 5 2 2 2 1 1 3 1 5 5 2 2 1 1 1 5 5 2 2 2 1 1 4 5 2 2 1 1 4 3 4 5 5 2 2 1 1
## [2184] 5 2 2 2 1 1 1 3 2 5 2 2 2 2 1 1 4 1 2 2 5 2 2 1 1 1 1 1 2 2 2 5 2 1 1 3 5
## [2221] 2 2 2 1 1 1 1 5 2 2 1 1 5 5 2 2 1 1 1 1 1 5 5 2 2 2 1 1 3 1 5 2 2 2 1 1 1
## [2258] 3 2 2 5 2 2 1 1 5 2 5 2 2 1 1 1 2 2 2 2 1 2 1 3 1 5 5 5 2 1 1 1 2 5 2 2 1
## [2295] 1 2 2 2 2 2 1 1 1 4 5 2 2 2 1 1 3 5 2 2 2 1 1 2 2 2 2 1 2 1 1 5 2 2 1 1 1
## [2332] 2 5 2 2 2 2 1 1 1 1 5 2 5 2 1 1 1 4 5 5 5 1 3 5 5 2 1 1 4 3 2 5 2 2 2 2 2
## [2369] 2 2 2 3 3 5 5 1 1 5 5 2 2 1 3 5 2 2 1 1 1 5 5 2 2 2 3 1 5 2 1 1 1 1 3 3 5
## [2406] 5 2 2 2 1 1 1 4 5 2 2 2 3 1 1 2 2 5 2 2 2 1 1 1 2 5 2 2 2 1 1 1 3 5 5 2 1
## [2443] 1 5 5 2 2 2 1 1 4 5 2 2 2 1 1 1 5 2 2 2 2 2 1 4 5 5 2 2 1 1 1 1 5 2 5 2 2
## [2480] 1 3 3 5 5 2 2 1 1 3 5 5 2 5 2 2 1 1 1 5 5 1 1 1 3 1 5 5 2 2 1 1 3 1 3 5 2
## [2517] 2 2 2 5 5 2 2 2 2 3 3 1 1 2 1 1 1 1 1 3 5 2 2 2 1 1 1 3 5 2 2 2 1 1 5 2 2
## [2554] 1 1 1 1 5 2 2 1 1 1 3 3 3 5 2 2 2 1 1 3 3 5 5 2 2 1 1 1 3 5 5 2 2 2 1 3 1
## [2591] 5 5 2 2 1 1 5 2 2 2 2 1 1 1 1 1 1 5 5 5 1 1 1 1 3 2 2 5 2 3 1 3 3 5 5 2 1
## [2628] 3 3 5 2 2 2 1 1 1 1 1 1 5 2 2 2 1 1 1 1
## 
## Within cluster sum of squares by cluster:
## [1] 1430.7781 1341.0495  319.4278  250.5515 1094.8283
##  (between_SS / total_SS =  72.1 %)
## 
## Available components:
## 
## [1] "cluster"      "centers"      "totss"        "withinss"     "tot.withinss"
## [6] "betweenss"    "size"         "iter"         "ifault"
\end{verbatim}

Interpreting this output, the very first thing you need to know is that \textbf{the cluster numbers are meaningless}. They aren't ranks. They aren't anything. After you have taken that on board, look at the cluster sizes at the top. Clusters 3 and 5 are pretty large compared to others. That's notable. Then we can look at the cluster means. For reference, 0 is going to be average. So group 1 are well above average on minutes played. Group 3 is slightly above, group 5 is slightly below. In fact, group 5 is below average on every metric. Group 3 is slightly above average on all metrics.

So which group is Cam Mack in? Well, first we have to put our data back together again. In K5, there is a list of cluster assignments in the same order we put them in, but recall we have no names. So we need to re-combine them with our original data. We can do that with the following:

\begin{Shaded}
\begin{Highlighting}[]
\NormalTok{playercluster <-}\StringTok{ }\KeywordTok{data.frame}\NormalTok{(playersselected, k5}\OperatorTok{$}\NormalTok{cluster) }
\end{Highlighting}
\end{Shaded}

Now we have a dataframe called playercluster that has our player names and what cluster they are in. The fastest way to find Cam Mack is to double click on the playercluster table in the environment and use the search in the top right of the table. Because this is based on some random selections of points to start the groupings, these may change from person to person, but Mack is in Group 1 in my data.

We now have a dataset and can plot it like anything else. Let's get Cam Mack and then plot him against the rest of college basketball on assists versus minutes played.

\begin{Shaded}
\begin{Highlighting}[]
\NormalTok{cm <-}\StringTok{ }\NormalTok{playercluster }\OperatorTok\StringTok{ }\KeywordTok{filter}\NormalTok{(Player }\OperatorTok{==}\StringTok{ "Cameron Mack"}\NormalTok{)}
\end{Highlighting}
\end{Shaded}

\begin{Shaded}
\begin{Highlighting}[]
\KeywordTok{ggplot}\NormalTok{() }\OperatorTok{+}\StringTok{ }
\StringTok{  }\KeywordTok{geom_point}\NormalTok{(}\DataTypeTok{data=}\NormalTok{playercluster, }\KeywordTok{aes}\NormalTok{(}\DataTypeTok{x=}\NormalTok{MP, }\DataTypeTok{y=}\NormalTok{AST, }\DataTypeTok{color=}\NormalTok{k5.cluster)) }\OperatorTok{+}\StringTok{ }
\StringTok{  }\KeywordTok{geom_point}\NormalTok{(}\DataTypeTok{data=}\NormalTok{cm, }\KeywordTok{aes}\NormalTok{(}\DataTypeTok{x=}\NormalTok{MP, }\DataTypeTok{y=}\NormalTok{AST), }\DataTypeTok{color=}\StringTok{"red"}\NormalTok{)}
\end{Highlighting}
\end{Shaded}

\includegraphics{SportsData_files/figure-latex/unnamed-chunk-291-1.pdf}

Not bad, not bad. But who are Cam Mack's peers? If we look at the numbers in Group 1, there's 335 of them. So let's limit them to just Big Ten guards. Unfortunately, my scraper didn't quite work and in the place of Conference is the coach's name. So I'm going to have to do this the hard way and make a list of Big Ten teams and filter on that. Then I'll sort by minutes played.

\begin{Shaded}
\begin{Highlighting}[]
\NormalTok{big10 <-}\StringTok{ }\KeywordTok{c}\NormalTok{(}\StringTok{"Nebraska Cornhuskers"}\NormalTok{, }\StringTok{"Iowa Hawkeyes"}\NormalTok{, }\StringTok{"Minnesota Golden Gophers"}\NormalTok{, }\StringTok{"Illinois Fighting Illini"}\NormalTok{, }\StringTok{"Northwestern Wildcats"}\NormalTok{, }\StringTok{"Wisconsin Badgers"}\NormalTok{, }\StringTok{"Indiana Hoosiers"}\NormalTok{, }\StringTok{"Purdue Boilermakers"}\NormalTok{, }\StringTok{"Ohio State Buckeyes"}\NormalTok{, }\StringTok{"Michigan Wolverines"}\NormalTok{, }\StringTok{"Michigan State Spartans"}\NormalTok{, }\StringTok{"Penn State Nittany Lions"}\NormalTok{, }\StringTok{"Rutgers Scarlet Knights"}\NormalTok{, }\StringTok{"Maryland Terrapins"}\NormalTok{)}

\NormalTok{playercluster }\OperatorTok\StringTok{ }\KeywordTok{filter}\NormalTok{(k5.cluster }\OperatorTok{==}\StringTok{ }\DecValTok{4}\NormalTok{) }\OperatorTok\StringTok{ }\KeywordTok{filter}\NormalTok{(Team }\OperatorTok\StringTok{ }\NormalTok{big10) }\OperatorTok\StringTok{ }\KeywordTok{arrange}\NormalTok{(}\KeywordTok{desc}\NormalTok{(MP))}
\end{Highlighting}
\end{Shaded}

\begin{verbatim}
##          Player                    Team Pos MP   FG.  X3P. AST TOV PTS
## 1   Cole Bajema     Michigan Wolverines   G 35 0.750 0.571   0   2  24
## 2    Reese Mona      Maryland Terrapins   G 30 1.000 1.000   2   1   9
## 3   Joey Downes Rutgers Scarlet Knights   G 10 0.667 1.000   0   1   8
## 4 Travis Valmon      Maryland Terrapins   G  8 0.500 1.000   0   0   4
## 5  Cooper Bybee        Indiana Hoosiers   G  4 1.000 1.000   0   0   3
##   k5.cluster
## 1          4
## 2          4
## 3          4
## 4          4
## 5          4
\end{verbatim}

So there are the 8 guards most like Cam Mack in the Big Ten. It'll be interesting to watch this evolve over the season. Fred Hoiberg and others think he might be one of the best guards in the league. We'll see, using cluster analysis.

\hypertarget{advanced-metrics}{%
\section{Advanced metrics}\label{advanced-metrics}}

How much does this change if we change the metrics? I used pretty standard box score metrics above. What if we did it using Player Efficiency Rating, True Shooting Percentage, Point Production, Assist Percentage, Win Shares Per 40 Minutes and Box Plus Minus (you can get definitions of all of them by \href{https://www.sports-reference.com/cbb/schools/nebraska/2020.html}{hovering over the stats on Nebraksa's stats page}).

We'll repeat the process. Filter out players who don't play, players with stats missing, and just focus on those stats listed above.

\begin{Shaded}
\begin{Highlighting}[]
\NormalTok{playersadvanced <-}\StringTok{ }\NormalTok{players }\OperatorTok\StringTok{ }
\StringTok{  }\KeywordTok{filter}\NormalTok{(MP}\OperatorTok{>}\DecValTok{0}\NormalTok{) }\OperatorTok\StringTok{ }
\StringTok{  }\KeywordTok{filter}\NormalTok{(Pos }\OperatorTok{==}\StringTok{ "G"}\NormalTok{) }\OperatorTok\StringTok{ }
\StringTok{  }\KeywordTok{select}\NormalTok{(Player, Team, Pos, PER, }\StringTok{`}\DataTypeTok{TS%}\StringTok{`}\NormalTok{, PProd, }\StringTok{`}\DataTypeTok{AST%}\StringTok{`}\NormalTok{, }\StringTok{`}\DataTypeTok{WS/40}\StringTok{`}\NormalTok{, BPM) }\OperatorTok\StringTok{ }
\StringTok{  }\KeywordTok{na.omit}\NormalTok{() }
\end{Highlighting}
\end{Shaded}

Now to scale them.

\begin{Shaded}
\begin{Highlighting}[]
\NormalTok{playersadvscaled <-}\StringTok{ }\NormalTok{playersadvanced }\OperatorTok\StringTok{ }
\StringTok{  }\KeywordTok{select}\NormalTok{(PER, }\StringTok{`}\DataTypeTok{TS%}\StringTok{`}\NormalTok{, PProd, }\StringTok{`}\DataTypeTok{AST%}\StringTok{`}\NormalTok{, }\StringTok{`}\DataTypeTok{WS/40}\StringTok{`}\NormalTok{, BPM) }\OperatorTok\StringTok{ }
\StringTok{  }\KeywordTok{mutate_all}\NormalTok{(scale) }\OperatorTok\StringTok{ }
\StringTok{  }\KeywordTok{na.omit}\NormalTok{()}
\end{Highlighting}
\end{Shaded}

Let's find the optimal number of clusters.

\begin{Shaded}
\begin{Highlighting}[]
\CommentTok{# function to compute total within-cluster sum of square }
\NormalTok{wss <-}\StringTok{ }\ControlFlowTok{function}\NormalTok{(k) \{}
  \KeywordTok{kmeans}\NormalTok{(playersadvscaled, k, }\DataTypeTok{nstart =} \DecValTok{10}\NormalTok{ )}\OperatorTok{$}\NormalTok{tot.withinss}
\NormalTok{\}}

\CommentTok{# Compute and plot wss for k = 1 to k = 15}
\NormalTok{k.values <-}\StringTok{ }\DecValTok{1}\OperatorTok{:}\DecValTok{15}

\CommentTok{# extract wss for 2-15 clusters}
\NormalTok{wss_values <-}\StringTok{ }\KeywordTok{map_dbl}\NormalTok{(k.values, wss)}
\end{Highlighting}
\end{Shaded}

\begin{verbatim}
## Warning: did not converge in 10 iterations

## Warning: did not converge in 10 iterations

## Warning: did not converge in 10 iterations

## Warning: did not converge in 10 iterations
\end{verbatim}

\begin{Shaded}
\begin{Highlighting}[]
\KeywordTok{plot}\NormalTok{(k.values, wss_values,}
       \DataTypeTok{type=}\StringTok{"b"}\NormalTok{, }\DataTypeTok{pch =} \DecValTok{19}\NormalTok{, }\DataTypeTok{frame =} \OtherTok{FALSE}\NormalTok{, }
       \DataTypeTok{xlab=}\StringTok{"Number of clusters K"}\NormalTok{,}
       \DataTypeTok{ylab=}\StringTok{"Total within-clusters sum of squares"}\NormalTok{)}
\end{Highlighting}
\end{Shaded}

\includegraphics{SportsData_files/figure-latex/unnamed-chunk-295-1.pdf}

Looks like 5 again.

\begin{Shaded}
\begin{Highlighting}[]
\NormalTok{advk5 <-}\StringTok{ }\KeywordTok{kmeans}\NormalTok{(playersadvscaled, }\DataTypeTok{centers =} \DecValTok{5}\NormalTok{, }\DataTypeTok{nstart =} \DecValTok{25}\NormalTok{)}
\end{Highlighting}
\end{Shaded}

What do we have here?

\begin{Shaded}
\begin{Highlighting}[]
\NormalTok{advk5}
\end{Highlighting}
\end{Shaded}

\begin{verbatim}
## K-means clustering with 5 clusters of sizes 104, 632, 766, 1253, 9
## 
## Cluster means:
##          PER        TS%      PProd       AST%      WS/40        BPM
## 1 -2.6922417 -2.6802561 -1.1497380 -1.0109009 -2.8398288 -2.8455990
## 2 -0.6089002 -0.6736304 -0.8322766 -0.2324539 -0.5651578 -0.6493960
## 3  0.5022214  0.2566711  1.2288970  0.7362004  0.4083263  0.5004325
## 4  0.1594366  0.3741799 -0.2279088 -0.2650562  0.2130658  0.2118878
## 5  8.9269330  4.3359455 -1.1326482  2.2478231  8.0858448  6.3926472
## 
## Clustering vector:
##    [1] 3 4 4 4 4 2 4 2 4 3 4 3 4 4 2 4 3 4 4 4 2 2 4 1 3 4 4 4 4 2 4 4 3 3 3 4 4
##   [38] 4 2 3 3 4 4 4 4 4 2 2 3 4 4 4 4 4 4 1 3 3 4 4 3 4 4 4 4 3 3 3 4 4 4 2 2 2
##   [75] 2 2 3 4 3 4 3 2 1 4 2 3 4 4 4 4 2 4 2 3 3 4 4 4 2 2 3 3 4 2 2 2 2 3 4 4 4
##  [112] 4 4 2 2 4 3 3 4 4 4 4 2 2 3 3 4 2 4 2 2 3 3 4 4 2 4 5 1 3 4 4 4 2 4 3 4 4
##  [149] 4 4 2 4 1 3 3 4 4 3 4 2 3 3 4 2 4 2 5 2 1 3 3 4 4 4 3 4 2 3 3 4 3 4 4 4 2
##  [186] 3 3 3 4 4 4 4 4 4 2 3 3 2 2 3 4 4 4 2 4 2 4 3 3 4 4 2 4 4 5 2 3 4 4 3 4 4
##  [223] 4 2 4 2 3 4 4 4 4 2 2 3 3 4 4 4 4 4 2 3 4 3 3 4 3 4 3 4 4 4 4 4 2 1 3 3 3
##  [260] 4 2 4 1 3 3 3 4 4 4 4 4 2 3 1 1 3 3 4 4 4 4 4 2 2 3 3 4 4 4 4 4 4 4 3 4 4
##  [297] 4 3 2 3 3 3 4 2 1 3 3 3 4 3 4 4 4 3 3 4 3 4 4 4 4 4 3 4 2 2 2 2 2 2 3 3 4
##  [334] 4 2 2 3 3 4 4 2 4 4 3 4 2 4 2 4 2 3 3 4 4 2 4 4 3 3 4 2 2 2 2 3 4 3 2 4 2
##  [371] 2 3 3 4 4 2 2 2 2 2 3 4 3 4 3 4 4 3 3 3 4 4 4 4 2 2 2 2 4 2 3 3 4 3 2 4 2
##  [408] 4 1 3 3 4 4 2 2 3 3 3 4 4 4 1 3 3 2 2 1 4 1 3 3 4 4 2 2 2 3 4 4 4 2 2 2 2
##  [445] 4 3 3 4 3 4 4 4 2 3 3 4 4 4 2 3 3 4 4 4 4 4 1 1 3 3 4 4 2 2 2 2 3 3 4 4 2
##  [482] 2 1 3 4 4 4 2 2 4 2 3 3 4 4 4 4 2 2 3 3 2 2 2 2 2 2 1 3 3 4 4 2 4 2 1 3 3
##  [519] 4 4 3 2 2 3 3 3 4 2 2 2 2 2 3 4 4 3 2 2 2 1 1 3 4 4 2 2 2 2 3 4 4 4 2 2 4
##  [556] 4 3 3 4 4 4 4 2 4 3 3 3 3 4 4 2 1 3 4 4 4 3 4 4 2 4 3 3 4 4 2 4 3 3 4 4 2
##  [593] 4 4 2 3 4 4 2 4 2 2 1 3 3 4 4 4 2 4 1 3 3 3 4 4 2 4 2 1 3 3 4 4 4 4 2 3 3
##  [630] 4 4 4 4 2 2 4 3 4 4 4 2 1 1 3 3 3 4 4 4 2 2 2 3 3 3 4 4 3 3 4 4 4 2 4 1 3
##  [667] 4 3 4 4 4 2 3 4 2 2 2 5 1 2 3 3 4 4 3 4 4 4 4 3 4 3 2 4 2 1 3 4 3 4 4 4 3
##  [704] 3 4 3 4 4 4 2 5 3 4 4 4 2 2 3 3 4 4 4 2 4 2 4 4 1 3 4 3 4 3 2 4 4 4 3 3 4
##  [741] 2 2 2 3 3 4 4 4 4 4 2 1 3 3 3 4 3 4 2 1 3 3 4 4 4 2 2 2 3 3 3 4 4 4 2 3 3
##  [778] 3 3 2 4 4 2 4 3 3 3 4 3 3 4 4 4 4 2 4 2 3 3 4 2 2 1 3 3 4 4 4 3 4 3 3 3 4
##  [815] 4 4 3 3 3 4 4 4 2 4 2 3 4 4 3 2 4 2 4 2 3 4 4 4 4 4 2 4 2 2 3 1 3 3 3 4 4
##  [852] 2 2 3 3 4 4 4 4 2 3 3 3 4 4 2 4 2 2 2 3 4 2 4 4 2 1 3 4 4 4 4 4 2 4 4 3 4
##  [889] 3 4 4 4 2 4 3 4 3 4 4 4 3 4 3 3 3 4 4 3 4 4 4 2 2 3 3 4 4 2 4 4 3 3 3 4 2
##  [926] 2 2 4 3 4 4 4 4 1 3 3 4 4 2 2 4 2 4 3 4 4 2 2 2 2 3 3 3 4 4 3 4 2 4 2 2 2
##  [963] 2 2 3 3 4 3 4 2 2 3 3 4 2 3 4 4 4 3 4 2 2 4 4 2 3 3 3 3 4 2 2 4 4 3 3 3 4
## [1000] 4 4 3 4 4 4 4 4 2 3 3 4 4 4 4 2 3 4 4 4 2 4 2 1 3 3 3 4 4 4 4 4 2 3 3 4 4
## [1037] 4 4 4 4 4 3 3 4 4 4 1 3 3 4 4 2 4 2 3 3 4 4 4 4 4 2 4 3 3 4 4 4 4 2 4 2 3
## [1074] 3 3 4 4 4 2 4 4 2 3 3 3 4 2 2 3 4 3 4 4 2 4 4 3 2 4 4 4 2 1 2 3 4 4 4 2 2
## [1111] 3 3 3 3 2 4 2 2 2 3 3 3 4 2 2 1 3 3 3 3 4 4 2 4 2 4 4 3 4 2 4 2 2 2 3 3 3
## [1148] 4 4 4 4 4 2 3 3 4 4 4 2 4 2 4 3 3 4 4 4 2 4 2 3 3 4 4 4 4 3 4 4 4 4 3 3 4
## [1185] 4 4 4 4 3 4 4 1 3 4 3 4 4 4 3 3 3 3 2 1 3 3 3 4 4 4 1 3 3 4 4 4 4 2 4 3 3
## [1222] 4 4 4 2 2 2 3 4 3 2 4 5 2 4 2 3 4 3 4 2 4 2 3 4 3 2 2 2 2 2 1 3 4 4 4 4 4
## [1259] 2 2 4 3 4 3 3 4 2 3 3 4 4 2 3 3 4 2 2 3 3 3 4 4 2 2 3 3 4 4 2 2 4 4 3 4 4
## [1296] 4 2 2 4 4 3 4 4 4 4 4 4 3 3 4 4 4 4 2 2 3 3 3 2 2 2 1 3 3 4 3 4 2 3 4 4 2
## [1333] 2 1 3 3 4 4 4 4 4 2 3 3 3 3 4 4 2 3 3 4 4 2 4 2 3 3 4 4 4 3 4 2 2 3 3 3 4
## [1370] 4 2 4 3 3 3 4 3 4 2 3 4 4 4 4 2 2 2 3 3 3 4 4 4 4 2 4 3 4 4 3 3 4 2 3 3 3
## [1407] 4 4 4 4 4 3 3 4 4 4 2 2 4 2 3 3 4 4 2 4 2 1 3 3 3 3 2 2 4 4 4 2 2 2 2 2 2
## [1444] 2 2 3 4 3 4 3 2 4 2 3 3 4 4 2 4 3 3 3 4 4 4 3 4 4 4 2 2 1 2 1 4 3 3 4 4 4
## [1481] 2 5 4 1 3 4 2 2 2 2 2 3 3 4 3 4 2 2 2 2 1 1 4 4 4 2 2 3 3 4 4 4 4 4 2 1 4
## [1518] 3 3 4 4 4 4 3 4 3 3 4 4 4 2 4 3 3 4 4 2 2 2 3 3 4 4 2 4 2 3 3 4 4 2 2 2 2
## [1555] 3 3 4 4 4 2 3 3 4 4 3 2 4 2 3 3 4 2 4 4 4 3 4 4 4 2 4 2 2 2 3 3 4 4 4 4 2
## [1592] 4 3 3 4 4 2 2 1 3 3 3 4 4 2 3 3 4 4 4 4 4 4 4 4 4 4 3 4 1 3 3 3 4 4 3 3 3
## [1629] 4 4 4 4 3 2 2 2 2 2 2 2 3 3 3 4 4 4 4 4 4 3 4 4 3 2 3 3 3 2 2 2 2 4 4 4 4
## [1666] 4 2 2 2 2 3 4 4 4 4 4 4 4 2 3 3 3 4 2 4 2 2 2 5 2 2 1 3 3 3 2 2 2 2 3 4 4
## [1703] 4 4 2 1 3 3 3 3 4 4 4 1 3 3 4 4 4 2 4 4 4 2 3 4 3 4 2 2 1 1 3 3 3 4 4 1 3
## [1740] 3 3 4 4 2 4 4 2 1 4 3 3 4 4 2 2 2 4 3 3 4 4 4 2 2 3 4 4 4 4 4 4 2 4 3 3 4
## [1777] 4 4 3 4 4 2 4 4 2 4 2 3 4 4 2 3 2 4 4 3 3 4 4 3 2 2 2 3 3 3 3 4 3 4 2 2 4
## [1814] 4 2 3 2 3 2 2 2 2 3 3 3 4 2 4 2 4 1 3 4 4 4 2 2 4 3 3 4 4 4 2 3 4 4 4 4 4
## [1851] 4 1 3 4 3 4 4 2 4 3 3 4 2 4 4 2 2 2 4 1 3 3 3 3 4 4 4 4 3 3 4 4 2 3 4 4 4
## [1888] 4 2 4 2 4 4 2 3 3 3 4 4 4 3 3 4 4 4 4 2 1 3 3 4 4 2 2 3 3 3 3 3 3 4 2 1 3
## [1925] 3 3 3 4 2 4 4 2 4 3 3 4 3 3 4 2 2 3 4 3 4 2 2 4 2 3 3 4 4 4 2 2 1 3 3 3 2
## [1962] 4 2 4 1 3 3 3 4 4 2 4 1 3 4 4 3 3 4 4 2 2 3 4 4 4 4 2 2 3 4 4 4 2 4 3 3 4
## [1999] 4 4 4 4 3 2 2 3 4 4 4 4 4 4 2 2 3 4 3 4 4 4 2 1 3 4 3 2 4 2 3 3 4 4 2 4 4
## [2036] 3 3 4 4 4 2 2 4 3 4 4 4 3 4 4 2 4 4 4 3 4 2 3 3 3 4 4 2 2 2 1 1 3 3 4 3 4
## [2073] 4 4 4 4 2 4 4 4 4 4 4 4 1 3 3 4 4 4 2 4 3 3 3 3 4 4 4 2 2 3 3 3 3 4 1 3 4
## [2110] 4 4 2 2 2 4 1 3 3 4 4 2 2 4 4 3 3 4 4 2 3 3 4 4 2 3 4 4 4 4 4 4 4 4 2 3 4
## [2147] 4 4 2 1 3 4 4 4 4 2 2 2 2 3 3 2 2 2 4 2 4 3 3 3 4 4 4 1 3 2 4 4 4 4 2 2 2
## [2184] 3 3 4 3 4 4 4 4 4 1 3 3 4 4 4 4 4 4 2 3 4 4 4 4 4 4 2 3 3 3 3 4 4 4 4 4 3
## [2221] 4 4 4 4 2 2 3 3 4 4 2 2 4 2 4 3 3 3 4 4 4 1 1 3 4 4 4 2 2 2 2 2 3 3 3 4 2
## [2258] 4 4 3 3 4 4 4 4 4 4 3 3 4 2 2 2 2 5 3 3 4 4 4 2 4 4 3 4 4 4 4 4 2 2 4 3 4
## [2295] 2 4 2 2 2 4 4 3 3 4 4 4 4 4 4 4 4 3 4 4 3 4 4 2 1 3 3 4 4 2 4 2 2 3 4 2 2
## [2332] 2 3 3 4 4 4 3 2 2 2 3 3 4 4 4 4 2 2 2 3 3 3 4 4 2 4 2 4 3 3 4 4 4 2 3 4 3
## [2369] 4 4 4 4 2 4 4 2 4 2 2 2 2 2 3 3 3 4 2 2 2 3 3 4 2 2 4 3 3 4 4 4 4 4 2 4 3
## [2406] 4 3 2 4 4 1 3 3 4 4 4 4 4 4 4 3 4 2 4 2 3 3 4 2 4 2 3 3 4 3 4 4 4 2 2 4 3
## [2443] 4 3 4 4 4 4 4 3 3 3 2 2 3 3 4 2 4 4 2 4 3 4 4 4 3 4 4 4 4 2 4 3 3 2 4 3 3
## [2480] 4 4 4 2 1 1 3 4 3 4 2 2 3 3 4 4 3 2 3 3 4 2 2 4 2 2 2 3 3 4 4 4 2 4 4 4 2
## [2517] 3 3 4 4 2 4 4 3 3 3 4 4 3 2 4 4 4 3 3 4 4 4 4 2 2 1 1 3 3 4 2 2 1 1 3 3 4
## [2554] 4 4 4 3 4 3 4 4 2 4 4 4 3 4 4 3 4 4 4 4 3 3 4 4 4 4 4 4 2 3 3 3 4 4 4 2 1
## [2591] 3 3 4 4 4 4 2 3 3 4 3 3 4 4 4 4 3 3 4 4 2 2 4 1 3 4 4 4 2 2 2 4 1 3 3 3 4
## [2628] 4 1 3 3 3 4 4 4 2 1 2 2 2 2 4 2 4 2 4 2 1 3 4 3 4 4 2 4 1 3 3 4 3 4 4 3 3
## [2665] 4 4 4 4 2 4 3 3 4 2 4 2 2 2 1 3 4 4 4 4 4 1 2 3 3 4 4 4 4 2 4 1 3 3 4 4 4
## [2702] 2 2 2 3 3 3 4 4 2 3 4 2 4 4 4 2 2 4 2 2 3 3 3 4 4 2 4 1 1 3 4 3 2 2 2 2 1
## [2739] 3 3 4 2 2 4 2 3 4 4 4 4 4 4 4 2 4 4 3 3 3 2 4 4 4 4
## 
## Within cluster sum of squares by cluster:
## [1]  737.6141 1248.0870 1849.5824 2624.4456  722.9391
##  (between_SS / total_SS =  56.7 %)
## 
## Available components:
## 
## [1] "cluster"      "centers"      "totss"        "withinss"     "tot.withinss"
## [6] "betweenss"    "size"         "iter"         "ifault"
\end{verbatim}

Looks like this time, cluster 1 is all below average and cluster 5 is all above. Which cluster is Cam Mack in?

\begin{Shaded}
\begin{Highlighting}[]
\NormalTok{playeradvcluster <-}\StringTok{ }\KeywordTok{data.frame}\NormalTok{(playersadvanced, advk5}\OperatorTok{$}\NormalTok{cluster) }
\end{Highlighting}
\end{Shaded}

Cluster 5 on my dataset. So three games in, we can say he's in a big group of players who are all above average on these advanced metrics.

Now who are his Big Ten peers?

\begin{Shaded}
\begin{Highlighting}[]
\NormalTok{playeradvcluster }\OperatorTok\StringTok{ }
\StringTok{  }\KeywordTok{filter}\NormalTok{(advk5.cluster }\OperatorTok{==}\StringTok{ }\DecValTok{5}\NormalTok{) }\OperatorTok\StringTok{ }
\StringTok{  }\KeywordTok{filter}\NormalTok{(Team }\OperatorTok\StringTok{ }\NormalTok{big10) }\OperatorTok\StringTok{ }
\StringTok{  }\KeywordTok{arrange}\NormalTok{(}\KeywordTok{desc}\NormalTok{(PProd))}
\end{Highlighting}
\end{Shaded}

\begin{verbatim}
##  [1] Player        Team          Pos           PER           TS.          
##  [6] PProd         AST.          WS.40         BPM           advk5.cluster
## <0 rows> (or 0-length row.names)
\end{verbatim}

Sorting on Points Produced, Cam Mack is eighth out of the 31 guards in the Big Ten who land in Cluster 5.

It's early goings, but watch this player. He's fun to watch and the stats back it up.

\hypertarget{rtweet-and-text-analysis}{%
\chapter{Rtweet and Text Analysis}\label{rtweet-and-text-analysis}}

\textbf{By Collin K. Berke, Ph.D.}

One of the best rivalries in college volleyball was played on Saturday, November 2, 2019. \href{https://journalstar.com/sports/huskers/volleyball/john-cook-on-the-radio-nebraska-penn-state-match-will/article_c1d5c426-e136-5ef2-b589-510a0f17da82.html}{The seventh-ranked Penn State Nittany Lions (16-3) took on the eigth-ranked Nebraska Cornhuskers (16-3)}. This match featured two of the best middle blockers in the country, Nebraska's Lauren Stivrins and Penn State's Kaitlyn Hord. Stivrins was ranked No.~1 in Big Ten hitting percentage, a .466 before the match up. Hord, close behind, had a .423 hitting percentage and was ranked towards the top as one of the league's top blockers.

Alongside being a competition between premier players, this match was set to be a battle between two of the winningest coaches in NCAA Women's Volleyball history. Russ Rose, head coach of the Nittany Lions, came into the match with a 1289-209 (.860) record, 17 Big Ten Conference Championships, and 7 NCAA National Championships. For the Nebraska Cornhuskers' head coach, John Cook came into the match with a 721-148 (.830) record, 9 Big 12 Conference Championships, 4 Big Ten Conference Championships, and 5 NCAA National Championships. Check out this article \href{https://www.ncaa.com/news/volleyball-women/article/2019-10-29/no-7-penn-state-vs-no-8-nebraska-volleyball-preview}{here} to get a better understanding of the significance of this game and rivalry.

Being a contest between two storied programs, premier players, and two of the most winningest coaches in NCAA volleyball history, this match was poised to be one of the premier Big Ten matches of the 2019 season. If history was to serve as a guide, this match would easily go into five exciting, nail-biting sets.

To no surprise--it did. Nebraska came out victorious, 3 sets to 2, winning the fifth set 15 - 13. Although we have commentators, analysts, and reporters to tell us the story of the game, wouldn't it be interesting to tell the story from the fan's perspective? Can what they say allow us to take a pulse of how the fan base feels during the game? We can answer this question using Twitter tweet data, which we will access with the \href{https://rtweet.info/}{\texttt{rtweet}} package.

\textbf{Question - How do people feel during a game? Positive? Negative? Neutral?}

This chapter will teach you how to extract, analyze, and visualize \href{www.twitter.com}{Twitter} text data to tell a story about peoples sentiments toward any sport team, player, or event. Although Twitter is conventionally thought of as a social media platform, at a general level, it can be thought of as a corpus of textual data, which is generated by millions of users, talking about a wide array of topics over time.

During this chapter, we will access text data held within the body of tweets, which we will extract and import into R through the use of an API (application programming interface). This can seem like a pretty technical term, but all it really is is a portal to which data can be shared between computers and humans. NPR has an \href{https://digitalservices.npr.org/post/api-101-what-api}{API 101 post} on their site, which you can read to get a rough idea of what an API is and how they are used.

In fact, many news organizations provide APIs for people to access and use their data. For example, many news services like \href{https://developer.nytimes.com/}{The New York Times}, \href{https://www.npr.org/api/index}{NPR}, \href{https://developer.ap.org/}{The Associated Press} and social media platforms like \href{https://developers.facebook.com/docs/apis-and-sdks/}{Facebook} have APIs that can be used to access content or varying types of data. Many of these just require you to: a). have a developer account; b) have the proper API keys; and c). use their API in accordance with their terms of service. Every API you come across should have documentation outlining its use.

Above was a pretty hand-wavy explanation of APIs. Indeed, APIs have many different uses beyond just extracting data, but such a discussion is beyond the scope of this chapter. Nevertheless, APIs can be a powerful, useful tool to access data not normally available on web pages or other statistical reporting services.

\hypertarget{prerequisites}{%
\section{Prerequisites}\label{prerequisites}}

You will need to have a Twitter account to access and extract data. If needed, you can sign up for an account \href{https://twitter.com/i/flow/signup}{here}.

\hypertarget{tools-for-text-analysis}{%
\subsection{Tools for text analysis}\label{tools-for-text-analysis}}

This chapter will also require you to load and acquaint yourself with functions in four packages, \texttt{rtweet}, \texttt{lubridate}, \texttt{stringr}, and \texttt{tidytext}. You may have used some of functions in other portions of this class. Others may be new to you.

\begin{itemize}
\item
  \href{https://rtweet.info/index.html}{\texttt{rtweet}} is a R package used to access Twitter data via the Twitter API.
\item
  \href{https://lubridate.tidyverse.org/}{\texttt{lubridate}} is a package that makes working with dates and times a bit easier.
\item
  \href{https://stringr.tidyverse.org/}{\texttt{stringr}} is a package that provides several functions to make working with string data a little easier.
\item
  \href{https://juliasilge.github.io/tidytext/}{\texttt{tidytext}} is a package used to tidy, analyze, and visualize textual analyses. We will use this package to calculate tweet sentiments (e.g., positive and negative feelings).
\end{itemize}

This chapter will also use other packages you have gained familiarity with throughout the class: \texttt{dplyr} and \texttt{ggplot2}. To install these packages and load them for use in our analysis session, run the following code:

\begin{Shaded}
\begin{Highlighting}[]
\KeywordTok{install.packages}\NormalTok{(}\StringTok{"rtweet"}\NormalTok{) }\CommentTok{# installs the rtweet package}

\KeywordTok{install.packages}\NormalTok{(}\StringTok{"tidytext"}\NormalTok{) }\CommentTok{# installs the tidytext package}

\KeywordTok{install.packages}\NormalTok{(}\StringTok{"tidyverse"}\NormalTok{) }\CommentTok{# A collection of packages, includes the stringr packages}

\KeywordTok{install.packages}\NormalTok{(}\StringTok{"lubridate"}\NormalTok{) }\CommentTok{# Provides functions to make working with dates/times easier}
\end{Highlighting}
\end{Shaded}

\begin{Shaded}
\begin{Highlighting}[]
\CommentTok{# Load the packages to be used in your analysis session}

\KeywordTok{library}\NormalTok{(rtweet)}

\KeywordTok{library}\NormalTok{(tidytext)}

\KeywordTok{library}\NormalTok{(tidyverse)}

\KeywordTok{library}\NormalTok{(lubridate)}

\KeywordTok{library}\NormalTok{(ggrepel)}
\end{Highlighting}
\end{Shaded}

\hypertarget{working-with-string-data}{%
\subsection{Working with string data}\label{working-with-string-data}}

String data is just basically letters, words, symbols, and even emojis. Take for example the following tweet:

\begin{Shaded}
\begin{Highlighting}[]
\NormalTok{knitr}\OperatorTok{::}\KeywordTok{include_graphics}\NormalTok{(}\KeywordTok{rep}\NormalTok{(}\StringTok{"images/volleyballTweet.png"}\NormalTok{))}
\end{Highlighting}
\end{Shaded}

\includegraphics[width=7.53in]{images/volleyballTweet}

Everything contained in the message portion of the tweet is string data, even the emojis. When it comes to emojis, most have a special textual code that is rendered by a browser or device that gets displayed as an image. For example, the ear of corn emoji is actually written as \texttt{:corn:}, but it gets rendered as an image when we view the tweet on our computers/devices. We can extract, analyze, and visualize this string data to tell a wide range of stories from users' tweets. Our goal being to show sentiment over the length of a Husker volleyball Match and football game.

\hypertarget{verifying-your-account-to-access-twitter-data}{%
\section{Verifying your account to access Twitter data}\label{verifying-your-account-to-access-twitter-data}}

Before you can access Twitter data, you will need to verify your account. The \texttt{rtweet} package makes this really easy to do. You will first need to run one of the package's functions for it to walk you through the authentication process. To do this, let's just search for the most recent 8000 (non-retweeted) tweets containing the \textbf{\#huskers} and \textbf{\#GBR} hashtags.

Before you run the following code chunk, though, be aware a few things will take place. First, a browser window will open up asking you to verify that \texttt{rtweet} is allowed to access Twitter data via the API on behalf of your account. Accept this request and enter your credentials if you are asked to. Once you do this, you should get a message in your browser stating you have successfully authenticated the \texttt{rtweet} package. The data will then begin to download. The amount of time needed to import this data will depend on how many tweets the hashtag(s) are associated with. More tweets generally means longer import times.

\begin{Shaded}
\begin{Highlighting}[]
\NormalTok{huskers <-}\StringTok{ }\KeywordTok{search_tweets}\NormalTok{(}
  \StringTok{"#huskers"}\NormalTok{, }\DataTypeTok{n =} \DecValTok{8000}\NormalTok{, }\DataTypeTok{include_rts =} \OtherTok{FALSE}
\NormalTok{)}

\NormalTok{gbr <-}\StringTok{ }\KeywordTok{search_tweets}\NormalTok{(}
  \StringTok{"#GBR"}\NormalTok{, }\DataTypeTok{n =} \DecValTok{8000}\NormalTok{, }\DataTypeTok{include_rts =} \OtherTok{FALSE}
\NormalTok{)}
\end{Highlighting}
\end{Shaded}

\textbf{Important Note}: Depending on when you run the above code chunk, the API will return different data then the data used for the examples later in this chapter. This is due to the query rate cap Twitter places on it's API. Twitter's API caps queries to 18,000 of the most recent tweets during the past couple of days. This cap resets every 15 minutes. The \texttt{rtweet} package does has functionality to pull data once your query limit resets. However, if you're looking to pull tweets for a very popular event (e.g., The Super Bowl), you may want to consider other options to extract this type of data. This is also important to understand because if you are looking to pull tweets for a specific event, you will need to make sure you are pulling this data within a reasonable time during or after the event. If you don't, these rate limits might not allow you access the data you need to do your analysis.

The data we will use later for the examples in the chapter can be found \href{https://unl.box.com/s/x4jjifc394gxfvsvbwb3csez2ne4l4rb}{here} and \href{https://unl.box.com/s/s4ej8khwi9ah9qvqpi2jviqomfocie3n}{here}. The first data set are tweets that use the \#huskers hashtag. The second has data of tweets that use the \#gbr hashtag. You will need to download both data sets, put them in the right directory, and import both for the below examples to work correctly. The code to import this data will look something like this:

\begin{Shaded}
\begin{Highlighting}[]
\NormalTok{huskerTweets <-}\StringTok{ }\KeywordTok{read_csv}\NormalTok{(}\StringTok{"data/huskerTweets.csv"}\NormalTok{)}
\end{Highlighting}
\end{Shaded}

\begin{verbatim}
## Parsed with column specification:
## cols(
##   .default = col_character(),
##   created_at = col_datetime(format = ""),
##   display_text_width = col_double(),
##   is_quote = col_logical(),
##   is_retweet = col_logical(),
##   favorite_count = col_double(),
##   retweet_count = col_double(),
##   quote_count = col_logical(),
##   reply_count = col_logical(),
##   symbols = col_logical(),
##   ext_media_type = col_logical(),
##   quoted_created_at = col_datetime(format = ""),
##   quoted_favorite_count = col_double(),
##   quoted_retweet_count = col_double(),
##   quoted_followers_count = col_double(),
##   quoted_friends_count = col_double(),
##   quoted_statuses_count = col_double(),
##   quoted_verified = col_logical(),
##   retweet_status_id = col_logical(),
##   retweet_text = col_logical(),
##   retweet_created_at = col_logical()
##   # ... with 21 more columns
## )
\end{verbatim}

\begin{verbatim}
## See spec(...) for full column specifications.
\end{verbatim}

\begin{Shaded}
\begin{Highlighting}[]
\NormalTok{gbrTweets <-}\StringTok{ }\KeywordTok{read_csv}\NormalTok{(}\StringTok{"data/gbrTweets.csv"}\NormalTok{)}
\end{Highlighting}
\end{Shaded}

\begin{verbatim}
## Parsed with column specification:
## cols(
##   .default = col_character(),
##   created_at = col_datetime(format = ""),
##   display_text_width = col_double(),
##   is_quote = col_logical(),
##   is_retweet = col_logical(),
##   favorite_count = col_double(),
##   retweet_count = col_double(),
##   quote_count = col_logical(),
##   reply_count = col_logical(),
##   symbols = col_logical(),
##   ext_media_type = col_logical(),
##   quoted_created_at = col_datetime(format = ""),
##   quoted_favorite_count = col_double(),
##   quoted_retweet_count = col_double(),
##   quoted_followers_count = col_double(),
##   quoted_friends_count = col_double(),
##   quoted_statuses_count = col_double(),
##   quoted_verified = col_logical(),
##   retweet_status_id = col_logical(),
##   retweet_text = col_logical(),
##   retweet_created_at = col_logical()
##   # ... with 21 more columns
## )
## See spec(...) for full column specifications.
\end{verbatim}

\begin{verbatim}
## Warning: 1 parsing failure.
##  row     col           expected actual                 file
## 4365 symbols 1/0/T/F/TRUE/FALSE    GBR 'data/gbrTweets.csv'
\end{verbatim}

This brings up a good point about saving any data you import from Twitter's API. \textbf{Always save your data.} Remember those rate limits? If you don't save your data and too many days pass, you will not be able to access that data again. To do this, you can use the \texttt{write\_as\_csv()} function from the \texttt{rtweet} package to save a \texttt{.csv} file of your data. The code to do this will look something like this:

\begin{Shaded}
\begin{Highlighting}[]
\KeywordTok{write_as_csv}\NormalTok{(huskerTweets, }\StringTok{"data/huskerTweets.csv"}\NormalTok{)}
\end{Highlighting}
\end{Shaded}

\textbf{Be aware that this function will overwrite data.} If you make changes to your \texttt{huskerTweets} object and then run the \texttt{write\_as\_csv()} function again, it will overwrite your saved file with the modifications you made to your object. The lesson then is to always save an extra copy of your data in a separate directory, just in case you do accidentally make a mistake in overwriting your data.

\hypertarget{the-data-used-here}{%
\section{The data used here}\label{the-data-used-here}}

To provide a little context, I pulled the data in this chapter on Sunday, November 3, 2019. This was the day after Nebraska Football lost to Purdue, and Nebraska Volleyball won against Penn State. You can follow the steps above to download this data for the following examples.

To make it easier to work with, I am going to combine these two data sets into one using the \texttt{bind\_rows()} function from \texttt{dplyr}. There is a slight problem though, some people may have had a tweet that contained both the \textbf{\#huskers} and \textbf{\#GBR} hashtags in their tweet. So if we combine these two data sets, there might be duplicate data. To dedupe the data, we can apply a \texttt{distinct(text,\ .keep\_all\ =\ TRUE)} to remove any duplicates. The \texttt{.keep\_all\ =\ TRUE} argument just tells R to keep all columns in the data frame after our data has been deduped.

\begin{Shaded}
\begin{Highlighting}[]
\NormalTok{tweet_data <-}\StringTok{ }\KeywordTok{bind_rows}\NormalTok{(huskerTweets, gbrTweets) }\OperatorTok\StringTok{ }
\StringTok{  }\KeywordTok{distinct}\NormalTok{(text, }\DataTypeTok{.keep_all =} \OtherTok{TRUE}\NormalTok{)}
\end{Highlighting}
\end{Shaded}

\hypertarget{data-exploration}{%
\section{Data Exploration}\label{data-exploration}}

Let's explore the data a bit. Run a \texttt{glimpse(tweet\_data)} to get a view of what data was returned from twitter. My query on November 3rd, 2019 returned 11,523 non-retweeted tweets from 3,917 accounts using the \texttt{\#huskers} and/or the \texttt{\#GBR} hashtag within the tweet's body (again if you ran the code above, your data will be different) .

It's important to remember these tweets can come from accounts that are people, organizations, and even bots. So when drawing conclusions from this data, make sure to keep in mind that these tweets may not represent the sentiment of just one person. Additionally, it is important to remember that not all fans of a sports team are on or use Twitter, so it surely is not a valid representation of all fan sentiment. Indeed, you could also have fans of other teams using your hashtags.

\begin{Shaded}
\begin{Highlighting}[]
\KeywordTok{glimpse}\NormalTok{(tweet_data)}
\end{Highlighting}
\end{Shaded}

\begin{verbatim}
## Observations: 11,523
## Variables: 90
## $ user_id                 <chr> "x17636179", "x17636179", "x17636179", "x15...
## $ status_id               <chr> "x1191100304940396544", "x11908526509059440...
## $ created_at              <dttm> 2019-11-03 21:10:16, 2019-11-03 04:46:11, ...
## $ screen_name             <chr> "SeanKeeler", "SeanKeeler", "SeanKeeler", "...
## $ text                    <chr> "ICYMI, #CSURams fans, a recap of @denverpo...
## $ source                  <chr> "Twitter Web App", "Twitter for iPhone", "T...
## $ display_text_width      <dbl> 269, 188, 182, 122, 135, 189, 172, 135, 94,...
## $ reply_to_status_id      <chr> NA, NA, NA, NA, NA, NA, NA, NA, NA, NA, NA,...
## $ reply_to_user_id        <chr> NA, NA, NA, NA, NA, NA, NA, NA, NA, NA, NA,...
## $ reply_to_screen_name    <chr> NA, NA, NA, NA, NA, NA, NA, NA, NA, NA, NA,...
## $ is_quote                <lgl> FALSE, TRUE, FALSE, FALSE, FALSE, FALSE, FA...
## $ is_retweet              <lgl> FALSE, FALSE, FALSE, FALSE, FALSE, FALSE, F...
## $ favorite_count          <dbl> 0, 0, 1, 116, 19, 2, 40, 23, 149, 5, 31, 15...
## $ retweet_count           <dbl> 0, 0, 0, 6, 1, 0, 1, 2, 5, 1, 3, 1, 0, 0, 0...
## $ quote_count             <lgl> NA, NA, NA, NA, NA, NA, NA, NA, NA, NA, NA,...
## $ reply_count             <lgl> NA, NA, NA, NA, NA, NA, NA, NA, NA, NA, NA,...
## $ hashtags                <chr> "CSURams Huskers ProudToBe AtThePeak CSU", ...
## $ symbols                 <lgl> NA, NA, NA, NA, NA, NA, NA, NA, NA, NA, NA,...
## $ urls_url                <chr> "tinyurl.com/y2b3kog9 tinyurl.com/yy7ntps4"...
## $ urls_t.co               <chr> "https://t.co/1FlAnRRgEq https://t.co/t8j1L...
## $ urls_expanded_url       <chr> "https://tinyurl.com/y2b3kog9 https://tinyu...
## $ media_url               <chr> NA, NA, NA, "http://pbs.twimg.com/ext_tw_vi...
## $ media_t.co              <chr> NA, NA, NA, "https://t.co/FSCP827hg0", "htt...
## $ media_expanded_url      <chr> NA, NA, NA, "https://twitter.com/HuskerSpor...
## $ media_type              <chr> NA, NA, NA, "photo", "photo", "photo", "pho...
## $ ext_media_url           <chr> NA, NA, NA, "http://pbs.twimg.com/ext_tw_vi...
## $ ext_media_t.co          <chr> NA, NA, NA, "https://t.co/FSCP827hg0", "htt...
## $ ext_media_expanded_url  <chr> NA, NA, NA, "https://twitter.com/HuskerSpor...
## $ ext_media_type          <lgl> NA, NA, NA, NA, NA, NA, NA, NA, NA, NA, NA,...
## $ mentions_user_id        <chr> "x8216772", "x8216772", "x24725032", "x1210...
## $ mentions_screen_name    <chr> "denverpost", "denverpost", "DPostSports", ...
## $ lang                    <chr> "en", "en", "en", "en", "en", "en", "en", "...
## $ quoted_status_id        <chr> NA, "x1190803840213209088", NA, NA, NA, NA,...
## $ quoted_text             <chr> NA, "From Nebraska to Fort Collins, how CSU...
## $ quoted_created_at       <dttm> NA, 2019-11-03 01:32:14, NA, NA, NA, NA, N...
## $ quoted_source           <chr> NA, "TweetDeck", NA, NA, NA, NA, NA, NA, NA...
## $ quoted_favorite_count   <dbl> NA, 5, NA, NA, NA, NA, NA, NA, NA, NA, 194,...
## $ quoted_retweet_count    <dbl> NA, 1, NA, NA, NA, NA, NA, NA, NA, NA, 16, ...
## $ quoted_user_id          <chr> NA, "x24725032", NA, NA, NA, NA, NA, NA, NA...
## $ quoted_screen_name      <chr> NA, "DPostSports", NA, NA, NA, NA, NA, NA, ...
## $ quoted_name             <chr> NA, "Denver Post Sports", NA, NA, NA, NA, N...
## $ quoted_followers_count  <dbl> NA, 34841, NA, NA, NA, NA, NA, NA, NA, NA, ...
## $ quoted_friends_count    <dbl> NA, 395, NA, NA, NA, NA, NA, NA, NA, NA, 60...
## $ quoted_statuses_count   <dbl> NA, 113697, NA, NA, NA, NA, NA, NA, NA, NA,...
## $ quoted_location         <chr> NA, "Denver, Colorado", NA, NA, NA, NA, NA,...
## $ quoted_description      <chr> NA, "Sports news & analysis from @denverpos...
## $ quoted_verified         <lgl> NA, TRUE, NA, NA, NA, NA, NA, NA, NA, NA, F...
## $ retweet_status_id       <lgl> NA, NA, NA, NA, NA, NA, NA, NA, NA, NA, NA,...
## $ retweet_text            <lgl> NA, NA, NA, NA, NA, NA, NA, NA, NA, NA, NA,...
## $ retweet_created_at      <lgl> NA, NA, NA, NA, NA, NA, NA, NA, NA, NA, NA,...
## $ retweet_source          <lgl> NA, NA, NA, NA, NA, NA, NA, NA, NA, NA, NA,...
## $ retweet_favorite_count  <lgl> NA, NA, NA, NA, NA, NA, NA, NA, NA, NA, NA,...
## $ retweet_retweet_count   <lgl> NA, NA, NA, NA, NA, NA, NA, NA, NA, NA, NA,...
## $ retweet_user_id         <lgl> NA, NA, NA, NA, NA, NA, NA, NA, NA, NA, NA,...
## $ retweet_screen_name     <lgl> NA, NA, NA, NA, NA, NA, NA, NA, NA, NA, NA,...
## $ retweet_name            <lgl> NA, NA, NA, NA, NA, NA, NA, NA, NA, NA, NA,...
## $ retweet_followers_count <lgl> NA, NA, NA, NA, NA, NA, NA, NA, NA, NA, NA,...
## $ retweet_friends_count   <lgl> NA, NA, NA, NA, NA, NA, NA, NA, NA, NA, NA,...
## $ retweet_statuses_count  <lgl> NA, NA, NA, NA, NA, NA, NA, NA, NA, NA, NA,...
## $ retweet_location        <lgl> NA, NA, NA, NA, NA, NA, NA, NA, NA, NA, NA,...
## $ retweet_description     <lgl> NA, NA, NA, NA, NA, NA, NA, NA, NA, NA, NA,...
## $ retweet_verified        <lgl> NA, NA, NA, NA, NA, NA, NA, NA, NA, NA, NA,...
## $ place_url               <chr> NA, NA, NA, NA, NA, NA, NA, NA, NA, NA, NA,...
## $ place_name              <chr> NA, NA, NA, NA, NA, NA, NA, NA, NA, NA, NA,...
## $ place_full_name         <chr> NA, NA, NA, NA, NA, NA, NA, NA, NA, NA, NA,...
## $ place_type              <chr> NA, NA, NA, NA, NA, NA, NA, NA, NA, NA, NA,...
## $ country                 <chr> NA, NA, NA, NA, NA, NA, NA, NA, NA, NA, NA,...
## $ country_code            <chr> NA, NA, NA, NA, NA, NA, NA, NA, NA, NA, NA,...
## $ geo_coords              <chr> "NA NA", "NA NA", "NA NA", "NA NA", "NA NA"...
## $ coords_coords           <chr> "NA NA", "NA NA", "NA NA", "NA NA", "NA NA"...
## $ bbox_coords             <chr> "NA NA NA NA NA NA NA NA", "NA NA NA NA NA ...
## $ status_url              <chr> "https://twitter.com/SeanKeeler/status/1191...
## $ name                    <chr> "Sean Keeler", "Sean Keeler", "Sean Keeler"...
## $ location                <chr> "Denver, CO", "Denver, CO", "Denver, CO", "...
## $ description             <chr> "@DenverPost staffer, dad, husband, drummer...
## $ url                     <chr> "https://t.co/z0eFbv9eaz", "https://t.co/z0...
## $ protected               <lgl> FALSE, FALSE, FALSE, FALSE, FALSE, FALSE, F...
## $ followers_count         <dbl> 5245, 5245, 5245, 30451, 30451, 30451, 3045...
## $ friends_count           <dbl> 1619, 1619, 1619, 701, 701, 701, 701, 701, ...
## $ listed_count            <dbl> 296, 296, 296, 247, 247, 247, 247, 247, 247...
## $ statuses_count          <dbl> 28124, 28124, 28124, 12801, 12801, 12801, 1...
## $ favourites_count        <dbl> 4498, 4498, 4498, 6033, 6033, 6033, 6033, 6...
## $ account_created_at      <dttm> 2008-11-25 23:53:13, 2008-11-25 23:53:13, ...
## $ verified                <lgl> FALSE, FALSE, FALSE, TRUE, TRUE, TRUE, TRUE...
## $ profile_url             <chr> "https://t.co/z0eFbv9eaz", "https://t.co/z0...
## $ profile_expanded_url    <chr> "http://www.seankeeler.tumblr.com", "http:/...
## $ account_lang            <lgl> NA, NA, NA, NA, NA, NA, NA, NA, NA, NA, NA,...
## $ profile_banner_url      <chr> "https://pbs.twimg.com/profile_banners/1763...
## $ profile_background_url  <chr> "http://abs.twimg.com/images/themes/theme10...
## $ profile_image_url       <chr> "http://pbs.twimg.com/profile_images/118130...
\end{verbatim}

As you can see, \texttt{glimpse()} returns a lot of columns that are not really relevant to our analysis. Let's apply a \texttt{select()} function to only retain the data relevant to our analysis, .

\begin{Shaded}
\begin{Highlighting}[]
\NormalTok{tweet_data <-}\StringTok{ }\NormalTok{tweet_data }\OperatorTok\StringTok{ }
\StringTok{  }\KeywordTok{select}\NormalTok{(user_id, status_id, created_at, screen_name, text, display_text_width,}
\NormalTok{         favorite_count, retweet_count, hashtags, description, followers_count)}
\end{Highlighting}
\end{Shaded}

\hypertarget{cleaning-data-for-analysis}{%
\section{Cleaning data for analysis}\label{cleaning-data-for-analysis}}

If you examine the data set, you will see this data needs some wrangling. First, we need to fix the \texttt{created\_at} variable. Right now it is represented in Greenwich Mean Time (GMT), but we need it to be in Central Standard Time (CST). We do this so we can make sense of when during the game things happened. Second, the data is outside of the time frame we are interested in examining, so we need to filter the data to be windowed during the time of the game. We will filter the data by date and time, examining tweets a little before and after the game.

\hypertarget{fixing-the-date-and-focusing-only-on-game-tweets}{%
\subsection{Fixing the date and focusing only on game tweets}\label{fixing-the-date-and-focusing-only-on-game-tweets}}

We will use the \texttt{with\_tz()} on our \texttt{created\_at} variable within our \texttt{mutate()} function to transform the \texttt{created\_at} column into Central Standard Time (CST). We do this by setting the \texttt{tzone} argument to \texttt{"America/Chicago"}. Once out time is adjusted, we need group tweets within a specific bin of time. For this example I have decided to bin tweets to the nearest 5 minute mark. We can do this by using the \texttt{round\_time()} function provided to us by the \texttt{lubridate} package.

Then, since we are only interested in tweets during the game, we can apply a \texttt{dplyr} \texttt{filter()} function to window our data set to tweets being posted around the start and end of the game.

\begin{Shaded}
\begin{Highlighting}[]
\NormalTok{volleyball_tweets <-}\StringTok{ }\NormalTok{tweet_data }\OperatorTok
\StringTok{  }\KeywordTok{mutate}\NormalTok{(}\DataTypeTok{created_at =} \KeywordTok{with_tz}\NormalTok{(created_at, }\DataTypeTok{tzone =} \StringTok{"America/Chicago"}\NormalTok{),}
         \DataTypeTok{created_at =} \KeywordTok{round_time}\NormalTok{(created_at, }\StringTok{"5 mins"}\NormalTok{, }\DataTypeTok{tz =} \StringTok{"America/Chicago"}\NormalTok{)) }\OperatorTok\StringTok{ }
\StringTok{  }\KeywordTok{filter}\NormalTok{(created_at }\OperatorTok{>=}\StringTok{ "2019-11-02 18:30:00"} \OperatorTok{&}\StringTok{ }\NormalTok{created_at }\OperatorTok{<=}\StringTok{ "2019-11-02 23:30:00"}\NormalTok{) }
\end{Highlighting}
\end{Shaded}

\hypertarget{number-of-tweets-throughout-the-game}{%
\section{Number of tweets throughout the game}\label{number-of-tweets-throughout-the-game}}

One question we might have pertains to the number of tweets that occur during the course of the match. To do this, all we need to do is \texttt{group\_by()} our tweets by our \texttt{created\_at} variable, and then use the \texttt{count()} function to count the number of tweets within each five minute bin. We then use \texttt{ggplot} to plot a line chart where \texttt{created\_at} is placed on the x-axis and \texttt{n}, number of tweets, is placed on the y-axis.

\begin{Shaded}
\begin{Highlighting}[]
\NormalTok{start_time <-}\StringTok{ }\KeywordTok{tibble}\NormalTok{(}\DataTypeTok{time =} \KeywordTok{as_datetime}\NormalTok{(}\StringTok{"2019-11-02 19:35:00"}\NormalTok{, }\DataTypeTok{tz =} \StringTok{"America/Chicago"}\NormalTok{), }\DataTypeTok{label =} \StringTok{"Start Time"}\NormalTok{) }

\NormalTok{volleyball_time <-}\StringTok{ }\NormalTok{volleyball_tweets }\OperatorTok
\StringTok{  }\KeywordTok{group_by}\NormalTok{(created_at) }\OperatorTok\StringTok{ }
\StringTok{  }\KeywordTok{count}\NormalTok{()}

\KeywordTok{ggplot}\NormalTok{() }\OperatorTok{+}
\StringTok{  }\KeywordTok{geom_line}\NormalTok{(}\DataTypeTok{data =}\NormalTok{ volleyball_time, }\KeywordTok{aes}\NormalTok{(}\DataTypeTok{x =}\NormalTok{ created_at, }\DataTypeTok{y =}\NormalTok{ n)) }\OperatorTok{+}
\StringTok{  }\KeywordTok{geom_point}\NormalTok{(}\DataTypeTok{data =}\NormalTok{ start_time, }\KeywordTok{aes}\NormalTok{(}\DataTypeTok{x =}\NormalTok{ time, }\DataTypeTok{y =} \DecValTok{5}\NormalTok{), }\DataTypeTok{color =} \StringTok{"red"}\NormalTok{) }\OperatorTok{+}
\StringTok{  }\KeywordTok{geom_text_repel}\NormalTok{(}\DataTypeTok{data =}\NormalTok{ start_time, }\KeywordTok{aes}\NormalTok{(}\DataTypeTok{x =}\NormalTok{ time, }\DataTypeTok{y =} \DecValTok{3}\NormalTok{, }\DataTypeTok{label =}\NormalTok{ label), }\DataTypeTok{nudge_x =} \DecValTok{-2}\NormalTok{) }\OperatorTok{+}
\StringTok{  }\KeywordTok{labs}\NormalTok{(}\DataTypeTok{y =} \StringTok{"Number of Tweets"}\NormalTok{,}
       \DataTypeTok{x =} \StringTok{"Central Standard Time (CST)"}\NormalTok{) }\OperatorTok{+}
\StringTok{  }\KeywordTok{theme_minimal}\NormalTok{() }\OperatorTok{+}
\StringTok{  }\KeywordTok{theme}\NormalTok{(}\DataTypeTok{axis.title.x =} \KeywordTok{element_blank}\NormalTok{())}
\end{Highlighting}
\end{Shaded}

\includegraphics{SportsData_files/figure-latex/unnamed-chunk-310-1.pdf}

There you have it. A trend line plotting tweet volume throughout the course of the event. Do you see any areas where the match might have had a significant number of tweets?

\hypertarget{tidying-the-text-data-for-analysis-applying-the-sentiment-scores}{%
\section{Tidying the text data for analysis, applying the sentiment scores}\label{tidying-the-text-data-for-analysis-applying-the-sentiment-scores}}

Okay, that's cool--but what we really want to know is what are peoples' sentiments throughout the game? Did they feel positive or negative throughout the event? Were there times that were more positive or negative? To achieve this, we are going to use the \texttt{tidytext} package to tidy up our text data and apply a sentiment score to each word held within each tweet. Let's break this down step-by-step.

First, we need to get the dictionary that contains the sentiment scoring for thousands of words used in the English language. \texttt{afinn\ \textless{}-\ get\_sentiments("afinn")} does just that for us. The development of these sentiment dictionaries is beyond this chapter. However, most of these dictionaries are crowd sourced by having people provide self-responses on how positive or negative a word is to them. For now, just understand the \texttt{affin} variable contains many words that have been rated for how positive or negative a word is on a scale that ranges from -5 to 5. -5 being the most negative, and 5 being the most positive. If you want to learn more about this dictionary or others, you can read more about them \href{https://www.tidytextmining.com/sentiment.html}{here}.

Second, now that we have our dictionary imported, we need to clean up our tweets data set so we can apply sentiment scores to each word used within each tweet. There's one problem, though. Each row in our data set is a complete tweet. For us to apply a sentiment score for each word, each word needs to get its own row. This is where the \texttt{unnest\_tokens()} function from the \texttt{tidytext} package comes into play. We use this function to create a data set that will create a new column called word, which will place every word from every tweet in our data set on its own row, which it knows which text data to this because we set the the second argument to the column name that holds our text data. In this case, we give it the \texttt{text} column. Once you run this code, if you look at the \texttt{volleyball\_tweets\_tidy} object, you should now have a data set where every row has its own word which was done for every tweet. This data frame should now be a super long data frame.

Lastly, the English language has many words that really don't mean anything in regards to sentiment. Take for example the word `the'. This article really doesn't represent a positive or negative sentiment. Thus, these types of words need to be taken out of our data set to enhance improve the accuracy of our analysis. To do this, we will apply the \texttt{anti\_join(stop\_words)} to our \texttt{dplyr} chain. All this does is get rid of the stop words in our data set that really don't contribute to the sentiment scores we are eventually going to calculate.

If you get an error on the next bit of code, you'll likely need to install the textdata package on the console with \texttt{install.packages("textdata")}. Next, if this next block of code hangs, it's because in the console you're being asked if you want to download some data. You do indeed want to do that, so type 1 and hit enter.

\begin{Shaded}
\begin{Highlighting}[]
\NormalTok{afinn <-}\StringTok{ }\KeywordTok{get_sentiments}\NormalTok{(}\StringTok{"afinn"}\NormalTok{)}

\NormalTok{volleyball_tweets_tidy <-}\StringTok{ }\NormalTok{volleyball_tweets }\OperatorTok\StringTok{ }
\StringTok{  }\KeywordTok{unnest_tokens}\NormalTok{(word, text) }\OperatorTok\StringTok{ }
\StringTok{  }\KeywordTok{anti_join}\NormalTok{(stop_words) }
\end{Highlighting}
\end{Shaded}

\begin{verbatim}
## Joining, by = "word"
\end{verbatim}

\hypertarget{fan-sentiment-over-the-game}{%
\subsubsection{Fan sentiment over the game}\label{fan-sentiment-over-the-game}}

Now that we have a tidied textual data set, all we need to do is apply our sentiment scores to these words using the \texttt{inner\_join()}, then \texttt{group\_by()} our \texttt{created\_at} variable, and calculate the mean sentiment for each five minute interval. At this point, we will use \texttt{ggplot} to plot sentiment of the tweets over time. We will do this by plotting the \texttt{created\_at} variable on the x-axis and the newly calculated \texttt{sentiment} variable on the y-axis. The rest is just adding annotations and styling, which we are already familiar with.

\begin{Shaded}
\begin{Highlighting}[]
\NormalTok{volleyball_tweets_sentiment <-}\StringTok{ }\NormalTok{volleyball_tweets_tidy }\OperatorTok
\StringTok{  }\KeywordTok{inner_join}\NormalTok{(afinn) }\OperatorTok\StringTok{ }
\StringTok{  }\KeywordTok{group_by}\NormalTok{(created_at) }\OperatorTok\StringTok{ }
\StringTok{  }\KeywordTok{summarise}\NormalTok{(}\DataTypeTok{sentiment =} \KeywordTok{mean}\NormalTok{(value))}
\end{Highlighting}
\end{Shaded}

\begin{verbatim}
## Joining, by = "word"
\end{verbatim}

\begin{Shaded}
\begin{Highlighting}[]
\KeywordTok{ggplot}\NormalTok{() }\OperatorTok{+}
\StringTok{  }\KeywordTok{geom_line}\NormalTok{(}\DataTypeTok{data =}\NormalTok{ volleyball_tweets_sentiment, }\KeywordTok{aes}\NormalTok{(}\DataTypeTok{x =}\NormalTok{ created_at, }\DataTypeTok{y =}\NormalTok{ sentiment)) }\OperatorTok{+}
\StringTok{  }\KeywordTok{geom_text}\NormalTok{(}\KeywordTok{aes}\NormalTok{(}\DataTypeTok{x =} \KeywordTok{as_datetime}\NormalTok{(}\StringTok{"2019-11-02 19:00:00"}\NormalTok{, }\DataTypeTok{tz =} \StringTok{"America/Chicago"}\NormalTok{), }\DataTypeTok{y =} \DecValTok{3}\NormalTok{), }\DataTypeTok{color =} \StringTok{"darkgreen"}\NormalTok{, }\DataTypeTok{label =} \StringTok{"Positive Sentiment"}\NormalTok{, }\DataTypeTok{size =} \DecValTok{5}\NormalTok{) }\OperatorTok{+}
\StringTok{  }\KeywordTok{geom_text}\NormalTok{(}\KeywordTok{aes}\NormalTok{(}\DataTypeTok{x =} \KeywordTok{as_datetime}\NormalTok{(}\StringTok{"2019-11-02 19:00:00"}\NormalTok{, }\DataTypeTok{tz =} \StringTok{"America/Chicago"}\NormalTok{), }\DataTypeTok{y =} \DecValTok{-3}\NormalTok{), }\DataTypeTok{color =} \StringTok{"red"}\NormalTok{, }\DataTypeTok{label =} \StringTok{"Negative Sentiment"}\NormalTok{, }\DataTypeTok{size =} \DecValTok{5}\NormalTok{) }\OperatorTok{+}
\StringTok{  }\KeywordTok{labs}\NormalTok{(}\DataTypeTok{title =} \StringTok{"People's Sentiment on Twitter Positive Towards}\CharTok{\textbackslash{}n}\StringTok{Husker Volleyball's Win Against Penn State"}\NormalTok{,}
       \DataTypeTok{subtitle =} \StringTok{"Sentiment mostly positive throughout the game"}\NormalTok{,}
       \DataTypeTok{caption =} \StringTok{"Source: #huskers and #GBR Tweets, 2019-11-02 | By Collin K. Berke"}\NormalTok{,}
       \DataTypeTok{y =} \StringTok{"Sentiment"}\NormalTok{,}
       \DataTypeTok{x =} \StringTok{"Central Standard Time (CST)"}\NormalTok{) }\OperatorTok{+}\StringTok{ }
\StringTok{  }\KeywordTok{scale_y_continuous}\NormalTok{(}\DataTypeTok{limits =} \KeywordTok{c}\NormalTok{(}\OperatorTok{-}\DecValTok{4}\NormalTok{, }\DecValTok{4}\NormalTok{)) }\OperatorTok{+}
\StringTok{  }\KeywordTok{theme_minimal}\NormalTok{() }\OperatorTok{+}
\StringTok{  }\KeywordTok{theme}\NormalTok{(}\DataTypeTok{axis.title.x =} \KeywordTok{element_blank}\NormalTok{(),}
        \DataTypeTok{plot.title =} \KeywordTok{element_text}\NormalTok{(}\DataTypeTok{size =} \DecValTok{16}\NormalTok{, }\DataTypeTok{face =} \StringTok{"bold"}\NormalTok{),}
    \DataTypeTok{axis.title =} \KeywordTok{element_text}\NormalTok{(}\DataTypeTok{size =} \DecValTok{10}\NormalTok{),}
    \DataTypeTok{axis.text =} \KeywordTok{element_text}\NormalTok{(}\DataTypeTok{size =} \DecValTok{7}\NormalTok{)}
\NormalTok{        )}
\end{Highlighting}
\end{Shaded}

\includegraphics{SportsData_files/figure-latex/unnamed-chunk-312-1.pdf}

When looking at this trend line, you can see that during the volleyball game, tweets using the \#husker and \#gbr hashtags had some wide variation in sentiment. Overall it seems that tweets during the volleyball match were mostly positive, where at times it dipped negative. Why might this be the case? Well, unfortunately, even though this was a big game for Husker volleyball, not many people were tweeting during the match (take a look at the number of tweets chart above). So if there was one word used in a tweet that was ranked as very negative in sentiment, it would have easily drove our average sentiment into the negative region quickly.

Also, we need to consider that this match took place after the Huskers loss to Purdue, which we will examine in the next example. This is important to know because people during the volleyball match may have also been tweeting about how poorly the football game went earlier in the day. Thus, low tweet volume mixed with the potential for tweets referencing something other than the match at hand may have had some influence on the sentiment scores.

There's also one last thing to keep in mind when you draw conclusions from this type of text data. Language is complex--it can have multiple meanings, which is highly influenced by context. Take for example the word `destroy', like its use in the following statement: ``This team is going to destroy the defense today.'' Although we clearly can see this is a positive statement, when a computer applies sentiment scores, the context of the statement is stripped away, and destroy will be scored as negative sentiment. In short, computers are not smart enough to include context when they calculate sentiment, yet. So, keep this limitation in mind when you draw conclusions from your sentiment analyses using text data.

\hypertarget{example-2---nebraskas-loss-to-purdue-what-were-fans-sentiments-towards-this-loss}{%
\subsection{Example 2 - Nebraska's loss to Purdue, what were fan's sentiments towards this loss?}\label{example-2---nebraskas-loss-to-purdue-what-were-fans-sentiments-towards-this-loss}}

The Nebraska Cornhuskers--\href{https://nebraska.rivals.com/news/nebraska-at-purdue-keys-to-victory-hol-score-predictions}{a 3-point favorite going into West Lafayette, IN}--squared off with the Purdue Boilermakers on November 2, 2019. Purdue was 2-6 on the season. Nebraska, with a 4-4 record coming off of a 38-31 home loss to Indiana, had many fans hoping Scott Frost could lead his team to a must needed win. \href{https://journalstar.com/sports/huskers/sipple/steven-m-sipple-frost-s-reaction-to-moos-six-win/article_bcebc067-865e-59a7-948d-c99290201294.html}{Especially given the expectation was the Cornhuskers would go 6-6 on the season}, and the team still had to play Wisconsin (6-2), Maryland (3-6), and Iowa (6-2) to get to those needed six wins to become bowl eligible. So, how did people particularly take this loss? Let's use our Twitter data to get an answer.

Again, we need to fix the time zone with the \texttt{with\_tz()} function so the data is represented in Central Standard Time (CST). Then we apply our \texttt{filter()} command to window our data to when the game was taking place.

\begin{Shaded}
\begin{Highlighting}[]
\NormalTok{football_tweets <-}\StringTok{ }\NormalTok{tweet_data }\OperatorTok
\StringTok{  }\KeywordTok{mutate}\NormalTok{(}\DataTypeTok{created_at =} \KeywordTok{with_tz}\NormalTok{(created_at, }\DataTypeTok{tzone =} \StringTok{"America/Chicago"}\NormalTok{),}
         \DataTypeTok{created_at =} \KeywordTok{round_time}\NormalTok{(created_at, }\StringTok{"5 mins"}\NormalTok{, }\DataTypeTok{tz =} \StringTok{"America/Chicago"}\NormalTok{)) }\OperatorTok\StringTok{ }
\StringTok{  }\KeywordTok{filter}\NormalTok{(created_at }\OperatorTok{>=}\StringTok{ "2019-11-02 11:00:00"} \OperatorTok{&}\StringTok{ }\NormalTok{created_at }\OperatorTok{<=}\StringTok{ "2019-11-02 15:30:00"}\NormalTok{) }
\end{Highlighting}
\end{Shaded}

Now, let's just get a sense of the number of tweets that occurred at certain points in the game. We again need to do some data wrangling with \texttt{group\_by()}, and then we use the \texttt{count()} function to add up all the tweets during each five minute interval. Once the data is wrangled, we can use our \texttt{ggplot} code to visualize tweet volume throughout the game.

\begin{Shaded}
\begin{Highlighting}[]
\NormalTok{football_time <-}\StringTok{ }\NormalTok{football_tweets }\OperatorTok
\StringTok{  }\KeywordTok{group_by}\NormalTok{(created_at) }\OperatorTok\StringTok{ }
\StringTok{  }\KeywordTok{count}\NormalTok{()}

\KeywordTok{ggplot}\NormalTok{() }\OperatorTok{+}
\StringTok{  }\KeywordTok{geom_line}\NormalTok{(}\DataTypeTok{data =}\NormalTok{ football_time, }\KeywordTok{aes}\NormalTok{(}\DataTypeTok{x =}\NormalTok{ created_at, }\DataTypeTok{y =}\NormalTok{ n)) }\OperatorTok{+}
\StringTok{  }\KeywordTok{theme_minimal}\NormalTok{() }\OperatorTok{+}
\StringTok{  }\KeywordTok{labs}\NormalTok{(}\DataTypeTok{y =} \StringTok{"Number of Tweets"}\NormalTok{)}
\end{Highlighting}
\end{Shaded}

\includegraphics{SportsData_files/figure-latex/unnamed-chunk-314-1.pdf}

Looking at this plot, we can see the tweet volume is a lot higher than that of the volleyball match. In fact, it looks like towards the end of the game there was a five minute interval where \textasciitilde{}80 or so tweets occurred. Given the outcome of the game, I assume people were not real happy during this spike in activity. Well we have the tools to answer this question.

As before, let's pull in our sentiment library with the \texttt{get\_sentiments()} function. Then lets tidy up our tweet data using the \texttt{unnest\_tokens()} and \texttt{anti\_join(stop\_words)}. Remember this step just places every word within a tweet on its own row and filters out any words that don't have any real meaning to the calculation of sentiment (i.e., and, the, a, etc.).

\begin{Shaded}
\begin{Highlighting}[]
\NormalTok{afinn <-}\StringTok{ }\KeywordTok{get_sentiments}\NormalTok{(}\StringTok{"afinn"}\NormalTok{)}

\NormalTok{football_tweets_tidy <-}\StringTok{ }\NormalTok{football_tweets }\OperatorTok\StringTok{ }
\StringTok{  }\KeywordTok{unnest_tokens}\NormalTok{(word, text) }\OperatorTok\StringTok{ }
\StringTok{  }\KeywordTok{anti_join}\NormalTok{(stop_words) }
\end{Highlighting}
\end{Shaded}

\begin{verbatim}
## Joining, by = "word"
\end{verbatim}

We now have our clean textual data, let's apply sentiment scores for each word using the \texttt{inner\_join(affin)} function, \texttt{group\_by(created\_at)} to create a group for each five minute interval, and then use \texttt{summarise()} to calculate then mean sentiment for each time period.

You can then use the \texttt{ggplot} code to plot sentiment over time, introduce annotations to highlight specific aspects within our plot, and then apply styling to the plot to move it closer to publication readiness. Now for the big reveal, how did people take the loss to a 2-6 Purdue? Run the code and find out.

\begin{Shaded}
\begin{Highlighting}[]
\NormalTok{football_tweets_sentiment <-}\StringTok{ }\NormalTok{football_tweets_tidy }\OperatorTok
\StringTok{  }\KeywordTok{inner_join}\NormalTok{(afinn) }\OperatorTok\StringTok{ }
\StringTok{  }\KeywordTok{group_by}\NormalTok{(created_at) }\OperatorTok\StringTok{ }
\StringTok{  }\KeywordTok{summarise}\NormalTok{(}\DataTypeTok{sentiment =} \KeywordTok{mean}\NormalTok{(value)) }\OperatorTok\StringTok{ }
\StringTok{  }\KeywordTok{arrange}\NormalTok{(sentiment)}
\end{Highlighting}
\end{Shaded}

\begin{verbatim}
## Joining, by = "word"
\end{verbatim}

\begin{Shaded}
\begin{Highlighting}[]
\KeywordTok{ggplot}\NormalTok{() }\OperatorTok{+}
\StringTok{  }\KeywordTok{geom_line}\NormalTok{(}\DataTypeTok{data =}\NormalTok{ football_tweets_sentiment, }\KeywordTok{aes}\NormalTok{(}\DataTypeTok{x =}\NormalTok{ created_at, }\DataTypeTok{y =}\NormalTok{ sentiment)) }\OperatorTok{+}
\StringTok{  }\KeywordTok{geom_text}\NormalTok{(}\KeywordTok{aes}\NormalTok{(}\DataTypeTok{x =} \KeywordTok{as_datetime}\NormalTok{(}\StringTok{"2019-11-02 14:30:00"}\NormalTok{, }\DataTypeTok{tz =} \StringTok{"America/Chicago"}\NormalTok{), }\DataTypeTok{y =} \DecValTok{2}\NormalTok{), }\DataTypeTok{color =} \StringTok{"darkgreen"}\NormalTok{, }\DataTypeTok{label =} \StringTok{"Positive Sentiment"}\NormalTok{, }\DataTypeTok{size =} \DecValTok{5}\NormalTok{) }\OperatorTok{+}
\StringTok{  }\KeywordTok{geom_text}\NormalTok{(}\KeywordTok{aes}\NormalTok{(}\DataTypeTok{x =} \KeywordTok{as_datetime}\NormalTok{(}\StringTok{"2019-11-02 14:30:00"}\NormalTok{, }\DataTypeTok{tz =} \StringTok{"America/Chicago"}\NormalTok{), }\DataTypeTok{y =} \DecValTok{-2}\NormalTok{), }\DataTypeTok{color =} \StringTok{"red"}\NormalTok{, }\DataTypeTok{label =} \StringTok{"Negative Sentiment"}\NormalTok{, }\DataTypeTok{size =} \DecValTok{5}\NormalTok{) }\OperatorTok{+}
\StringTok{  }\KeywordTok{scale_y_continuous}\NormalTok{(}\DataTypeTok{limits =} \KeywordTok{c}\NormalTok{(}\OperatorTok{-}\FloatTok{2.5}\NormalTok{, }\FloatTok{2.5}\NormalTok{)) }\OperatorTok{+}
\StringTok{  }\KeywordTok{labs}\NormalTok{(}\DataTypeTok{title =} \StringTok{"People's Sentiment on Twitter Negative}\CharTok{\textbackslash{}n}\StringTok{Towards Husker Football's Loss to Purdue"}\NormalTok{,}
       \DataTypeTok{subtitle =} \StringTok{"People were positive at the start and part of the first half,}\CharTok{\textbackslash{}n}\StringTok{then negative throughout"}\NormalTok{,}
       \DataTypeTok{caption =} \StringTok{"Source: #huskers and #GBR Tweets, 2019-11-02 | By Collin K. Berke"}\NormalTok{,}
       \DataTypeTok{y =} \StringTok{"Sentiment"}\NormalTok{,}
       \DataTypeTok{x =} \StringTok{"Central Standard Time (CST)"}\NormalTok{) }\OperatorTok{+}\StringTok{ }
\StringTok{  }\KeywordTok{theme_minimal}\NormalTok{() }\OperatorTok{+}
\StringTok{  }\KeywordTok{theme}\NormalTok{(}\DataTypeTok{axis.title.x =} \KeywordTok{element_blank}\NormalTok{(),}
        \DataTypeTok{plot.title =} \KeywordTok{element_text}\NormalTok{(}\DataTypeTok{size =} \DecValTok{16}\NormalTok{, }\DataTypeTok{face =} \StringTok{"bold"}\NormalTok{),}
    \DataTypeTok{axis.title =} \KeywordTok{element_text}\NormalTok{(}\DataTypeTok{size =} \DecValTok{10}\NormalTok{),}
    \DataTypeTok{axis.text =} \KeywordTok{element_text}\NormalTok{(}\DataTypeTok{size =} \DecValTok{7}\NormalTok{)}
\NormalTok{  )}
\end{Highlighting}
\end{Shaded}

\includegraphics{SportsData_files/figure-latex/unnamed-chunk-316-1.pdf}

As you can see, it started out pretty positive. Then, it started to go negative throughout the first half. However, there was a bump, which was around the time D-Lineman, Darrion Daniels almost scored a pick six. Around halftime, we can see a little bit of a bump towards positive sentiment. This was probably most likely due to people cheering on the Huskers to come out strong after the half. As the second half progressed, you can see things turned for the worst again, and sentiment became negative up until the end of the game, most likely because people realized they were going to get another L on the schedule. You can relive all this excitement again by catching the game recap \href{https://www.youtube.com/watch?v=m0hKH6Zb0vY\&feature=onebox}{here}.

\hypertarget{arranging-multiple-plots-together}{%
\chapter{Arranging multiple plots together}\label{arranging-multiple-plots-together}}

Sometimes you have two or three (or more) charts that are really just one chart that you need to merge them together. It would be nice to be able to arrange them programmatically and not have to mess with it in illustrator.

Good news.

There is.

It's called \texttt{cowplot}, and it's pretty easy to use. First install cowplot with install.packages(``cowplot''). Then let's load tidyverse and cowplot.

\begin{Shaded}
\begin{Highlighting}[]
\KeywordTok{library}\NormalTok{(tidyverse)}
\KeywordTok{library}\NormalTok{(cowplot)}
\end{Highlighting}
\end{Shaded}

What follows is just stuff for me to set up a couple of bar charts. You can run it -- it'll work on your machine without changing a thing -- but what I'm doing here isn't important. The stuff you need to do is below.

\begin{Shaded}
\begin{Highlighting}[]
\NormalTok{attendance <-}\StringTok{ }\KeywordTok{read_csv}\NormalTok{(}\StringTok{"https://raw.githubusercontent.com/mattwaite/sportsdatabook/Master/data/attendance.csv"}\NormalTok{)}
\end{Highlighting}
\end{Shaded}

\begin{verbatim}
## Parsed with column specification:
## cols(
##   Institution = col_character(),
##   Conference = col_character(),
##   `2013` = col_double(),
##   `2014` = col_double(),
##   `2015` = col_double(),
##   `2016` = col_double(),
##   `2017` = col_double(),
##   `2018` = col_double()
## )
\end{verbatim}

Making a quick percent change.

\begin{Shaded}
\begin{Highlighting}[]
\NormalTok{attendance <-}\StringTok{ }\NormalTok{attendance }\OperatorTok\StringTok{ }\KeywordTok{mutate}\NormalTok{(}\DataTypeTok{change =}\NormalTok{ ((}\StringTok{`}\DataTypeTok{2018}\StringTok{`}\OperatorTok{-}\StringTok{`}\DataTypeTok{2017}\StringTok{`}\NormalTok{)}\OperatorTok{/}\StringTok{`}\DataTypeTok{2017}\StringTok{`}\NormalTok{)}\OperatorTok{*}\DecValTok{100}\NormalTok{)}
\end{Highlighting}
\end{Shaded}

Let's chart the top 10 and bottom 10 of college football ticket growth \ldots{} and shrinkage.

\begin{Shaded}
\begin{Highlighting}[]
\NormalTok{top10 <-}\StringTok{ }\NormalTok{attendance }\OperatorTok\StringTok{ }\KeywordTok{top_n}\NormalTok{(}\DecValTok{10}\NormalTok{, }\DataTypeTok{wt=}\NormalTok{change)}
\NormalTok{bottom10 <-}\StringTok{ }\NormalTok{attendance }\OperatorTok\StringTok{ }\KeywordTok{top_n}\NormalTok{(}\DecValTok{10}\NormalTok{, }\DataTypeTok{wt=}\OperatorTok{-}\NormalTok{change)}
\end{Highlighting}
\end{Shaded}

Okay, now to do this I need to save my plots to an object. We do this the same way we save things to a dataframe -- with the arrow. We'll make two identical bar charts, one with the top 10 and one with the bottom 10.

\begin{Shaded}
\begin{Highlighting}[]
\NormalTok{bar1 <-}\StringTok{ }\KeywordTok{ggplot}\NormalTok{() }\OperatorTok{+}\StringTok{ }\KeywordTok{geom_bar}\NormalTok{(}\DataTypeTok{data=}\NormalTok{top10, }\KeywordTok{aes}\NormalTok{(}\DataTypeTok{x=}\KeywordTok{reorder}\NormalTok{(Institution, change), }\DataTypeTok{weight=}\NormalTok{change)) }\OperatorTok{+}\StringTok{ }\KeywordTok{coord_flip}\NormalTok{()}
\end{Highlighting}
\end{Shaded}

\begin{Shaded}
\begin{Highlighting}[]
\NormalTok{bar2 <-}\StringTok{ }\KeywordTok{ggplot}\NormalTok{() }\OperatorTok{+}\StringTok{ }\KeywordTok{geom_bar}\NormalTok{(}\DataTypeTok{data=}\NormalTok{bottom10, }\KeywordTok{aes}\NormalTok{(}\DataTypeTok{x=}\KeywordTok{reorder}\NormalTok{(Institution, change), }\DataTypeTok{weight=}\NormalTok{change)) }\OperatorTok{+}\StringTok{ }\KeywordTok{coord_flip}\NormalTok{()}
\end{Highlighting}
\end{Shaded}

With cowplot, we can use a function called \texttt{plot\_grid} to arrange the charts:

\begin{Shaded}
\begin{Highlighting}[]
\KeywordTok{plot_grid}\NormalTok{(bar1, bar2) }
\end{Highlighting}
\end{Shaded}

\includegraphics{SportsData_files/figure-latex/unnamed-chunk-323-1.pdf}

We can also stack them on top of each other:

\begin{Shaded}
\begin{Highlighting}[]
\KeywordTok{plot_grid}\NormalTok{(bar1, bar2, }\DataTypeTok{ncol=}\DecValTok{1}\NormalTok{) }
\end{Highlighting}
\end{Shaded}

\includegraphics{SportsData_files/figure-latex/unnamed-chunk-324-1.pdf}

To make these publishable, we should add headlines, chatter, decent labels, credit lines, etc. But to do this, we'll have to figure out which labels go on which charts, so we can make it look decent. For example -- both charts don't need x or y labels. If you don't have a title and subtitle on both, the spacing is off, so you need to leave one blank or the other blank. You'll just have to fiddle with it until you get it looking right.

\begin{Shaded}
\begin{Highlighting}[]
\NormalTok{bar1 <-}\StringTok{ }\KeywordTok{ggplot}\NormalTok{() }\OperatorTok{+}\StringTok{ }\KeywordTok{geom_bar}\NormalTok{(}\DataTypeTok{data=}\NormalTok{top10, }\KeywordTok{aes}\NormalTok{(}\DataTypeTok{x=}\KeywordTok{reorder}\NormalTok{(Institution, change), }\DataTypeTok{weight=}\NormalTok{change)) }\OperatorTok{+}\StringTok{ }\KeywordTok{coord_flip}\NormalTok{() }\OperatorTok{+}\StringTok{ }\KeywordTok{labs}\NormalTok{(}\DataTypeTok{title=}\StringTok{"College football winners..."}\NormalTok{, }\DataTypeTok{subtitle =} \StringTok{"Not every football program saw attendance shrink in 2018. But some really did."}\NormalTok{,  }\DataTypeTok{x=}\StringTok{""}\NormalTok{, }\DataTypeTok{y=}\StringTok{"Percent change"}\NormalTok{, }\DataTypeTok{caption =} \StringTok{""}\NormalTok{) }\OperatorTok{+}\StringTok{ }\KeywordTok{theme_minimal}\NormalTok{() }\OperatorTok{+}\StringTok{ }
\StringTok{  }\KeywordTok{theme}\NormalTok{(}
    \DataTypeTok{plot.title =} \KeywordTok{element_text}\NormalTok{(}\DataTypeTok{size =} \DecValTok{16}\NormalTok{, }\DataTypeTok{face =} \StringTok{"bold"}\NormalTok{),}
    \DataTypeTok{axis.title =} \KeywordTok{element_text}\NormalTok{(}\DataTypeTok{size =} \DecValTok{8}\NormalTok{), }
    \DataTypeTok{plot.subtitle =} \KeywordTok{element_text}\NormalTok{(}\DataTypeTok{size=}\DecValTok{10}\NormalTok{), }
    \DataTypeTok{panel.grid.minor =} \KeywordTok{element_blank}\NormalTok{()}
\NormalTok{    )}
\end{Highlighting}
\end{Shaded}

\begin{Shaded}
\begin{Highlighting}[]
\NormalTok{bar2 <-}\StringTok{ }\KeywordTok{ggplot}\NormalTok{() }\OperatorTok{+}\StringTok{ }\KeywordTok{geom_bar}\NormalTok{(}\DataTypeTok{data=}\NormalTok{bottom10, }\KeywordTok{aes}\NormalTok{(}\DataTypeTok{x=}\KeywordTok{reorder}\NormalTok{(Institution, change), }\DataTypeTok{weight=}\NormalTok{change)) }\OperatorTok{+}\StringTok{ }\KeywordTok{coord_flip}\NormalTok{() }\OperatorTok{+}\StringTok{  }\KeywordTok{labs}\NormalTok{(}\DataTypeTok{title =} \StringTok{"... and losers"}\NormalTok{, }\DataTypeTok{subtitle=} \StringTok{""}\NormalTok{, }\DataTypeTok{x=}\StringTok{""}\NormalTok{, }\DataTypeTok{y=}\StringTok{""}\NormalTok{,  }\DataTypeTok{caption=}\StringTok{"Source: NCAA | By Matt Waite"}\NormalTok{) }\OperatorTok{+}\StringTok{ }\KeywordTok{theme_minimal}\NormalTok{() }\OperatorTok{+}\StringTok{ }
\StringTok{  }\KeywordTok{theme}\NormalTok{(}
    \DataTypeTok{plot.title =} \KeywordTok{element_text}\NormalTok{(}\DataTypeTok{size =} \DecValTok{16}\NormalTok{, }\DataTypeTok{face =} \StringTok{"bold"}\NormalTok{),}
    \DataTypeTok{axis.title =} \KeywordTok{element_text}\NormalTok{(}\DataTypeTok{size =} \DecValTok{8}\NormalTok{), }
    \DataTypeTok{plot.subtitle =} \KeywordTok{element_text}\NormalTok{(}\DataTypeTok{size=}\DecValTok{10}\NormalTok{), }
    \DataTypeTok{panel.grid.minor =} \KeywordTok{element_blank}\NormalTok{()}
\NormalTok{    )}
\end{Highlighting}
\end{Shaded}

Saving a cowplot plot\_grid is the same as anything else we did in the class:

\begin{Shaded}
\begin{Highlighting}[]
\KeywordTok{plot_grid}\NormalTok{(bar1, bar2) }\OperatorTok{+}\StringTok{ }\KeywordTok{ggsave}\NormalTok{(}\StringTok{"test.png"}\NormalTok{)}
\end{Highlighting}
\end{Shaded}

\begin{verbatim}
## Saving 6.5 x 4.5 in image
\end{verbatim}

\includegraphics{SportsData_files/figure-latex/unnamed-chunk-327-1.pdf}

\hypertarget{assignments}{%
\chapter{Assignments}\label{assignments}}

This is a collection of assignments I've used in my Sports Data Analysis and Visualization course at the University of Nebraska-Lincoln. The overriding philosophy is to have students do lots of small assignments that directly apply what they learned, and often pulling from other assignments. Each small assignment is just a few points each -- I make them 5 points each and make the grading a yes/no decision on 5 different questions -- so a bad grade on one doesn't matter. Then, twice during the semester, I have them create blog posts with visualizations on a topic of their choosing. The topic must have a point of view -- Nebraska's woes on third down are why the team is struggling, for example -- and must be backed up with data. They have to write a completely documented R Notebook explaining what they did and why; they have to write a publicly facing blog post for a general audience and that post has to have at least three graphs backing up their point; and they have to give a lightning talk (no more than five minutes) in class about what they have found. Those two assignments are typically worth 50 percent of the course grade.

I think rubrics are crap, but I give students these questions as a guide to what I'm expecting:

\begin{enumerate}
\def\labelenumi{\arabic{enumi}.}
\tightlist
\item
  Did you read the data into a dataframe?
\item
  Did you use the skill discussed in the chapter correctly?
\item
  Did you answer all the questions posed by the assignment?
\item
  Did you use Markdown comments to explain your steps, what you did and why?
\end{enumerate}

\textbf{Chapter 1: Intro}

\begin{itemize}
\tightlist
\item
  Install \href{https://slack.com/get}{Slack} on your computer and your phone.
\item
  If on a Mac, \href{http://osxdaily.com/2014/02/12/install-command-line-tools-mac-os-x/}{install the Command Line Tools}.
\item
  Install \href{https://rweb.crmda.ku.edu/cran/}{R for your computer}.
\item
  Install \href{https://www.rstudio.com/products/rstudio/download/\#download}{R Studio Desktop for your computer} ONLY AFTER YOU HAVE INSTALLED R
\end{itemize}

\textbf{Chapter 2: Basics}

Part 1:

In the console, type \texttt{install.packages("swirl")}

Then \texttt{library("swirl")}

Then \texttt{swirl()}

Follow instructions on the screen. Each time you are asked if which one you want, you want the first one. The basics, the beginning, the first parts. All the first ones. Then just follow the instructions on the screen.

Part 2:

Create an R notebook (which you should have done if you were following along). In it, delete all the generated text from it, so you have a blank document. Then, write a sentence in the document telling me what the last thing you did in Swirl was. Then add a code block (see about inserting code in the chapter) and add two numbers together. Any two numbers. Run that code block. Save the file and submit the .Rmd file created when you save it to Canvas. That's it. Simple.

\textbf{Chapter 3: Data, structures and types}

Using what you learned in the chapter, fetch \href{http://www.cfbstats.com/2018/leader/827/player/split01/category19/sort01.html}{the list of the Big Ten's leading tacklers}. Submit the CSV file to Canvas. In the comments, label each field type. What are they? Dates? Characters? Numeric?

\textbf{Chapter 4: Aggregates}

Import \href{https://unl.box.com/s/a8m91bro10t89watsyo13yjegb1fy009}{this dataset of every college basketball game in the 2018-19 season}. Using what you learned in the chapter, answer the following questions:

\begin{enumerate}
\def\labelenumi{\arabic{enumi}.}
\tightlist
\item
  What team shot the most shots?
\item
  What team averaged the most shots?
\item
  What team had the highest median number of shots?
\item
  How much difference is there between the top average shots team and the top median shots team? Why do you think that is?
\end{enumerate}

\textbf{Chapter 5: Mutating Data}

Import \href{https://unl.box.com/s/a8m91bro10t89watsyo13yjegb1fy009}{this dataset of every college basketball game in the 2018-19 season}. Using what you learned in the chapter, mutate a new variable: differential. Differential is the difference between the team score and the opponent score. A positive number means the team in question won. A negative number means the team in question lost. After creating the differential, average them together and sort them in descending order. Which team had the highest average point differential in college basketball? In other words, which team consistently won by the largest margins?

\textbf{Chapter 6: Filters and selections}

Import the data of \href{https://unl.box.com/s/s1wzw61u9ia50qmirfhuvprgpmmah9rj}{every college basketball player's season stats in 2018-19 season}. Using this data, let's get closer to a real answer to where the cutoff for true shooting season should be from the chapter. First, find the median number of shots attempted in the season, then set the cutoff filter for who had the best true shooting percentage using that number.

\textbf{Chapter 7: Transforming data}

Import this dataset of \href{https://unl.box.com/s/fs3rj0dns1xh2y1dx0c2yc0adh4u3zsy}{college football attendance data from 2013-2018}. This data is long data -- one team, one year, one row. We need it to be wide data. Hint: it'll be much easier if you select only the columns you need to make it wide instead of using them all. Submit your notebook.

\textbf{Chapter 8: Simulations}

On Feb.~6, Nebraska's basketball team had a nightmare night shooting the ball. They attempted 57 shots \ldots{} and made only 12. The team shot .429 on the season. Simulate 1000 games of them taking 57 shots using their season long .429 as the probability that they'll make a shot. How many times do they make just 12?

\textbf{Chapter 9: Correlations and regressions}

Do the same thing described in the chapter, but for defense. Report your R-squared number, your p-value, what those mean and from that, how close does it come to predicting the Iowa Nebraska game?

\textbf{Chapter 10: Multiple regression}

You have been hired by Fred Hoiberg to build a team. He's interested in the model started in the chapter, but wants more.

There are more predictors to be added to our model. You are to find two. Two that contribute to the predictive quality of the model without largely overlapping another predictor.

In your notebook, report the adjusted r-squared you achieved.

You are to generate a new set of coefficients, a new formula and a new set of numbers of what a conference champion would expect in terms of differential. I've done a lot of work for you. Continue it. Add two more predictors and complete the prediction. And compare that to Nebraska of this season.

Turn in your notebook with these answers and comments to the code you added, making sure to add WHY you are doing things. Why did you select those two variables.

\textbf{Chapter 11: Residuals}

Using the same data from the chapter, model defensive third down percentage and defensive points allowed. Which teams are overperforming that model given the residual analysis?

\textbf{Chapter 12: Z scores}

Refine the composite Z Score I started in the chapter. Add two more elements to it. What else do you think is important to the success of a basketball team? I've got shooting, rebounds and the opponents shooting. What else would you add?

In an R Notebook, make your case for the two elements you are adding -- what is your logic? Why these two?. Then, follow my steps here until you get to the \texttt{teamquality} dataframe step, where you'll need to add the fields you are going to add to the composite. Then you'll need to add your fields to the \texttt{teamtotals} dataframe. Then you'll need to adjust \texttt{teamzscore}.

Finally, look at your ranking of Big Ten teams and compare it to mine. Did adding more elements to the composite change anything? Explain the differences in your notebook. Which one do you think is more accurate? I won't be offended if you say yours, but why do you feel that way?

\textbf{Chapter 13: Intro to ggplot}

\href{https://unl.box.com/s/hvxmnxhr41x4ikgt3vk38aczcbrf97pn}{Take this same attendance data}. I want you to produce a bar chart of the top 10 schools by percent change in attendance between 2018 and 2013. I want you to change the title and the labels and I want you to apply a theme different from the ones I used above. You can find \href{https://ggplot2.tidyverse.org/reference/ggtheme.html}{more themes in the ggplot documentation}.

\textbf{Chapter 14: Stacked bar charts}

I want you to make this same chart, except I want you to make the weight the percentage of the total number of graduates that gender represents. You'll be mutating a new field to create that percentage. You'll then chart it with the fill. The end result should be a stacked bar chart allowing you to compare genders between universities. Answer the following question: Which schools have the largest gender imbalances?

\textbf{Chapter 15: Waffle charts}

Compare Nebraska and Michgan's night on the basketball court using a Waffle chart and another metric than what I've done above for the game.

\href{https://github.com/hrbrmstr/waffle}{Here's the library's documentation}.
\href{https://www.sports-reference.com/cbb/boxscores/2019-02-28-19-michigan.html}{Here's the stats from the game}.

Turn in your notebook with your waffle chart. It must contain these two things:

\begin{itemize}
\tightlist
\item
  Your waffle chart
\item
  A written narrative of what it says. What does your waffle chart say about how that game turned out?
\end{itemize}

\textbf{Chapter 16: Line Charts}

Import \href{https://unl.box.com/s/a8m91bro10t89watsyo13yjegb1fy009}{this dataset of every college basketball game in the 2018-19 season}.

\begin{itemize}
\item
  How does Nebraska's shooting percentage compare to the Big Ten over the season? Put the Big Ten on the same chart as Nebraska, you'll need two dataframes, two geoms and with your Big Ten dataframe, you need to use \texttt{group} in the aesthetic.
\item
  After working on this chart, your boss comes in and says they don't care about field goal percentage anymore. They just care about three-point shooting because they read on some blog that three-point shooting was all the rage. Change what you need to change to make your line chart now about how the season has gone behind the three-point line. How does Nebraska compare to the rest of the Big Ten?
\end{itemize}

\textbf{Chapter 17: Step charts}

Re-make the chart in the chapter, but with rebounding. I want you to visualize the differential between our rebounds and their rebounds, and then plot the step chart showing over the course of the season. Highlight Nebraska. Highlight the top team. Add annotation layers to label both of them.

\textbf{Chapter 18: Ridge charts}

You've been hired by Fred Hoiberg to tell him how to win the Big Ten. He's not impressed with that I came up with. So what you need to do is look for a \emph{composite} measure that produces a meaningful ridgeplot. What that means is you're going to mutate \texttt{wintotalgroupinglogs} one more time. Is the differential between rebounding meaningful instead of just the total? Or assists? Or something else? Your call. Your goal is to produce a ridgeplot that tells The Mayor he needs to focus on doing X better than the opponent to win a Big Ten title.

\textbf{Chapter 19: Lollipop charts}

You've been hired by Fred Hoiberg to tell him how to win the Big Ten. He's not impressed with that I came up with. So what else could you look at with lollipop charts? Your call. Your goal is to produce a lollipop chart that tells The Mayor he needs to focus on the gap between X and Y if he wants to win a Big Ten title.

\textbf{Chapter 20: Scatterplot}

Using the data from the walkthrough, model and graph two other elements of Nebraska's season versus wins. How much does your choices of metrics predict the season? What do the scatterplots of what you chose look like? What do the linear models say (r-squared, p-values)? How predictive are they, i.e.~using y=mx+b, how close to Nebraska's win total do your models get to?

\textbf{Chapter 21: Facet Wraps}

Which Big Ten teams were good a shooting three point shots? Which teams weren't? Using a facet grid, chart each teams three point shooting season against the league average.

\textbf{Chapter 22}

Import \href{https://unl.box.com/s/a8m91bro10t89watsyo13yjegb1fy009}{this dataset of every college basketball game in the 2018-19 season}.

Create a dataframe that shows the 10 best or 10 worst at something. Or rank the Big Ten. Your choice. Then use formattable to show in both a table and visually how the best were different from the worst.

Export it to a PNG using the example above. Then, in Illustrator, add a headline, chatter, source and credit lines. Turn in the PNG file you export from Illustrator.

\textbf{New Chapter}

Import \href{https://unl.box.com/s/a8m91bro10t89watsyo13yjegb1fy009}{this dataset of every college basketball game in the 2018-19 season}. Create a bubble chart looking at two stats to make your scatterplot and a third making the size of your bubble. Make the color the conference name.

I want to see your bubble chart, but more importantly, I want you to discuss if what you came up with makes an effective bubble chart. Does it tell a story? Does the size of the bubble enhance understanding? No is an acceptable answer. But explain why it did or didn't work.

\textbf{New Chapter}

Your turn: Let's evaluate the second part of the quote from the chapter: November basketball tells you where you are.

We've looked at wins. What else could you look at over the course of the season that tells you where you are? Pick a metric. Explain your choice. Make a circular bar chart. Evaluate the result. What does it say?

\textbf{Chapter 23: Rvest}

I am a huge Premiere League fan, so I want data on the league. \href{https://fbref.com/en/comps/9/stats/Premier-League-Stats}{For now, I just want teams}. Scrape the team data at the top, but before you do, look at the header. Is it one row? Does that make it standard? Nope. So what now?

\textbf{Chapter 24: Advanced Rvest}

I don't usually assign an advanced rvest assignment because I don't want to turn 30 students loose on some poor provider's servers.

\textbf{Note:} There are no assignments for annotations and finishing touches. In my classes, I have students present two major visual stories where they have to incorporate the elements of those assignments as part of their grade.

\textbf{Chapter 30: Plotly}

First, create a simple ggplot like we did above exploring WRAA -- \href{http://m.mlb.com/glossary/advanced-stats/weighted-runs-above-average}{weighted runs above average} -- as your x value and and plate appearances (PA) as your y variable.

Next, create a plotly visualization using the same two variables. Alter the hover elements to show relevant data. If you leave it the same from the chapter, you lose points.

Export your plotly visualization to plotly's website. Include a link of your viz in your notebook. In your notebook, discuss the relative advantages and disadvantages of this interactive plot versus the static plots we've been doing.

\textbf{Chapter 31: Clustering}

We looked at who Cam Mack's peers are, but what about the team? Use k-means clustering on a \href{https://unl.box.com/s/qdqu5rbz7f9jtk04fhuiqrb9p0nx5a0z}{dataset of every college basketball team's season stats} and determine who Nebraska's peers are.

To complete this assignment, you'll need to pick the metrics you want to measure teams by. One note -- teams haven't played the same number of games, so it would be wise to either focus on the per game or percentage metrics, either using them or creating them yourself. You'll then need to scale them. You'll need to decide the optimal number of clusters (k) and then run them. Combine the data back, determine which cluster Nebraska is in in your clustering and then show Nebraska's peers.

In your notebook, write a few sentences and answering this question: Is the peer group Nebraska is in fair?

\textbf{Chapter 32: Rtweet}

Using the Rtweet library, gather tweets about the Maryland game Saturday night. WARNING: The API only lets you go back 18000 tweets -- we settled on 8000 in the walkthrough. If you wait too long to run the scrape after the game, you won't get the tweets you need to complete it. If you don't intend to do the assignment Saturday night, at least scrape the data and save it as a CSV as detailed in the chapter. You can then analyze it any time you want.

Analyze sentiment and chart it as in the chapter. Compare the Maryland game to the Purdue game. Are they different? Is sentiment better or worse for one or the other? Describe the differences in your notebook.

\hypertarget{appendix}{%
\chapter{Appendix}\label{appendix}}

These are some additional materials I use in my classes.

\hypertarget{how-to-get-help-in-this-class}{%
\section{How to get help in this class}\label{how-to-get-help-in-this-class}}

\begin{quote}
This is the contents of a document I send out every semester. I use Slack to help students with code problems outside of class. It's much easier than other options, such as email. A suggestion: set ground rules on when you will and won't answer Slack messages.
\end{quote}

\begin{enumerate}
\def\labelenumi{\arabic{enumi}.}
\tightlist
\item
  \textbf{Use Slack}. Email is a miserable way to handle technological questions. I'm not answering code questions via email. On Slack we can have a back and forth where we solve this quickly instead of waiting on each other to respond to an email.
\item
  \textbf{Don't use screenshots}. Tell me what you're trying to do and then copy and paste your code into the Slack message.
\item
  \textbf{Always copy and paste the error message you are getting}. There is a near infinite number of things you could have done and a nearly limitless number of errors you could be getting. Both help.
\end{enumerate}

\textbf{Slack tips}

\begin{itemize}
\tightlist
\item
  Slack uses Markdown in messages. Did you know your code blocks in R Notebook are Markdown themselves? If you copy the whole block -- with the ``` and everything, Slack will format it like this:
\end{itemize}

\begin{verbatim}
simulations <- rbinom(n = 1000, size = 39, prob = .309)

hist(simulations)

table(simulations)
\end{verbatim}

\begin{itemize}
\tightlist
\item
  Using the Slack app will also mean getting alerted to messages right away. If you are logging in through a web browser, you won't know when I've responded. If we're having a Slack conversation and I see you log in, send me a message, then disappear right away, I know you're using a browser and when that conversation drags because you aren't getting messages, \textbf{I'm going to get frustrated}. You likely aren't the only one asking for help at that moment.
\end{itemize}

\bibliography{packages.bib}

\end{document}
